% 122B_HW1.tex

\documentclass{article}
\usepackage{graphicx} % Required for inserting images
\usepackage[margin = 2.54cm]{geometry}
\usepackage[most]{tcolorbox}

\newtcolorbox{myBox}[3]{
arc=5mm,
lower separated=false,
fonttitle=\bfseries,
%colbacktitle=green!10,
%coltitle=green!50!black,
enhanced,
attach boxed title to top left={xshift=0.5cm,
        yshift=-2mm},
colframe=blue!50!black,
colback=blue!10
}

\usepackage{amsmath}
\usepackage{amssymb}
\usepackage{verbatim}
\usepackage[utf8]{inputenc}
\linespread{1.2}

\newtheorem{definition}{Definition}
\newtheorem{proposition}{Proposition}
\newtheorem{theorem}{Theorem}
\newtheorem{question}{Question}

\title{Math CS 122B HW1}
\author{Zih-Yu Hsieh}

\begin{document}
\maketitle

\section*{1}
\begin{myBox}[]{}
    \begin{question}
        Ahlfors Pg. 178 Problem 2:

        Show that the series
        $$\zeta(z)=\sum_{n=1}^{\infty}n^{-z}$$
        converges for $Re(z)>1$, and represent its derivative in series form.
    \end{question}
\end{myBox}

\textbf{Pf:}

The following proof would assume the domain of the above series is the half plane $Re(z)>1$.

\hfil

\textbf{The series converges pointwise:}

For all $z\in\mathbb{C}$ satisfying $Re(z)>1$, $z=a+bi$ for $a,b\in\mathbb{R}$, and $a>1$. Then, for any $n\in\mathbb{N}$,
the number $n^{-z}=e^{-z\log(n)} = e^{-(a+bi)\ln(n)} = e^{-a\ln(n)}\cdot e^{i(-b\ln(n))} = n^{-a}\cdot e^{i(-b\ln(a))}$. Hence, if taken the modulus $e^{-a\ln(n)}$,
since $a>1$, by p-series test, the following series converges:
$$\sum_{n=1}^{\infty}n^{-a}$$
Which, since the original series satisfies the following:
$$\sum_{n=1}^{\infty}|n^{-z}| = \sum_{n=1}^{\infty}\left|n^{-a}\cdot e^{i(-b\ln(n))}\right| = \sum_{n=1}^{\infty}n^{-a}$$
Hence, the series absolutely converges, which $\zeta(z)=\sum_{n=1}^{\infty}n^{-z}$ is defined on $Re(z)>1$.

\hfil

\textbf{Partial Sum converges uniformly on any Compact Subset:}

Suppose $K\subset \mathbb{C}$ is a compact subset of the plane $Re(z)>1$, each component $n^{-z}=e^{-z\ln(n)}=n^{-a}\cdot e^{i(-b\ln(n))}$ is analytic on the half plane (also on $K$), 
hence there exists $z_0\in K$, such that $|n^{-z_0}|$ yields the maximum.

Since for $z_0=a+bi$, $|n^{-z_0}|=\left|n^{-a}\cdot e^{i(-b\ln(n))}\right|=n^{-a}$, and since $z_0$ is in the half plane, so $Re(z_0)=a>1$. Then, the series $\sum_{n=1}^{\infty}n^{-a}$ converges.

Now, notice that for each $n\in\mathbb{N}$, $M_n=\sup_{z\in K}|n^{-z}| = \max_{z\in K}|n^{-z}|=n^{-a}$ satisfies $\sum_{n=1}^{\infty}M_n$ converges, then by Weierstrass M-Test,
the series $\sum_{n=1}^{\infty}n^{-z}$ in fact converges uniformly on $K$. 

Then, because the series $\zeta(z)=\sum_{n=1}^{\infty}n^{-z}$ converges absolutely on $Re(z)>1$, converging uniformly on all compact subset of the half plane, and each component is analytic on the half plane,
then by the theorem in Ahlfors pg. 176, $\zeta(z)$ is analytic, and the partial sum $\sum_{n=1}^{N}n^{-z}$ (for $N\in\mathbb{N}$) has derivative converges to  $\zeta'(z)$ uniformly on all compact subsets of the half plane.
Hence, based on the same theorem again, we can claim the folloing on the chosen half plane:
$$\zeta'(z)=\lim_{N\rightarrow\infty}\frac{d}{dz}\sum_{n=1}^{N}n^{-z} = \lim_{N\rightarrow\infty}\sum_{n=1}^{N}\frac{d}{dz}(e^{-z\ln(n)}) = \lim_{N\rightarrow\infty}\sum_{n=1}^{N}-\ln(n)e^{-z\ln(n)} = -\sum_{n=1}^{\infty}ln(n)n^{-z}$$

\hfil

\hfil

\section*{2}
\begin{myBox}[]{}
    \begin{question}
        Ahlfors Pg. 184 Problem 5:

        The Fibonacci numbers are defined by $c_0=0$, $c_1=1$, and $c_n=c_{n-1}+c_{n-2}$ for all $n\geq 2$.
        
        Show that the $c_n$ are Taylor Coefficients of a rational function, and determine a closed expression for $c_n$.
    \end{question}
\end{myBox}

\textbf{Pf:}

Consider the generating function, a formal power series defined as $F(z)=\sum_{n=0}^{\infty}c_nz^n$.

\hfil

\textbf{Radius of Convergence of the Power Series:}

Recall that radius of convergence of power series $R=\limsup(|c_n|^{\frac{1}{n}})^{-1}\in [0,\infty]$, where $c_n$ is the coefficients of each degree.

First, we can verify that for all $n\in\mathbb{N}$, $c_n<2^n$:

For base case $n=1$, $c_1=1<2^1$.

Now, suppose for given $n\geq 1$, $c_n<2^n$, then for case $(n+1)\geq 2$, since $c_{n+1}=c_n+c_{n-1} < 2\cdot c_n$ (since $c_{n}>c_{n-1}$), then $c_{n+1}<2\cdot c_n<2\cdot 2^n=2^{n+1}$
by induction hypothesis, which this completes the induction.

Since all $n\in\mathbb{N}$ has $0<c_n<2^n$, then $|c_n|^{\frac{1}{n}}=c_n^{\frac{1}{n}}<(2^n)^{\frac{1}{n}}<2$, so $\limsup(|c_n|^{\frac{1}{n}})\leq 2$, hence $R=\limsup(|c_n|^{\frac{1}{n}})^{-1}\geq\frac{1}{2}$.
Thus, we can claim that the power series $F(z)=\sum_{n=0}^{\infty}c_nz^n$ in fact converges absolutely for disk $|z|<\frac{1}{2}$ (since $|z|<\frac{1}{2}$ is contained in the radius of convergence).

\hfil

\textbf{Closed Expression of $c_n$:}

Now, consider power series $F(z)$ on $|z|<\frac{1}{2}$: Since $F(z)$ can be rewritten as $c_0+c_1z+\sum_{n=2}^{\infty}c_nz^n = 1+z+\sum_{n=2}^{\infty}c_nz^n$.
Then, based on the definition of Fibonnaci numbers, it can be rewritten as:
$$F(z)=1+z+\sum_{n=2}^{\infty}(c_{n-1}+c_{n-2})z^n = 1+z+\sum_{n=2}^{\infty}c_{n-1}z^n+\sum_{n=2}^{\infty}c_{n-2}z^n$$
$$ = 1+z+\sum_{n=1}^{\infty}c_nz^{n+1}+\sum_{n=0}^{\infty}c_nz^{n+2} = 1+z\left(1+\sum_{n=1}^{\infty}c_nz^n\right)+z^2\sum_{n=0}^{\infty}c_nz^n$$
$$=1+z\sum_{n=0}^{\infty}c_nz^n+z^2F(z) = 1+zF(z)+z^2F(z)$$
Then, we can yield the following:
$$F(z)=1+zF(z)+z^2F(z),\quad F(z)(1-z-z^2)=1,\quad F(z)=\frac{1}{1-z-z^2}$$
Now, if $1-z-z^2=0$ (or $z^2+z-1=0$), we have $z=\frac{-1\pm\sqrt{5}}{2}$. Hence, $1-z-z^2=-\left(\frac{-1+\sqrt{5}}{2}-z\right)\left(\frac{-1-\sqrt{5}}{2}-z\right)$.
Then, $F(z)$ can be decomposed using partial fraction:
$$F(z)=\frac{A}{\frac{-1+\sqrt{5}}{2}-z}+\frac{B}{\frac{-1-\sqrt{5}}{2}-z}=\frac{1}{1-z-z^2},\quad B\left(\frac{-1+\sqrt{5}}{2}-z\right)+A\left(\frac{-1-\sqrt{5}}{2}-z\right)=-1$$
So, from the above expression, we get:
$$B\cdot\frac{-1+\sqrt{5}}{2}+A\cdot \frac{-1-\sqrt{5}}{2}=-1,\quad -B-A=0$$
$$\implies A=-B,\quad B\cdot\frac{-1+\sqrt{5}}{2}-B\cdot \frac{-1-\sqrt{5}}{2}=-1$$
$$\implies B\left(\frac{-1+\sqrt{5}}{2}-\frac{-1-\sqrt{5}}{2}\right)=B\cdot \sqrt{5}=-1,\quad B=-\frac{1}{\sqrt{5}},\quad A=\frac{1}{\sqrt{5}}$$
So, $F(z)$ can be expressed as:
$$F(z)=\frac{1}{\sqrt{5}}\left(\frac{1}{\frac{-1+\sqrt{5}}{2}-z}-\frac{1}{\frac{-1-\sqrt{5}}{2}-z}\right)$$
Now, notice that for any $k\neq 0$, on $|z|<|k|$, since $|z/k|<1$, then $\sum_{n=0}^{\infty}(z/k)^n$ converges absolutely to $\frac{1}{1-z/k} = \frac{k}{k-z}$, which $\frac{1}{k-z}=\frac{1}{k}\sum_{n=0}^{\infty}(z/k)^n$.

Because both $\frac{-1+\sqrt{5}}{2},\frac{-1-\sqrt{5}}{2}$ has abslute values greater than $\frac{1}{2}$ (first one is approximately $0.618$, the second one is approximately $-1.618$),
hence, on the disk $|z|<\frac{1}{2}$, both equations below are true based on the above formula:
$$\frac{1}{\frac{-1+\sqrt{5}}{2}-z}=\frac{1}{\frac{-1+\sqrt{5}}{2}}\sum_{n=0}^{\infty}\left(\frac{z}{\frac{-1+\sqrt{5}}{2}}\right)^n,\quad \frac{1}{\frac{-1-\sqrt{5}}{2}-z}=\frac{1}{\frac{-1-\sqrt{5}}{2}}\sum_{n=0}^{\infty}\left(\frac{z}{\frac{-1-\sqrt{5}}{2}}\right)^n$$
Hence, $F(z)$ can be expressed as:
$$F(z)=\frac{1}{\sqrt{5}}\left(\sum_{n=0}^{\infty}\left(\frac{1}{\frac{-1+\sqrt{5}}{2}}\right)^{n+1}z^n-\sum_{n=0}^{\infty}\left(\frac{1}{\frac{-1-\sqrt{5}}{2}}\right)^{n+1}z^n\right)$$
$$=\frac{1}{\sqrt{5}}\sum_{n=0}^{\infty}\frac{\left(\frac{-1-\sqrt{5}}{2}\right)^{n+1}-\left(\frac{-1+\sqrt{5}}{2}\right)^{n+1}}{\left(\frac{-1+\sqrt{5}}{2}\right)^{n+1}\left(\frac{-1-\sqrt{5}}{2}\right)^{n+1}}z^n =\frac{1}{\sqrt{5}}\sum_{n=0}^{\infty}\frac{\left(\frac{-1-\sqrt{5}}{2}\right)^{n+1}-\left(\frac{-1+\sqrt{5}}{2}\right)^{n+1}}{\left(\frac{(-1)^2-(\sqrt{5})^2}{4}\right)^{n+1}}z^n$$
$$=\frac{1}{\sqrt{5}}\sum_{n=0}^{\infty}\frac{\left(\frac{-1-\sqrt{5}}{2}\right)^{n+1}-\left(\frac{-1+\sqrt{5}}{2}\right)^{n+1}}{(-1)^{n+1}}z^n = \sum_{n=0}^{\infty}\frac{\left(\frac{1+\sqrt{5}}{2}\right)^{n+1}-\left(\frac{1-\sqrt{5}}{2}\right)^{n+1}}{\sqrt{5}}z^n$$
Then, by the uniqueness of Taylor Series, the following is the closed expression of $c_n$:
$$\forall n\in\mathbb{N},\quad c_n=\frac{\left(\frac{1+\sqrt{5}}{2}\right)^{n+1}-\left(\frac{1-\sqrt{5}}{2}\right)^{n+1}}{\sqrt{5}}$$

\break

\section*{3}
\begin{myBox}[]{}
    \begin{question}
        Ahlfors Pg. 186 Problem 4:

        Show that the Laurent development of $(e^z-1)^{-1}$ at the origin is of the form
        $$\frac{1}{z}-\frac{1}{2}+\sum_{k=1}^{\infty}(-1)^{k-1}\frac{B_k}{(2k)!}z^{2k-1}$$
        where the numbers $B_k$ are known as the Bernoulli numbers. Calculate $B_1,B_2,B_3$.
    \end{question}
\end{myBox}

\textbf{Pf:}

Given the function $f(z)=(e^z-1)^{-1}$, it is analytic on $\mathbb{C}\setminus\{0\}$ (with $0<|z|<\infty$), hence there exists a laurent development that agrees on the whole $\mathbb{C}\setminus\{0\}$:
$$f(z)=\sum_{n=-\infty}^{\infty}A_nz^n$$
And, for all index $n$, the formula of $A_n$ is given as follow:
$$n\geq 1,\quad A_{-n}=\frac{1}{2\pi i}\int_{C}f(\zeta)\zeta^{n-1}d\zeta\quad \quad \quad \quad \quad n\geq 0,\quad A_n=\frac{1}{2\pi i}\int_{C}\frac{f(\zeta)}{\zeta^{n+1}}d\zeta$$
Where $C$ is a circle with radius $r>0$, centered at $z=0$.

\hfil

\textbf{Coefficients of Negative Degree:}

First, for $n=1$, the coefficient $A_{-1}$ is given as:
$$A_{-1}=\frac{1}{2\pi i}\int_{C}f(\zeta)\zeta^{1-1}d\zeta = \frac{1}{2\pi i}\int_{C}\frac{1}{e^{\zeta}-1}d\zeta=\frac{1}{2\pi i}\int_{C}\frac{1}{\zeta}\cdot\frac{\zeta}{e^{\zeta}-1}d\zeta$$
Now, notice that for $\frac{\zeta}{e^\zeta-1}$ with an isolated singularity at $\zeta=0$, since $\lim_{\zeta\rightarrow 0}\frac{\zeta}{e^\zeta-1}=1$ (since the limit of its reciprical is $\lim_{\zeta\rightarrow 0}\frac{e^\zeta-e^0}{\zeta}=1$, 
which is the derivative of $e^z$ at $0$),
then, $\lim_{\zeta\rightarrow 0}\zeta \cdot \frac{\zeta}{e^{\zeta}-1}=0$, which is a sufficient and necessary condition for $\zeta=0$ to be a removable singularity of $\frac{\zeta}{e^{\zeta}-1}$.

Hence, $\frac{\zeta}{e^\zeta-1}$ has an analytic extension onto the whole $\mathbb{C}$, with the function being $1$ at $\zeta=0$. Which, by Cauchy's Integral Formula, $A_{-1}$ of the above form, is the evaluation of $\frac{\zeta}{e^\zeta-1}$ at $0$ (more precisely, evaluation of its extension at $0$),
which provides $A_{-1}=1$.

\hfil

Then, for $n>1$, the coefficient $A_{-n}$ is given by:
$$A_{-n}=\frac{1}{2\pi i}\int_{C}f(\zeta)\zeta^{n-1}d\zeta=\frac{1}{2\pi i}\int_{C}\frac{1}{\zeta}\cdot \frac{\zeta^n}{e^\zeta-1}d\zeta$$
Notice that for $n>1$, the function $\frac{\zeta^n}{e^\zeta-1}$ has isolated singularity at $0$, since $\lim_{\zeta\rightarrow 0}\zeta\cdot \frac{\zeta^n}{e^\zeta-1}=0$, then the singularity at $0$ is in fact removable.
Hence, it has an analytic extension onto $\mathbb{C}$, where the evaluation at $\zeta=0$ is $\lim_{\zeta\rightarrow 0}\frac{\zeta^n}{e^\zeta-1} = 0$ (since it can be broken down into $\zeta^{n-1}$ and $\frac{\zeta}{e^\zeta-1}$; the first one has limit $0$ since $n-1\geq 1$, while the second one has limit $1$, hence the product has limit $0$).

Again, the integral form of $A_{-n}$ is in fact the evaluation of the extension of $\frac{\zeta^n}{e^\zeta-1}$ at $\zeta=0$ by Cauchy's Integral Formula, hence $A_{-n}=0$.

\hfil

\hfil

\textbf{Coefficients of Nonnegative Degree:}

For $n\geq 0$, the coefficient $A_n$ is given as:
$$A_n=\frac{1}{2\pi i}\int_{C}\frac{f(\zeta)}{\zeta^{n+1}}d\zeta=\frac{1}{2\pi i}\int_{C}\frac{1}{\zeta^{n+1}(e^\zeta-1)}d\zeta = \frac{1}{2\pi i}\int_{C}\frac{1}{\zeta^{n+2}}\cdot \frac{\zeta}{e^\zeta-1}d\zeta$$
Since we've verified above, that $\frac{z}{e^z-1}$ has an analytic extension onto $\mathbb{C}$, it has a power series expansion about $0$, $\frac{z}{e^z-1}=\sum_{n=0}^{\infty}C_nz^n$, and it agrees with the function on the whole $\mathbb{C}$.
We'll find the coefficient to help calculate $A_n$.

Notice that since $\frac{e^z-1}{z}\cdot \frac{z}{e^z-1}=1$, while $\frac{e^z-1}{z}$ also can be extended analytically onto $\mathbb{C}$,
given the following power series expansion of $\frac{e^z-1}{z}$:
$$e^z=1+\sum_{n=1}^{\infty}\frac{z^n}{n!},\quad \frac{e^z-1}{z}=\frac{1}{z}\sum_{n=1}^{\infty}\frac{z^n}{n!}=\sum_{n=1}^{\infty}\frac{z^{n-1}}{n!}=\sum_{n=0}^{\infty}\frac{z^n}{(n+1)!}$$
Which we can conclude the following:
$$\frac{e^z-1}{z}\cdot \frac{z}{e^z-1}=\left(\sum_{n=0}^{\infty}C_nz^n\right)\left(\sum_{n=0}^{\infty}\frac{z^n}{(n+1)!}\right)=1 = 1 + \sum_{n=1}^{\infty}0\cdot z^n$$
Since regardless of $z\in\mathbb{C}$, the above equation is true, $\sum_{n=0}^{\infty}C_nz^n$ is in fact the inverse of the formal power series $\sum_{n=0}^{\infty}\frac{z^n}{(n+1)!}\in \mathbb{C}[[z]]$. Which, they satisfy the following relationship:
\begin{itemize}
    \item For coefficient of degree $0$, we have $C_0\cdot \frac{1}{(0+1)!}=1$, hence $C_0=1$.
    \item For coefficient of degree $1$, we have $C_1\cdot \frac{1}{(0+1)!}+C_0\cdot\frac{1}{(1+1)!}=0$, hence $C_1+\frac{1}{2!}=0,\quad C_1=-\frac{1}{2}$.
    \item For coefficient of degree $n\geq 2$, we have $\sum_{k=0}^{n}C_k\cdot \frac{1}{((n-k)+1)!}=0$, hence: 
    $$C_n=C_n\cdot\frac{1}{(0+1)!} = -\sum_{k=0}^{n-1}C_k\cdot\frac{1}{((n-k)+1)!}$$
\end{itemize}

\hfil

Since $g(z)=\frac{z}{e^z-1}$ has its power series $\sum_{n=0}^{\infty}C_nz^n$ converge to itself on the whole $\mathbb{C}$,then its $n^{th}$ derivative at $0$ is given as $g^{(n)}(0)=n!C_n$. Which, $A_n$ can be rewritten as the following using Cauchy's Integral Formula:
$$A_n=\frac{1}{2\pi i}\int_{C}\frac{1}{\zeta^{n+2}}\cdot\frac{\zeta}{e^\zeta-1}d\zeta=\frac{1}{2\pi i}\int_{C}\frac{g(\zeta)}{\zeta^{n+2}}d\zeta = \frac{g^{(n+1)}(0)}{(n+1)!}=\frac{(n+1)!C_{n+1}}{(n+1)!}=C_{n+1}$$

\hfil

\textbf{Forms of Laurent Series of $(e^z-1)^{-1}$:}

With the information from previous two sections, we can express the laurent series as the following:
$$\sum_{n=-\infty}^{\infty}A_nz^n = \sum_{n=1}^{\infty}A_{-n}z^{-n}+\sum_{n=0}^{\infty}A_nz^n = \frac{1}{z}+\sum_{n=0}^{\infty}C_{n+1}z^n = \frac{1}{z}-\frac{1}{2}+\sum_{n=1}^{\infty}C_{n+1}z^n$$

Now, recall that $\frac{z}{e^z-1}=\sum_{n=0}^{\infty}C_nz^n$, then for all $z\in\mathbb{C}$, consider the expression with $-z$, we get:
$$\frac{-z}{e^{-z}-1}=\sum_{n=0}^{\infty}C_n(-z)^n = \sum_{n=0}^{\infty}(-1)^nC_nz^n$$
Which, consider the difference of the two terms, we get:
$$\frac{-z}{e^{-z}-1}-\frac{z}{e^z-1} = \frac{-ze^z}{1-e^z}-\frac{z}{e^z-1}=\frac{ze^z}{e^z-1}-\frac{z}{e^z-1}=\frac{z(e^z-1)}{e^z-1}=z$$
(Note: for $z=0$, consider its extension, where evaluation at $z=0$ is the limit $\lim_{z\rightarrow 0}\frac{z}{e^z-1}=1$, then the above difference is $0$, which agrees with the formula).

Hence, in power series form, we get:
$$\sum_{n=0}^{\infty}(-1)^nC_nz^n-\sum_{n=0}^{\infty}C_nz^n = \sum_{n=0}^{\infty}((-1)^n-1)C_nz^n=z$$
Then, all the even terms have $(-1)^n-1=0$, we're left with the odd terms. Hence:
$$\sum_{k=1}^{\infty}((-1)^{2k-1}-1)C_{2k-1}z^{2k-1}=\sum_{k=1}^{\infty}-2C_{2k-1}z^{2k-1}=-2C_1z+\sum_{k=2}^{\infty}-2C_{2k-1}z^{2k-1}=z$$
By the uniqueness of taylor series, we need $-2C_1=1,\ C_1=\frac{-1}{2}$ (which agrees with our previous calculation), and $-2C_{2k-1}=0,\ C_{2k-1}=0$ for all $k\geq 2$.
Therefore, all the odd terms of $C_n$ is $0$.

Hence, the laurent series can be expressed as follow:
$$\frac{1}{z}-\frac{1}{2}+\sum_{n=1}^{\infty}C_{n+1}z^n=\frac{1}{z}-\frac{1}{2}+\sum_{k=1}^{\infty}C_{(2k-1)+1}z^{2k-1}=\frac{1}{z}-\frac{1}{2}+\sum_{k=1}^{\infty}C_{2k}z^{2k-1}$$
(Note: Since now the odd terms of $C_n$ appears as the even degrees' coefficients).

Now, if we do some modification, let $B_n' = n!C_n$ for all $n\in\mathbb{N}$ (or $C_n = \frac{B_n'}{n!}$), then the laurent series can be expressed as:
$$(e^z-1)^{-1}=\frac{1}{z}-\frac{1}{2}+\sum_{k=1}^{\infty}\frac{B_{2k}'}{(2k)!}z^{2k-1}$$
Then, let $B_k=(-1)^kB_{2k}'$, we get the desired form:
$$(e^z-1)^{-1}=\frac{1}{z}-\frac{1}{2}+\sum_{k=1}^{\infty}(-1)^k\frac{B_k}{(2k)!}z^{2k-1}$$

\hfil

\textbf{Calculation of Bernoulli Numbers:}

For $k=1,2,3$, we'll convert it back into $C_n$ for simplicity. Which, $B_k=(-1)^kB_{2k}=(-1)^k(2k)!C_{2k}$.

For $k=1, 2k=2$, we have $B_1$ given as:
$$C_2=-\sum_{k=0}^{1}C_k\frac{1}{(2-k+1)!}=-\left(\frac{C_0}{3!}+\frac{C_1}{2!}\right)=-\left(1\cdot\frac{1}{6}-\frac{1}{2}\cdot\frac{1}{2}\right)=\frac{1}{12}$$
$$B_1=(-1)^1\cdot 2!\cdot C_2 = -\frac{1}{6}$$
For $k=2,2k=4$, we have $B_2$ given as:
$$C_4=-\sum_{k=0}^{3}C_k\frac{1}{(4-k+1)!}=-\left(\frac{C_0}{5!}+\frac{C_1}{4!}+\frac{C_2}{3!}+\frac{C_3}{2!}\right)=-\left(\frac{1}{120}-\frac{1}{2\cdot 24}+\frac{1}{12\cdot 6}\right)$$
$$= -\frac{1}{12}\left(\frac{1}{10}-\frac{1}{4}+\frac{1}{6}\right)=-\frac{1}{12}\cdot\frac{1}{60}$$
$$B_2=(-1)^2\cdot 4!\cdot C_4=-\frac{24}{12\cdot 60}=-\frac{1}{30}$$
(Note: recall that for $k\geq 2$, all $C_{2k-1}=0$, hence all odd index $n\geq 3$ has $C_3=0$).

For $k=3,2k=6$, we have $B_3$ given as:
$$C_6=-\sum_{k=0}^{5}C_k\frac{1}{(6-k+1)!}=-\left(\frac{C_0}{7!}+\frac{C_1}{6!}+\frac{C_2}{5!}+\frac{C_3}{4!}+\frac{C_4}{3!}+\frac{C_5}{2!}\right)$$
$$=-\left(\frac{1}{7!}-\frac{1}{2\cdot 6!}+\frac{1}{12\cdot 5!}-\frac{1}{12\cdot 60\cdot 3!}\right)$$

$$B_3=(-1)^3\cdot 6!\cdot C_6 = -6!\cdot \left(-\left(\frac{1}{7!}-\frac{1}{2\cdot 6!}+\frac{1}{12\cdot 5!}-\frac{1}{12\cdot 60\cdot 3!}\right)\right)$$
$$=\frac{1}{7}-\frac{1}{2}+\frac{6}{12}-\frac{6!}{12\cdot 60 \cdot 3!}=\frac{1}{7}-\frac{1}{6}=\frac{1}{42}$$

So, we have the following:
$$B_1=-\frac{1}{6},\quad B_2=-\frac{1}{30},\quad B_3=\frac{1}{42}$$

\break

\section*{4}
\begin{myBox}[]{}
    \begin{question}
        Stein and Shakarchi Pg. 86 Problem 2:

        Let 
        $$F(z)=\sum_{n=1}^{\infty}d(n)z^n$$
        for $|z|<1$, where $d(n)$ denotes the number of divisors of $n$. Observe that the radius of convergence of this series is $1$.
        Verify the identity
        $$\sum_{n=1}^{\infty}d(n)z^n=\sum_{n=1}^{\infty}\frac{z^n}{1-z^n}$$
        Using this identity, show that if $z=r$ with $0<r<1$, then
        $$|F(r)|\geq c\frac{1}{1-r}\log(1/(1-r))$$
        as $r\rightarrow 1$. Similarly, if $\theta = 2\pi p/q$ where $p$ and $q$ are positive integers and $z=re^{i\theta}$, then
        $$|F(re^{i\theta})|\geq c_{p/q}\frac{1}{1-r}\log(1/(1-r))$$
        as $r\rightarrow 1$. Conclude that $F$ cannot be continued analytically past the unit disk.
    \end{question}
\end{myBox}

\textbf{Pf:}


\end{document}