% 122B_HW1.tex

\documentclass{article}
\usepackage{graphicx} % Required for inserting images
\usepackage[margin = 2.54cm]{geometry}
\usepackage[most]{tcolorbox}

\newtcolorbox{myBox}[3]{
arc=5mm,
lower separated=false,
fonttitle=\bfseries,
%colbacktitle=green!10,
%coltitle=green!50!black,
enhanced,
attach boxed title to top left={xshift=0.5cm,
        yshift=-2mm},
colframe=blue!50!black,
colback=blue!10
}

\usepackage{amsmath}
\usepackage{amssymb}
\usepackage{verbatim}
\usepackage[utf8]{inputenc}
\linespread{1.2}

\newtheorem{definition}{Definition}
\newtheorem{proposition}{Proposition}
\newtheorem{theorem}{Theorem}
\newtheorem{question}{Question}

\title{Math CS 122B HW1}
\author{Zih-Yu Hsieh}

\begin{document}
\maketitle

\section*{1}
\begin{myBox}[]{}
    \begin{question}
        Ahlfors Pg. 178 Problem 2:

        Show that the series
        $$\zeta(z)=\sum_{n=1}^{\infty}n^{-z}$$
        converges for $Re(z)>1$, and represent its derivative in series form.
    \end{question}
\end{myBox}

\textbf{Pf:}

The following proof would assume the domain of the above series is the half plane $Re(z)>1$.

\hfil

\textbf{The series converges pointwise:}

For all $z\in\mathbb{C}$ satisfying $Re(z)>1$, $z=a+bi$ for $a,b\in\mathbb{R}$, and $a>1$. Then, for any $n\in\mathbb{N}$,
the number $n^{-z}=e^{-z\log(n)} = e^{-(a+bi)\ln(n)} = e^{-a\ln(n)}\cdot e^{i(-b\ln(n))} = n^{-a}\cdot e^{i(-b\ln(a))}$. Hence, if taken the modulus $e^{-a\ln(n)}$,
since $a>1$, by p-series test, the following series converges:
$$\sum_{n=1}^{\infty}n^{-a}$$
Which, since the original series satisfies the following:
$$\sum_{n=1}^{\infty}|n^{-z}| = \sum_{n=1}^{\infty}\left|n^{-a}\cdot e^{i(-b\ln(n))}\right| = \sum_{n=1}^{\infty}n^{-a}$$
Hence, the series absolutely converges, which $\zeta(z)=\sum_{n=1}^{\infty}n^{-z}$ exists on $Re(z)>1$.

\hfil

\textbf{Partial Sum converges uniformly on any Compact Subset:}

Suppose $K\subset \mathbb{C}$ is a compact subset of the plane $Re(z)>1$, each component $n^{-z}=e^{-z\ln(n)}=n^{-a}\cdot e^{i(-b\ln(n))}$ is analytic on the half plane (also on $K$), 
hence there exists $z_0\in K$, such that $|n^{-z_0}|$ yields the maximum.

Since for $z_0=a+bi$, $|n^{-z_0}|=\left|n^{-a}\cdot e^{i(-b\ln(n))}\right|=n^{-a}$, and since $z_0$ is in the half plane, so $Re(z_0)=a>1$. Then, the series $\sum_{n=1}^{\infty}n^{-a}$ converges.

Now, notice that for each $n\in\mathbb{N}$, $M_n=\sup_{z\in K}|n^{-z}| = \max_{z\in K}|n^{-z}|=n^{-a}$ satisfies $\sum_{n=1}^{\infty}M_n$ converges, then by Weierstrass M-Test,
the series $\sum_{n=1}^{\infty}n^{-z}$ in fact converges uniformly on $K$. 

Then, because the series $\zeta(z)=\sum_{n=1}^{\infty}n^{-z}$ converges absolutely on $Re(z)>1$, converging uniformly on all compact subset of the half plane, and each component is analytic on the half plane,
then by the theorem in Ahlfors pg. 176, $\zeta(z)$ is analytic, and the partial sum $\sum_{n=1}^{N}n^{-z}$ (for $N\in\mathbb{N}$) has derivative converges to  $\zeta'(z)$ uniformly on all compact subsets of the half plane.
Hence, based on the same theorem again, we can claim the folloing on the chosen half plane:
$$\zeta'(z)=\lim_{N\rightarrow\infty}\frac{d}{dz}\sum_{n=1}^{N}n^{-z} = \lim_{N\rightarrow\infty}\sum_{n=1}^{N}\frac{d}{dz}(e^{-z\ln(n)}) = \lim_{N\rightarrow\infty}\sum_{n=1}^{N}-\ln(n)e^{-z\ln(n)} = -\sum_{n=1}^{\infty}ln(n)n^{-z}$$


\break

\section*{2}
\begin{myBox}[]{}
    \begin{question}
        Ahlfors Pg. 184 Problem 5:

        The Fibonacci numbers are defined by $c_0=0$, $c_1=1$, and $c_n=c_{n-1}+c_{n-2}$ for all $n\geq 2$.
        
        Show that the $c_n$ are Taylor Coefficients of a rational function, and determine a closed expression for $c_n$.
    \end{question}
\end{myBox}

\textbf{Pf:}

Consider the generating function, a formal power series defined as $F(z)=\sum_{n=0}^{\infty}c_nz^n$.

\hfil

\textbf{Radius of Convergence of the Power Series:}

Recall that radius of convergence of power series $R=\limsup(|c_n|^{\frac{1}{n}})^{-1}\in [0,\infty]$, where $c_n$ is the coefficients of each degree.

First, we can verify that for all $n\in\mathbb{N}$, $c_n<2^n$:

For base case $n=1$, $c_1=1<2^1$.

Now, suppose for given $n\geq 1$, $c_n<2^n$, then for case $(n+1)\geq 2$, since $c_{n+1}=c_n+c_{n-1} < 2\cdot c_n$ (since $c_{n}>c_{n-1}$), then $c_{n+1}<2\cdot c_n<2\cdot 2^n=2^{n+1}$
by induction hypothesis, which this completes the induction.

Since all $n\in\mathbb{N}$ has $0<c_n<2^n$, then $|c_n|^{\frac{1}{n}}=c_n^{\frac{1}{n}}<(2^n)^{\frac{1}{n}}<2$, so $\limsup(|c_n|^{\frac{1}{n}})\leq 2$, hence $R=\limsup(|c_n|^{\frac{1}{n}})^{-1}\geq\frac{1}{2}$.
Thus, we can claim that the power series $F(z)=\sum_{n=0}^{\infty}c_nz^n$ in fact converges absolutely for disk $|z|<\frac{1}{2}$ (since $|z|<\frac{1}{2}$ is contained in the radius of convergence).

\hfil

\textbf{Closed Expression of $c_n$:}

Now, consider power series $F(z)$ on $|z|<\frac{1}{2}$: Since $F(z)$ can be rewritten as $c_0+c_1z+\sum_{n=2}^{\infty}c_nz^n = 1+z+\sum_{n=2}^{\infty}c_nz^n$.
Then, based on the definition of Fibonnaci numbers, it can be rewritten as:
$$F(z)=1+z+\sum_{n=2}^{\infty}(c_{n-1}+c_{n-2})z^n = 1+z+\sum_{n=2}^{\infty}c_{n-1}z^n+\sum_{n=2}^{\infty}c_{n-2}z^n$$
$$ = 1+z+\sum_{n=1}^{\infty}c_nz^{n+1}+\sum_{n=0}^{\infty}c_nz^{n+2} = 1+z\left(1+\sum_{n=1}^{\infty}c_nz^n\right)+z^2\sum_{n=0}^{\infty}c_nz^n$$
$$=1+z\sum_{n=0}^{\infty}c_nz^n+z^2F(z) = 1+zF(z)+z^2F(z)$$
Then, we can yield the following:
$$F(z)=1+zF(z)+z^2F(z),\quad F(z)(1-z-z^2)=1,\quad F(z)=\frac{1}{1-z-z^2}$$
Now, if $1-z-z^2=0$ (or $z^2+z-1=0$), we have $z=\frac{-1\pm\sqrt{5}}{2}$. Hence, $1-z-z^2=-\left(\frac{-1+\sqrt{5}}{2}-z\right)\left(\frac{-1-\sqrt{5}}{2}-z\right)$.
Then, $F(z)$ can be decomposed using partial fraction:
$$F(z)=\frac{A}{\frac{-1+\sqrt{5}}{2}-z}+\frac{B}{\frac{-1-\sqrt{5}}{2}-z}=\frac{1}{1-z-z^2},\quad B\left(\frac{-1+\sqrt{5}}{2}-z\right)+A\left(\frac{-1-\sqrt{5}}{2}-z\right)=-11$$
So, from the above expression, we get:
$$B\cdot\frac{-1+\sqrt{5}}{2}+A\cdot \frac{-1-\sqrt{5}}{2}=-1,\quad -B-A=0$$
$$\implies A=-B,\quad B\cdot\frac{-1+\sqrt{5}}{2}-B\cdot \frac{-1-\sqrt{5}}{2}=-1$$
$$\implies B\left(\frac{-1+\sqrt{5}}{2}-\frac{-1-\sqrt{5}}{2}\right)=B\cdot \sqrt{5}=-1,\quad B=-\frac{1}{\sqrt{5}},\quad A=\frac{1}{\sqrt{5}}$$
So, $F(z)$ can be expressed as:
$$F(z)=\frac{1}{\sqrt{5}}\left(\frac{1}{\frac{-1+\sqrt{5}}{2}-z}-\frac{1}{\frac{-1-\sqrt{5}}{2}-z}\right)$$
Now, notice that for any $k\neq 0$, on $|z|<|k|$, since $|z/k|<1$, then $\sum_{n=0}^{\infty}(z/k)^n$ converges absolutely to $\frac{1}{1-z/k} = \frac{k}{k-z}$, which $\frac{1}{k-z}=\frac{1}{k}\sum_{n=0}^{\infty}(z/k)^n$.

Because both $\frac{-1+\sqrt{5}}{2},\frac{-1-\sqrt{5}}{2}$ has abslute values greater than $\frac{1}{2}$ (first one is approximately $0.618$, the second one is approximately $-1.618$),
hence, on the disk $|z|<\frac{1}{2}$, both equations below are true based on the above formula:
$$\frac{1}{\frac{-1+\sqrt{5}}{2}-z}=\frac{1}{\frac{-1+\sqrt{5}}{2}}\sum_{n=0}^{\infty}\left(\frac{z}{\frac{-1+\sqrt{5}}{2}}\right)^n,\quad \frac{1}{\frac{-1-\sqrt{5}}{2}-z}=\frac{1}{\frac{-1-\sqrt{5}}{2}}\sum_{n=0}^{\infty}\left(\frac{z}{\frac{-1-\sqrt{5}}{2}}\right)^n$$
Hence, $F(z)$ can be expressed as:
$$F(z)=\frac{1}{\sqrt{5}}\left(\sum_{n=0}^{\infty}\left(\frac{1}{\frac{-1+\sqrt{5}}{2}}\right)^{n+1}z^n-\sum_{n=0}^{\infty}\left(\frac{1}{\frac{-1-\sqrt{5}}{2}}\right)^{n+1}z^n\right)$$
$$=\frac{1}{\sqrt{5}}\sum_{n=0}^{\infty}\frac{\left(\frac{-1-\sqrt{5}}{2}\right)^{n+1}-\left(\frac{-1+\sqrt{5}}{2}\right)^{n+1}}{\left(\frac{-1+\sqrt{5}}{2}\right)^{n+1}\left(\frac{-1-\sqrt{5}}{2}\right)^{n+1}}z^n =\frac{1}{\sqrt{5}}\sum_{n=0}^{\infty}\frac{\left(\frac{-1-\sqrt{5}}{2}\right)^{n+1}-\left(\frac{-1+\sqrt{5}}{2}\right)^{n+1}}{\left(\frac{(-1)^2-(\sqrt{5})^2}{4}\right)^{n+1}}z^n$$
$$=\frac{1}{\sqrt{5}}\sum_{n=0}^{\infty}\frac{\left(\frac{-1-\sqrt{5}}{2}\right)^{n+1}-\left(\frac{-1+\sqrt{5}}{2}\right)^{n+1}}{(-1)^{n+1}}z^n = \sum_{n=0}^{\infty}\frac{\left(\frac{1+\sqrt{5}}{2}\right)^{n+1}-\left(\frac{1-\sqrt{5}}{2}\right)^{n+1}}{\sqrt{5}}z^n$$

\break

\section*{3}
\begin{myBox}[]{}
    \begin{question}
        Ahlfors Pg. 186 Problem 4:
    \end{question}
\end{myBox}

\textbf{Pf:}


\end{document}