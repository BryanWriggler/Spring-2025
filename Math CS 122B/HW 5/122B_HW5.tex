\documentclass{article}
\usepackage{graphicx} % Required for inserting images
\usepackage[margin = 2.54cm]{geometry}
\usepackage[most]{tcolorbox}

\newtcolorbox{myBox}[3]{
arc=5mm,
lower separated=false,
fonttitle=\bfseries,
%colbacktitle=green!10,
%coltitle=green!50!black,
enhanced,
attach boxed title to top left={xshift=0.5cm,
        yshift=-2mm},
colframe=blue!50!black,
colback=blue!10
}

\usepackage{amsmath}
\usepackage{amssymb}
\usepackage{verbatim}
\usepackage[utf8]{inputenc}
\linespread{1.2}

\newtheorem{definition}{Definition}
\newtheorem{proposition}{Proposition}
\newtheorem{theorem}{Theorem}
\newtheorem{question}{Question}

\title{Math CS 122B HW5}
\author{Zih-Yu Hsieh}

\begin{document}
\maketitle

\section*{1}
\begin{myBox}[]{}
    \begin{question}
        Freitag Chap. V.3 Exercise 5:

        The algebraic differential equation of the $\wp$-function can berewritten as:
        $$(\wp')^2 = 4(\wp-e_1)(\wp-e_2)(\wp-e_3)$$
        Here, $e_j,\ 1\leq j\leq 3$, are the three half lattice values of the $\wp$-function.
    \end{question}
\end{myBox}

\textbf{Pf:}

Given the algebraic differential equation of the $\wp$-function as follow:
$$(\wp'(z))^2=4(\wp(z))^3-g_2\wp(z)-g_3$$
Within the fundamental region $P$, there are $3$ points with the value of $\wp'$ to be zero, which is given by $\frac{w_1}{2},\ \frac{w_2}{2},\ \frac{w_1+w_2}{2}$ (and points congruent to these points $\mod\ L$)
when the lattice $L=w_1\mathbb{Z}+w_2\mathbb{Z}$.

Then, by definition, the given points have the evaluation to be the following:
$$e_1=\wp\left(\frac{w_1}{2}\right),\quad e_2=\wp\left(\frac{w_2}{2}\right),\quad e_3=\wp\left(\frac{w_1+w_2}{2}\right)$$
Which, let $w=\wp(z)$, then the polynomial $4w^3-g_2w-g_3=0$ iff $\wp'(z)=0$, which within the fundamental region, only the three distinct points mentioned above are the solution,
so the values of $\wp$ of these points are the zeros of the polynomial $4w^3-g_2w-g_3$.

Then, since $e_1,e_2,e_3$ are all distinct, while $4w^3-g_2w-g_3$ has at most $3$ distinct zeroes, then they must be all the zeros of the polynomial.
Hence, $4w^3-g_2w-g_3 = 4(w-e_1)(w-e_2)(w-e_3)$, which we get the following:
$$(\wp'(z))^3=4(\wp(z))^3-g_2\wp(z)-g_3 = 4(\wp(z)-e_1)(\wp(z)-e_2)(\wp(z)-e_3)$$

\break

\section*{2}
\begin{myBox}[]{}
    \begin{question}
        Freitag Chap. V.3 Exercise 6:

        Show the following recursion formulas for the Eisenstein series $G_{2m}$ for $m\geq 4$:
        $$(2m+1)(m-3)(2m-1)G_{2m}=3\sum_{j=2}^{m-2}(2j-1)(2m-2j-1)G_{2j}G_{2m-2j}$$
        for instance $G_{10}=\frac{5}{11}G_4G_6$. Any Eisenstein series $G_{2m}$, $m\geq 4$, 
        is thus representable as a polynomial in $G_4$ and $G_6$ with nonnegative coefficients.
    \end{question}
\end{myBox}

\textbf{Pf:}

First, the $\wp$-function is given as follow:
$$\wp(z)=\frac{1}{z^2}+\sum_{m=1}^{\infty}(2m+1)G_{2(m+1)}z^{2m}$$
With the formula of $\wp$-function as series of functions, since it converges normally within $\mathbb{C}\setminus L$ (with $L$ being the lattice),
then differentiation can be performed termwise. Hence, its second derivative is given by:
$$\wp''(z)=\frac{d^2}{dz^2}\left(\frac{1}{z^2}\right)+\sum_{m=1}^{\infty}\frac{d^2}{dz^2}\left((2m+1)G_{2(m+1)}z^{2m}\right) = \frac{6}{z^4}+\sum_{m=1}^{\infty}(2m+1)(2m)(2m-1)G_{2(m+1)}z^{2m-2}$$
$$= \frac{6}{z^4}+\sum_{m=2}^{\infty}(2m-1)(2m-2)(2m-3)G_{2m}z^{2m-4}$$

Recall the following second order differential equation of $\wp$-function:
$$2\wp''(z)=12(\wp(z))^2-g_2,\quad \wp''(z)=6(\wp(z))^2-\frac{g_2}{2}$$
The goal is to get a recursive relation of the coefficient of each power of $\wp''(z)$.

\hfil

With the expression of $\wp''$ in power series from above, to get an expression of $G_{2m}$ for $m\geq 4$, it suffices to find the coefficient of $z^{2m-4}$ within $6(\wp(z))^2-\frac{g_2}{2}$.
There are two casees to consider:
\begin{itemize}
    \item[1.] $z^{2m-4}$ can be expressed as $\frac{1}{z^2}\cdot z^{2m-2}$, within $\wp(z)$, the coefficient of $\frac{1}{z^2}$ is $1$, while the coefficient of $z^{2m-2}=z^{2(m-1)}$ is $(2(m-1)+1)G_{2((m-1)+1)} = (2m-1)G_{2m}$.
    Hence, since $(\wp(z))^2$ has two copies of the above expression, then the coefficient of $\frac{1}{z^2}\cdot z^{2m-2}$ is:
    $$2\cdot 1\cdot (2m-1)G_{2m} = 2(2m-1)G_{2m}$$
    \item[2.] Since $\wp(z)$ also has all power $z^{2m}$ for $m\geq 1$, then $z^{2m-4}=z^{2(m-2)}$ can also be expressed as $z^{2k}\cdot z^{2(m-k-2)}$, for integers $k\geq 1$ and $(m-k-2)\geq 1 $
    (or $k\leq (m-3)$). Hence, for the convolution of power series of $(\wp(z))^2$ (excluding the negative powers mentioned above), $z^{2m-4}$ term has the following coefficient:
    $$\sum_{k=1}^{m-3}(2k+1)G_{2(k+1)}\cdot (2(m-k-2)+1)G_{2((m-k-2)+1)} = \sum_{k=1}^{m-3}(2(k+1)-1)(2m-2(k+1)-1)G_{2(k+1)}G_{2(m-(k+1))}$$
    $$ = \sum_{k=2}^{m-2}(2k-1)(2m-2k-1)G_{2k}G_{2(m-k)}$$
    (Note: recall that $z^{2k}$ term has coefficient $(2k+1)G_{2(k+1)}$, while $z^{2(m-k-2)}$ term has coefficient given as $(2(m-k-2)+1)G_{2((m-k-2)+1)}$).
\end{itemize}
So, the coefficient of $z^{2m-4}$ in $(\wp(z))^2$ is recorded as:
$$2(2m-1)G_{2m}+\sum_{k=2}^{m-2}(2k-1)(2m-2k-1)G_{2k}G_{2m-2k}$$
Hence, based on the equation $\wp''(z)=6(\wp(z))^2-\frac{g_2}{2}$, for all $m\geq 4$, the coefficient of $z^{2m-4}$ is given as the following two forms:
$$\textmd{Coefficient of } z^{2m-4} \textmd{ in } \wp''(z):\quad (2m-1)(2m-2)(2m-3)G_{2m}$$
$$\textmd{Coefficient of } z^{2m-4} \textmd{ in } 6(\wp(z))^2-\frac{g_2}{2}:\quad 6\left(2(2m-1)G_{2m}+\sum_{k=2}^{m-2}(2k-1)(2m-2k-1)G_{2k}G_{2m-2k}\right)$$
Which, for the two to be equal, we get the following equality:
$$(2m-1)(2m-2)(2m-3)G_{2m}=12(2m-1)G_{2m}+6\sum_{k=2}^{m-2}(2k-1)(2m-2k-1)G_{2k}G_{2m-2k}$$
$$(2m-1)(4m^2-10m+6)G_{2m}-12(2m-1)G_{2m}=6\sum_{k=2}^{m-2}(2k-1)(2m-2k-1)G_{2k}G_{2m-2k}$$
$$(2m-1)(4m^2-10m-6)G_{2m}=6\sum_{k=2}^{m-2}(2k-1)(2m-2k-1)G_{2k}G_{2m-2k}$$
$$(2m-1)(2m-6)(2m+1)G_{2m}=6\sum_{k=2}^{m-2}(2k-1)(2m-2k-1)G_{2k}G_{2m-2k}$$
$$\implies (2m+1)(m-3)(2m-1)G_{2m}=3\sum_{k=2}^{m-2}(2k-1)(2m-2k-1)G_{2k}G_{2m-2k}$$
Which, this equation is the desired recursive form.

\break

\section*{3}
\begin{myBox}[]{}
    \begin{question}
        Freitag Chap. V.4 Exercise 3:

        Let $L\subset \mathbb{C}$ be a lattice with the property $g_2(L)=8$ and $g_3(L)=0$. The point $(2,4)$ lies on the affine elliptic curve $y^2=4x^3-8x$.
        Let $+$ be the addition (for points on the corresponding projective curve). SHow that $2\cdot (2,4):= (2,4)+(2,4)$ is the point $(\frac{9}{4},\frac{21}{4})$.
    \end{question}
\end{myBox}

\textbf{Pf:}

Consider the tangent of $(2,4)$ on the given elliptic curve $y^2=4x^3-8x$: By implicit differentiation, we get the following relationship:
$$2y\frac{dy}{dx}=12x^2-8$$
which, for $(x,y)=(2,4)$, $\frac{dy}{dx}\bigm|_{(2,4)}=\frac{12x^2-8}{2y}\bigm|_{(2,4)} = \frac{12\cdot 2^2-8}{2\cdot 4} = 5$. Hence, the tangent is expressed as the following equation:
$$(y-4)=5(x-2),\quad y=5x-6$$

\hfil

Now, to solve for the third point, it must satisfy the following equations:
$$\begin{cases}
    y=5x-6\\
    y^2=4x^3-8x
\end{cases}$$
Hence, $(5x-6)^2 = 4x^3-8x$, which $25x^2-60x+36 = 4x^3-8x$, so $4x^3-25x^2+52x-36 = 0$. Which, consider the fact that $(x,y)=(2,4)$ appears on the tangent twice (with multiplicity 2),
then $(x-2)^2$ is presumably a factor of the above equation. The above polynomial in fact has the following factorization:
$$4x^3-25x^2+52x-36 = (x-2)^2(4x-9)$$
This indicates that the third zero happesn when $x=\frac{9}{4}$. Which, the only point lying on the defined tangent above is given as:
$$y=5\cdot\frac{9}{4}-6 = \frac{21}{4}$$
So, the third point lying on the tangent is $(\frac{9}{4},\frac{21}{4})$.


\break

\section*{4}
\begin{myBox}[]{}
    \begin{question}
        Stein and Shakarchi Pg. 281 Problem 3:

        Suppose $\Omega$ is a simply connected domain that excludes the three roots of the polynomial $4z^3-g_2z-g_3$.
        For $w_0\in\Omega$ fixed, define the function $I$ on $\Omega$ by 
        $$I(w)=\int_{w_0}^{w}\frac{dz}{\sqrt{4z^3-g_2z-g_3}},\quad w\in\Omega$$
        Then the function $I$ has an inverse given by $\wp(z+\alpha)$ for some constant $\alpha$; that is:
        $$I(\wp(z+\alpha))=z$$
        for appropriate $\alpha$.
    \end{question}
\end{myBox}

\textbf{Pf:}

Given that $\Omega$ is a simply connected domain that excludes the roots $e_1,e_2,e_3$ of $4z^3-g_2z-g_3$,
then since this simply connected open region doesn't include the zeros for the polynomial, hence there exists a well-defined square root for the function
(can be denoted by $\sqrt{4z^3-g_2z-g_3}$).

Then, given the definition of $I(w)$ above (as an antiderivative of $\frac{1}{\sqrt{4z^3-g_2z-g_3}}$), its derivative $I'(w)=\frac{1}{\sqrt{4z^3-g_2z-g_3}}$.

\hfil

Now, since $\wp:\mathbb{C}\setminus L\rightarrow\mathbb{C}$ is an order 2 even elliptic function, then threre exists $\alpha_1\in \mathbb{C}\setminus L$, such that $\wp(\alpha_1)=\wp(-\alpha_1)=w_0$,
while $\wp'(\alpha_1) = -\wp'(-\alpha_1)$.

Then, given the algebraic differential equation $(\wp'(z))^2=4(\wp(z))^3-g_2\wp(z)-g_3$, then for the defined square root, we have $(\wp'(\alpha_1))^2=(\wp'(-\alpha_1))^2 = 4w_0^3-g_2w_0-g_3$. 
Which, for the defined square root, there are two cases: either $\sqrt{4w_0^3-g_2w_0-g_3} = \wp'(\alpha_1)$, or $\sqrt{4w_0^3-g_2w_0-g_3} = -\wp'(\alpha_1) = \wp'(-\alpha_1)$.
In either case, we can choose $\alpha\in \{\alpha_1,-\alpha_1\}$, such that $\sqrt{4w_0^3-g_2w_0-g_3} = \sqrt{(\wp'(\alpha))^2} = \wp'(\alpha)$ (and it still satisfies $\wp(\alpha)=w_0$).

Hence, given the function $I(\wp(z+\alpha))$ with the domain being the preimage of $\Omega$ (which is containing $0$, since $\wp(0+\alpha)=\wp(\alpha)=w_0\in\Omega$), we have the following:
$$I(\wp(0+\alpha)) = I(w_0)=\int_{w_0}^{w_0}\frac{dz}{\sqrt{4z^3-g_2z-g_3}} = 0$$
Also, if differentiate this composition of function, we get:
$$(I(\wp(z+\alpha)))' = I'(\wp(z+\alpha))\wp'(z+\alpha) = \frac{\wp'(z+\alpha)}{\sqrt{4(\wp(z+\alpha))^3-g_2(\wp(z+\alpha))-g_3}} = \frac{\wp'(z+\alpha)}{\sqrt{(\wp'(z+\alpha))^2}} = \pm 1$$
Notice that since both $I$ and $\wp$ are analytic function within the given domain, hence the composition and its derivative are both analytic;
on the other hand, since $(I(\wp(z+\alpha)))'$ has the value at $z=0$ being the following:
$$(I(\wp(z+\alpha)))'\bigm|_{z=0} = \frac{\wp'(0+\alpha)}{\sqrt{(\wp'(0+\alpha))^2}} = \frac{\wp'(\alpha)}{\sqrt{(\wp'(\alpha))^2}} = \frac{\wp'(\alpha)}{\wp'(\alpha)}=1$$
then in case for $(I(\wp(z+\alpha)))'$ to be continuous (in particular, continuous), we need $(I(\wp(z+\alpha)))' = 1$, which implies that $I(\wp(z+\alpha))=z$.
So, $\alpha$ is the desired constant, such that $\wp(z+\alpha)$ is the inverse of $I$.

\break

\section*{5}
\begin{myBox}[]{}
    \begin{question}
        Stein and Shakarchi Pg. 282 Problem 4:

        Suppose $\mathcal{T}$ is purely imaginary, say $\mathcal{T}=it$ with $t>0$. 
        Consider the division of the complex plane into congruent rectangles obtained by considering the lines $x=n/2$, $y=tm/2$ as $n$ and $m$ range over the integers.
        \begin{itemize}
            \item[(a)] Show that $\wp$ is real-valued on all these lines, and hence on the boundaries of all these rectangles.
            \item[(b)] Prove that $\wp$ maps the interior of each rectangle conformally to the uppoer (or lower) half-plane. 
        \end{itemize}
    \end{question}
\end{myBox}

\textbf{Pf:}

Assume the lattice is given by $L=\mathbb{Z}+\mathbb{Z}it$ for the $\wp$-function. Which, for all $w=n+i\cdot tm\in L$, its conjugate $\overline{w}=n-i\cdot tm\in L$. On the other hand, $-w = -n-i\cdot tm\in L$.
\begin{itemize}
    \item[(a)] \textbf{Horizontal Line:}
    
    For all point (that's not a lattice point) on the horizontal line (the line $y=\frac{tm}{2}$ for some $m\in\mathbb{Z}$), $z=x+i\cdot\frac{tm}{2}$ for some $x\in\mathbb{R}$. 
    Which, since $itm\in L$, then $\wp(x-i\cdot\frac{tm}{2})=\wp((x+i\cdot\frac{tm}{2})-itm) = \wp(x+i\cdot\frac{tm}{2})$. Then, consider the expression $2\wp(x+i\cdot\frac{tm}{2})$, we get:
    $$2\wp\left(x+i\cdot\frac{tm}{2}\right) = \wp\left(x+i\cdot\frac{tm}{2}\right)+\wp\left(x-i\cdot\frac{tm}{2}\right)$$
    $$=\left[\frac{1}{(x+itm/2)^2}+\sum_{\substack{w\in L\\w\neq 0}}\left(\frac{1}{((x+itm/2)-w)^2}-\frac{1}{w^2}\right)\right]+\left[\frac{1}{(x-itm/2)^2}+\sum_{\substack{w\in L\\w\neq 0}}\left(\frac{1}{((x-itm/2)-\overline{w})^2}-\frac{1}{\overline{w}^2}\right)\right]$$
    $$=\left(\frac{1}{(x+itm/2)^2}+\frac{1}{(x+\overline{itm/2})^2}\right)+\sum_{\substack{w\in L\\w\neq 0}}\left[\left(\frac{1}{(x+itm/2-w)^2}-\frac{1}{w^2}\right)+\left(\frac{1}{(x+\overline{itm/2}-\overline{w})^2}-\frac{1}{\overline{w}^2}\right)\right]$$
    $$=\left(\frac{1}{(x+itm/2)^2}+\frac{1}{\overline{(x+itm/2)}^2}\right)+\sum_{\substack{w\in L\\w\neq 0}}\left[\left(\frac{1}{(x+itm/2-w)^2}-\frac{1}{w^2}\right)+\left(\frac{1}{\overline{(x+itm/2-w)}^2}-\frac{1}{\overline{w}^2}\right)\right]$$
    $$= 2\textmd{Re}\left(\frac{1}{(x+itm/2)^2}\right)+\sum_{\substack{w\in L\\w\neq 0}}2\textmd{Re}\left(\frac{1}{(x+itm/2-w)^2}-\frac{1}{w^2}\right)$$
    $$2\textmd{Re}\left(\frac{1}{(x+itm/2)^2}\right)+2\sum_{\substack{w\in L\\w\neq 0}}\textmd{Re}\left(\frac{1}{(x+itm/2-w)^2}-\frac{1}{w^2}\right)$$
    (Note: the above term converges, because for each component $z$ of the series, $|Re(z)|\leq |z|$, hence if the original series converges absolutely, the above series also converges; and, the original series $\wp(x+i\cdot\frac{tm}{2})$ is absolutely convergent).
    
    Then, since $2\wp(x+i\cdot\frac{tm}{2})$ is real, so does $\wp(x+i\cdot\frac{tm}{2})$. This proves that $\wp$ is purely real on the line $y=\frac{tm}{2},\ m\in\mathbb{Z}$ with the given lattice.

    \hfil

    \textbf{Vertical Line:}

    For all non-lattice point on the vertical line (the line $x=\frac{n}{2}$ for some $n\in\mathbb{Z}$), $z=\frac{n}{2}+iy$ for some $y\in\mathbb{R}$. Which, since $n\in L$, then $\wp(-\frac{n}{2}+iy)=\wp((\frac{n}{2}+iy)-n)=\wp(\frac{n}{2}+iy)$. Then, if we consider the term $\wp(\frac{n}{2}+iy)-\overline{\wp(\frac{n}{2}+iy)}=2\textmd{Im}(\wp(\frac{n}{2}+iy))$, we get:
    $$\wp\left(\frac{n}{2}+iy\right)-\overline{\wp\left(\frac{n}{2}+iy\right)}=\wp\left(\frac{n}{2}+iy\right)-\overline{\wp\left(-\frac{n}{2}+iy\right)}$$
    $$ = \left[\frac{1}{(n/2+iy)^2}+\sum_{\substack{w\in L\\w\neq 0}}\left(\frac{1}{(n/2+iy-w)^2}-\frac{1}{w^2}\right)\right]-\overline{\left[\frac{1}{(-n/2+iy)^2}+\sum_{\substack{w\in L\\w\neq 0}}\left(\frac{1}{(-n/2+iy-(-\overline{w}))^2}-\frac{1}{(-\overline{w}) ^2}\right)\right]}$$
    $$=\left(\frac{1}{(n/2+iy)^2}-\frac{1}{\overline{(-n/2+iy)}^2}\right)+\sum_{\substack{w\in L\\w\neq 0}}\left[\left(\frac{1}{(n/2+iy-w)^2}-\frac{1}{w^2}\right)-\overline{\left(\frac{1}{(-n/2+iy+\overline{w})^2}-\frac{1}{\overline{w}^2}\right)}\right]$$
    $$=\left(\frac{1}{(n/2+iy)^2}-\frac{1}{(-n/2-iy)^2}\right)+\sum_{\substack{w\in L\\w\neq 0}}\left[\left(\frac{1}{(n/2+iy-w)^2}-\frac{1}{w^2}\right)-\left(\frac{1}{\overline{(-n/2+iy+\overline{w})}^2}-\frac{1}{\overline{(\overline{w})}^2}\right)\right]$$
    $$=\left(\frac{1}{(n/2+iy)^2}-\frac{1}{(n/2+iy)^2}\right)+\sum_{\substack{w\in L\\w\neq 0}}\left[\left(\frac{1}{(n/2+iy-w)^2}-\frac{1}{w^2}\right)-\left(\frac{1}{(-n/2-iy+w)^2}-\frac{1}{w^2}\right)\right]$$
    $$=0+\sum_{\substack{w\in L\\w\neq 0}}\left[\left(\frac{1}{(n/2+iy-w)^2}-\frac{1}{w^2}\right)-\left(\frac{1}{(n/2+iy-w)^2}-\frac{1}{w^2}\right)\right]=0$$
    This shows that $2\cdot \textmd{Im}(\wp(\frac{n}{2}+iy)) = 0$, hence $\textmd{Im}(\wp(\frac{n}{2}+iy)) = 0$, which shows that $\wp(\frac{n}{2}+iy)$ is purely real. 
    
    Then, this proves that $\wp$ is purely real on the line $x=\frac{n}{2},\ n\in\mathbb{Z}$ with the given lattice.
    
    \hfil

    \hfil

    \item[(b)] To prove the problem, we'll consider only the fundamental region given in the graph (which is made up of $4$ rectangles).

    \textbf{1. The Boundary of the rectangle is injective:}
    
    \begin{figure}[h!]
        \begin{center}
            \includegraphics*[width=55mm]{image 1.jpg}
            \caption{Illustration of the rectangles in the fundamental region}
        \end{center}
    \end{figure}

    Based on the above graph, within the fundamental region (with vertices $0,\ 1,\ (1+it),$ and $it$), the region $B_1,\ B_2$ are the two rectangles with distinct characterization (which, based on the relation of $\wp$, the two points $z,w$ in the fundamental region have the same image iff $z\equiv w$ or $z\equiv -w$ uner modulo $L$). Also, because $\wp$ has order $2$, at most $2$ distinct points in the fundamental region can be evaluated to be the same. Hence, for all points $z\in B_1$, the other point $w$ in the fundamental region with $\wp(z)=\wp(z)$ must occur in $(1+it)-B_1$ (the same case applies for $B_2$ and $(1+it)-B_2$).

    Then, since the boundary of $B_1$ and $(1+it)-B_1$ only intersects at the midpoint of the fundamental region (which by the property of $\wp$, it has order $2$, so no other points evaluated to be the same as the midpoint), then for the other point on the boundary of $B_1$, since the corresponding point with the same value lies in the boundary of $(1+it)-B_1$ (so they are not in the same boundary), then restricting to $\partial B_1$, the function $\wp$ is in fact injective (and same logic applies to $\partial B_2$).

    \hfil

    \textbf{2. Boundary surjects onto $\mathbb{R}$ by $\wp$:}

    \begin{figure}[h!]
        \begin{center}
            \includegraphics*[width=55mm]{image 2.jpg}
        \end{center}
    \end{figure}

    Given a rectangle with boundary, WLOG, up to certain rotation and reflection, can assume under this orientation, the bottom left corner is a point in the lattice (so $\wp(w)=\infty$), $p_1,p_2,p_3\notin L$ are the midpoints, with $2p_i\in L$ for each index $i$ (which corresponds to the values $\wp(p_1)=e_1,\ \wp(p_2)=e_2,$ and $\wp(p_3)=e_3$ respectively, and $\wp'$ evaluated to be $0$ at these points), and $e_1<e_3$.

    Which, for each $c_i$, since it is a closed straight line, can generate continuous path $f_i:[0,1]\rightarrow c_i$ that satisfies the given orientation in the graph, and $f_i'$ being a nonzero constant in $(0,1)$ (i.e. can view each $c_i$ as a unit interval). And, since $c_i$ is contained in the boundary of the rectangle, then $\wp(c_i)\subseteq \mathbb{R}\cup\{\infty\}$. Hence, if exclude the point $w$, when restricting the domain to each $c_i$, can view $\wp$ as a real valued function from interval $[0,1]$ to $\mathbb{R}$ (so we're treating each $c_i$ as an interval in $\mathbb{R}$). Then, there are some properties we can derive:
    \begin{itemize}
        \item $e_2\in (e_1,e_3)$: Suppose the contrary that this is false, then either $e_2<e_1,e_3$ or $e_2>e_1,e_3$ (for definiteness, consider the first case). Yet, if we choose $y\in\mathbb{R}$ such that $y\in (e_2,e_1)$ and $y\in (e_2,e_3)$, since $p_1,p_2,p_3$ maps to $e_1,e_2,e_3$ respectively, while they're the endpoints of $c_2$ and $c_3$, then by Intermediate Value Theorem, there exists $z_2\in c_2$ and $z_3\in c_3$ (which are not the endpoints $p_1,p_2,p_3$), such that $\wp(z_2)=\wp(z_3)=y$ (since each $c_i$ can be mapped to by the unit interval $[0,1]$ in a linear manner, can treat $c_i$ as an interval in $\mathbb{R}$). But, since $z_2\neq z_3$ (because they're not the endpoints, while $c_2,c_3$ only intersect at the endpoint), this violates the injectivity of $\wp$ on the boundary of the rectangle.
        Hence, $e_2\in (e_1,e_3)$ is enforced.

        \item $\wp$ is monotonic on each $c_i$: Since $\wp'$ only evaluates to be $0$ at the midpoints (the points with $a\notin L$, but $2a\in L$), then on the boundary, the only part with $\wp'=0$ is $p_1,p_2,p_3$. Which, if viewing each $c_i$ as an interval in $\mathbb{R}$, since $\wp'\neq 0$ on these intervals except at the endpoints, then the derivative (in form of $\wp' \cdot f_i'$) is either $>0$ or $<0$ for all points in the interior of $c_i$. Hence, the function must be monotonic.
        
        Also, based on the fact that $\wp(p_1)=e_1<e_2=\wp(p_2)$ and $\wp(p_2)=e_2<e_3=\wp(p_3)$ derived above, on $c_2$ and $c_3$ with the specified orientation, $\wp$ is monotonically increasing.

        \item $\wp$ is also increasing on $c_1$ and $c_4$:
    \end{itemize}
\end{itemize}

\end{document}