\documentclass{article}
\usepackage[margin = 2.54cm]{geometry} % set margin to traditional doc

%packages
\usepackage{graphicx} % Required for inserting images
\usepackage[most]{tcolorbox} %for creating environments
\usepackage{amsmath}
\usepackage{amssymb}
\usepackage{verbatim}
\usepackage[utf8]{inputenc}
\usepackage[dvipsnames]{xcolor} %for importing multiple colors
\usepackage{hyperref} %for creating links to different sections

\linespread{1.2} %controlling line spread

%define colors i like
\definecolor{myTeal}{RGB}{0,128,128}
\definecolor{myGreen}{RGB}{34,170,34}
\definecolor{mySapphire}{RGB}{15,82,186}
\definecolor{myEmerald}{RGB}{50.4, 130, 90}

%create math environments, can add [section] or [subsection] to add index counter based on sections/subsections
\newtheorem{define}{Definition}
\newtheorem{prop}{Proposition}
\newtheorem{thm}{Theorem}
\newtheorem{question}{Question}
\newtheorem{lemma}{Lemma}

%setup colored box environment for each math env above
\tcolorboxenvironment{define}{
    enhanced, colframe=myTeal!50!teal, colback=myTeal!10,
    arc=5mm, lower separated=false, fonttitle=\bfseries
}
\tcolorboxenvironment{prop}{
    enhanced, colframe=myGreen!50!black, colback=myGreen!15,
    arc=5mm, lower separated=false, fonttitle=\bfseries
}
\tcolorboxenvironment{thm}{
    enhanced, colframe=mySapphire!50!mySapphire, colback=mySapphire!15,
    arc=5mm, lower separated=false, fonttitle=\bfseries
}
\tcolorboxenvironment{question}{
    enhanced, colframe=blue!50!black, colback=blue!10,
    arc=5mm, lower separated=false, fonttitle=\bfseries
}
\tcolorboxenvironment{lemma}{
    enhanced, colframe=myEmerald!50!myEmerald, colback=myEmerald!10,
    arc=5mm, lower separated=false, fonttitle=\bfseries
}

%setup hyperlink within pdf
\hypersetup{
    colorlinks=true,
    linkcolor=blue,
    filecolor=magenta,      
    urlcolor=cyan,
    pdftitle={Overleaf Example},
    pdfpagemode=FullScreen,
}

%common command (add to template)
%general
\newcommand{\FF}{\mathbb{F}}
\newcommand{\NN}{\mathbb{N}}
\newcommand{\ZZ}{\mathbb{Z}}
\newcommand{\QQ}{\mathbb{Q}}
\newcommand{\RR}{\mathbb{R}}
\newcommand{\CC}{\mathbb{C}}

%algebra
\newcommand{\Gal}{\textmd{Gal}}
\newcommand{\Aut}{\textmd{Aut}}

%analysis
\newcommand{\Vol}{\textmd{Vol}}

%complex
\newcommand{\Real}{\textmd{Re}}
\newcommand{\Imag}{\textmd{Im}}

\title{Math CS 122B HW8 Part 2}
\author{Zih-Yu Hsieh}

\begin{document}
\maketitle

%1
\section{}
\begin{question}\label{q:1}
    Stein and Shakarchi Pg. 200-201 Exercise 4:

    Suppose $\{a_n\}_{n\in\mathbb{N}}$ is a sequence of complex numbers such that $a_n=a_m$ iff $n\equiv m\ mod\ q$ for some positive integer $q$. Define the \textbf{Dirichlet $L$-series} associated to $\{a_n\}$ by 
    $$L(s)=\sum_{n=1}^{\infty}\frac{a_n}{n^s}\quad \textmd{for Re}(s)>1$$
    Also, with $a_0=a_q$, let 
    $$Q(x)=\sum_{m=0}^{q-1}a_{q-m}e^{mx}$$
    Show, as in Exercises $15$ and $16$ of the previous chapter, that 
    $$L(s)=\frac{1}{\Gamma(s)}\int_{0}^{\infty}\frac{Q(x)x^{s-1}}{e^{qx}-1}dx\quad \textmd{for Re}(s)>1$$
    Prove as a result that $L(s)$ is continuable into the complex plane, with the only possible singularity a pole at $s=1$. In fact, $L(s)$ is regular at $s=1$ if and only if $\sum_{m=0}^{q-1}a_m=0$. Note the conection with the Direchlet $L(s,\chi)$ series, taken up to Book I Chapter 8, and that as a consequence, $L(s,\chi)$ is regular at $s=1$ if and only if $\chi$ is a non-trivial character.
\end{question}

\textbf{Pf:}

\subsection{Integral Representation of $L(s)$:}
Given $\textmd{Re}(s)>1$, and $x\in (0,\infty)$, notice that $\frac{1}{e^{qx}-1} = \frac{e^{-qx}}{1-e^{-qx}}$, with the fact that $-qx <0$, then $e^{-qx}<1$. Hence, the following expression is absolutely convergent, and converging normally for any compact subset of $(0,\infty)$:
\begin{equation}
    \label{eq:1}
    \frac{1}{e^{qx}-1}=\frac{e^{-qx}}{1-e^{-qx}} = \sum_{n=1}^{\infty}(e^{-qx})^n
\end{equation}
Since it converges normally within any compact subset of $(0,\infty)$ (the domain of integration), then the integral expression in the question can be rewritten as:
\begin{equation}
    \label{eq:2}
    \begin{split}
        \frac{1}{\Gamma(s)}\int_{0}^{\infty}\frac{Q(x)x^{s-1}}{e^{qx}-1}dx &= \frac{1}{\Gamma(s)}\int_{0}^{\infty}Q(x)x^{s-1}\left(\sum_{n=1}^{\infty}e^{-qx}\right)dx\\
        &= \frac{1}{\Gamma(s)}\sum_{n=1}^{\infty}\int_{0}^{\infty}\left(\sum_{m=0}^{q-1}a_{q-m}e^{mx}\right)x^{s-1}\cdot e^{-nqx}dx\\
        &= \frac{1}{\Gamma(s)}\sum_{n=1}^{\infty}\sum_{m=0}^{q-1}a_{q-m}\int_{0}^{\infty}x^{s-1}e^{-(nq-m)x}dx
    \end{split}
\end{equation}
Which, by swapping $r=q-m$ (where $r$ ranges from $1$ to $q$), extending from (\ref{eq:2}), we get the following:
\begin{equation}
    \label{eq:3}
    \begin{split}
        \frac{1}{\Gamma(s)}\int_{0}^{\infty}\frac{Q(x)x^{s-1}}{e^{qx}-1}dx &= \frac{1}{\Gamma(s)}\sum_{n=1}^{\infty}\sum_{r=1}^{q}a_r\int_{0}^{\infty}x^{s-1}e^{-(nq-(q-r))x}dx\\
        &= \frac{1}{\Gamma(s)}\sum_{n=1}^{\infty}\sum_{r=1}^{q}a_r\int_{0}^{\infty}x^{s-1}e^{-((n-1)q+r)x}dx\\
        &=\frac{1}{\Gamma(s)}\sum_{n=0}^{\infty}\sum_{r=1}^{q}a_r\int_{0}^{\infty}x^{s-1}e^{-(nq+r)x}dx\\
    \end{split}
\end{equation}
Then, performing substitution $u = (nq+r)x$ for each index $n$ and $r$, $du = (nq+r)dx$, which (\ref{eq:3}) becomes:
\begin{equation}
    \label{eq:4}
    \begin{split}
        \frac{1}{\Gamma(s)}\int_{0}^{\infty}\frac{Q(x)x^{s-1}}{e^{qx}-1}dx &= \frac{1}{\Gamma(s)}\sum_{n=0}^{\infty}\sum_{r=1}^{q}a_r\int_{0}^{\infty}\left(\frac{u}{nq+r}\right)^{s-1}\cdot e^{-u}\frac{du}{nq+r}\\
        &= \frac{1}{\Gamma(s)}\sum_{n=0}^{\infty}\sum_{r=1}^{q}a_r\cdot\frac{1}{(nq+r)^s}\int_{0}^{\infty}u^{s-1}e^{-u}du\\
        &= \frac{1}{\Gamma(s)}\sum_{n=0}^{\infty}\sum_{r=1}^{q}\frac{a_{r}}{(nq+r)^s}\cdot\Gamma(s) = \sum_{n=0}^{\infty}\sum_{r=1}^{q}\frac{a_{r}}{(nq+r)^s}
    \end{split}
\end{equation}
Now, in terms of the original $L(s)$, recall that $a_n=a_m$ iff $n\equiv m\mod\ q$, so the original series expression can be rearranged as:
\begin{equation}
    \label{eq:5}
    \begin{split}
        L(s)&=\sum_{k=1}^{\infty}\frac{a_k}{k^s} = \sum_{n=1}^{\infty}\frac{a_{nq}}{(nq)^s}+\sum_{n=0}^{\infty}\sum_{r=1}^{q-1}\frac{a_{nq+r}}{(nq+r)^s}\\
        &=\sum_{n=0}^{\infty}\frac{a_{q}}{(nq+q)^s}+\sum_{n=0}^{\infty}\sum_{r=1}^{q-1}\frac{a_{r}}{(nq+r)^s} = \sum_{n=0}^{\infty}\sum_{r=1}^{q}\frac{a_r}{(nq+r)^s}
    \end{split}
\end{equation}
Then, combining the results in (\ref{eq:4}) and (\ref{eq:5}), we get $L(s)=\frac{1}{\Gamma(s)}\int_{0}^{\infty}\frac{Q(x)x^{s-1}}{e^{qx}-1}dx$ (for $\textmd{Re}(s)>1$).

\subsection{Continuation to $\mathbb{C}\setminus\{1\}$:}
With the above integral expression, one can separate the integration as follow:
\begin{equation}
    \label{eq:6}
    \begin{split}
        L_1(s):=&\frac{1}{\Gamma(s)}\int_{0}^{\frac{1}{q}}\frac{Q(x)x^{s-1}}{e^{qx}-1}dx,\quad L_2(s):=\frac{1}{\Gamma(s)}\int_{\frac{1}{q}}^{\infty}\frac{Q(x)x^{s-1}}{e^{qx}-1}dx\\
        &L(s) = \frac{1}{\Gamma(s)}\int_{0}^{\infty}\frac{Q(x)x^{s-1}}{e^{qx}-1}dx = L_1(s)+L_2(s)
    \end{split}
\end{equation}
Since $Q(x)=\sum_{m=0}^{q-1}a_{q-m}e^{mx}$, it is dominated by $e^{(q-1)x}$. Hence, there exists $K>0$, such that $|Q(x)|\leq \sum_{m=0}^{q-1}|a_{q-m}|e^{mx}\leq Ke^{q-1}x$. Then, for $x>\frac{1}{q}$ and any $s\in\mathbb{C}$, since $qx >1$, then $e^{qx}>e>2$, so $\frac{1}{2}e^{qx}>1$, or $\frac{1}{2}e^{qx}=e^{qx}-\frac{1}{2}e^{qx}<e^{qx}-1$. Then, $L_2(s)$ satisfies the following inequality:
\begin{equation}
    \label{eq:7}
    \begin{split}
        |L_2(s)|&\leq \frac{1}{|\Gamma(s)|}\int_{\frac{1}{q}}^{\infty}\frac{|Q(x)|\cdot|x^{s-1}|}{|e^{qx}-1|}dx \leq \frac{1}{|\Gamma(s)|}\int_{\frac{1}{q}}^{\infty}\frac{Ke^{(q-1)x}\cdot x^{\textmd{Re}(s)-1}}{e^{qx}-1}dx\\
        &\leq \frac{1}{|\Gamma(s)|}\int_{\frac{1}{q}}^{\infty}\frac{Ke^{(q-1)x}\cdot x^{Re(s)-1}}{\frac{1}{2}e^{qx}}dx = \frac{2K}{|\Gamma(s)|}\int_{\frac{1}{q}}^{\infty}x^{\Real(s)-1}\cdot e^{-x}dx < \infty
    \end{split}
\end{equation}
Hence, $L_2(s)$ is an entire function.

Now, if consider $L_1(s)$, we'll utilize the power series of $e^{mz}$ (for all $z\in\CC$) and $\frac{z}{e^z-1}$ (for $|z|<2\pi$) given as follow:
\begin{equation}
    \label{eq:8}
    e^{mz} = \sum_{k=0}^{\infty}\frac{(mz)^n}{n!},\quad \frac{z}{e^z-1}=\sum_{k=0}^{\infty}\frac{B_k}{k!}z^k
\end{equation}
(Note: $\{B_k\}_{k\in\NN}$ denotes the sequence of \textbf{Bernoulli Numbers}, and $B_0=1$).

Then, under the power series, the integral in $L_1(s)$ can be expressed as follow for $\Real(s)>1$:
\begin{equation}
    \label{eq:9}
    \begin{split}
        \int_{0}^{\frac{1}{q}}\frac{Q(x)x^{s-1}}{e^{qx}-1}dx  &= \sum_{m=0}^{q-1}a_{q-m}\int_{0}^{\frac{1}{q}}\frac{e^{mx}x^{s-1}}{e^{qx}-1}dx\\
        &= a_q\int_{0}^{\frac{1}{q}}\frac{x^{s-1}}{e^{qx}-1}dx + \sum_{m=1}^{q-1}a_{q-m}\left(\sum_{n=0}^{\infty}\frac{1}{n!}\int_{0}^{\frac{1}{q}}\frac{(mx)^nx^{s-1}}{e^{qx}-1}dx\right)\\
        &= a_q\int_{0}^{1}\frac{(u/q)^{s-1}}{e^u-1}\frac{du}{q} + \sum_{m=1}^{q-1}\sum_{n=0}^{\infty}\frac{a_{q-m}\cdot m^n}{n!}\int_{0}^{\frac{1}{q}}\frac{x^{s+n-1}}{e^{qx}-1}dx\\
        &= \frac{a_q}{q^s}\int_{0}^{1}\frac{u\cdot u^{s-2}}{e^u-1}du + \sum_{m=1}^{q-1}\sum_{n=0}^{\infty}\frac{a_{q-m}\cdot m^n}{n!}\int_{0}^{1}\frac{(u/q)^{s+n-1}}{e^u-1}\frac{du}{q}\\
        &=\frac{a_q}{q^s}\sum_{k=0}^{\infty}\frac{B_k}{k!}\int_{0}^{1}u^{s+k-2}du + \frac{1}{q^s}\sum_{m=1}^{q-1}\sum_{n=0}^{\infty}\frac{a_{q-m}\cdot m^n}{n!\cdot q^n}\int_{0}^{1}\frac{u\cdot u^{s+n-2}}{e^u-1}du\\
        &= \frac{a_q}{q^s}\sum_{k=0}^{\infty}\frac{B_k}{k!(s+k-1)}u^{s+k-1}\bigg|_{0}^{1} + \frac{1}{q^s}\sum_{m=1}^{q-1}\sum_{n=0}^{\infty}\frac{a_{q-m}\cdot m^n}{n!\cdot q^n}\left(\sum_{k=0}^{\infty}\int_{0}^{1}u^{s+n+k-2}du\right)\\
        &= \frac{a_q}{q^s}\sum_{k=0}^{\infty}\frac{B_k}{k!(s+k-1)}+\frac{1}{q^s}\sum_{m=1}^{q-1}\sum_{n=0}^{\infty}\sum_{k=0}^{\infty}\frac{a_{q-m}\cdot m^n}{n!\cdot q^n(s+n+k-1)}u^{s+n+k-1}\bigg|_{0}^{1}\\
        &= \frac{a_q}{q^s}\sum_{k=0}^{\infty}\frac{B_k}{k!(s+k-1)}+\frac{1}{q^s}\sum_{m=1}^{q-1}\sum_{n=0}^{\infty}\sum_{k=0}^{\infty}\frac{a_{q-m}\cdot (m/q)^n}{n!(s+n+k-1)}\\
        &= \frac{a_q}{q^s(s-1)} + \frac{a_q}{q^s}\sum_{k=1}^{\infty}\frac{B_k}{k!(s+k-1)} + \frac{1}{q^s}\sum_{m=1}^{q-1}\left(\frac{a_{q-m}}{s-1}+\sum_{\substack{n,k=0\\(n,k)\neq (0,0)}}^{\infty}\frac{a_{q-m}(m/q)^n}{n!(s+n+k-1)}\right)\\
        &= \sum_{m=0}^{q-1}\frac{a_{q-m}}{q^s(s-1)} + \frac{a_q}{q^s}\sum_{k=1}^{\infty}\frac{B_k}{k!(s+k-1)} + \frac{1}{q^s}\sum_{m=1}^{q-1}\sum_{\substack{n,k=0\\(n,k)\neq (0,0)}}^{\infty}\frac{a_{q-m}(m/q)^n}{n!(s+n+k-1)}
    \end{split}
\end{equation}
Which, using the infinite series expression above as the analytic continuation, we see that the integral as potential simple pole at $s=1$ (the first summation), and has simple pole at all integer $\ZZ_{\leq 0}$ (the second and third summation with $(s+k-1)$ or $(s+n+k-1)$ as denominator, for $k\geq 1$ and $(n,k)\neq (0,0)$).

Then, because $\frac{1}{\Gamma(s)}$ only has simple zeros at all $\ZZ_{\leq 0}$, the expression $L_1(s)= \frac{1}{\Gamma(s)}\int_{1}^{\frac{1}{q}}\frac{Q(x)x^{s-1}}{e^{qx}-1}dx$ has all the simple poles of the integral at $\ZZ_{\leq 0}$ cancelled out by $\frac{1}{\Gamma(s)}$, and left with a potential pole at $s=1$.

\hfil

Finally, for the summation expression of the integral in (\ref{eq:9}), we get that the potential pole at $s=1$ is described by $\sum_{m=0}^{q-1}\frac{a_{q-m}}{q^s(s-1)}=\frac{1}{q^s(s-1)}\sum_{m=0}^{q-1}$. Since at $s=1$, $\frac{1}{\Gamma(s)}\neq 0$, then after multiplying with $\frac{1}{\Gamma(s)}$, this expression is regular at $s=1$ iff the coefficient $\sum_{m=0}^{q-1}a_{q-m}=\sum_{r=1}^{q}a_r = 0$. 

Hence, we can conclude that $L(s)$ can be continued onto the whole plane, except possibly at $s=1$. And, with the last information from above, $L(s)$ is regular at $s=1$ iff $\sum_{r=1}^{q}a_r=0$.

\break

%2
\section{}
\begin{question}\label{q:2}
    Stein and Shakarchi Pg. 204 Problem 4:

    One can combine ideas from the prime number theorem with the proof of Dirichlet's Theorem about primes in arithmetic progression (given in Book I) to prove the following: Let $q$ and $l$ be relatively prime integers. We consider the primes belonging to the arithmetic progression $\{qk+ll\}_{k\in\mathbb{N}}$, and let $\pi_{q,l}(x)$ denote the number of such primes $\leq x$. Then one has 
    $$\pi_{q,l}(x)\sim \frac{x}{\varphi(q)\log(x)}\quad \textmd{as }x\rightarrow\infty$$
    where $\varphi(q)$ denotes the number of positive integers less than $q$ and relatively prime to $q$ (i.e. the Euler Totient Function).
\end{question}

\textbf{Pf:}

Given $q,l\in\mathbb{N}$ with $\gcd(q,l)=1$. Find an expression of Dirichlet Series, that produces the following formula:
$$L(s):=\prod_{p}\frac{1}{1-\delta_l(p)p^{-s}}$$
Where $p$ ranges through all primes, and $\delta_l:\mathbb{N}\rightarrow\mathbb{N}$ is defined as follow:
$$\delta_l(n)=\begin{cases}
    1 & n\equiv l\mod\ q\\
    0 & \textmd{otherwise}
\end{cases}$$

\end{document}