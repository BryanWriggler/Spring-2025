\documentclass{article}
\usepackage{graphicx} % Required for inserting images
\usepackage[margin = 2.54cm]{geometry}
\usepackage[most]{tcolorbox}

\newtcolorbox{myBox}[3]{
arc=5mm,
lower separated=false,
fonttitle=\bfseries,
%colbacktitle=green!10,
%coltitle=green!50!black,
enhanced,
attach boxed title to top left={xshift=0.5cm,
        yshift=-2mm},
colframe=blue!50!black,
colback=blue!10
}

\usepackage{amsmath}
\usepackage{amssymb}
\usepackage{verbatim}
\usepackage[utf8]{inputenc}
\linespread{1.2}

\newtheorem{definition}{Definition}
\newtheorem{proposition}{Proposition}
\newtheorem{theorem}{Theorem}
\newtheorem{question}{Question}

\title{Math CS 122B HW7}
\author{Zih-Yu Hsieh}

\begin{document}
\maketitle

\section*{1}
\begin{myBox}[]{}
    \begin{question}
        The functional equation of the $\zeta$-function can also be written in the following form:
        $$\zeta(1-s)=2(2\pi)^{-s}\Gamma(s)\cos\left(\frac{\pi s}{2}\right)\zeta(s)$$
        Deduce from this: In the half-plane $\sigma\leq 0$, the function $\zeta(s)$ has exactly the zeros $s=-2k,\ k\in\mathbb{N}$. All other zeros of the $\zeta$-function are located in the vertical strip $0<\textmd{Re} s<1$.
    \end{question}
\end{myBox}

\textbf{Pf:}

First, recall that for the half plane $\sigma>1$, the following inequality is given:
$$\left|\frac{\zeta(\sigma+it)}{\sigma-1}\right|^4|\zeta(\sigma+2it)|[\zeta(\sigma)(\sigma-1)]^3\geq (\sigma-1)^{-1}$$
Since for $\sigma>1$, the expressiong $(\sigma-1)^{-1}>0$, this enforces all $s = \sigma+it$ in the half plane to have $\zeta(s)\neq 0$ (or else the left side of the inequality is $0$, which violates the inequality). Similarly, this inequality can be extended onto the line $\textmd{Re}(s)=1$, where $\zeta(s)$ has no zeros on this line also. So, for $\sigma\geq 1$, $\zeta(s)$ has no zero.

\hfil

Now, in the half plane $\sigma\leq 0$, for all $s'\neq 0$, since it can be written as $s'=1-s$, where $s=1-s'$ has $\textmd{Re}(s)=1-\textmd{Re}(s') \geq 1$ (and since $s'\neq 0$, then $s\neq 1$). Then, by the functional equation, we get the following:
$$\zeta(s')=\zeta(1-s)=2(2\pi)^{-s}\Gamma(s)\cos\left(\frac{\pi s}{2}\right)\zeta(s)$$
Since $\textmd{Re}(s)\geq 1$ with $s\neq 1$, then $\zeta(s)\neq 0$ based on what is mentioned during the start; also, $\Gamma(s)\neq 0$ for all $s\in \mathbb{C}\setminus \{0,-1,-2,...\}$, while $2(2\pi)^{-s}\neq 0$ for all $s\in\mathbb{C}$. Hence, in case for $\zeta(1-s)=0$, we must have $\cos(\frac{\pi s}{2})=0$, which enforces $\frac{\pi s}{2} = k\pi + \frac{\pi}{2}$ for some $k\in\mathbb{Z}$, or $s = 2k+1$ fo some $k\in\mathbb{Z}$. Now, under this assumption, since $\textmd{Re}(s)\geq 1$ while $s\neq 1$, then $k\geq 1$. So, when transfering back to $s'=1-s$, we get $s' = 1-(2k+1)=-2k$ for integer $k\geq 1$.

Hence, for $\textmd{Re}(s')\leq 0$, if $\zeta(s')=0$, then $s' = -2k$ for some $k\in\mathbb{N}$ (this is an iff since at all these points, $s=2-s'$ satisfies $\cos(\frac{\pi s}{2})=0$, which $\zeta(s')=\zeta(1-s)=0$).

\hfil

Finally, for $s'=0$ (where if $s'=1-s$, $s=1$). Recall that $\zeta(s)$ has a simple pole at $s=1$, while $\cos(\frac{\pi s}{2})$ has a simple zero at $s=1$ (where the input is $\frac{\pi}{2}$, where $\cos$ is $0$). Hence, $\cos(\frac{\pi s}{2}) = (s-1)h(z)$ for some analytic function $h$ where $h(1)\neq 0$. Also, we know $\lim_{s\rightarrow 1}(s-1)\zeta(s) = 1$ (has been given in the textbook). Then, we get the following:
$$\lim_{s\rightarrow 1}\zeta(1-s) = \lim_{s\rightarrow 1}2(2\pi)^{-s}\Gamma(s)h(s)(s-1)\zeta(s) = 2(2\pi)^{-1}\Gamma(1)h(1)\cdot \lim_{s\rightarrow 1}(s-1)\zeta(s) = 2(2\pi)^{-1}\Gamma(1)h(1)\neq 0$$
Hence, we can deduce that at $s=1$ (where $s'=1-s=0$), $\zeta(s')$ has a removable singularity that has limit not being $0$, so $\zeta(s')$ as an extension has $\zeta(0)\neq 0$.

\hfil

The above casees proves that when $\sigma\geq 1$ or $\sigma\leq 0$, $\zeta(s)=0$ iff $s=-2k$ for some $k\in\mathbb{N}$, where for any other input $\zeta$ is nonzero.

Hence, if there are any other zeros, it must exist in the vertical strip $0<\textmd{Re}(s)<1$.

\break

\section*{2}
\begin{myBox}[]{}
    \begin{question}
        The following special case of the Hecke Theorem was already known to B. Riemann (1859):
        $$\xi(s):=\pi ^{-s/2}\Gamma\left(\frac{s}{2}\right)\zeta(s)=\sum_{n=1}^{\infty}\int_{0}^{\infty}e^{-\pi n^2t}t^{s/2}\frac{dt}{t}$$
        $$=\frac{1}{2}\int_{1}^{\infty}(\vartheta(it)-1)(t^{s/2}+t^{(1-s)/2})\frac{dt}{t}-\frac{1}{s}-\frac{1}{1-s}$$
        Deduce directly this special case, and use it to prove the meromorphic continuation and the functional equation.
    \end{question}
\end{myBox}

\textbf{Pf:}

For this problem, first assume $\textmd{Re}(s)>1$ (where $\zeta(s)$ is defined with the original series form). We'll break down into two different equations:

\hfil

\textbf{Equation 1:}

We'll first prove the following:
$$\pi^{-s/2}\Gamma\left(\frac{s}{2}\right)\zeta(s)=\sum_{n=1}^{\infty}\int_{0}^{\infty}e^{-\pi n^2t}t^{s/2}\frac{dt}{t}$$
For each $n\in\mathbb{N}$, the integral within the right hand side summation, after doing the substitution $u = \pi n^2t$ (where $du=\pi n^2dt$), we get the following:
$$\int_{0}^{\infty}e^{-\pi n^2t}t^{s/2}\frac{dt}{t} = \int_{0}^{\infty}e^{-u}\left(\frac{u}{\pi n^2}\right)^{s/2}\frac{du/(\pi n^2)}{u/(\pi n^2)} = (\pi n^2)^{-s/2}\int_{0}^{\infty}e^{-u}u^{s/2}\frac{du}{u}$$
$$ = \pi^{-s/2}n^{-s}\int_{0}^{\infty}e^{-u}u^{s/2-1}du = \pi^{-s/2}n^{-s}\Gamma\left(\frac{s}{2}\right)$$
Hence, for the series, since $\zeta(s)=\sum_{n=1}^{\infty}n^{-s}$ converges normally within $\textmd{Re}(s)>1$, we get the following:
$$\sum_{n=1}^{\infty}\int_{0}^{\infty}e^{-\pi n^2t}t^{s/2}\frac{dt}{t} = \sum_{n=1}^{\infty}\pi^{-s/2}n^{-s}\Gamma\left(\frac{s}{2}\right) = \pi^{-s/2}\Gamma\left(\frac{s}{2}\right)\sum_{n=1}^{\infty}n^{-s} = \pi^{-s/2}\Gamma\left(\frac{s}{2}\right)\zeta(s)$$
So, the first equatity holds.

\hfil

\textbf{Equation 2:}

Our second goal is to prove the following:
$$\sum_{n=1}^{\infty}\int_{0}^{\infty}e^{-\pi n^2t}t^{s/2}\frac{dt}{t}=\frac{1}{2}\int_{1}^{\infty}(\vartheta(it)-1)(t^{s/2}+t^{(1-s)/2})\frac{dt}{t}-\frac{1}{s}-\frac{1}{1-s}$$

Since the summation is absolutely convergent for $\textmd{Re}(s)>1$ (based on how $\xi(s)$ is defined), then swapping the order of summation causes no issue. Hence, for all $n\in\mathbb{N}$, since $n^2=(-n)^2$, the series can also be decomposed as:
$$\sum_{n=1}^{\infty}\int_{0}^{\infty}e^{-\pi n^2t}t^{s/2}\frac{dt}{t} = \frac{1}{2}\sum_{n=1}^{\infty}\int_{0}^{\infty}e^{-\pi n^2t}t^{s/2}\frac{dt}{t}+\frac{1}{2}\sum_{n=-\infty}^{-1}\int_{0}^{\infty}e^{-\pi n^2t}t^{s/2}\frac{dt}{t}$$
$$= \frac{1}{2}\sum_{\substack{n\in\mathbb{Z}\\n\neq 0}}\int_{0}^{\infty}e^{-\pi n^2t}t^{s/2}\frac{dt}{t} =\frac{1}{2}\sum_{\substack{n\in\mathbb{Z}\\n\neq 0}}\int_{1}^{\infty}e^{-\pi n^2t}t^{s/2}\frac{dt}{t}+\frac{1}{2}\sum_{\substack{n\in\mathbb{Z}\\n\neq 0}}\int_{0}^{1}e^{-\pi n^2t}t^{s/2}\frac{dt}{t}$$
For the first summation, recall that the Theta Series is defined as $\vartheta:\mathbb{H}\rightarrow\mathbb{C},\ \vartheta(\tau) = \sum_{n=-\infty}^{\infty}e^{i\pi n^2\tau}$.
Since for $n=0$, $e^{i\pi n^2\tau} = e^0=1$, then we get the following:
$$\sum_{\substack{n\in\mathbb{Z}\\n\neq 0}}e^{i\pi n^2\tau} = \vartheta(\tau)-1$$
If consider $\tau=it$ for $t\in(0,\infty)$, we get the following:
$$\vartheta(it)-1 = \sum_{\substack{n\in\mathbb{Z}\\n\neq 0}}e^{i\pi n^2\cdot it} = \sum_{\substack{n\in\mathbb{Z}\\n\neq 0}}e^{-\pi n^2t}$$
So, for the first summation in the equation, since $\vartheta$ is a normally convergent series of function, the summation and integral can change the order of operation. Hence:
$$\frac{1}{2}\sum_{\substack{n\in\mathbb{Z}\\n\neq 0}}\int_{1}^{\infty}e^{-\pi n^2t}t^{s/2}\frac{dt}{t} = \frac{1}{2}\int_{1}^{\infty}\left(\sum_{\substack{n\in\mathbb{Z}\\n\neq 0}}e^{-\pi n^2t}\right)t^{s/2}\frac{dt}{t} = \frac{1}{2}\int_{1}^{\infty}(\vartheta(it)-1)t^{s/2}\frac{dt}{t}$$
Now, to work on the second summation, for all $n\in\mathbb{Z}$ with $n\neq 0$, given the substitution $u=\frac{1}{t}$, $du=-\frac{1}{t^2}dt$ (or $\frac{dt}{t}=-tdu = -\frac{du}{u}$), we get the following integral representation:
$$\int_{0}^{1}e^{-\pi n^2t}t^{s/2}\frac{dt}{t} = -\int_{\infty}^{1}e^{-\pi n^2\cdot 1/u}u^{-s/2}\frac{du}{u} = \int_{1}^{\infty}e^{-\pi n^2\cdot 1/u}u^{-s/2}\frac{du}{u}$$
Which, based on the above change of variable and the property of Theta Series, the second summation in the equation can be rewrite as:
$$\frac{1}{2}\sum_{\substack{n\in\mathbb{Z}\\n\neq 0}}\int_{0}^{1}e^{-\pi n^2t}t^{s/2}\frac{dt}{t} = \frac{1}{2}\sum_{\substack{n\in\mathbb{Z}\\n\neq 0}}\int_{1}^{\infty}e^{-\pi n^2\cdot 1/u}u^{-s/2}\frac{du}{u} = \frac{1}{2}\int_{1}^{\infty}\left(\sum_{\substack{n\in\mathbb{Z}\\n\neq 0}}e^{-\pi n^2\cdot 1/u}\right)u^{-s/2}\frac{du}{u}$$
$$ = \frac{1}{2}\int_{1}^{\infty}\left(\vartheta\left(\frac{i}{u}\right)-1\right)u^{-s/2}\frac{du}{u}$$
Which, there is a property of Theta Series given below:
$$\vartheta\left(-\frac{1}{z}\right)=\sqrt{\frac{z}{i}}\vartheta(z)$$
Hence, for $u\in(0,\infty)$, let $z=iu$, we get the following:
$$\vartheta\left(\frac{i}{u}\right)=\vartheta\left(-\frac{1}{iu}\right) = \sqrt{\frac{iu}{i}}\vartheta(iu) = u^{1/2}\vartheta(iu)$$
So, the summation can be modified as follow:
$$\frac{1}{2}\sum_{\substack{n\in\mathbb{Z}\\n\neq 0}}\int_{0}^{1}e^{-\pi n^2t}t^{s/2}\frac{dt}{t} = \frac{1}{2}\int_{1}^{\infty}\left(u^{1/2}\vartheta(iu)-1\right)u^{-s/2}\frac{du}{u}$$
$$ = \frac{1}{2}\int_{1}^{\infty}(\vartheta(iu)-1)u^{1/2}u^{-s/2}\frac{du}{u} + \frac{1}{2}\int_{1}^{\infty}u^{1/2}u^{-s/2}\frac{du}{u}-\frac{1}{2}\int_{1}^{\infty}u^{-s/2}\frac{du}{u}$$
For the integrals at the middle and the right, since the power of $u$ is given by $\frac{1-s}{2}-1$ and $\frac{-s}{2}-1$, then because $\textmd{Re}(s)>1$, the two powers of $u$ both have the real parts being less than $-1$. Hence, the integral absolutely converges, and using power rule, we yield:
$$\frac{1}{2}\int_{1}^{\infty}u^{1/2}u^{-s/2}\frac{du}{u} = \frac{1}{2}\int_{1}^{\infty}u^{\frac{1-s}{2}-1}du = \frac{1}{2}\cdot\frac{2}{1-s}u^{\frac{1-s}{2}}\bigg|_{1}^{\infty} = -\frac{1}{1-s}$$
$$\frac{1}{2}\int_{1}^{\infty}u^{-s/2}\frac{du}{u}=\frac{1}{2}\int_{1}^{\infty}u^{-s/2-1}du = -\frac{1}{2}\cdot\frac{2}{s}u^{-s/2}\bigg|_{1}^{\infty} = \frac{1}{s}u^{-s/2}\bigg|_{\infty}^{1} = \frac{1}{s}$$
(Note: since $\textmd{Re}(s)>1$, then $\textmd{Re}(\frac{1-s}{2})<0$ and $\textmd{Re}(-\frac{s}{2})<0$, so $\lim_{u\rightarrow\infty}u^{\frac{1-s}{2}}=0$ and $\lim_{u\rightarrow\infty}u^{-s/2}=0$).
So, combining the pieces for the second summation, we get:
$$\frac{1}{2}\sum_{\substack{n\in\mathbb{Z}\\n\neq 0}}\int_{0}^{1}e^{-\pi n^2t}t^{s/2}\frac{dt}{t} = \frac{1}{2}\int_{1}^{\infty}(\vartheta(iu)-1)u^{1/2}u^{-s/2}\frac{du}{u} + \frac{1}{2}\int_{1}^{\infty}u^{1/2}u^{-s/2}\frac{du}{u}-\frac{1}{2}\int_{1}^{\infty}u^{-s/2}\frac{du}{u}$$
$$ = \frac{1}{2}\int_{1}^{\infty}(\vartheta(it)-1)t^{(1-s)/2}\frac{dt}{t}+\left(-\frac{1}{1-s}\right) - \frac{1}{s} = \frac{1}{2}\int_{1}^{\infty}(\vartheta(it)-1)t^{(1-s)/2}\frac{dt}{t} - \frac{1}{s}-\frac{1}{1-s}$$
Finally, the original equation can be obtained as follow:
$$\sum_{n=1}^{\infty}\int_{0}^{\infty}e^{-\pi n^2t}t^{s/2}\frac{dt}{t} = \frac{1}{2}\sum_{\substack{n\in\mathbb{Z}\\n\neq 0}}\int_{1}^{\infty}e^{-\pi n^2t}t^{s/2}\frac{dt}{t}+\frac{1}{2}\sum_{\substack{n\in\mathbb{Z}\\n\neq 0}}\int_{0}^{1}e^{-\pi n^2t}t^{s/2}\frac{dt}{t}$$
$$ = \left(\frac{1}{2}\int_{1}^{\infty}(\vartheta(it)-1)t^{s/2}\frac{dt}{t}\right)+\left(\frac{1}{2}\int_{1}^{\infty}(\vartheta(it)-1)t^{(1-s)/2}\frac{dt}{t} - \frac{1}{s}-\frac{1}{1-s}\right)$$
$$ = \frac{1}{2}\int_{1}^{\infty}(\vartheta(it)-1)(t^{s/2}+t^{(1-s)/2})\frac{dt}{t}-\frac{1}{s}-\frac{1}{1-s}$$
This verifies the second equation. Together with the first and the second equation proven, the function $\xi(s)$ defined in the question for $\textmd{Re}(s)>1$ satisfies the following equation:
$$\xi(s):=\pi ^{-s/2}\Gamma\left(\frac{s}{2}\right)\zeta(s)=\sum_{n=1}^{\infty}\int_{0}^{\infty}e^{-\pi n^2t}t^{s/2}\frac{dt}{t}$$
$$=\frac{1}{2}\int_{1}^{\infty}(\vartheta(it)-1)(t^{s/2}+t^{(1-s)/2})\frac{dt}{t}-\frac{1}{s}-\frac{1}{1-s}$$

\hfil

\textbf{Meromorphic Continuation \& Functional equation of $\zeta$:}

The textbook had introduced the meromorphic continuation of $\zeta$ onto the half plane $\textmd{Re}(s)>0$. Now, consider any $s$ satisfying $\frac{1}{2}<\textmd{Re}(s)<1$, then $(1-s)$ satisfies $\textmd{Re}(1-s)=1-\textmd{Re}(s)$, $0<\textmd{Re}(1-s)<\frac{1}{2}$. Since both $s,(1-s)$ are within the domain of the extended $\zeta$, then if plug in the $\xi$-function defined above, using the integral expression, we get:
$$\xi(s) = \int_{1}^{\infty}(\vartheta(it)-1)(t^{s/2}+t^{(1-s)/2})\frac{dt}{t}-\frac{1}{s}-\frac{1}{1-s}$$

$$\xi(1-s)\int_{1}^{\infty}(\vartheta(it)-1)(t^{(1-s)/2}+t^{(1-(1-s))/2})\frac{dt}{t}-\frac{1}{1-s}-\frac{1}{1-(1-s)}$$
$$ = \int_{1}^{\infty}(\vartheta(it)-1)(t^{s/2}+t^{(1-s)/2})\frac{dt}{t}-\frac{1}{s}-\frac{1}{1-s} = \xi(s)$$
Hence, for $s$ satisfying $\frac{1}{2}<\textmd{Re}(s)<1$, $\xi(1-s)=\xi(s)$, which leads to the following functional equation:
$$\pi^{-(1-s)/2}\Gamma\left(\frac{1-s}{2}\right)\zeta(1-s)=\xi(1-s)=\xi(s)=\pi^{-s/2}\Gamma\left(\frac{s}{2}\right)\zeta(s)$$
$$\implies \zeta(1-s) = \pi ^{1/2-s}\frac{\Gamma\left(\frac{s}{2}\right)}{\Gamma\left(\frac{1-s}{2}\right)}\zeta(s)$$
Now, recall the Duplication Formula of $\Gamma$-function:
$$\forall z\in \mathbb{C}\setminus\{0,-1,-2,...\},\quad \Gamma\left(\frac{z}{2}\right)\Gamma\left(\frac{1+z}{2}\right)=\frac{\sqrt{\pi}}{2^{z-1}}\Gamma(z)$$
Then, take $z=s$ (since $\frac{1}{2}<\textmd{Re}(s)<1$ by assumption now, it is valid), we get:
$$\Gamma\left(\frac{s}{2}\right)\Gamma\left(\frac{1+s}{2}\right)=\frac{\sqrt{\pi}}{2^{s-1}}\Gamma(s)\implies \Gamma\left(\frac{s}{2}\right)=\frac{\sqrt{\pi }\cdot\Gamma(s)}{2^{s-1}\Gamma\left(\frac{1+s}{2}\right)}$$
Hence, the expression $\zeta(1-s)$ becomes:

$$\zeta(1-s)=\pi ^{1/2-s}\frac{\Gamma\left(\frac{s}{2}\right)}{\Gamma\left(\frac{1-s}{2}\right)}\zeta(s) = \frac{\sqrt{\pi}}{\pi^s\Gamma\left(\frac{1-s}{2}\right)}\cdot\frac{\sqrt{\pi}\cdot \Gamma(s)}{2^{s-1}\Gamma\left(\frac{1+s}{2}\right)}\zeta(s)$$
$$ = \frac{2\pi \cdot\Gamma(s)}{(2\pi)^{s}}\cdot \frac{1}{\Gamma\left(\frac{1+s}{2}\right)\Gamma\left(1-\left(\frac{1+s}{2}\right)\right)}\zeta(s)$$
Which, this expression allows the use of another property of $\Gamma$-function:
$$\forall z\in \mathbb{C},\quad \frac{1}{\Gamma(z)\Gamma(1-z)}=\frac{\sin(\pi z)}{\pi}$$
Take $z=\frac{1+s}{2}$, we get:
$$\frac{1}{\Gamma\left(\frac{1+s}{2}\right)\Gamma\left(1-\left(\frac{1+s}{2}\right)\right)} =\frac{\sin\left(\pi\left(\frac{1+s}{2}\right)\right)}{\pi} = \frac{\sin\left(\frac{\pi s}{2}+\frac{\pi}{2}\right)}{\pi} = \frac{\cos\left(\frac{\pi s}{2}\right)}{\pi}$$
So, the following functional equation of $\zeta$ (on $\frac{1}{2}<\textmd{Re}(s)<1$) can be deduced:
$$\zeta(1-s)=\frac{2\pi \cdot\Gamma(s)}{(2\pi)^{s}}\cdot \frac{1}{\Gamma\left(\frac{1+s}{2}\right)\Gamma\left(1-\left(\frac{1+s}{2}\right)\right)}\zeta(s)$$
$$=\frac{2\pi \cdot\Gamma(s)}{(2\pi)^s}\cdot\frac{\cos\left(\frac{\pi s}{2}\right)}{\pi}\zeta(s) = 2(2\pi)^{-s}\Gamma(s)\cos\left(\frac{\pi s}{2}\right)\zeta(s)$$
This finishes the proof for the functional equation. Moreover, for all $s'$ with $\textmd{Re}(s')<\frac{1}{2}$, since given $s'=1-s$, or $s=1-s'$, we have $\textmd{Re}(s)=\textmd{Re}(1-s')=1-\textmd{Re}(s') > \frac{1}{2}$, then the following expression is well-defined:
$$2(2\pi)^{-s}\Gamma(s)\cos\left(\frac{\pi s}{2}\right)\zeta(s)$$
Hence, the meromorphic continuation of $\zeta$ onto the whole plane can be followed by the functional equation:
$$\forall s'\in\mathbb{C},\ \textmd{Re}(s')<\frac{1}{2},\ \textmd{let }s'=1-s,\ \textmd{Re}(s)>\frac{1}{2},\quad \textmd{Define } \zeta(1-s):=2(2\pi)^{-s}\Gamma(s)\cos\left(\frac{\pi s}{2}\right)\zeta(s)$$

\end{document}