\documentclass{article}
\usepackage{graphicx} % Required for inserting images
\usepackage[margin = 2.54cm]{geometry}
\usepackage[most]{tcolorbox}

\newtcolorbox{myBox}[3]{
arc=5mm,
lower separated=false,
fonttitle=\bfseries,
%colbacktitle=green!10,
%coltitle=green!50!black,
enhanced,
attach boxed title to top left={xshift=0.5cm,
        yshift=-2mm},
colframe=blue!50!black,
colback=blue!10
}

\usepackage{amsmath}
\usepackage{amssymb}
\usepackage{verbatim}
\usepackage[utf8]{inputenc}
\linespread{1.2}

\newtheorem{definition}{Definition}
\newtheorem{proposition}{Proposition}
\newtheorem{theorem}{Theorem}
\newtheorem{question}{Question}

\title{Math CS 122B HW2}
\author{Zih-Yu Hsieh}

\begin{document}
\maketitle

\section*{1}
\begin{myBox}[]{}
    \begin{question}
        The \textbf{Beta function} is defined for $Re(\alpha)>0$ and $\Re(\beta)>0$ by
        $$B(\alpha,\beta)=\int_{0}^{1}(1-t)^{\alpha-1}t^{\beta-1}dt$$
        \begin{itemize}
            \item[(a)] Prove that 
            $$B(\alpha,\beta)=\frac{\Gamma(\alpha)\Gamma(\beta)}{\Gamma(\alpha+\beta)}$$
            \item[(b)] Show that
            $$B(\alpha,\beta)=\int_{0}^{\infty}\frac{u^{\alpha-1}}{(1+u)^{\alpha+\beta}}du$$
        \end{itemize}
    \end{question}
\end{myBox}

\textbf{Pf:}

\begin{itemize}
    \item[(a)] First, we'll consider $\Gamma(\alpha)\Gamma(\beta)$:
    $$\Gamma(\alpha)\Gamma(\beta)=\int_{0}^{\infty}t^{\alpha-1}e^{-t}dt\int_{0}^{\infty}s^{\beta-1}e^{-s}ds=\int_{0}^{\infty}\int_{0}^{\infty}t^{\alpha-1}s^{\beta-1}e^{-s-t}dsdt$$
    If we consider the change of variable $f:(0,1)\times (0,\infty)\rightarrow (0,\infty)\times (0,\infty)$ by $f(r,u)=(ur,u(1-r))=(s,t)$, since this is a differentiable function, and its derivative is given by:
    $$Df = \begin{pmatrix}
        \frac{\partial f_1}{\partial r}&\frac{\partial f_1}{\partial u}\\
        \frac{\partial f_2}{\partial r}&\frac{\partial f_2}{\partial u}
    \end{pmatrix}=\begin{pmatrix}
        \frac{\partial}{\partial r}(ur)&\frac{\partial}{\partial u}(ur)\\
        \frac{\partial}{\partial r}(u(1-r))&\frac{\partial}{\partial u}(u(1-r))
    \end{pmatrix}=\begin{pmatrix}
        u&r\\
        -u&(1-r) 
    \end{pmatrix}$$
    $$\frac{\partial (s,t)}{\partial (r,u)}=|\begin{vmatrix}
        u&r\\-u&(1-r)
    \end{vmatrix}|=u(1-r)-(-u)r=u$$
    Then, the above integral can be given as:
    $$\Gamma(\alpha)\Gamma(\beta)=\int_{0}^{\infty}\int_{0}^{\infty}t^{\alpha-1}s^{\beta-1}e^{-s-t}dsdt = \int_{0}^{\infty}\int_{0}^{1}(u(1-r))^{\alpha-1}(ur)^{\beta-1}e^{-(ur+u(1-r))}\left|\frac{\partial (s,t)}{\partial (r,u)}\right|drdu$$
    $$=\int_{0}^{\infty}\int_{0}^{1}u^{\alpha+\beta-2}e^{-u}\cdot (1-r)^{\alpha-1}r^{\beta-1}udrdu=\int_{0}^{\infty}u^{\alpha+\beta-1}e^{-u}\cdot (1-r)^{\alpha-1}r^{\beta-1}drdu$$
    $$=\int_{0}^{\infty}u^{\alpha+\beta-1}e^{-u}du\cdot\int_{0}^{1}(1-r)^{\alpha-1}r^{\beta-1}dr = \Gamma(\alpha+\beta)\cdot B(\alpha,\beta)$$
    Hence, we can rewrite the above equality to be as follow:
    $$\Gamma(\alpha)\Gamma(\beta)=\Gamma(\alpha+\beta)\cdot B(\alpha,\beta),\quad B(\alpha,\beta)=\frac{\Gamma(\alpha)\Gamma(\beta)}{\Gamma(\alpha+\beta)}$$
    (Recall: $\Gamma(z)\neq 0$ for all $z\in\mathbb{C}\setminus S$, $S=\{0,-1,-2,...\}$).
    
    \hfil 

    \item[(b)] Consider the following expression:
    $$\int_{0}^{\infty}\frac{u^{\alpha-1}}{(1+u)^{\alpha+\beta}}du$$
    First, if we do the substitution $(1+u)=e^t$, $du=e^tdt$, which $u=0\implies e^t=1,\ t=0$, and $\lim_{t\rightarrow\infty}e^t=\infty$, so $\lim_{t\rightarrow\infty}u=\infty$. Then, the integral can be rewrite as:
    $$\int_{0}^{\infty}\frac{u^{\alpha-1}}{(1+u)^{\alpha+\beta}}du=\int_{0}^{\infty}\frac{(e^t-1)^{\alpha-1}}{(e^t)^{\alpha+\beta}}e^tdt=\int_{0}^{\infty}((1-e^{-t})e^t)^{\alpha-1}(e^{-t})^{\alpha+\beta}\cdot e^tdt$$
    $$=\int_{0}^{\infty}(1-e^{-t})^{\alpha-1}\cdot (e^{t})^{\alpha-1}\cdot (e^{t})^{-\alpha}\cdot (e^t)^{-\beta}\cdot e^tdt = \int_{0}^{\infty}(1-e^{-t})^{\alpha-1}(e^{-t})^\beta dt$$
    Then, for the above expression, if we do the second substitution $r=e^{-t}$, $dr=-e^{-t}dt$, $dt=-e^tdt = -r^{-1}dr$. Which $t=0\implies r=e^0,\ r=1$, and $\lim_{t\rightarrow\infty}e^{-t}=\lim_{t\rightarrow\infty}r=0$. So, the integral can be rewrite as:
    $$\int_{0}^{\infty}(1-e^{-t})^{\alpha-1}(e^{-t})^\beta dt=\int_{1}^{0}(1-r)^{\alpha-1}r^{\beta}\cdot (-r^{-1})dr = \int_{0}^{1}(1-r)^{\alpha-1}r^{\beta-1}dr = B(\alpha,\beta)$$
    Hence, we can conclude the following:
    $$\int_{0}^{\infty}\frac{u^{\alpha-1}}{(1+u)^{\alpha+\beta}}du=\int_{0}^{\infty}(1-e^{-t})^{\alpha-1}(e^{-t})^\beta dt=B(\alpha,\beta)$$
\end{itemize}

\hfil

\hfil

\section*{2}
\begin{myBox}[]{}
    \begin{question}
        The hypergeometric series $F(\alpha,\beta,\gamma; z)$ was defined as 
        $$F(\alpha,\beta,\gamma;z)=1+\sum_{n=1}^{\infty}\frac{\alpha(\alpha+1)...(\alpha+n-1)\cdot \beta(\beta+1)...(\beta+n-1)}{n!\cdot \gamma(\gamma+1)...(\gamma+n-1)}z^n$$
        Here $\alpha>0,\beta>0,\gamma>\beta$, and $|z|<1$. Show that
        $$F(\alpha,\beta,\gamma;z)=\frac{\Gamma(\gamma)}{\Gamma(\beta)\Gamma(\gamma-\beta)}\int_{0}^{1}t^{\beta-1}(1-t)^{\gamma-\beta-1}(1-zt)^{-\alpha}dt$$
        Show as a result that the hypergeometric function, initially defined by a power series convergent in the unit disc, can be continued analytically to the complex plane slit along the half-line $[1,\infty)$.
    \end{question}
\end{myBox}

\textbf{Pf:}

\textbf{Properties of Gamma function:}

First, we can use induction to verity that given $z\in \mathbb{C}\setminus\{0,-1,-2,...\}$, all $n\in\mathbb{N}$ satisfies $\Gamma(z+n)=(z+n-1)...(z+1)z\Gamma(z)$. 

For base case $n=1$, by the identity of gamma function, $\Gamma(z+1)=z\Gamma(z)$, so the formula is true.

Then, suppose for given $n\in\mathbb{N}$, we have $\Gamma(z+n)=(z+n-1)...(z+1)z\Gamma(z)$, which for $(z+n+1)$, it satisfies:
$$\Gamma(z+n+1)=(z+n)\Gamma(z+n)=(z+n)(z+n-1)...(z+1)z\Gamma(z)$$
Hence, this completes the induction.

So, for all $n\in\mathbb{N}$, we also have the following identity:
$$(z+n-1)...(z+1)z = \frac{\Gamma(z+n)}{\Gamma(z)}$$

By this property, we can rewrite the hypergeometric series as follow:
$$F(\alpha,\beta,\gamma;z)=1+\sum_{n=1}^{\infty}\frac{\alpha(\alpha+1)...(\alpha+n-1)\cdot \beta(\beta+1)...(\beta+n-1)}{n!\cdot \gamma(\gamma+1)...(\gamma+n-1)}z^n $$
$$= 1+\sum_{n=1}^{\infty}\frac{(\Gamma(\alpha+n)/\Gamma(\alpha))(\Gamma(\beta+n)/\Gamma(\beta))}{n!\cdot (\Gamma(\gamma+n)/\Gamma(\gamma))}z^n = 1+\frac{\Gamma(\gamma)}{\Gamma(\alpha)\Gamma(\beta)}\sum_{n=1}^{\infty}\frac{\Gamma(\alpha+n)\Gamma(\beta+n)}{n!\cdot \Gamma(\gamma+n)}z^n$$

\hfil

\textbf{Power series of $(1-\zeta)^{-\alpha}$:}

Given the above function, it is analytic within the disk $|\zeta|<1$. Then, consider its derivatives at $\zeta=0$, we get:
$$\frac{d}{d\zeta}(1-\zeta)^{-\alpha}=\alpha(1-\zeta)^{-\alpha-1}$$
$$\forall n\in\mathbb{N},\ \frac{d^n}{d\zeta^n}(1-\zeta)^{-\alpha}=(\alpha+n-1)...(\alpha+1)\alpha(1-\zeta)^{-\alpha-n}=\frac{\Gamma(\alpha+n)}{\Gamma(\alpha)}(1-\zeta)^{-\alpha-n}$$
So, let $f(\zeta)=(1-\zeta)^{-\alpha}$, $f^{(n)}(0)=\frac{\Gamma(\alpha+n)}{\Gamma(\alpha)}$. Which, the power series about $\zeta=0$ is:
$$f(\zeta)=\sum_{n=0}^{\infty}\frac{f^{(n)}(0)}{n!}\zeta^n = \sum_{n=0}^{\infty}\frac{\Gamma(\alpha+n)}{n!\cdot\Gamma(\alpha)}\zeta^n$$

\hfil

\textbf{The Integral:}

Then, since the power series converges uniformly for any compact region within the unit disk $|\zeta|<1$, while the integral of the function with $(1-zt)^{-\alpha}$ being defined with $|z|<1$, $t\in (0,1)$, hence the function is integrated over a region that can be covered by a compact set (choose closed disk with radius $|\zeta|\leq R<1$, where $|z|<R$).
As the power series converges uniformly on this region, integral and summation in fact commutes, hence the integral mentioned in the question can be represented as:
$$\frac{\Gamma(\gamma)}{\Gamma(\beta)\Gamma(\gamma-\beta)}\int_{0}^{1}t^{\beta-1}(1-t)^{\gamma-\beta-1}(1-zt)^{-\alpha}dt = \frac{\Gamma(\gamma)}{\Gamma(\beta)\Gamma(\gamma-\beta)}\int_{0}^{1}t^{\gamma-\beta-1}(1-t)^{\beta-1}\left(\sum_{n=0}^{\infty}\frac{\Gamma(\alpha+n)}{n!\cdot\Gamma(\alpha)}(zt)^n\right)dt$$
$$=\frac{\Gamma(\gamma)}{\Gamma(\beta)\Gamma(\gamma-\beta)\Gamma(\alpha)}\sum_{n=0}^{\infty}\frac{\Gamma(\alpha+n)}{n!}z^n\int_{0}^{1}t^{\beta+n-1}(1-t)^{\gamma-\beta-1}dt$$
With the identity verified in \textbf{Question 1}, we know the following:
$$\forall \alpha,\beta\in\mathbb{C},\ Re(\alpha),Re(\beta)>0,\quad B(\alpha,\beta)=\int_{0}^{1}(1-t)^{\alpha-1}t^{\beta-1}dt=\frac{\Gamma(\alpha)\Gamma(\beta)}{\Gamma(\alpha+\beta)}$$
Hence, the above form of integral becomes:
$$\frac{\Gamma(\gamma)}{\Gamma(\beta)\Gamma(\gamma-\beta)\Gamma(\alpha)}\sum_{n=0}^{\infty}\frac{\Gamma(\alpha+n)}{n!}z^n\int_{0}^{1}t^{\beta+n-1}(1-t)^{\gamma-\beta-1}dt$$
$$=\frac{\Gamma(\gamma)}{\Gamma(\beta)\Gamma(\gamma-\beta)\Gamma(\alpha)}\sum_{n=0}^{\infty}\frac{\Gamma(\alpha+n)}{n!}z^n\cdot B(\gamma-\beta, \beta+n)$$
$$=\frac{\Gamma(\gamma)}{\Gamma(\beta)\Gamma(\gamma-\beta)\Gamma(\alpha)}\sum_{n=0}^{\infty}\frac{\Gamma(\alpha+n)}{n!}z^n\cdot\frac{\Gamma(\gamma-\beta)\Gamma(\beta+n)}{\Gamma(\gamma+n)}$$
$$=\frac{\Gamma(\gamma)}{\Gamma(\beta)\Gamma(\alpha)}\sum_{n=0}^{\infty}\frac{\Gamma(\alpha+n)\Gamma(\beta+n)}{n!\Gamma(\gamma+n)}z^n = \frac{\Gamma(\gamma)}{\Gamma(\beta)\Gamma(\alpha)}\cdot\frac{\Gamma(\alpha+0)\Gamma(\beta+0)}{0!\Gamma(\gamma+0)}+\frac{\Gamma(\gamma)}{\Gamma(\beta)\Gamma(\alpha)}\sum_{n=1}^{\infty}\frac{\Gamma(\alpha+n)\Gamma(\beta+n)}{n!\Gamma(\gamma+n)}z^n$$
$$=1+\frac{\Gamma(\gamma)}{\Gamma(\beta)\Gamma(\alpha)}\sum_{n=1}^{\infty}\frac{\Gamma(\alpha+n)\Gamma(\beta+n)}{n!\Gamma(\gamma+n)}z^n$$
Hence, we can conclude the following:
$$\frac{\Gamma(\gamma)}{\Gamma(\beta)\Gamma(\gamma-\beta)}\int_{0}^{1}t^{\beta-1}(1-t)^{\gamma-\beta-1}(1-zt)^{-\alpha}dt$$
$$=\frac{\Gamma(\gamma)}{\Gamma(\beta)\Gamma(\gamma-\beta)\Gamma(\alpha)}\sum_{n=0}^{\infty}\frac{\Gamma(\alpha+n)}{n!}z^n\int_{0}^{1}t^{\beta+n-1}(1-t)^{\gamma-\beta-1}dt$$
$$=1+\frac{\Gamma(\gamma)}{\Gamma(\beta)\Gamma(\alpha)}\sum_{n=1}^{\infty}\frac{\Gamma(\alpha+n)\Gamma(\beta+n)}{n!\Gamma(\gamma+n)}z^n = F(\alpha,\beta,\gamma;z)$$
The identity proposed in the question is showned above.

\hfil

\textbf{Analytic Continuation:}

For all $z\in\mathbb{C}\setminus[1,\infty)$ and all $t\in (0,1)$, then since $z\notin [1,\infty)$, then $tz\notin [1,\infty)$ (since if $tz\in [1,\infty)$, $z\in [1/t,\infty)\subseteq [1,\infty)$, which is a contradiction), hence $(1-tz)\notin (-\infty,0]$.
So, if define a $\log(z)$ to have a branch cut on $(-\infty,0]$, then $\log(1-tz)$ is analytic.

Which, on this new domain, the following function is defined, and analytic:
$$\bar{F}(\alpha,\beta,\gamma;z)=\int_{0}^{1}t^{\beta-1}(1-t)^{\gamma-\beta-1}e^{-\alpha\log(1-zt)}dt$$
Which, on the unit disk $|z|<1$, the above function agrees with the hypergeometric functions. Hence, it is an analytic continuation of the hypergeometric function on the domain $\mathbb{C}\setminus[1,\infty)$.

\break

\section*{3}
\begin{myBox}[]{}
    \begin{question}
        Prove that
        $$\frac{d^2}{ds^2}(\log(\Gamma(s)))=\sum_{n=0}^{\infty}\frac{1}{(s+n)^2}$$
        whenever $s$ is a positive number. Show that if the left-hand side is interpreted
        as $(\Gamma'/\Gamma)'$, then the above formula also holds for all complex numbers $s$ with
        $s\neq 0,-1,-2,...$
    \end{question}
\end{myBox}

\textbf{Pf:}

We'll directly prove the case for viewing it as $\Gamma'/\Gamma$ (which applies to the case for positive real inputs).
First, recall the following characterization of $\Gamma$:
$$\forall z\in\mathbb{C},\quad \frac{1}{\Gamma(z)}=G(z)=ze^{\gamma z}H(z),\quad H(z)=\prod_{n=1}^{\infty}\left(1+\frac{z}{n}\right)e^{-z/n}$$
(Note: $\gamma$ is the Euler-Mascheroni Constant).

Which, the derivative of $1/\Gamma(z)$ given as $-\frac{\Gamma'(z)}{(\Gamma(z))^2}$, while the derivative of $G(z)$ is given as follow:
$$G'(z)=\left(e^{\gamma z}+\gamma ze^{\gamma z}\right)H(z)+ze^{\gamma z}H'(z) = \frac{1}{z}\cdot ze^{\gamma z}H(z)+\gamma\cdot ze^{\gamma z}H(z)+ze^{\gamma z}H'(z)$$
$$ = \left(\frac{1}{z}+\gamma\right)G(z)+ze^{\gamma z}H'(z)$$
Since the derivatives match up, the only thing left is finding a precise formula for $H'(z)$.

\hfil

\textbf{Expression of $H'(z)$:}

For all $z\in\mathbb{C}$, choose $N\in\mathbb{N}$, such that for all $n\in\mathbb{N}$, $n\geq N$ implies $\left|\frac{z}{n}\right|\leq \frac{1}{2}$ (in other words, we're working in the disk $|z|\leq \frac{N}{2}$, which is compact). 
Then, we can define a single-valued branch for $\log(1+\zeta)$ for $|\zeta|<1$.
Then, by grouping the components of the product in $H(z)$, we get the following:
$$H(z)=\prod_{n=1}^{\infty}\left(1+\frac{z}{n}\right)e^{-z/n}=\left(\prod_{n=1}^{N}\left(1+\frac{z}{n}\right)e^{-z/n}\right)\cdot\left(\prod_{n=N+1}^{\infty}\left(1+\frac{z}{n}\right)e^{-z/n}\right)$$
$$=\left(\prod_{n=1}^{N}\left(1+\frac{z}{n}\right)\right)\cdot \exp\left(\sum_{n=1}^{N}-\frac{z}{n}\right)\cdot\left(\prod_{n=N+1}^{\infty}\exp\left(\log\left(1+\frac{z}{n}\right)-\frac{z}{n}\right)\right)$$
$$=\left(\prod_{n=1}^{N}\left(1+\frac{z}{n}\right)\right)\cdot \exp\left(\sum_{n=1}^{N}-\frac{z}{n}\right)\cdot\exp\left(\sum_{n=N+1}^{\infty}\left(\log\left(1+\frac{z}{n}\right)-\frac{z}{n}\right)\right)$$
$$=\left(\prod_{n=1}^{N}\left(1+\frac{z}{n}\right)\right)\cdot\exp\left(-\sum_{n=1}^{N}\frac{z}{n}+\sum_{n=N+1}^{\infty}\left(\log\left(1+\frac{z}{n}\right)-\frac{z}{n}\right)\right)$$
Before continuing, we need to argue why the infinite series of function in the above exponent converges normally in the disk:
Since $\left|\frac{z}{n}\right|\leq \frac{1}{2}$ for all $n\geq N$, then the power series of $\log(1+\frac{z}{n})$ is $-\sum_{k=1}^{\infty}\frac{(-1)^k}{k}\left(\frac{z}{n}\right)^k$. Then, each index $n\geq N$ satisfies the following:
$$\left|\log\left(1+\frac{z}{n}\right)-\frac{z}{n}\right|=\left|-\sum_{k=1}^{\infty}\frac{(-1)^k}{k}\left(\frac{z}{n}\right)^k-\frac{z}{n}\right| = \left|\sum_{k=2}^{\infty}\frac{(-1)^k}{k}\left(\frac{z}{n}\right)^k\right| \leq \sum_{k=2}^{\infty}\frac{1}{k}\left|\frac{z}{n}\right|^k$$
$$\leq \sum_{k=2}^{\infty}\left|\frac{z}{n}\right|^k\leq \left|\frac{z}{n}\right|^2\sum_{k=2}^{\infty}\left|\frac{z}{n}\right|^{k-2} = \left|\frac{z}{n}\right|^2\sum_{k=0}^{\infty}\left|\frac{z}{n}\right|^k\leq \left|\frac{z}{n}\right|^2\sum_{k=0}^{\infty}\left(\frac{1}{2}\right)^k = 2\left|\frac{z}{n}\right|^2$$
With the assumption that we're working over the disk $|z|\leq \frac{N}{2}$, the above bound can be simplified as:
$$\left|\log\left(1+\frac{z}{n}\right)-\frac{z}{n}\right|\leq 2\left|\frac{z}{n}\right|^2 \leq 2\left(\frac{N}{2}\right)^2\cdot\frac{1}{n^2} = \frac{N^2}{2}\cdot\frac{1}{n^2}$$
Hence, the series of function converges normally in the disk because of the following inequality:
$$\sum_{n=N+1}^{\infty}\left|\log\left(1+\frac{z}{n}\right)-\frac{z}{n}\right|\leq \sum_{n=N+1}^{\infty}\frac{N^2}{2}\cdot\frac{1}{n^2}<\infty$$
So, it's valid to talk about the way we organize the infinite product in $H(z)$ (and more conveniently, the above infinite series can be differentiated term by term).

\hfil

Now, define the two functions $A(z),B(z)$ on the disk $|z|\leq \frac{N}{2}$ as follow:
$$A(z)=\prod_{n=1}^{N}\left(1+\frac{z}{n}\right),\quad B(z)=\exp\left(-\sum_{n=1}^{N}\frac{z}{n}+\sum_{n=N+1}^{\infty}\left(\log\left(1+\frac{z}{n}\right)-\frac{z}{n}\right)\right)$$
Then, the function $H=AB$, hence the derivative is given by $H'=A'B+B'A$.

For $A'(z)$, it is expressed as follow:
$$A'(z)=\sum_{n=1}^{N}\left(\frac{d}{dz}\left(1+\frac{z}{n}\right)\right)\cdot \left(\prod_{k=1,\ k\neq n}^{N}\left(1+\frac{z}{k}\right)\right)=\sum_{n=1}^{N}\frac{1}{n}\cdot\left(\prod_{k=1,\ k\neq n}^{N}\left(1+\frac{z}{k}\right)\right)$$
$$=\sum_{n=1}^{N}\frac{1}{n}\cdot\frac{1}{1+\frac{z}{n}}\cdot\left(\prod_{k=1}^{N}\left(1+\frac{z}{k}\right)\right) = \sum_{n=1}^{N}\frac{1}{z+n}\cdot A(z)$$
For $B'(z)$, it is expressed as follow:
$$B'(z)=\frac{d}{dz}\exp\left(-\sum_{n=1}^{N}\frac{z}{n}+\sum_{n=N+1}^{\infty}\left(\log\left(1+\frac{z}{n}\right)-\frac{z}{n}\right)\right)$$
$$=\exp\left(-\sum_{n=1}^{N}\frac{z}{n}+\sum_{n=N+1}^{\infty}\left(\log\left(1+\frac{z}{n}\right)-\frac{z}{n}\right)\right)\cdot \frac{d}{dz}\left(-\sum_{n=1}^{N}\frac{z}{n}+\sum_{n=N+1}^{\infty}\left(\log\left(1+\frac{z}{n}\right)-\frac{z}{n}\right)\right)$$
$$= B(z)\left(-\sum_{n=1}^{N}\frac{1}{n}+\sum_{n=N+1}^{\infty}\left(\frac{1/n}{1+z/n}-\frac{1}{n}\right)\right) =B(z)\left(-\sum_{n=1}^{N}\frac{1}{n}+\sum_{n=N+1}^{\infty}\left(\frac{1}{z+n}-\frac{1}{n}\right)\right)$$
Then, $H'(z)$ is then given by:
$$H'(z)=A'B+B'A = \left(\sum_{n=1}^{N}\frac{1}{z+n}\right)\cdot A(z)\cdot B(z)+B(z)\left(-\sum_{n=1}^{N}\frac{1}{n}+\sum_{n=N+1}^{\infty}\left(\frac{1}{z+n}-\frac{1}{n}\right)\right)\cdot A(z)$$
$$=A(z)B(z)\cdot\left(\sum_{n=1}^{N}\left(\frac{1}{z+n}-\frac{1}{n}\right)+\sum_{n=N+1}^{\infty}\left(\frac{1}{z+n}-\frac{1}{n}\right)\right)= H(z)\left(\sum_{n=1}^{\infty}\left(\frac{1}{z+n}-\frac{1}{n}\right)\right)$$

\hfil

\textbf{Expression of $\Gamma'/\Gamma$:}

Now, plug back into the original expression, we get the following:
$$\frac{-\Gamma'(z)}{(\Gamma(z))^2}=\left(\frac{1}{\Gamma(z)}\right)'=G'(z)=\left(\frac{1}{z}+\gamma\right)G(z)+ze^{\gamma z}H'(z)$$

\break

\section*{4}
\begin{myBox}[]{}
    \begin{question}
        
    \end{question}
\end{myBox}

\textbf{Pf:}

\break

\section*{5}
\begin{myBox}[]{}
    \begin{question}
        
    \end{question}
\end{myBox}

\textbf{Pf:}

\end{document}