\documentclass{article}
\usepackage{graphicx} % Required for inserting images
\usepackage[margin = 2.54cm]{geometry}
\usepackage[most]{tcolorbox}

\newtcolorbox{myBox}[3]{
arc=5mm,
lower separated=false,
fonttitle=\bfseries,
%colbacktitle=green!10,
%coltitle=green!50!black,
enhanced,
attach boxed title to top left={xshift=0.5cm,
        yshift=-2mm},
colframe=blue!50!black,
colback=blue!10
}

\usepackage{amsmath}
\usepackage{amssymb}
\usepackage{verbatim}
\usepackage[utf8]{inputenc}
\linespread{1.2}

\newtheorem{definition}{Definition}
\newtheorem{proposition}{Proposition}
\newtheorem{theorem}{Theorem}
\newtheorem{question}{Question}

\title{Math CS 122b HW8 Part 1}
\author{Zih-Yu Hsieh}

\begin{document}
\maketitle

\section*{1 (not done)}
\begin{myBox}[]{}
    \begin{question}
        Stein and Shakarchi Pg. 201-202 Exercise 8:

        The function $\zeta$ has infinitely many zeros in the critical strip. This can be seen as follows.
        \begin{itemize}
            \item[(a)] Let $F(s)=\xi(1/2+s)$, where $\xi(s)=\pi^{-s/2}\Gamma(s/2)\zeta(s)$. 
            Show that $F(s)$ is an even function of $s$, and as a result, there exists $G$ so that $G(s^2)=F(s)$.
            \item[(b)] Show that the function $(s-1)\zeta(s)$ is an entire function of growth order $1$, that is 
            $$|(s-1)\zeta(s)|\leq A_\epsilon e^{a_\epsilon |s|^{1+\epsilon}}$$
            As a consequence $G(s)$ is of growth order $1/2$.
            \item[(c)] Deduce from the above that $\zeta$ has infinitely many zeros in the critical strip.
        \end{itemize}
        [Hint: To prove (a) and (b) use the functional equation for $\zeta(s)$. For (c), use a result of Hadamard, which states that an entire function with fractional order has infinitely many zeros (Exercise $14$ in Chapter $5$)].
    \end{question}
\end{myBox}

\textbf{Pf:}
\begin{itemize}
    \item[(a)] Recall that in \textbf{HW 7 Question 1} (\textbf{Freitag Chap. VII.5 Problem 5}), to deduce the functional equation of $\zeta$, we've proven the functional equation $\xi(s')=\xi(1-s')$. As a result, for any $s\in\mathbb{C}$, if treating $F$ as a meromorphic function, we get:
    $$F(s)=\xi\left(\frac{1}{2}+s\right) = \xi\left(1-\left(\frac{1}{2}+s\right)\right) = \xi\left(\frac{1}{2}-s\right) = F(-s)$$
    Hence, this proves that $F(s)$ is an even function.

    \hfil

    \item[(b)] Recall that $\zeta(s)$ is analytic on $\mathbb{C}\setminus\{1\}$, with a simple pole at $s=1$ with residue $1$, then $(s-1)\zeta(s)$ is in fact having a removable singularity at $s=1$, hence can be extended to an entire function.
    
    \hfil

    \textbf{1. $(s-1)\zeta(s)$ Has growth order $1$ for $\textmd{Re}(s)\geq \frac{1}{2}$:}

    In \textbf{Freitag Lemma VII.5.2}, the following functions are well defined:
    $$\forall t\in\mathbb{R},\quad \beta(t)=t-[t]-\frac{1}{2},\quad [t]:=\max{n\in\mathbb{Z},\ n\leq t}$$
    $$\forall s\in\mathbb{C},\ \textmd{Re}(s)>0,\quad F(s):=\int_{1}^{\infty}t^{-s-1}\beta(t)dt$$
    Then as a result, the following equation is true for $\textmd{Re}(s)>1$, hence defines an analytic continuation for $\zeta(s)$ on $\textmd{Re}(s)>0$:
    $$\forall s\in\mathbb{C},\ \textmd{Re}(s)>1,\quad \zeta(s)=\frac{1}{2}+\frac{1}{s-1}-sF(s)$$
    So, if multiply with $(s-1)$, for $\textmd{Re}(s)\geq \frac{1}{2}$, $(s-1)\zeta(s)$ is well-defined, and can be given as the following formula:
    $$(s-1)\zeta(s)=\frac{(s-1)}{2}+1-(s-1)sF(s)$$

    Which, let $s = x+iy$ for $x,y\in\mathbb{R}$, on $\textmd{Re}(s) = x \geq \frac{1}{2}$ (which $\frac{1}{x}\leq 2$), $F(s)$ can be bounded as follow:
    $$|F(s)| = \left|\int_{1}^{\infty}t^{-s-1}\beta(t)dt\right| \leq \int_{1}^{\infty}|t^{-(x+iy)-1}\beta(t)|dt \leq \int_{1}^{\infty}|t^{-x-1}\cdot t^{iy}|dt = \int_{1}^{\infty}t^{-x-1}dt$$
    $$ = \frac{-1}{x}t^{-x}\bigg|_{1}^{\infty} = \frac{1}{x}\leq 2$$
    (Note: for any $t\in\mathbb{R}$, $|\beta(t)|\leq \frac{1}{2}<1$, and since $x\geq \frac{1}{2}$, then the integral of $t^{-x-1}$ has power $<-1$, which is absolutely convergent).

    So, if considering the modulus of $(s-1)\zeta(s)$ on $\textmd{Re}(s)\geq \frac{1}{2}$, we get the following:
    $$|(s-1)\zeta(s)| = \left|\frac{(s-1)}{2}+1-(s-1)sF(s)\right|\leq \frac{|s-1|}{2} + 1 + |(s-1)s|\cdot |F(s)| \leq \frac{|s|+1}{2} + 1 + 2(|s|^2+|s|)$$
    $$\leq 2|s|^2 + \frac{3}{2}|s|+\frac{3}{2}$$
    Which, take $4e^{|s|} = 4 + 4|s| + 2|s|^2 + \sum_{n=3}^{\infty}\frac{4}{n!}|s|^n$, since for any $s\in\mathbb{C}$ each term is nonnegative, then we can deduce:
    $$|(s-1)\zeta(s)| \leq 2|s|^2 + \frac{3}{2}|s|+\frac{3}{2} \leq 4+4|s| + 2|s|^2 \leq 4 + 4|s| + 2|s|^2 + \sum_{n=3}^{\infty}\frac{4}{n!}|s|^n = 4e^{|s|}$$
    This shows that $(s-1)\zeta(s)$ has growth order $1$ on the half plane $\textmd{Re}(s)\geq \frac{1}{2}$.

    \hfil

    \textbf{2. $(s-1)\zeta(s)$ Has growth order $1$ for the whole plane:}

    In the previous part the growth order is verified for $\textmd{Re}(s)\geq \frac{1}{2}$. so the rest suffices to show it for the half plane $\textmd{Re}(s')<\frac{1}{2}$.
    
    Recall that in \textbf{HW 7}, we've proven the following functional equation of $\zeta$:
    $$\zeta(1-s) = 2(2\pi)^{-s}\Gamma(s)\cos\left(\frac{\pi s}{2}\right)\zeta(s)$$
    Hence, for any $s'$ with $\textmd{Re}(s')<\frac{1}{2}$, let $s' = 1-s$ for some $s\in\mathbb{C}$, then $s = 1-s'$, so $\textmd{Re}(s) = \textmd{Re}(1-s') > \frac{1}{2}$. Then, the equation $(s'-1)\zeta(s')$ becomes:
    $$(s'-1)\zeta(s') = ((1-s)-1)\zeta(1-s) = -s\cdot 2(2\pi)^{-s}\Gamma(s)\cos\left(\frac{\pi s}{2}\right)\zeta(s)$$
    Which, $|s| = |1-s'| \leq |s'|+1$, so the growth order in terms of $|s|$ can be replaced using $|s'|$ instead. From the above equality, we do need to talk about the growth order or different components of the functions:
    \begin{itemize}
        \item For $(2\pi)^{-s} = e^{-\log(2\pi)s} = e^{-\log(2\pi)(x+iy)} = e^{-\log(2\pi)x}\cdot e^{-\log(2\pi)iy}$, it satisfies $|(2\pi)^{-s}| =  e^{-\log(2\pi)x}$
    \end{itemize}
    
    \hfil

    \item[(c)] In \textbf{Part (b)}, it was proven that $F$ has growth order $1$, while $G$ has growth order $1/2$. So based on Hadamard's result, 
\end{itemize}

\break

\section*{2}
\begin{myBox}[]{}
    \begin{question}
        Stein and Shakarchi Pg. 202-203 Exercise 10:

        In the theory of primes, a better approximation fo $\pi(x)$ (instead of $x/\log(x)$) turns out to be $\textmd{Li}(x)$ defined by 
        $$\textmd{Li}(x)=\int_{2}^{x}\frac{dt}{\log(t)}$$
        \begin{itemize}
            \item[(a)] Prove that 
            $$\textmd{Li}(x)=\frac{x}{\log(x)}+O\left(\frac{x}{(\log(x))^2}\right)\quad \textmd{as }x\rightarrow\infty$$
            and that as a consequence 
            $$\pi(x)\sim \textmd{Li}(x)\quad \textmd{as }x\rightarrow\infty$$
            \item[(b)] Refine the previous analysis by showing that for every integer $N>0$ one has the following asymptotic expansion 
            $$\textmd{Li}(x)=\frac{x}{\log(x)}+\frac{x}{(\log(x))^2}+2\frac{x}{(\log(x))^3}+\cdots (N-1)!\frac{x}{(\log(x))^N}+O\left(\frac{x}{(\log(x))^{N+1}}\right)$$
            as $x\rightarrow\infty$.
        \end{itemize}
    \end{question}
\end{myBox}

\textbf{Pf:}
\begin{itemize}
    \item[(a)] First, using integration by parts, for all $x\geq 4$ (where $x\geq \sqrt{x}\geq 2$), $\textmd{Li}(x)$ can be expressed as follow:
    $$\textmd{Li}(x)=\int_{2}^{x}\frac{dt}{\log(t)} = \frac{t}{\log(t)}\bigg|_{2}^{x}-\int_{2}^{x}t\cdot \frac{d}{dt}\left(\frac{1}{\log(t)}\right)dt$$
    $$ = \frac{x}{\log(x)}-\frac{2}{\log(2)}-\int_{2}^{x}t\cdot\frac{-1}{(\log(t))^2}\cdot\frac{1}{t}dt$$
    $$ = \frac{x}{\log(x)}-\frac{2}{\log(2)}+\int_{2}^{x}\frac{1}{(\log(t))^2}dt$$
    Which, for the last integral expression, it can be reformulate as follow:
    $$\int_{2}^{x}\frac{dt}{(\log(t))^2} = \int_{2}^{\sqrt{x}}\frac{dt}{(\log(t))^2}+\int_{\sqrt{x}}^{x}\frac{dt}{(\log(t))^2}$$
    Since $\log(t)$ is a strictly increasing function on $(1,\infty)$ and is strictly positive, then $\frac{1}{(\log(t))^2}$ is a strictly decreasing function on this interval instead. Hence, for all $t\in [2,\sqrt{x}]$, $\frac{1}{(\log(t))^2}\leq \frac{1}{(\log(2))^2}$, while any $t\in [\sqrt{x},x]$ satisfies $\frac{1}{(\log(t))^2}\leq \frac{1}{(\log(\sqrt{x}))^2} = \frac{4}{(\log(x))^2}$. Hence, the above expression satisfies:
    $$\int_{2}^{x}\frac{dt}{(\log(t))^2} = \int_{2}^{\sqrt{x}}\frac{dt}{(\log(t))^2}+\int_{\sqrt{x}}^{x}\frac{dt}{(\log(t))^2}\leq \int_{2}^{\sqrt{x}}\frac{dt}{(\log(2))^2}+\int_{\sqrt{x}}^{x}\frac{4dt}{(\log(x))^2}$$
    $$ = \frac{\sqrt{x}-2}{(\log(2))^2} + \frac{4(x-\sqrt{x})}{(\log(x))^2}\leq \frac{4x}{(\log(x))^2}+\frac{\sqrt{x}}{(\log(2))^2}$$
    Which, if evaluate the following limit, we get:
    $$\lim_{x\rightarrow\infty}\frac{\sqrt{x}}{x/(\log(x))^2} = \lim_{x\rightarrow\infty}\frac{(\log(x))^2}{\sqrt{x}} = \lim_{x\rightarrow\infty}\frac{2\log(x)/x}{1/(2\sqrt{x})} = \lim_{x\rightarrow\infty}\frac{4\log(x)}{\sqrt{x}}$$
    $$= \lim_{x\rightarrow\infty}\frac{4/x}{1/(2\sqrt{x})} = \lim_{x\rightarrow\infty}\frac{8}{\sqrt{x}}=0$$
    Hence, for some $x_1>4$ and $A_1>0$, we have $x>x_1$ implies $\sqrt{x}\leq A_1\frac{x}{(\log(x))^2}$. So, the integral follows the inequality below for $x>x_0$:
    $$\int_{2}^{x}\frac{dt}{(\log(t))^2}\leq \frac{\sqrt{x}}{(\log(2))^2} + \frac{4x}{(\log(x))^2} \leq \frac{1}{(\log(2))^2}\cdot\frac{A_1x}{(\log(x))^2}+\frac{4x}{(\log(x))^2}$$
    $$\leq \left(\frac{A_1}{(\log(2))^2}+4\right)\frac{x}{(\log(x))^2}$$
    So, this shows that $\int_{2}^{x}\frac{dt}{(\log(t))^2}=O\left(\frac{x}{(\log(x))^2}\right)$. Hence:
    $$\textmd{Li}(x)=\frac{x}{\log(x)}-\frac{2}{\log(2)}+\int_{2}^{x}\frac{dt}{(\log(t))^2}\leq \frac{x}{\log(x)}+\int_{2}^{x}\frac{dt}{(\log(t))^2} = \frac{x}{\log(x)}+O\left(\frac{x}{(\log(x))^2}\right)$$
    This shows that $\textmd{Li}(x)=\frac{x}{\log(x)}+O\left(\frac{x}{(\log(x))^2}\right)$.

    \hfil

    \item[(b)] First, we'll consider the following formula about the integral of $\frac{1}{(\log(t))^n}$ using integration by parts:
    $$\forall n\in\mathbb{N},\quad \int_{2}^{x}\frac{dt}{(\log(t))^n} = \frac{t}{(\log(t))^n}\bigg|_{2}^{x}-\int_{2}^{x}t\cdot\frac{d}{dt}\left(\frac{1}{(\log(t))^n}\right)dt$$
    $$ = \frac{x}{(\log(x))^n}-\frac{2}{(\log(2))^n}-\int_{2}^{x}t\cdot\frac{-n}{(\log(t))^{n+1}}\cdot\frac{1}{t}dt = \frac{x}{(\log(x))^n}-\frac{2}{(\log(2))^n}+n\int_{2}^{x}\frac{dt}{(\log(t))^{n+1}}$$
    Which, using the same argument used in \textbf{part (a)} about $\frac{1}{(\log(t))^n}$ is a decreasing function for all $n\in\mathbb{N}$, for all $x\geq 4$ (where $x>\sqrt{x}\geq 2$), we get:
    $$\int_{2}^{x}\frac{dt}{(\log(t))^{n+1}} = \int_{2}^{\sqrt{x}}\frac{dt}{(\log(t))^{n+1}}+\int_{\sqrt{x}}^{x}\frac{dt}{(\log(t))^{n+1}} \leq \int_{2}^{\sqrt{x}}\frac{dt}{(\log(2))^{n+1}}+\int_{\sqrt{x}}^{x}\frac{dt}{(\log(\sqrt{x}))^{n+1}}$$
    $$=\frac{(\sqrt{x}-2)}{(\log(2))^{n+1}} + \frac{2^{n+1}(x-\sqrt{x})}{(\log(x))^{n+1}} \leq \frac{2^{n+1}x}{(\log(x))^{n+1}}+\frac{\sqrt{x}}{(\log(2))^{n+1}}$$
    Now, since the base case $\lim_{x\rightarrow\infty}\frac{\sqrt{x}}{x/(\log(x))^2}=0$ is proven in \textbf{part (a)}, using induction, we can get the following relationship:
    $$\forall n\in\mathbb{N},\quad \lim_{x\rightarrow\infty}\frac{\sqrt{x}}{x/(\log(x))^{n+1}} = \lim_{x\rightarrow\infty}\frac{(\log(x))^{n+1}}{\sqrt{x}} = \lim_{x\rightarrow\infty}\frac{(n+1)(\log(x))^n/x}{1/(2\sqrt{x})}$$
    $$ = \lim_{x\rightarrow\infty}2(n+1)\frac{\sqrt{x}}{x/(\log(x))^n} = 0$$
    Hence, there exists $x_n>4$ and $A_n>0$, such that $x>x_n$ implies $\sqrt{x}\leq A_n\frac{x}{(\log(x))^{n+1}}$. Hence, we get:
    $$\int_{2}^{x}\frac{dt}{(\log(t))^{n+1}} \leq \frac{2^{n+1}x}{(\log(x))^{n+1}}+\frac{\sqrt{x}}{(\log(2))^{n+1}}$$
    $$ \leq \frac{2^{n+1}x}{(\log(x))^{n+1}}+\frac{A_n}{(\log(2))^{n+1}}\frac{x}{(\log(x))^{n+1}} = \left(2^{n+1}+\frac{A_n}{(\log(2))^{n+1}}\right)\frac{x}{(\log(x))^{n+1}}$$
    This shows that $\int_{2}^{x}\frac{dt}{(\log(t))^{n+1}}=O\left(\frac{x}{(\log(x))^{n+1}}\right)$.

    \hfil

    Finally, using the case proven in \textbf{part (a)}, we know $\textmd{Li}(x)=\frac{x}{\log(x)}-\frac{2}{\log(2)}+\int_{2}^{x}\frac{dt}{(\log(t))^2}$. Which utilizing the above equation, by induction, one can show that for any integer $n\geq 2$, the following formula holds:
    $$\textmd{Li}(x)=\sum_{k=1}^{n}(k-1)!\frac{x}{(\log(x))^k} - \sum_{k=1}^{n}(k-1)!\frac{2}{(\log(2))^k}+n!\int_{2}^{x}\frac{dt}{(\log(t))^{n+1}} \leq \sum_{k=1}^{n}(k-1)!\frac{x}{(\log(x))^k}+n!\int_{2}^{x}\frac{dt}{(\log(t))^{n+1}}$$
    Then, with the statement that $\int_{2}^{x}\frac{dt}{(\log(t))^{n+1}}=O\left(\frac{x}{(\log(x))^{n+1}}\right)$ deduced previously, for any $n\in\mathbb{N}$, we get the following:
    $$\textmd{Li}(x)=\sum_{k=1}^{n}(k-1)!\frac{x}{(\log(x))^{k}}+O\left(\frac{x}{(\log(x))^{n+1}}\right)$$
    $$ = \frac{x}{\log(x)}+\frac{x}{(\log(x))^2}+2!\frac{x}{(\log(x))^3}+\cdots +(n-1)!\frac{x}{(\log(x))^n}+O\left(\frac{x}{(\log(x))^{n+1}}\right)$$

\end{itemize}

\break

\section*{3}
\begin{myBox}[]{}
    \begin{question}
        Stein and Shakarchi Pg. 204 Problem 2:

        One of the "explicit formulas" in the theory of primes is as follows: if $\psi_1$ is the integrated Tchebychev function considered in Section $2$, then 
        $$\psi_1(x)=\frac{x^2}{2}-\sum_{\rho}\frac{x^{\rho+1}}{\rho(\rho+1)}-E(x)$$
        where the sum is taken over all zeros $\rho$ of the $\zeta$-function in the critical strip. The error term is given by $E(x)=c_1x+c_0+\sum_{k=1}^{\infty}x^{1-2k}/(2k(2k-1))$, where $c_1=\zeta'(0)/\zeta(0)$ and $c_0=\zeta'(-1)/\zeta(-1)$. Note that $\sum_{\rho}1/|\rho|^{1+\epsilon}<\infty$ for every $\epsilon>0$, because $(1-s)\zeta(s)$ has order of growth $1$. Also, obviously $E(x)=O(x)$ as $x\rightarrow\infty$.
    \end{question}
\end{myBox}

\textbf{Pf:}

First, recall that the following formula of $\psi_1(x)$ holds for any $x>1$ and $c>1$:
$$\psi_1(x)=\frac{1}{2\pi i}\int_{c-i\infty}^{c+i\infty}\frac{x^{s+1}}{s(s+1)}\left(-\frac{\zeta'(s)}{\zeta(s)}\right)ds$$
Which, to get a closed expression, we'll utilize a product formula for $\zeta$ under product formula, and residue theorem.

\hfil

\textbf{1. Formula for $\zeta$:}

Assume we get the following, based on the functional equation of $\zeta$:
$$\zeta(s)=\frac{1}{2(s-1)}(\pi e^\gamma)^{s/2}\prod_{n=1}^{\infty}\left(1+\frac{s}{2k}\right)e^{-s/(2k)}\prod_{\textmd{Im}(\rho)>0}\left(1-\frac{s}{\rho}\right)\left(1-\frac{s}{(\overline{\rho})}\right)$$
Where $\rho$ represents all the zeros of $\zeta$ within the critical strip. (Note: $\zeta$ has zero at)  

\break

\section*{4}
\begin{myBox}[]{}
    \begin{question}
        Stein and Shakarchi Pg. 204 Problem 3:

        Using the previous problem one can show that 
        $$\pi(x)-\textmd{Li}(x)=O(x^{\alpha+\epsilon})\quad \textmd{as }x\rightarrow\infty$$
        for every $\epsilon>0$, where $\alpha$ is fixed and $1/2\leq \alpha<1$ if and only if $\zeta(s)$ has no zeros in the strip $\alpha<\textmd{Re}(s)<1$. The case $\alpha=1/2$ corresponds to the Riemann Hypothesis.     
    \end{question}
\end{myBox}

\textbf{Pf:}

\end{document}