\documentclass{article}
\usepackage{graphicx} % Required for inserting images
\usepackage[margin = 2.54cm]{geometry}
\usepackage[most]{tcolorbox}

\newtcolorbox{myBox}[3]{
arc=5mm,
lower separated=false,
fonttitle=\bfseries,
%colbacktitle=green!10,
%coltitle=green!50!black,
enhanced,
attach boxed title to top left={xshift=0.5cm,
        yshift=-2mm},
colframe=blue!50!black,
colback=blue!10
}

\usepackage{amsmath}
\usepackage{amssymb}
\usepackage{verbatim}
\usepackage[utf8]{inputenc}
\linespread{1.2}

\newtheorem{definition}{Definition}
\newtheorem{proposition}{Proposition}
\newtheorem{theorem}{Theorem}
\newtheorem{question}{Question}

\title{Math CS 122b HW8 Part 1}
\author{Zih-Yu Hsieh}

\begin{document}
\maketitle

\section*{1 (need slight modification)}
\begin{myBox}[]{}
    \begin{question}
        Stein and Shakarchi Pg. 201-202 Exercise 8:

        The function $\zeta$ has infinitely many zeros in the critical strip. This can be seen as follows.
        \begin{itemize}
            \item[(a)] Let $F(s)=\xi(1/2+s)$, where $\xi(s)=\pi^{-s/2}\Gamma(s/2)\zeta(s)$. 
            Show that $F(s)$ is an even function of $s$, and as a result, there exists $G$ so that $G(s^2)=F(s)$.
            \item[(b)] Show that the function $(s-1)\zeta(s)$ is an entire function of growth order $1$, that is 
            $$|(s-1)\zeta(s)|\leq A_\epsilon e^{a_\epsilon |s|^{1+\epsilon}}$$
            As a consequence $G(s)$ is of growth order $1/2$.
            \item[(c)] Deduce from the above that $\zeta$ has infinitely many zeros in the critical strip.
        \end{itemize}
        [Hint: To prove (a) and (b) use the functional equation for $\zeta(s)$. For (c), use a result of Hadamard, which states that an entire function with fractional order has infinitely many zeros (Exercise $14$ in Chapter $5$)].
    \end{question}
\end{myBox}

\textbf{Pf:}
\begin{itemize}
    \item[(a)] Recall that in \textbf{HW 7 Question 1} (\textbf{Freitag Chap. VII.5 Problem 5}), to deduce the functional equation of $\zeta$, we've proven the functional equation $\xi(s')=\xi(1-s')$. As a result, for any $s\in\mathbb{C}$, if treating $F$ as a meromorphic function, we get:
    $$F(s)=\xi\left(\frac{1}{2}+s\right) = \xi\left(1-\left(\frac{1}{2}+s\right)\right) = \xi\left(\frac{1}{2}-s\right) = F(-s)$$
    Hence, this proves that $F(s)$ is an even function.

    \hfil

    \item[(b)] Recall that $\zeta(s)$ is analytic on $\mathbb{C}\setminus\{1\}$, with a simple pole at $s=1$ with residue $1$, then $(s-1)\zeta(s)$ is in fact having a removable singularity at $s=1$, hence can be extended to an entire function.
    
    \hfil

    \textbf{1. $(s-1)\zeta(s)$ Has growth order $1$ for $\textmd{Re}(s)\geq \frac{1}{2}$:}

    In \textbf{Freitag Lemma VII.5.2}, the following functions are well defined:
    $$\forall t\in\mathbb{R},\quad \beta(t)=t-[t]-\frac{1}{2},\quad [t]:=\max{n\in\mathbb{Z},\ n\leq t}$$
    $$\forall s\in\mathbb{C},\ \textmd{Re}(s)>0,\quad F(s):=\int_{1}^{\infty}t^{-s-1}\beta(t)dt$$
    Then as a result, the following equation is true for $\textmd{Re}(s)>1$, hence defines an analytic continuation for $\zeta(s)$ on $\textmd{Re}(s)>0$:
    $$\forall s\in\mathbb{C},\ \textmd{Re}(s)>1,\quad \zeta(s)=\frac{1}{2}+\frac{1}{s-1}-sF(s)$$
    So, if multiply with $(s-1)$, for $\textmd{Re}(s)\geq \frac{1}{2}$, $(s-1)\zeta(s)$ is well-defined, and can be given as the following formula:
    $$(s-1)\zeta(s)=\frac{(s-1)}{2}+1-(s-1)sF(s)$$

    Which, let $s = x+iy$ for $x,y\in\mathbb{R}$, on $\textmd{Re}(s) = x \geq \frac{1}{2}$ (which $\frac{1}{x}\leq 2$), $F(s)$ can be bounded as follow:
    $$|F(s)| = \left|\int_{1}^{\infty}t^{-s-1}\beta(t)dt\right| \leq \int_{1}^{\infty}|t^{-(x+iy)-1}\beta(t)|dt \leq \int_{1}^{\infty}|t^{-x-1}\cdot t^{iy}|dt = \int_{1}^{\infty}t^{-x-1}dt$$
    $$ = \frac{-1}{x}t^{-x}\bigg|_{1}^{\infty} = \frac{1}{x}\leq 2$$
    (Note: for any $t\in\mathbb{R}$, $|\beta(t)|\leq \frac{1}{2}<1$, and since $x\geq \frac{1}{2}$, then the integral of $t^{-x-1}$ has power $<-1$, which is absolutely convergent).

    So, if considering the modulus of $(s-1)\zeta(s)$ on $\textmd{Re}(s)\geq \frac{1}{2}$, we get the following:
    $$|(s-1)\zeta(s)| = \left|\frac{(s-1)}{2}+1-(s-1)sF(s)\right|\leq \frac{|s-1|}{2} + 1 + |(s-1)s|\cdot |F(s)| \leq \frac{|s|+1}{2} + 1 + 2(|s|^2+|s|)$$
    $$\leq 2|s|^2 + \frac{3}{2}|s|+\frac{3}{2}$$
    Which, take $4e^{|s|} = 4 + 4|s| + 2|s|^2 + \sum_{n=3}^{\infty}\frac{4}{n!}|s|^n$, since for any $s\in\mathbb{C}$ each term is nonnegative, then we can deduce:
    $$|(s-1)\zeta(s)| \leq 2|s|^2 + \frac{3}{2}|s|+\frac{3}{2} \leq 4+4|s| + 2|s|^2 \leq 4 + 4|s| + 2|s|^2 + \sum_{n=3}^{\infty}\frac{4}{n!}|s|^n = 4e^{|s|}$$
    This shows that $(s-1)\zeta(s)$ has growth order $1$ on the half plane $\textmd{Re}(s)\geq \frac{1}{2}$.

    \hfil

    \textbf{2. $(s-1)\zeta(s)$ Has growth order $1$ for the whole plane:}

    In the previous part the growth order is verified for $\textmd{Re}(s)\geq \frac{1}{2}$. so the rest suffices to show it for the half plane $\textmd{Re}(s')<\frac{1}{2}$. (And, we'll utilize the fact that for all $s\in\mathbb{C}$, $|e^s|\leq e^{|s|}$, which can be seen using Taylor Series).
    
    Recall that in \textbf{HW 7}, we've proven the following functional equation of $\zeta$:
    $$\zeta(1-s) = 2(2\pi)^{-s}\Gamma(s)\cos\left(\frac{\pi s}{2}\right)\zeta(s)$$
    Hence, for any $s'$ with $\textmd{Re}(s')<\frac{1}{2}$, let $s' = 1-s$ for some $s\in\mathbb{C}$, then $s = 1-s'$, so $\textmd{Re}(s) = \textmd{Re}(1-s') > \frac{1}{2}$. Then, the equation $(s'-1)\zeta(s')$ becomes:
    $$(s'-1)\zeta(s') = ((1-s)-1)\zeta(1-s) = -s\cdot 2(2\pi)^{-s}\Gamma(s)\cos\left(\frac{\pi s}{2}\right)\zeta(s)$$
    And, since $\cos(\frac{\pi}{2})=0$, $\cos(\frac{\pi s}{2}$ has a zero at $s=1$, then $\cos\left(\frac{\pi s}{2}\right) = (s-1)h(s)$ for some analytic function $h$. So, the above formula can be further written as:
    $$((1-s)-1)\zeta(1-s) = -s\cdot 2(2\pi)^{-s}\Gamma(s)h(s)\cdot (s-1)\zeta(s)$$
    Which, $|s| = |1-s'| \leq |s'|+1$, so the growth order in terms of $|s|$ can be replaced using $|s'|$ instead. From the above equality, we do need to talk about the growth order of different components:
    \begin{itemize}
        \item For $(2\pi)^{-s} = e^{-\log(2\pi)s} = e^{-\log(2\pi)(x+iy)} = e^{-\log(2\pi)x}\cdot e^{-\log(2\pi)iy}$, it satisfies the following: 
        $$|(2\pi)^{-s}| =  |e^{-\log(2\pi)s}| \leq e^{\log(2\pi)|s|}$$
        This proves that $(2\pi)^{-s}$ has growth order $1$.
        \item For $\Gamma(s)$, since we're working with the half plane $\textmd{Re}(s)>\frac{1}{2}$, then it's valid to apply \textbf{Stir
    ing's Formula} (given in \textbf{Freitag Proposition IV.1.14}): 
        
        Let $H(s)=\sum_{n=0}^{\infty}\left(\left(s+n+\frac{1}{2}\right)\log\left(1+\frac{1}{s+n}\right)-1\right)$, then $\Gamma(s)$ can be expressed as:
        $$\Gamma(s)=\sqrt{2\pi}s^{s-\frac{1}{2}}e^{-s}e^{H(z)} = \sqrt{2\pi}e^{(s-1/2)\log(s)-s+H(s)}$$
        and $s\rightarrow\infty$ implies $H(s)\rightarrow 0$ (within the given half plane $\textmd{Re}(s)>\frac{1}{2}$).

        Which, notice that for $s$ in the half plane, since $s\rightarrow\infty$ implies $H(s)\rightarrow 0$, then there exists $M>0$, such that $|s|>M$ implies $|H(s)|<1$. And, since for all $\epsilon>0$ (specifically, can limit to $\epsilon<1$), there exists $M'>0$, such that $|\log(s)| \leq |s|^\epsilon$, then for all $s$ in the half plant satisfies $|s|>M,M'$, we get:
        $$\left|\left(s-\frac{1}{2}\right)\log(s)-s+H(s)\right| \leq \left(|s|+\frac{1}{2}\right)|\log(s)| + |s| + |H(s)| \leq \left(|s|+\frac{1}{2}\right)|s|^{\epsilon} + |s| + 1$$
        $$\leq |s|^{1+\epsilon}+\frac{1}{2}|s|^\epsilon+|s|^{1+\epsilon}+1 \leq \frac{5}{2}|s|^{1+\epsilon}+1$$
        Hence, $\Gamma(s)$ satisfies:
        $$|\Gamma(s)| = \left|\sqrt{2\pi}e^{(s-1/2)\log(s)-s+H(s)}\right| \leq \sqrt{2\pi}\exp\left(\left|\left(s-\frac{1}{2}\right)\log(s)-s+H(s)\right|\right)$$
        $$\leq \sqrt{2\pi}\exp\left(\frac{5}{2}|s|^{1+\epsilon}+1\right) = e\sqrt{2\pi}e^{\frac{5}{2}|s|^{1+\epsilon}}$$
        Hence, for any $\epsilon>0$, with suitable constant $A_\epsilon,a_\epsilon>0$, on the half plane $\textmd{Re}(s)>\frac{1}{2}$, $|\Gamma(s)|\leq A_\epsilon e^{a_\epsilon|s|^{1+\epsilon}}$, showing that $\Gamma(s)$ has growth order $1$.

        \item For $h(s)$ mentioned above, since $(s-1)h(s)=\cos\left(\frac{\pi s}{2}\right)$, and $\cos\left(\frac{\pi s}{2}\right)$ can be written as:
        $$\cos\left(\frac{\pi s}{2}\right) = \frac{e^{i\frac{\pi s}{2}}+e^{-i\frac{\pi s}{2}}}{2}$$
        Hence, the following inequality is true:
        $$\left|\cos\left(\frac{\pi s}{2}\right)\right| \leq \frac{1}{2}\left(|e^{i\frac{\pi s}{2}}| + |e^{-i\frac{\pi s}{2}}|\right) \leq \frac{1}{2}\left(e^{\frac{\pi}{2}|s|}+e^{\frac{\pi}{2}|s|}\right) = e^{\frac{\pi}{2}|s|}$$
        Hence, $\cos\left(\frac{\pi s}{2}\right)$ is with growth order $1$, which also implies that $h(s)$ is with growth order $1$.
    \end{itemize}
    Finally, back to the original equation, since for any $s'$ with $\textmd{Re}(s')<\frac{1}{2}$, writing $s' = 1-s$ for $\textmd{Re}(s)>\frac{1}{2}$ yields the following expresion:
    $$(s'-1)\zeta(s')=((1-s)-1)\zeta(1-s) = -s\cdot 2(2\pi)^{-s}\Gamma(s)h(s)\cdot (s-1)\zeta(s)$$
    Then, since $s,\ (2\pi)^{-s},\ \Gamma(s),\ h(s)$ are all with growth order $1$, and $(s-1)\zeta(s)$ has been proven to have growth order $1$ also in the previous part, then the whole product $(s'-1)\zeta(s')$ is with growth order $1$ (with input $s$). However, since $|s| = |1-s'| \leq |s'|+1$ as mentioned before, then it is also with growth order $1$ with respect to $s'$.

    \hfil

    Regardless of the choise of $s$ (either $\textmd{Re}(s)\geq \frac{1}{2}$ or $\textmd{Re}(s)<\frac{1}{2}$), we eventually get that $(s-1)\zeta(s)$ is with growth order $1$. 
    \begin{comment}
    Then, since $F(s)$ defined in the question satisfies the following equation:
    $$F(s)=\xi(1/2+s) = \pi^{-(1/2+s)/2}\Gamma\left(\frac{(1/2+s)}{2}\right)\zeta\left(\frac{1}{2}+s\right)$$
    $$\left(\left(\frac{1}{2}+s\right)-1\right)F(s) = \pi^{-(1/2+s)/2}\Gamma\left(\frac{(1/2+s)}{2}\right)\left(\left(\frac{1}{2}+s\right)-1\right)\zeta\left(\frac{1}{2}+s\right)$$
    Then, since $\left(\left(\frac{1}{2}+s\right)-1\right)\zeta\left(\frac{1}{2}+s\right)$, $\Gamma\left(\frac{(1/2+s)}{2}\right)$ and $\pi^{-(1/2+s)/2}$ are all well-defined on $\textmd{Re}(s)\geq 0$ (or $\textmd{Re}\left(\frac{1}{2}+s\right)\geq \frac{1}{2}$), and $F(s)$ is an even function , this implies 
    \end{comment}
    
    \hfil

    \item[(c)] In \textbf{Part (b)}, it was proven that $F$ has growth order $1$, while $G$ (after being modified into an entire function) has growth order $1/2$. So based on Hadamard's result, it has infinitely many zeros, which also implies that $F(s)=G(s^2)$ has infinitely many zeros. However, since $F(s)$ is given as:
    $$F(s)=\xi(1/2+s) =  \pi^{-(1/2+s)/2}\Gamma\left(\frac{(1/2+s)}{2}\right)\zeta\left(\frac{1}{2}+s\right)$$
    Which, because $F(s)$ is even, it is enough to consider the half plane $\textmd{Re}(s)\geq 0$: Because $\pi^{z},\ \Gamma(z)$ are both nonzero functions, then these zeros of $F$ must be contributed by $\zeta(1/2+s)$; On the other hand, it is well-known that $\zeta(z)$ has no zeros for $\textmd{Re}(z)\geq 1$, hence for $\textmd{Re}(1/2+s)\geq 1$, or $\textmd{Re}(s)\geq \frac{1}{2}$, since $\zeta(1/2+s)$ has no zeros, then $F(s)$ has no zeros.
    Hence, the zeros of $F(s)$ (on the half plane $\textmd{Re}(s)\geq 0$) must appear in the range $0\leq \textmd{Re}(s) < \frac{1}{2}$, which eventually implies that there are infinitely many $s$ in this strip (which satisfies $\frac{1}{2}\leq \textmd{Re}(1/2+s)<1$, with $(1/2+s)$ being in the critical strip) satisfying $\zeta(1/2+s) = 0$. 

    So, we can conclude that $\zeta(s)$ has infinitely many zeros in the critical strip.
\end{itemize}

\break

\section*{2}
\begin{myBox}[]{}
    \begin{question}
        Stein and Shakarchi Pg. 202-203 Exercise 10:

        In the theory of primes, a better approximation fo $\pi(x)$ (instead of $x/\log(x)$) turns out to be $\textmd{Li}(x)$ defined by 
        $$\textmd{Li}(x)=\int_{2}^{x}\frac{dt}{\log(t)}$$
        \begin{itemize}
            \item[(a)] Prove that 
            $$\textmd{Li}(x)=\frac{x}{\log(x)}+O\left(\frac{x}{(\log(x))^2}\right)\quad \textmd{as }x\rightarrow\infty$$
            and that as a consequence 
            $$\pi(x)\sim \textmd{Li}(x)\quad \textmd{as }x\rightarrow\infty$$
            \item[(b)] Refine the previous analysis by showing that for every integer $N>0$ one has the following asymptotic expansion 
            $$\textmd{Li}(x)=\frac{x}{\log(x)}+\frac{x}{(\log(x))^2}+2\frac{x}{(\log(x))^3}+\cdots (N-1)!\frac{x}{(\log(x))^N}+O\left(\frac{x}{(\log(x))^{N+1}}\right)$$
            as $x\rightarrow\infty$.
        \end{itemize}
    \end{question}
\end{myBox}

\textbf{Pf:}
\begin{itemize}
    \item[(a)] First, using integration by parts, for all $x\geq 4$ (where $x\geq \sqrt{x}\geq 2$), $\textmd{Li}(x)$ can be expressed as follow:
    $$\textmd{Li}(x)=\int_{2}^{x}\frac{dt}{\log(t)} = \frac{t}{\log(t)}\bigg|_{2}^{x}-\int_{2}^{x}t\cdot \frac{d}{dt}\left(\frac{1}{\log(t)}\right)dt$$
    $$ = \frac{x}{\log(x)}-\frac{2}{\log(2)}-\int_{2}^{x}t\cdot\frac{-1}{(\log(t))^2}\cdot\frac{1}{t}dt$$
    $$ = \frac{x}{\log(x)}-\frac{2}{\log(2)}+\int_{2}^{x}\frac{1}{(\log(t))^2}dt$$
    Which, for the last integral expression, it can be reformulate as follow:
    $$\int_{2}^{x}\frac{dt}{(\log(t))^2} = \int_{2}^{\sqrt{x}}\frac{dt}{(\log(t))^2}+\int_{\sqrt{x}}^{x}\frac{dt}{(\log(t))^2}$$
    Since $\log(t)$ is a strictly increasing function on $(1,\infty)$ and is strictly positive, then $\frac{1}{(\log(t))^2}$ is a strictly decreasing function on this interval instead. Hence, for all $t\in [2,\sqrt{x}]$, $\frac{1}{(\log(t))^2}\leq \frac{1}{(\log(2))^2}$, while any $t\in [\sqrt{x},x]$ satisfies $\frac{1}{(\log(t))^2}\leq \frac{1}{(\log(\sqrt{x}))^2} = \frac{4}{(\log(x))^2}$. Hence, the above expression satisfies:
    $$\int_{2}^{x}\frac{dt}{(\log(t))^2} = \int_{2}^{\sqrt{x}}\frac{dt}{(\log(t))^2}+\int_{\sqrt{x}}^{x}\frac{dt}{(\log(t))^2}\leq \int_{2}^{\sqrt{x}}\frac{dt}{(\log(2))^2}+\int_{\sqrt{x}}^{x}\frac{4dt}{(\log(x))^2}$$
    $$ = \frac{\sqrt{x}-2}{(\log(2))^2} + \frac{4(x-\sqrt{x})}{(\log(x))^2}\leq \frac{4x}{(\log(x))^2}+\frac{\sqrt{x}}{(\log(2))^2}$$
    Which, if evaluate the following limit, we get:
    $$\lim_{x\rightarrow\infty}\frac{\sqrt{x}}{x/(\log(x))^2} = \lim_{x\rightarrow\infty}\frac{(\log(x))^2}{\sqrt{x}} = \lim_{x\rightarrow\infty}\frac{2\log(x)/x}{1/(2\sqrt{x})} = \lim_{x\rightarrow\infty}\frac{4\log(x)}{\sqrt{x}}$$
    $$= \lim_{x\rightarrow\infty}\frac{4/x}{1/(2\sqrt{x})} = \lim_{x\rightarrow\infty}\frac{8}{\sqrt{x}}=0$$
    Hence, for some $x_1>4$ and $A_1>0$, we have $x>x_1$ implies $\sqrt{x}\leq A_1\frac{x}{(\log(x))^2}$. So, the integral follows the inequality below for $x>x_0$:
    $$\int_{2}^{x}\frac{dt}{(\log(t))^2}\leq \frac{\sqrt{x}}{(\log(2))^2} + \frac{4x}{(\log(x))^2} \leq \frac{1}{(\log(2))^2}\cdot\frac{A_1x}{(\log(x))^2}+\frac{4x}{(\log(x))^2}$$
    $$\leq \left(\frac{A_1}{(\log(2))^2}+4\right)\frac{x}{(\log(x))^2}$$
    So, this shows that $\int_{2}^{x}\frac{dt}{(\log(t))^2}=O\left(\frac{x}{(\log(x))^2}\right)$. Hence:
    $$\textmd{Li}(x)=\frac{x}{\log(x)}-\frac{2}{\log(2)}+\int_{2}^{x}\frac{dt}{(\log(t))^2}\leq \frac{x}{\log(x)}+\int_{2}^{x}\frac{dt}{(\log(t))^2} = \frac{x}{\log(x)}+O\left(\frac{x}{(\log(x))^2}\right)$$
    This shows that $\textmd{Li}(x)=\frac{x}{\log(x)}+O\left(\frac{x}{(\log(x))^2}\right)$.

    \hfil

    \item[(b)] First, we'll consider the following formula about the integral of $\frac{1}{(\log(t))^n}$ using integration by parts:
    $$\forall n\in\mathbb{N},\quad \int_{2}^{x}\frac{dt}{(\log(t))^n} = \frac{t}{(\log(t))^n}\bigg|_{2}^{x}-\int_{2}^{x}t\cdot\frac{d}{dt}\left(\frac{1}{(\log(t))^n}\right)dt$$
    $$ = \frac{x}{(\log(x))^n}-\frac{2}{(\log(2))^n}-\int_{2}^{x}t\cdot\frac{-n}{(\log(t))^{n+1}}\cdot\frac{1}{t}dt = \frac{x}{(\log(x))^n}-\frac{2}{(\log(2))^n}+n\int_{2}^{x}\frac{dt}{(\log(t))^{n+1}}$$
    Which, using the same argument used in \textbf{part (a)} about $\frac{1}{(\log(t))^n}$ is a decreasing function for all $n\in\mathbb{N}$, for all $x\geq 4$ (where $x>\sqrt{x}\geq 2$), we get:
    $$\int_{2}^{x}\frac{dt}{(\log(t))^{n+1}} = \int_{2}^{\sqrt{x}}\frac{dt}{(\log(t))^{n+1}}+\int_{\sqrt{x}}^{x}\frac{dt}{(\log(t))^{n+1}} \leq \int_{2}^{\sqrt{x}}\frac{dt}{(\log(2))^{n+1}}+\int_{\sqrt{x}}^{x}\frac{dt}{(\log(\sqrt{x}))^{n+1}}$$
    $$=\frac{(\sqrt{x}-2)}{(\log(2))^{n+1}} + \frac{2^{n+1}(x-\sqrt{x})}{(\log(x))^{n+1}} \leq \frac{2^{n+1}x}{(\log(x))^{n+1}}+\frac{\sqrt{x}}{(\log(2))^{n+1}}$$
    Now, since the base case $\lim_{x\rightarrow\infty}\frac{\sqrt{x}}{x/(\log(x))^2}=0$ is proven in \textbf{part (a)}, using induction, we can get the following relationship:
    $$\forall n\in\mathbb{N},\quad \lim_{x\rightarrow\infty}\frac{\sqrt{x}}{x/(\log(x))^{n+1}} = \lim_{x\rightarrow\infty}\frac{(\log(x))^{n+1}}{\sqrt{x}} = \lim_{x\rightarrow\infty}\frac{(n+1)(\log(x))^n/x}{1/(2\sqrt{x})}$$
    $$ = \lim_{x\rightarrow\infty}2(n+1)\frac{\sqrt{x}}{x/(\log(x))^n} = 0$$
    Hence, there exists $x_n>4$ and $A_n>0$, such that $x>x_n$ implies $\sqrt{x}\leq A_n\frac{x}{(\log(x))^{n+1}}$. Hence, we get:
    $$\int_{2}^{x}\frac{dt}{(\log(t))^{n+1}} \leq \frac{2^{n+1}x}{(\log(x))^{n+1}}+\frac{\sqrt{x}}{(\log(2))^{n+1}}$$
    $$ \leq \frac{2^{n+1}x}{(\log(x))^{n+1}}+\frac{A_n}{(\log(2))^{n+1}}\frac{x}{(\log(x))^{n+1}} = \left(2^{n+1}+\frac{A_n}{(\log(2))^{n+1}}\right)\frac{x}{(\log(x))^{n+1}}$$
    This shows that $\int_{2}^{x}\frac{dt}{(\log(t))^{n+1}}=O\left(\frac{x}{(\log(x))^{n+1}}\right)$.

    \hfil

    Finally, using the case proven in \textbf{part (a)}, we know $\textmd{Li}(x)=\frac{x}{\log(x)}-\frac{2}{\log(2)}+\int_{2}^{x}\frac{dt}{(\log(t))^2}$. Which utilizing the above equation, by induction, one can show that for any integer $n\geq 2$, the following formula holds:
    $$\textmd{Li}(x)=\sum_{k=1}^{n}(k-1)!\frac{x}{(\log(x))^k} - \sum_{k=1}^{n}(k-1)!\frac{2}{(\log(2))^k}+n!\int_{2}^{x}\frac{dt}{(\log(t))^{n+1}} \leq \sum_{k=1}^{n}(k-1)!\frac{x}{(\log(x))^k}+n!\int_{2}^{x}\frac{dt}{(\log(t))^{n+1}}$$
    Then, with the statement that $\int_{2}^{x}\frac{dt}{(\log(t))^{n+1}}=O\left(\frac{x}{(\log(x))^{n+1}}\right)$ deduced previously, for any $n\in\mathbb{N}$, we get the following:
    $$\textmd{Li}(x)=\sum_{k=1}^{n}(k-1)!\frac{x}{(\log(x))^{k}}+O\left(\frac{x}{(\log(x))^{n+1}}\right)$$
    $$ = \frac{x}{\log(x)}+\frac{x}{(\log(x))^2}+2!\frac{x}{(\log(x))^3}+\cdots +(n-1)!\frac{x}{(\log(x))^n}+O\left(\frac{x}{(\log(x))^{n+1}}\right)$$

\end{itemize}

\break

\section*{3}
\begin{myBox}[]{}
    \begin{question}
        Stein and Shakarchi Pg. 204 Problem 2:

        One of the "explicit formulas" in the theory of primes is as follows: if $\psi_1$ is the integrated Tchebychev function considered in Section $2$, then 
        $$\psi_1(x)=\frac{x^2}{2}-\sum_{\rho}\frac{x^{\rho+1}}{\rho(\rho+1)}-E(x)$$
        where the sum is taken over all zeros $\rho$ of the $\zeta$-function in the critical strip. The error term is given by $E(x)=c_1x+c_0+\sum_{k=1}^{\infty}x^{1-2k}/(2k(2k-1))$, where $c_1=\zeta'(0)/\zeta(0)$ and $c_0=\zeta'(-1)/\zeta(-1)$. Note that $\sum_{\rho}1/|\rho|^{1+\epsilon}<\infty$ for every $\epsilon>0$, because $(1-s)\zeta(s)$ has order of growth $1$. Also, obviously $E(x)=O(x)$ as $x\rightarrow\infty$.
    \end{question}
\end{myBox}

\textbf{Pf:}

First, recall that the following formula of $\psi_1(x)$ holds for any $x>1$ and $c>1$:
$$\psi_1(x)=\frac{1}{2\pi i}\int_{c-i\infty}^{c+i\infty}\frac{x^{s+1}}{s(s+1)}\left(-\frac{\zeta'(s)}{\zeta(s)}\right)ds$$
Which, to get a closed expression, we'll utilize Hadamard's product formula for $\zeta$ and Residue Theorem.

\hfil

\textbf{1. Product Formula for $\zeta$ and $-\frac{\zeta'}{\zeta}$:}

Based on \textbf{Question 1 part (b)} in this assignment, we've proven that $(s-1)\zeta(s)$ is an entire function with growth order $1$, and it is zero precisely at all the zeros of $\zeta(s)$ since at $s=1$, $\zeta(s)$ has residue $1$. Which, $(s-1)\zeta(s)$ has zeros at $-2k$ for $k\in\mathbb{N}$, and all zeros of $\zeta$, denoted as $\rho$ in the critical strip.

Then, based on \textbf{Hadamard's Factorization Theorem} (can be seen in \textbf{Stein and Shakarchi Chapter 5.5}), since $(s-1)\zeta(s)$ has growth order $1$ with the zeros mentioned above (which the zeros are all nonzero), then there exists polynomial $P(s) = ls+d$ with degree $1$ (at most the growth order), such that the following holds:
$$(s-1)\zeta(s)=e^{ls+d}\left(\prod_{k=1}^{\infty}E_1\left(\frac{s}{2k}\right)\right)\left(\prod_{\rho}E_1\left(\frac{s}{\rho}\right)\right)$$
$$ = e^{ls+d}\left(\prod_{k=1}^{\infty}\left(1-\frac{s}{2k}\right)e^{s/(2k)}\right)\left(\prod_{\rho}\left(1-\frac{s}{\rho}\right)e^{s/\rho}\right)$$
Where the second product contains all nontrivial zeros of $\zeta$ in the critical strip. Hence, the following is a formula for $\zeta(s)$ in terms of products of zeros and poles:
$$\zeta(s) = (s-1)^{-1}e^{ls+d}\left(\prod_{k=1}^{\infty}\left(1-\frac{s}{2k}\right)e^{s/(2k)}\right)\left(\prod_{\rho}\left(1-\frac{s}{\rho}\right)e^{s/\rho}\right)$$
Then, utilizing logarithmic derivative, we get the following:
$$\frac{\zeta'(s)}{\zeta(s)} = -\frac{1}{s-1}+l+\sum_{k=1}^{\infty}\left(\frac{1}{s-2k}+\frac{1}{2k}\right) + \sum_{\rho}\left(\frac{1}{s-\rho}+\frac{1}{\rho}\right)$$
$$-\frac{\zeta'(s)}{\zeta(s)}=\frac{1}{s-1}-l-\sum_{k=1}^{\infty}\frac{s}{(s-2k)2k}-\sum_{\rho}\frac{s}{(s-\rho)\rho}$$
And, this formula is normally convergent within any compact subset of the domain (not containing the zeros and the poles of $\zeta$), so integration can be exchanged with summation.

\hfil

\textbf{2. Contour Integration:}

Now, choose any $c_0>1$, and restrict the domain to the open half plane $\textmd{Re}(s)<c_0$. CHoose any $c\in\mathbb{R}$ such that $1<c<c_0$, and define the contour $\gamma_{r}$ as the following semicircle for any $r\in\mathbb{R}_{>0}$:

\textbf{Insert Image}

Which, $\gamma_r$ is involved with semicircle $c_r$ with radius $r$, and the straight line $\ell_r$ parametrized by $c+it$, for $t\in [-r,r]$.

Temporarily, assume $r$ is chosen so that $\gamma_{r}$ contains no zeros or poles of $\zeta$, and let $D_{r}$ be the region enclosed by $\gamma_{r}$. Then, if perform the contour integration, we get the following:
$$\frac{1}{2\pi i}\int_{\gamma_{r}}\frac{x^{s+1}}{s(s+1)}\left(-\frac{\zeta'(s)}{\zeta(s)}\right)ds = \frac{1}{2\pi i}\int_{\gamma_{r}}\frac{x^{s+1}}{s(s+1)}\left(\frac{1}{s-1}-l-\sum_{k=1}^{\infty}\frac{s}{(s-2k)2k}-\sum_{\rho}\frac{s}{(s-\rho)\rho}\right)ds$$
$$ = \frac{1}{2\pi i}\int_{\gamma_{r}}\frac{x^{s+1}}{s(s+1)(s-1)}ds-\frac{1}{2\pi i}\int_{\gamma_{r}}\frac{lx^{s+1}}{s(s+1)}ds - \sum_{k=1}^{\infty}\frac{1}{2\pi i}\int_{\gamma_{r}}\frac{x^{s+1}}{(s+1)(s-2k)2k}ds - \sum_{\rho}\frac{1}{2\pi i}\int_{\gamma_{r}}\frac{x^{s+1}}{(s+1)(s-\rho)\rho}ds$$

Now, since we've restricted the domain to $\textmd{Re}(s)<c_0$, then for any $s=u+iv$ in this region (which $u<c_0$), for any fixed $x>1$, we get $x^{s+1} = x^{(u+iv)+1} = x^{u+1}\cdot x^{iv} = x^{u+1}\cdot e^{iv\log(x)}$, which $|x^{s+1}| = x^{u+1} < x^{c_0+1}$.

\hfil

Then, for the first integral above, there involves some fixed $\alpha\in \mathbb{C}$, where the denominator involve the terms $(s-\alpha)$. Then, one can choose radius $R>0$, such that for all radius $r>R$, the involved term $(s-\alpha)$ satisfies $|s-\alpha| > \frac{r}{2}$ (which $\frac{1}{|s-\alpha|}<\frac{2}{r}$). So, for the integration over the semicircle $c_r$, we get the following:
$$\left|\frac{1}{2\pi i}\int_{c_{r}}\frac{x^{s+1}}{s(s+1)(s-1)}ds\right| \leq \frac{1}{2\pi}\int_{c_r}\frac{|x^{s+1}|}{|s(s+1)(s-1)|}|ds|< \frac{1}{2\pi}\int_{c_r}\frac{2^3\cdot x^{c_0+1}}{r^3}|ds|$$
$$ = \frac{1}{2\pi}\cdot\frac{2^3\cdot x^{c_0+1}}{r^3}\cdot \pi r = \frac{2^2\cdot x^{c_0+1}}{r^2}$$
(Note: We're integrating over a semicircle, so eventually the integration of a constnat multiplies by $\pi r$).

Then, take $r\rightarrow\infty$, since $\frac{1}{r^2}\rightarrow 0$, the above inequality proves that $\left|\frac{1}{2\pi i}\int_{c_{r}}\frac{x^{s+1}}{s(s+1)(s-1)}ds\right|\rightarrow 0$, or $\frac{1}{2\pi i}\int_{c_{r}}\frac{x^{s+1}}{s(s+1)(s-1)}ds\rightarrow 0$. Hence, we get the following:
$$\lim_{r\rightarrow\infty}\frac{1}{2\pi i}\int_{\gamma_r}\frac{x^{s+1}}{s(s+1)(s-1)}ds = \lim_{r\rightarrow\infty}\left(\frac{1}{2\pi i}\int_{c_r}\frac{x^{s+1}}{s(s+1)(s-1)}ds+\frac{1}{2\pi i}\int_{\ell_r}\frac{x^{s+1}}{s(s+1)(s-1)}ds\right)$$
$$ = \lim_{r\rightarrow\infty}\frac{1}{2\pi i}\int_{c-ir}^{c+ir}\frac{x^{s+1}}{s(s+1)(s-1)}ds = \frac{1}{2\pi i}\int_{c-i\infty}^{c+i\infty}\frac{x^{s+1}}{s(s+1)(s-1)}ds$$

Now, if we apply similar methods to the other integrals (integrating functions with polynomial of degree $2$ on the denominator), then we get the following instead (we'll use the second one as an example):
$$\left|\frac{1}{2\pi i}\int_{c_{r}}\frac{lx^{s+1}}{s(s+1)}ds\right|\leq \frac{1}{2\pi}\int_{c_r}\frac{l|x^{s+1}|}{|s(s+1)|}|ds| < \frac{1}{2\pi}\int_{c_r}\frac{2^2\cdot l\cdot x^{c_0+1}}{r^2}|ds|$$
$$ = \frac{1}{2\pi}\cdot \frac{2^2\cdot l\cdot x^{c_0+1}}{r^2}\cdot \pi r = \frac{2l\cdot x^{c_0+1}}{r}$$
Take $r\rightarrow\infty$, since $\frac{1}{r}\rightarrow 0$, the above inequality proves that $\left|\frac{1}{2\pi i}\int_{c_{r}}\frac{lx^{s+1}}{s(s+1)}ds\right|\rightarrow 0$, or $\frac{1}{2\pi i}\int_{c_{r}}\frac{lx^{s+1}}{s(s+1)}ds\rightarrow 0$. Hence, we again get the following:
$$\lim_{r\rightarrow\infty}\frac{1}{2\pi i}\int_{\gamma_{r}}\frac{lx^{s+1}}{s(s+1)}ds = \lim_{r\rightarrow\infty}\left(\frac{1}{2\pi i}\int_{c_{r}}\frac{lx^{s+1}}{s(s+1)}ds+\frac{1}{2\pi i}\int_{\ell_{r}}\frac{lx^{s+1}}{s(s+1)}ds\right)$$
$$ = \lim_{r\rightarrow\infty}\frac{1}{2\pi i}\int_{c-ir}^{c+ir}\frac{lx^{s+1}}{s(s+1)}ds = \frac{1}{2\pi i}\int_{c-i\infty}^{c+i\infty}\frac{lx^{s+1}}{s(s+1)}ds$$
Apply the similar formulas to the other two sums, then we get the following:
$$\lim_{r\rightarrow\infty}\frac{1}{2\pi i}\int_{\gamma_{r}}\frac{x^{s+1}}{s(s+1)}\left(-\frac{\zeta'(s)}{\zeta(s)}\right)ds$$

\hfil

$$ = \lim_{r\rightarrow\infty}\frac{1}{2\pi i}\int_{\gamma_{r}}\frac{x^{s+1}}{s(s+1)(s-1)}ds-\lim_{r\rightarrow\infty}\frac{1}{2\pi i}\int_{\gamma_{r}}\frac{lx^{s+1}}{s(s+1)}ds$$
$$- \lim_{r\rightarrow\infty}\sum_{k=1}^{\infty}\frac{1}{2\pi i}\int_{\gamma_{r}}\frac{x^{s+1}}{(s+1)(s-2k)2k}ds - \lim_{r\rightarrow\infty}\sum_{\rho}\frac{1}{2\pi i}\int_{\gamma_{r}}\frac{x^{s+1}}{(s+1)(s-\rho)\rho}ds$$

\hfil

$$ = \int_{c-i\infty}^{c+i\infty}\frac{x^{s+1}}{s(s+1)}\cdot \frac{1}{s-1}ds-\frac{1}{2\pi i}\int_{c-i\infty}^{c+i\infty}\frac{x^{s+1}}{s(s+1)}\cdot l ds$$
$$-\sum_{k=1}^{\infty}\frac{1}{2\pi i}\int_{c-i\infty}^{c+i\infty}\frac{x^{s+1}}{s(s+1)}\cdot \frac{s}{(s-2k)2k}ds -\sum_{\rho}\frac{1}{2\pi i}\int_{c-i\infty}^{c+i\infty}\frac{x^{s+1}}{s(s+1)}\cdot \frac{s}{(s-\rho)\rho}ds$$

\hfil

$$ = \frac{1}{2\pi i}\int_{c-i\infty}^{c+i\infty}\frac{x^{s+1}}{s(s+1)}\left(\frac{1}{s-1}-l-\sum_{k=1}^{\infty}\frac{s}{(s-2k)2k}-\sum_{\rho}\frac{s}{(s-\rho)\rho}\right)ds$$
$$ = \frac{1}{2\pi i}\int_{c-i\infty}^{c+i\infty}\frac{x^{s+1}}{s(s+1)}\left(-\frac{\zeta'(s)}{\zeta(s)}\right)ds = \psi_1(x)$$
So, it suffices to show that the above limit provides the explicit formula mentioned in the question.

\hfil

\textbf{3. Value of each integration:}

As a quick recap, we get the following formula:
$$\psi_1(x) = \frac{1}{2\pi i}\int_{c-i\infty}^{c+i\infty}\frac{x^{s+1}}{s(s+1)}\left(-\frac{\zeta'(s)}{\zeta(s)}\right)ds$$

\hfil

$$ = \lim_{r\rightarrow\infty}\frac{1}{2\pi i}\int_{\gamma_{r}}\frac{x^{s+1}}{s(s+1)(s-1)}ds-\lim_{r\rightarrow\infty}\frac{1}{2\pi i}\int_{\gamma_{r}}\frac{lx^{s+1}}{s(s+1)}ds$$
$$- \lim_{r\rightarrow\infty}\sum_{k=1}^{\infty}\frac{1}{2\pi i}\int_{\gamma_{r}}\frac{x^{s+1}}{(s+1)(s-2k)2k}ds - \lim_{r\rightarrow\infty}\sum_{\rho}\frac{1}{2\pi i}\int_{\gamma_{r}}\frac{x^{s+1}}{(s+1)(s-\rho)\rho}ds$$
Which, for each integration, there exists finitely many 


\break

\section*{4}
\begin{myBox}[]{}
    \begin{question}
        Stein and Shakarchi Pg. 204 Problem 3:

        Using the previous problem one can show that 
        $$\pi(x)-\textmd{Li}(x)=O(x^{\alpha+\epsilon})\quad \textmd{as }x\rightarrow\infty$$
        for every $\epsilon>0$, where $\alpha$ is fixed and $1/2\leq \alpha<1$ if and only if $\zeta(s)$ has no zeros in the strip $\alpha<\textmd{Re}(s)<1$. The case $\alpha=1/2$ corresponds to the Riemann Hypothesis.     
    \end{question}
\end{myBox}

\textbf{Pf:}

\end{document}