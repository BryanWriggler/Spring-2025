\documentclass{article}
\usepackage{graphicx} % Required for inserting images
\usepackage[margin = 2.54cm]{geometry}
\usepackage[most]{tcolorbox}

\newtcolorbox{myBox}[3]{
arc=5mm,
lower separated=false,
fonttitle=\bfseries,
%colbacktitle=green!10,
%coltitle=green!50!black,
enhanced,
attach boxed title to top left={xshift=0.5cm,
        yshift=-2mm},
colframe=blue!50!black,
colback=blue!10
}

\usepackage{amsmath}
\usepackage{amssymb}
\usepackage{verbatim}
\usepackage[utf8]{inputenc}
\linespread{1.2}

\newtheorem{definition}{Definition}
\newtheorem{proposition}{Proposition}
\newtheorem{theorem}{Theorem}
\newtheorem{question}{Question}

\title{Math CS 122B HW6}
\author{Zih-Yu Hsieh}

\begin{document}
\maketitle

\section*{1}
\begin{myBox}[]{}
    \begin{question}
        Freitag Chap. V.6 Exercise 5:

        Lef $f$ be an elliptic function for the lattice $L$. We choose $b_1,...,b_n$ to be a system of representatives modulo $L$ for the poles of $f$, and we consider for each $j$ the principal part of $f$ in the pole $b_j$:
        $$\sum_{v=1}^{l_j}\frac{a_{v,j}}{(z-b_j)^v}$$
        The Second Liiouville Theorem ensures the relation 
        $$\sum_{j=1}^{n}a_{1,j}=0$$
        Show:
        \begin{itemize}
            \item[(a)] Let $c_1,...,c_n\in\mathbb{C}$ b given numbers, and let $b_1,...,b_n$ modulo $L$ be a set of different points in $\mathbb{C}/L$. The function 
            $$h(z):=\sum_{j=1}^{n}c_j\zeta(z-b_j)$$
            constructed by means of the Weierstrass $\zeta$-function, is then elliptic, iff 
            $$\sum_{j=1}^{n}c_j=0$$
            \item[(b)] Let $b_1,...,b_n$ be pairwise different modulo $L$, and let $l_1,...,l_n$ be prescribed natural numbers. Let $a_{v,j}$ ($1\leq j\leq n$, $1\leq v\leq l_j$) be complex numbers such that $\sum_{j=1}^{n}a_{1,j}=0$ and $a_{l_j,j}\neq 0$ for all $j$.
            
            Then, there exists an elliptic function for the lattice $L$, having poles modulo $L$ exactly in the points $b_1,...,b_n$, and having the corresponding principal parts respectively equal to 
            $$\sum_{v=1}^{l_j}\frac{a_{v,j}}{(z-b_j)^v}$$
        \end{itemize}
    \end{question}
\end{myBox}

\textbf{Pf:}
\begin{itemize}
    \item[(a)] Given the Weierstrass $\sigma$-function below ($\sigma:\mathbb{C}\rightarrow\mathbb{C}$), the Weierstrass $\zeta$-function ($\zeta:\mathbb{C}\setminus L\rightarrow\mathbb{C}$) is defined as:
    $$\sigma(z)=z\prod_{\substack{w\in L\\w\neq 0}}\left(1-\frac{z}{w}\right)\exp\left(\frac{z}{w}+\frac{1}{2}\left(\frac{z}{w}\right)^2\right),\quad \zeta(z)=\frac{\sigma'(z)}{\sigma(z)}$$
    Based on the formula of $\sigma$, it has simple zeros only at all $w\in L$; and, it implies that $\zeta$ is not defined only on $L$. Now, to prove the statement, consider the following:
    \begin{itemize}
        \item[$\implies:$] Suppose the defined $h(z)$ is elliptic. Then, since for each index $j\in\{1,...,n\}$, $\sigma(z-b_j)$ has a simple zero at $(w+b_j)$ for each $w\in L$ (which the set $b_j+L$ contains all the simple zeros of $\sigma(z-b_j)$, which is discrete). Then, since $\bigcup_{j=1}^{n}(b_j+L)$ is also discrete, choose the fundamental region $P$ of lattice $L$ such that $\partial P$ contains no points from $\bigcup_{j=1}^{n}(b_j+L)$ (the set containing all the zeros of each $\sigma(z-b_j)$, also the set of all undefined points of all $\zeta(z-b_j)$), by the Second Liouville's Theorem, we get the following:
        $$0=\frac{1}{2\pi i}\int_{\partial P}h(z)dz = \frac{1}{2\pi i}\int_{\partial P}\sum_{j=1}^{n}c_j\zeta(z-b_j)dz = \sum_{j=1}^{n}c_j\cdot\frac{1}{2\pi i}\int_{\partial P}\frac{\sigma'(z-b_j)}{\sigma(z-b_j)}dz$$
        For each $j\in\{1,...,n\}$, since $P$ only contains one representative of $b_j\in\mathbb{C}/L$, then it only contains one zero of $\sigma(z-b_j)$. Hence, by argument principle, we get the following:
        $$\frac{1}{2\pi i}\int_{\partial P}\frac{\sigma'(z-b_j)}{\sigma(z-b_j)}dz = 1 = \textmd{Number of zeros of } \sigma(z-b_j) \textmd{ in } P$$
        Hence, the original integral becomes:
        $$0=\frac{1}{2\pi i}\int_{\partial P}h(z)dz = \sum_{j=1}^{n}c_j\cdot\frac{1}{2\pi i}\int_{\partial P}\frac{\sigma'(z-b_j)}{\sigma(z-b_j)}dz = \sum_{j=1}^{n}c_j$$
        So, $\sum_{j=1}^{n}c_j = 0$.

        \hfil

        \item[$\impliedby:$] Now, suppose $\sum_{j=1}^{n}c_j = 0$. For all $w\in L$, since $\sigma(z+w)$ and $\sigma(z)$ both only have simple zeros at any $w'\in L$, then $\frac{\sigma(z+w)}{\sigma(z)}$ can be extended to an entire function with no zeros in $\mathbb{C}$ (since the zeros cancel out at each $w'\in L$). Hence, there exists an analytic function $h:\mathbb{C}\rightarrow\mathbb{C}$, with $\frac{\sigma(z+w)}{\sigma(z)}=e^{h(z)}$. Then, apply derivatives, we get:
        $$\frac{\sigma'(z+w)\sigma(z)-\sigma'(z)\sigma(z+w)}{(\sigma(z))^2} = h'(z)e^{h(z)} = h'(z)\cdot\frac{\sigma(z+w)}{\sigma(z)}$$
        $$\frac{\sigma'(z+w)\sigma(z+w)}{\sigma(z+w)\sigma(z)}-\frac{\sigma'(z)\sigma(z+w)}{(\sigma(z))^2}=h'(z)\cdot\frac{\sigma(z+w)}{\sigma(z)}$$
        $$\frac{\sigma'(z+w)}{\sigma(z+w)}-\frac{\sigma'(z)}{\sigma(z)}=h'(z)$$
        On the other hand, since $\left(\frac{\sigma'(z)}{\sigma(z)}\right)' = -\wp(z)$, then:
        $$h''(z)=\left(\frac{\sigma'}{\sigma}\right)'(z+w)-\left(\frac{\sigma'}{\sigma}\right)'(z) = (-\wp(z+w))-(-\wp(z))=0$$
        Hence, $h(z)$ is in fact a degree $1$ polynomial. So, there exists $a_w,b_w\in\mathbb{C}$, such that:
        $$\frac{\sigma(z+w)}{\sigma(z)}=e^{h(z)}=e^{a_wz+b_w},\quad \sigma(z+w)=e^{a_wz+b_w}\sigma(z)$$
        Then, apply the derivative, and take its quotient with $\sigma(z+w)$, we get:
        $$\sigma'(z+w)=a_we^{a_wz+b_w}\sigma(z)+e^{a_wz+b_w}\sigma'(z)$$
        $$\zeta(z+w)=\frac{\sigma'(z+w)}{\sigma(z+w)}=\frac{a_we^{a_wz+b_w}\sigma(z)+e^{a_wz+b_w}\sigma'(z)}{e^{a_wz+b_w}\sigma(z)} = a_w+\frac{\sigma'(z)}{\sigma(z)}=a_w+\zeta(z)$$
        Which, apply it to the definition of $h(z)$, we get:
        $$h(z+w)=\sum_{j=1}^{n}c_j\zeta(z-b_j+w) = \sum_{j=1}^{n}c_j(a_w+\zeta(z-b_j)) = a_w\sum_{j=1}^{n}c_j+\sum_{j=1}^{n}c_j\zeta(z-b_j) = h(z)$$

        (Note: recall that $\sum_{j=1}^{n}c_j$ is assumed to be $0$).

        Hence, $h(z)$ is an elliptic function.
    \end{itemize}
    he above two implication shows that $h(z)$ is an elliptic function iff $\sum_{j=1}^{n}c_j=0$.
    
    \hfil

    \item[(b)] To construct the desired principal part for each point $b_1,...,b_n$ modulo $L$, we need to consider the order 1 case separately from the other poles:
    
    \hfil
    
    For order $1$, we have the condition that $\sum_{j=1}^{n}a_{1,j}=0$, so we can utilize the statement proven in \textbf{part (a)}. Notice that $\zeta(z)=\frac{\sigma'(z)}{\sigma(z)}$ is the logarithmic derivative of $\sigma(z)$, with the formula given in \textbf{part (a)}, we get the following:
    $$\zeta(z)=\frac{\sigma'(z)}{\sigma(z)}=\frac{1}{z}+\sum_{\substack{w\in L\\w\neq 0}}\left(\frac{-1/w}{1-z/w}+\frac{d}{dz}\left(\frac{z}{w}+\frac{1}{2}\cdot\frac{z^2}{w^2}\right)\right) = \frac{1}{z}+\sum_{\substack{w\in L\\w\neq 0}}\left(\frac{1}{z-w}+\frac{1}{w}+\frac{z}{w^2}\right)$$
    This demonstrates that $\zeta(z)$ has its principal part given as $\frac{1}{z-w}$ at all $w\in L$. Hence, $\zeta(z-b_j)$ would have its principal part given as $\frac{1}{z-b_j}$ for all point equivalent to $b_j\mod\ L$. 
    Which, using the statement in \textbf{part (a)}, we know since $\sum_{j=1}^{n}a_{1,j}=0$, it implies that $\sum_{j=1}^{n}a_{1,j}\zeta(z-b_j)$ is an elliptic function; moreover, since each $b_j$ is distinct, its principal part is governed by only $a_{1,j}\zeta(z-b_j)$ for each index $j$, hence this is an elliptic function describing the principal part up to the simple poles at each point.

    \hfil

    For order $\geq 2$, we could utilize the fact that $\wp(z)$ has a double pole at all $w\in L$. Recall the formula of $\wp(z)$ in series form:
    $$\wp(z) = \frac{1}{z^2}+\sum_{\substack{w\in L\\w\neq 0}}\left(\frac{1}{(z-w)^2}-\frac{1}{w^2}\right)$$
    Which, its principal part is given by $\frac{1}{(z-w)^2}$ at all $w\in L$. So, for any index $j$ with $l_j\geq 2$, to describe the principal part with $\frac{a_{2,j}}{(z-b_j)^2}$ at each point equivalent to $b_j\mod\ L$, we can use $a_{2,j}\wp(z-b_j)$ (shift the double poles to each point in $b_j+L$).
    
    Besides that, for any $n> 0$, since $\wp(z)$ converges normally within $\mathbb{C}\setminus L$, then its $n^{th}$ order derivative can be performed term by term:
    $$\wp^{(n)}(z)=\frac{d^n}{dz^n}\left(\frac{1}{z^2}\right)+\sum_{\substack{w\in L\\w\neq 0}}\frac{d^n}{dz^n}\left(\frac{1}{(z-w)^2}-\frac{1}{w^2}\right) = \frac{(-1)^n\cdot (n+1)!}{z^{n+2}}+\sum_{\substack{w\in L\\w\neq 0}}\frac{(-1)^n\cdot (n+1)!}{(z-w)^{n+2}}$$
    $$\frac{(-1)^n}{(n+1)!}\wp^{(n)}(z) = \frac{1}{z^{n+2}}+\sum_{\substack{w\in L\\w\neq 0}}\frac{1}{(z-w)^{(n+2)}}$$
    This shows that the function $\frac{(-1)^n}{(n+1)!}\wp^{(n)}(z)$ has principal part $\frac{1}{(z-w)^{n+2}}$ at all $w\in L$. So, for all index $j$ with $l_j >2$, any $2<v\leq l_j$ with its principal part given by $\frac{a_{v,j}}{(z-b_j)^{v}}$ at each point equivalent to $b_j\mod\ L$, could be given by $a_{v,j}\cdot\frac{(-1)^{(v-2)}}{(v-1)!}\wp^{(v-2)}(z-b_j)$, based on similar logic as above.

    \hfil

    In general, since the finite sum of elliptic function of the same lattice stays as an elliptic function, to create an elliptic function with the prescribed principal parts described in the question, one explicit formula can be given as:
    $$\sum_{j=1}^{n}a_{1,j}\zeta(z-b_j)+\sum_{j=1}^{n}\sum_{v=2}^{l_j}a_{v,j}\cdot\frac{(-1)^{v-2}}{(v-1)!}\wp^{(v-2)}(z-b_j)$$
    (Note: if $l_j<2$, simply ignore the corresponding index $j$ for the second sum).
\end{itemize}

\break

\section*{2}
\begin{myBox}[]{}
    \begin{question}
        Freitag Chap. V.6 Exercise 7:

        We are interested in alternating $\mathbb{R}$-bilinear maps (forms)
        $$A:\mathbb{C}\times\mathbb{C}\rightarrow\mathbb{R}$$
        Show:
        \begin{itemize}
            \item[(a)] Any such map $A$ is of the form 
            $$A(z,w)=h\textmd{Im}(z\overline{w})$$
            with a uniquely determined real number $h$. We have explicitly $h=A(1,i)$.
            \item[(b)] Let $L\subset \mathbb{C}$ be a lattice. Then $A$ is called a \emph{Riemannian form} with respect to $L$ iff $h$ is positive, and $A$ only takes integral values on $L\times L$. If 
            $$L=\mathbb{Z}w_1+\mathbb{Z}w_2,\quad \textmd{Im}\left(\frac{w_2}{w_1}\right)>0$$
            then the formula 
            $$A(t_1w_1+t_2w_2,s_1w_1+s_2w_2):=\det\begin{pmatrix}t_1&s_1\\t_2&s_2\end{pmatrix}$$ defines a Riemannian form $A$ on $L$.
            \item[(c)] A non-constant analytic function $\Theta:\mathbb{C}\rightarrow\mathbb{C}$ is called a \emph{theta function} for the lattice $L\subset \mathbb{C}$, iff it satisfies an equation of the type 
            $$\Theta(z+w)=e^{a_wz+b_2}\cdot\Theta(z)$$ for all $z\in\mathbb{C}$, and all $w\in L$. Here, $a_w$ and $b_w$ are onstants that may depend on $w$, but not on $z$. 

            Show the existence of a Riemannian form $A$ with respect to $L$, such that 
            $$A(w,\lambda)=\frac{1}{2\pi i}(a_w\lambda - wa_\lambda)$$
            for all $w,\lambda\in L$.
        \end{itemize}
    \end{question}
\end{myBox}

\textbf{Pf:}
\begin{itemize}
    \item[(a)] For any $z,w\in \mathbb{C}$, there exists $a,b,c,d\in\mathbb{R}$, with $z=a+bi$ and $w=c+di$. Then, by the property of a bilinear form, we get:
    $$A(z,w)=A(a+bi,c+di) = A(a,c+di)+A(bi,c+di) = A(a,c)+A(a,di)+A(bi,c)+A(bi,di)$$
    $$ = acA(1,1)+adA(1,i)+bcA(i,1)+bdA(i,i)$$
    Then, because of the property of alternating form, $A(z,w)=-A(w,z)$, which any $u\in\mathbb{C}$ satisfies $A(u,u)=-A(u,u)$, so $A(u,u)=0$. Hence, we can further reduce the equation to the following:
    $$A(z,w)=acA(1,1)+adA(1,i)+bcA(i,1)+bdA(i,i) = adA(1,i)-bcA(1,i) = (ad-bc)A(1,i)$$
    Now, notice that if we take $z\overline{w}$, we get:
    $$z\overline{w} = (a+bi)\overline{(c+di)} = (a+bi)(c-di) = (ac+bd)+(bc-ad)i$$
    Which, $\textmd{Im}(z\overline{w}) = bc-ad$. So in fact, we get the following formula:
    $$A(z,w) = (ad-bc)A(1,i) = -A(1,i)\cdot \textmd{Im}(z\overline{w})$$
    So, let $h = -A(1,i)=A(i,1)$ (which is uniquely determined by the alternating form), we get:
    $$A(z,w)=A(i,1)\cdot\textmd{Im}(z\overline{w})=h\cdot\textmd{Im}(z\overline{w})$$
    To make $h=A(1,i)$, another way is using the formula $A(z,w)=h\cdot \textmd{Im}(\overleftarrow{z}w)$ instead (since $\textmd{Im}(\overleftarrow{z}w)=-\textmd{Im}(z\overleftarrow{w})$).

    \hfil

    \hfil

    \item[(b)] If view $\mathbb{C}$ as an $\mathbb{R}$-vector space, it is a two-dimensional vector space. Which, the basis $w_1,w_2$ of the lattice $L$ is also a basis for $\mathbb{C}$. Then, for all $z,w\in\mathbb{C}$. Then, for all $z,w\in\mathbb{C}$, there exists $t_1,t_2,s_1,s_2\in\mathbb{R}$, such that $z=t_1w_1+t_2w_2$, and $w=s_1w_2+s_2w_2$.
    
    \hfil
    
    First, we'll check that the given form is an alternating bilinear form: 
    
    If consider $A(z,w)$ and $A(w,z)$, we get:
    $$A(z,w)=A(t_1w_1+t_2w_2,s_1w_1+s_2w_2) = \det\begin{pmatrix}t_1&s_1\\t_2&s_2\end{pmatrix}$$ 
    $$=-\det\begin{pmatrix}s_1&t_1\\s_2&t_2\end{pmatrix} = -A(s_1w_2+s_2w_2,t_1w_2+t_2w_2) = -A(w,z)$$
    So, the alternating property is checked. Now, if given $u\in\mathbb{C}$, with $k_1,k_2\in\mathbb{R}$ satisfying $u=k_1w_1+k_2w_2$, then given arbitrary $k,l\in\mathbb{R}$, we get the following:
    $$A(kz+lu,w)=A(k(t_1w_1+t_2w_2)+l(k_1w_1+k_2w_2),s_1w_1+s_2w_2)$$
    $$A((kt_1+lk_1)w_1+(kt_2+lk_2)w_2,s_1w_1+s_2w_2) = \det\begin{pmatrix}
        (kt_1+lk_1) & s_1\\
        (kt_2+lk_2) & s_2
    \end{pmatrix}$$
    $$ = (kt_1+lk_1)s_2 - (kt_2+lk_2)s_1 = k(t_1s_2-t_2s_1)+l(k_1s_2-k_2s_1)$$
    $$=k\det\begin{pmatrix}t_1&s_1\\t_2&s_2\end{pmatrix} + l\det\begin{pmatrix}k_1&s_1\\k_2&s_2\end{pmatrix}$$
    $$ = kA(t_1w_1+t_2w_2,s_1w_1+s_2w_2)+ lA(k_1w_1+k_2w_2,s_1w_1+s_2w_2)$$
    $$= kA(z,w)+lA(u,w)$$
    This proves the bilinearity (including the alternating property, this also proves the linearity of the second column).

    So, $A$ defined in the question is an alternating bilinear form.

    \hfil

    Now, for all $z,w\in L\times L$, since there exists $t_1,t_2,s_1,s_2\in\mathbb{Z}$, with $z=t_1w_1+t_2w_2$ and $w=s_1w_1+s_2w_2$, we get:
    $$A(z,w)=A(t_1w_1+t_2w_2,s_1w_1+s_2w_2)=\det\begin{pmatrix}t_1&s_1\\t_2&s_2\end{pmatrix} = t_1s_2-t_2s_1 \in\mathbb{Z}$$
    So, $A$ yields integer value for all elements in $L\times L$.

    \hfil

    Lastly, consider $h=A(1,i)$ given in \textbf{part (a)}. Given that $w_1=a+bi$, $w_2=c+di$ for some $a,b,c,d\in\mathbb{R}$, and $\textmd{Im}(w_2/w_1)>0$, we get:
    $$\frac{w_2}{w_1}=\frac{c+di}{a+bi}=\frac{(c+di)(a-bi)}{(a+bi)(a-bi)} = \frac{(ac+bd)+(ad-bc)i}{a^2+b^2},\quad \textmd{Im}\left(\frac{w_2}{w_1}\right)=\frac{ad-bc}{a^2+b^2}>0$$
    $$\implies ad-bc > 0$$
    Then, given the definition of $A$, we know the following:
    $$A(w_1,w_2) = \det\begin{pmatrix}1&0\\0&1\end{pmatrix} = 1$$
    $$A(w_1,w_2)=A(a+bi,c+di) = acA(1,1)+ adA(1,i)+bcA(i,1)+bdA(i,i)$$
    $$ = adA(1,i)-bcA(1,i) = (ad-bc)h$$
    Hence, we derived the following:
    $$(ad-bc)h = 1 >0,\quad ad-bc >0\implies h = \frac{1}{ad-bc}>0$$
    Then, since $A$ is an alternating bilinear form, takes integer values on $L\times L$, and has $h>0$, $A$ is a Riemannian Form.

    \hfil

    \hfil

    \item[(c)] Let $L=\mathbb{Z}w_1+\mathbb{Z}w_2$, with $\textmd{Im}(\frac{w_2}{w_1})>0$. 
    
    We'll first investigate the $\Theta$-function: Given the definition of $\Theta$ function, we know for any $z\in \mathbb{C}$, if $\Theta(z)=0$, then for all $w\in L$, $\Theta(z+w)=e^{a_wz+b_w}\Theta(z) = 0$. Hence, let $b_1,...,b_n$ represent the zeros of $\Theta$ in a fundamental region $P$, then for all $z\in\mathbb{C}$, we get $\Theta(z)=0$ iff $z\equiv b_j\mod\ L$ for some $j\in\{1,...,n\}$ (since if $z\in P$ satisfies $z\neq b_j$ for all index $j$, then for all $w\in L$, $\Theta(z+w)=e^{a_wz+b_w}\Theta(z)\neq 0$).
    
    On the other hand, for all $w\in L$, if consider the derivative $\Theta'(z+w)$, we get:
    $$\Theta'(z+w) = a_we^{a_wz+b_w}\Theta(z) + e^{a_wz+b_w}\Theta'(z)$$
    Which, the following is true:
    $$\frac{\Theta'(z+w)}{\Theta(z+w)}=\frac{a_we^{a_wz+b_w}\Theta(z) + e^{a_wz+b_w}\Theta'(z)}{e^{a_wz+b_w}\Theta(z)} = a_w+\frac{\Theta'(z)}{\Theta(z)}$$

    \hfil

    \textbf{1. Relations of $a_w$ with basis $w_1,w_2$:}

    Notice the following relation of $w_1$ and $\Theta$:
    $$\frac{\Theta'(z+w_1)}{\Theta(z+w_1)}=a_{w_1}+\frac{\Theta'(z)}{\Theta(z)}$$
    Which, by induction, any $k\in\mathbb{Z}$ with $k\geq 0$ satisfies:
    $$\frac{\Theta'(z+kw_1)}{\Theta(z+kw_1)}=ka_{w_1}+\frac{\Theta'(z)}{\Theta(z)}$$
    Then, for $k<0$, since $z=(z+kw_1)-kw_1$ with $-k>0$, we get the following relation:
    $$\frac{\Theta'(z)}{\Theta(z)}=\frac{\Theta'((z+kw_1)-kw_1)}{\Theta((z+kw_1)-kw_1)} = (-k)a_{w_1}+\frac{\Theta'(z+kw_1)}{\Theta(z+kw_1)},\quad \frac{\Theta'(z+kw_1)}{\Theta(z+kw_1)} = ka_{w_1}+\frac{\Theta'(z)}{\Theta(z)}$$
    Hence, the above formula can be generalize to any $k\in\mathbb{Z}$. Then, apply similar logic to $w_2$, we also get the following:
    $$\forall l\in\mathbb{Z},\quad \frac{\Theta'(z+lw_2)}{\Theta(z+lw_2)} = la_{w_2}+\frac{\Theta'(z)}{\Theta(z)}$$
    So, for arbitrary $w\in L$, since there exists $k,l\in\mathbb{Z}$, with $w=kw_1+lw_2$, then the following relation is true:
    $$a_w+\frac{\Theta'(z)}{\Theta(z)}=\frac{\Theta'(z+w)}{\Theta(z+w)} = \frac{\Theta'(z+kw_1+lw_2)}{\Theta(z+kw_1+lw_2)}$$
    $$ = la_{w_2}+\frac{\Theta'(z+kw_1)}{\Theta'(z+kw_1)} = ka_{w_1}+la_{w_2}+\frac{\Theta'(z)}{\Theta(z)}$$
    
    $$\implies a_w = ka_{w_1}+la_{w_2}$$

    \hfil

    \textbf{2. Define the Riemannian Form:}

    Since $L$ is a lattice, $w_1$ and $w_2$ are linearly independent when viewing $\mathbb{C}$ as an $\mathbb{R}$-vector space, hence $w_1,w_2$ forms a basis of $\mathbb{C}$. Which, for all $u,v\in\mathbb{C}$, there exists $t_1,t_2,s_1,s_2\in\mathbb{R}$, such that $u=t_1w_1+t_2w_2$ and $v=s_1w_1+s_2w_2$. So, define the map $A:\mathbb{C}\times\mathbb{C}\rightarrow\mathbb{C}$ as follow:
    $$A(u,v) = A(t_1w_1+t_2w_2,s_1w_1+s_2w_2) = \frac{1}{2\pi i}((t_1a_{w_1}+t_2a_{w_2})v - (s_1a_{w_1}+s_2a_{w_2})u)$$
    Notice that the image isn't guaranteed to be in $\mathbb{R}$. Temporarily, we'll postpone the proof of $A(\mathbb{C}\times \mathbb{C})\subseteq \mathbb{R}$, and verify that $A$ satisfies all the other properties of being a Riemannian Form first (except the part that $h>0$, since it requires the fact that image of $A$ is a subset of $\mathbb{R}$).
    \begin{itemize}
        \item \textbf{Alternating Property:}
        
        Given the definition of $A$, we get:
        $$A(v,u)=A(s_1w_1+s_2w_2,t_1w_1+t_2w_2) = \frac{1}{2\pi i}((s_1a_{w_1}+s_2a_{w_2})u-(t_1a_{w_1}+t_2a_{w_2})v)$$
        $$ = -\frac{1}{2\pi i}((t_1a_{w_1}+t_2a_{w_2})v-(s_1a_{w_1}+s_2a_{w_2})u) = -A(u,v)$$
        \item \textbf{Bilinearity:}
        
        Given arbitrary $w \in\mathbb{C}$, there exists $r_1,r_2\in\mathbb{R}$, with $w=r_1w_1+r_2w_2$. Which, for arbitrary $k,l\in\mathbb{R}$ we get:
        $$A(ku+lw,v)=A(k(t_1w_1+t_2w_2)+l(r_1w_1+r_2w_2),s_1w_1+s_2w_2)$$
        $$ = A((kt_1+lr_1)w_1+(kt_2+lr_2)w_2,s_1w_1+s_2w_2)$$
        $$= \frac{1}{2\pi i}(((kt_1+lr_1)a_{w_1}+(kt_2+lr_2)a_{w_2})v-(s_1a_{w_1}+s_2a_{w_2})(ku+lw))$$
        $$ = \frac{1}{2\pi i}\left(k(t_1a_{w_1}+t_2a_{w_2})v - (s_1a_{w_1}+s_2a_{w_2})ku\right)+\frac{1}{2\pi i}\left(l(r_1a_{w_1}+r_2a_{w_2})v - (s_1a_{w_1}+s_2a_{w_2})lw\right)$$
        $$ = kA(u,v)+lA(w,v)$$
        Combining the alternating property, the linearity in the second column is also given, which proves the bilinearity.

        \item \textbf{$A$ yields integer values on $L\times L$:}
        
        Given any $w,\lambda\in L$, there exists $t_1,t_2,s_1,s_2\in \mathbb{Z}$, $w=t_1w_1+t_2w_2$, and $\lambda = s_1w_1+s_2w_2$. Which, with $a_w=t_1a_{w_1}+t_2a_{w_2}$ and $a_{\lambda}=s_1a_{w_1}+a_2a_{w_2}$ proven in statement $\textbf{1}$, we get:
        $$A(w,\lambda) = A(t_1a_{w_1}+t_2a_{w_2},s_1a_{w_1}+s_2a_{w_2})$$
        $$=\frac{1}{2\pi i}((t_1a_{w_1}+t_2a_{w_2})\lambda - (s_1a_{w_1}+s_2a_{w_2})w) = \frac{1}{2\pi i}(a_w\lambda -a_\lambda w)$$
        This is the desired formula for any $w,\lambda\in L$. To prove that the value is an integer, consider the parallelagram $P$ spanned by $w$ and $\lambda$. Which, given $\partial P$, since it's closed and $\Theta$ has discrete zeros, then there existts $a\in\mathbb{C}$, such that $P'=a+P$ with $\partial P'$ containing no zeros of $\Theta$ (hence $\frac{\Theta'}{\Theta}$ is well-defined on $\partial P'$). 
        
        \begin{figure}[h!]
            \begin{center}
                \includegraphics*[width=60mm]{q2 orient 1.jpg}
                \caption{Example Orientation}
            \end{center}
        \end{figure}
        
        Which, WLOG, assume the orientation of $\partial P'$ is given by $a\rightarrow (a+w)\rightarrow (a+w+\lambda)\rightarrow (a+\lambda)\rightarrow a$, along with argument principal, we get the following:
        $$\pm\textmd{Number of zeros of } \Theta \textmd{ in } P' = \frac{1}{2\pi i}\int_{\partial P'}\frac{\Theta'(z)}{\Theta(z)}dz$$
        $$ = \frac{1}{2\pi i}\left(\int_{a}^{a+w}\frac{\Theta'(z)}{\Theta(z)}dz+\int_{a+w}^{a+w+\lambda}\frac{\Theta'(z)}{\Theta(z)}dz+\int_{a+w+\lambda}^{a+\lambda}\frac{\Theta'(z)}{\Theta(z)}dz+\int_{a+\lambda}^{a}\frac{\Theta'(z)}{\Theta(z)}dz\right)$$
        $$=\frac{1}{2\pi i}\left(\left(\int_{a}^{a+w}\frac{\Theta'(z)}{\Theta(z)}dz - \int_{a}^{a+w}\frac{\Theta'(z+\lambda)}{\Theta(z+\lambda)}dz\right)+\left(\int_{a}^{a+\lambda}\frac{\Theta'(z+w)}{\Theta(z+w)}dz - \int_{a}^{a+\lambda}\frac{\Theta'(z)}{\Theta(z)}dz\right)\right)$$
        $$=\frac{1}{2\pi i}\left(\int_{a}^{a+w}\left(\frac{\Theta'(z)}{\Theta(z)}-\left(a_\lambda+\frac{\Theta'(z)}{\Theta(z)}\right)\right)dz + \int_{a}^{a+\lambda}\left(a_w+\frac{\Theta'(z)}{\Theta(z)}-\frac{\Theta'(z)}{\Theta(z)}\right)dz\right)$$
        $$ = \frac{1}{2\pi i}\left(\int_{a}^{a+\lambda}a_wdz - \int_{a}^{a+w}a_\lambda dz\right) = \frac{1}{2\pi i}(a_w\lambda - a_\lambda w) = A(w,\lambda)$$
        This shows that $A(w,\lambda)$ is in fact an integer (the sign depends on the orientation of $P'$ described above).
    \end{itemize}

    Second to last, to prove the image is contained in $\mathbb{R}$, we'll utilize the continuity of $A$ (since fixing one entry, $A$ becomes a linear map that is continuous). 

    First, we'll prove that any $w,\lambda\in(\mathbb{Q}w_1+\mathbb{Q}w_2)$ satisfies $A(w,\lambda) \in\mathbb{R}$: Since both $w,\lambda$ have the coefficients of $w_1,w_2$ being rational, then for large enough $k,l\in\mathbb{N}$, $kw,l\lambda\in L$ (EX: choose $k,l$ to be the multiples of the denominators of the ratioal coefficients of $w,\lambda$ respectively, then each coefficient of $kw,l\lambda$ is an integer).
    So, evaluate in $A$, we get:
    $$A(kw,l\lambda)\in\mathbb{Z},\quad A(w,\lambda) = \frac{1}{kl}A(kw,l\lambda) \in\mathbb{R}$$
    
    Now, let $L' = \mathbb{Q}w_1+\mathbb{Q}w_2$, since $L'$ is a dense set in $\mathbb{C}$ (due to the denseness of $\mathbb{Q}$ in $\mathbb{R}$), then $L'\times L'$ is a dense set in $\mathbb{C}\times \mathbb{C}$. So, for any $(u,v)\in\mathbb{C}\times \mathbb{C}$, it is a limit point of $L'\times L'$, hence there exists a sequence $(u_n,v_n)_{n\in\mathbb{N}}\subset L'\times L'$, with $\lim_{n\rightarrow\infty}(u_n,v_n) = (u,v)$. Therefore, by continuity of $A$, we get:
    $$\lim_{n\rightarrow\infty}A(u_n,v_n) = A(u,v)$$
    And, since each index $n$ satisfies $A(u_n,v_n)\in \mathbb{R}$ (recall that $(u_n,v_n)\in L'\times L'$), then by completeness of $\mathbb{R}$, $A(u,v)$ as a limit of sequence in $\mathbb{R}$, must also belong to $\mathbb{R}$. So, $A(u,v)\in\mathbb{R}$.

    This proves that $A$ has an image in $\mathbb{R}$, hence it's in fact an alternating bilinear form $A:\mathbb{C}\times\mathbb{C}\rightarrow\mathbb{R}$.

    \hfil

    The last task is to verify that $h$ corresponding to $A$ is in fact positive. Recall that $h=A(1,i)$, with the assumption that $w_1=a+bi$, $w_2=c+di$, and $\textmd{Im}(w_2/w_1)>0$, we know $ad-bc > 0$. Hence, we get the following:
    $$A(w_1,w_2) =A(a+bi,c+di) = acA(1,1)+adA(1,i)+bcA(i,1)+bdA(i,i) = (ad-bc)A(1,i) = (ad-bc)h$$
    Which, given the assumption that $\textmd{Im}(w_2/w_1)>0$, the orientation of the parallelagram is given as follow:

    \begin{figure}[h!]
        \begin{center}
            \includegraphics*[width=60mm]{q2 orient 2.jpg}
            \caption{Orientation of the Parallelagram (Counter Clockwise Contour)}
        \end{center}
    \end{figure}

    Hence, $A(w_1,w_2)$ when representing as the integral form:
    $$A(w_1,w_2)=\frac{1}{2\pi i}\int_{\partial P}\frac{\Theta'(z)}{\Theta(z)}dz = \textmd{Number of zeros of }\Theta \textmd{ in } P \geq 0$$
    It is nonnegative (since the integration is along a counterclockwise contour like above). So, given $(ad-bc)h = A(w_1,w_2)\geq 0$, then $h \geq 0$.

    In terms for $h$ to be positive, we need $A(w_1,w_2)>0$, hence an extra condition imposed is that $\Theta$ needs to have at least a zero (so there is a zero within the fundamental region of lattice $L$, causing the integral form of $A(w_1,w_2)$ to be nonzero).

    \hfil

    Compiling all the information, the defined $A:\mathbb{C}\times \mathbb{C}\rightarrow\mathbb{R}$ is in fact an alternating bilinear form, that yields integer values on $L\times L$ (with formula $A(w,\lambda)=\frac{1}{2\pi i}(a_w\lambda-a_\lambda w)$), and also $h>0$ (under a nontrivial case), hence it is a Riemannian Form. This proves the existence of Riemannian Form satisfying the given formula on $L\times L$.
\end{itemize}

\break

\section*{3}
\begin{myBox}[]{}
    \begin{question}
        Freitag Chap. V.7 Exercise 5:

        Show:
        \begin{itemize}
            \item[(a)] For the lattice $L_i=\mathbb{Z}+\mathbb{Z}i$ we have $g_3(i)=0$ and $g_2(i)\in\mathbb{R}^\times$, in particular $\Delta(i)=g_2^3(i)>0$.
            \item[(b)] For the lattice $L_w=\mathbb{Z}+\mathbb{Z}w$, $w:=e^{2\pi i/3}$, we have $g_2(w)=0$ and $g_3(w)\in\mathbb{R}^\times$, in particular $\Delta(w)=-27g_3^2(w)$. 
        \end{itemize}
    \end{question}
\end{myBox}

\textbf{Pf:}

Given any lattice $L=\mathbb{Z}+\mathbb{Z}\mathcal{T}$ with $\mathcal{T}\in\mathbb{H}$, recall that $g_2(\mathcal{T}) = 60G_4(\mathcal{T})$ and $g_3(\mathcal{T})=140G_6(\mathcal{T})$, and $\Delta(\mathcal{T}) \neq 0$ (since in case to have lattice, the half lattices yield distinct values $e_1,e_2,e_3$, which are roots of the polynomial $4w^3-g_2w-g_3$, when talking about the algebraic differential equation of $\wp$-function). Hence as $\Delta(\mathcal{T})=g_2^3(\mathcal{T})-27g_3^2(\mathcal{T})$ denotes the discriminant of the cubic polynomial above, having distinct roots implies $\Delta(\mathcal{T})\neq 0$.

To prove that $g_2,g_3$ yield real values for specific lattices, it suffices to prove the case for $G_4$ and $G_6$.
\begin{itemize}
    \item[(a)] Given lattice $L_i = \mathbb{Z}+\mathbb{Z}i$, there are two properties:
    \begin{itemize}
        \item For all $w=a+bi\in L_i$, its conjugate $\overline{w}=a-bi\in L_i$ (since both $a,b\in\mathbb{Z}$).
        \item Given same $w$, $iw = -b+ai \in L_i$ based on the same reason above.
    \end{itemize}
    Notice that the above pairing is a one-to-one correspondance. Hence, for $k\geq 3$, we get the following formula for $2G_k(i)$:
    $$2G_k(i) = \sum_{\substack{w\in L_i\\w\neq 0}}\frac{1}{w^k}+\sum_{\substack{w\in L_i\\w\neq 0}}\frac{1}{\overline{w}^k} = \sum_{\substack{w\in L_i\\w\neq 0}}2\textmd{Re}\left(\frac{1}{w^k}\right)\in \mathbb{R}$$
    (Note: since for $k\geq 3$, $G_k(z)$ converges normally within $\mathbb{H}$, hence changing the order of summation doesn't matter).

    So, both $G_4(i)$ and $G_6(i)$ are real, implying that $g_2(i)=60G_4(i),\ g_3(i)=140G_6(i)$ are also real.

    Then, consider the similar formulation with $iw$ instead, we get:
    $$2G_6(i) = \sum_{\substack{w\in L_i\\w\neq 0}}\frac{1}{w^6}+\sum_{\substack{w\in L_i\\w\neq 0}}\frac{1}{(iw)^6} = \sum_{\substack{w\in L\\w\neq 0}}\left(\frac{1}{w^6}-\frac{1}{w^6}\right) = 0$$
    This implies that $G_6(i) = 0$, which $g_3(i) = 140G_6(i)=0$.

    Finally, given the discriminant formula, since the following is true:
    $$\Delta(i)=g_2^3(i)-27g_3^2(i) = g_2^3(i)$$
    On the other hand, $\Delta(i)\neq 0$ (since $L_i$ is a lattice), which the above equation implies that $g_2(i)\neq 0$.
    Also, recall that from \textbf{HW 5 Problem 5}, we've proven that given the lattice formed by $\mathbb{Z}+\mathbb{Z}ti$ (where $t\in\mathbb{R}$), $\wp(z)$ yields real values on all the half lines $x=\frac{n}{2}$ and $y=\frac{tm}{2}$ ($m,n\in\mathbb{Z}$); specifically, on all the half points (the intersection of half lines), the value of $\wp$ (denoted as $e_1,e_2,e_3$) are all real. Since these three values are the roots of the cubic polynomial $4w^3-g_2(i)w-g_3(i)$, then because the cubic polynomial have three real distinct roots, its discriminant $\Delta(i) >0$, showing that $g_2^3(i)>0$, or $g_2(i)>0$.

    Hence, for lattice $L_i$, we have $g_2(i)>0$, and $g_3(i)=0$ (so, $g_2(i)\in\mathbb{R}^\times$).

    \hfil
    
    \item[(b)] Given $L_w=\mathbb{Z}+\mathbb{Z}w$ with $w=e^{2\pi i/3} = -\frac{1}{2}+\frac{\sqrt{3}}{2}i$ (which $w^3 = 1$), here are several of its properties:
    \begin{itemize}
        \item Since $e^{2\pi i/6} = \frac{1}{2}+\frac{\sqrt{3}}{2}i = 1+(-\frac{1}{2}+\frac{\sqrt{3}}{2}i)= 1+w\in L_w$, while $-\overline{w} = -(-\frac{1}{2}-\frac{\sqrt{3}}{2}i) = \frac{1}{2}+\frac{\sqrt{3}}{2}i = e^{2\pi i/6}$, so $-\overline{w}\in L$. Then, for all $\lambda = a+bw\in L_w$ (for some $a,b\in\mathbb{Z}$), we get:
        $$-\overline{\lambda} = -(a+b\overline{w}) = -a+b(-\overline{w})\in L_w\implies\overline{\lambda}\in L_w$$
        \item Since $e^{2\pi i/6}\in L_w$, then $-e^{2\pi i/6} = e^{2\pi i\cdot2/3} = w^2\in L_w$. Hence, given the same $\lambda$ above, we also get:
        $$w\lambda = w(a+bw) = aw + bw^2 \in L_w$$
        $$w^2\lambda = w^2(a+bw) = aw^2+bw^3 = b+aw^2\in L_2$$
    \end{itemize}
    Notice that all pairings above are one-to-one correspondance. Hence, for all $k\geq 3$, we get the following formula for $2G_k(w)$:
    $$2G_k(w)=\sum_{\substack{\lambda\in L_w\\\lambda\neq 0}}\frac{1}{\lambda^k}+\sum_{\substack{\lambda\in L_w\\\lambda\neq 0}}\frac{1}{\overline{\lambda}^k} = \sum_{\substack{\lambda\in L_w\\\lambda\neq 0}}2\textmd{Re}\left(\frac{1}{\lambda^k}\right)\in \mathbb{R}$$ 
    Hence, $G_k(w)\in\mathbb{R}$ for all $k\geq 3$. In particular, $G_4(w),G_6(w)\in\mathbb{R}$, implying that $g_2(w)=60G_4(w)$ and $g_3(w)=140G_6(w)$ are also real.
    
    Now, if we consider $3G_4(w)$ specifically, based on the second property of $L_w$ listed above (where $\lambda\in L_w$ implies $w\lambda$ and $w^2\lambda$ are in $L_w$), we get:
    $$3G_4(w) = \sum_{\substack{\lambda\in L_w\\w\neq 0}}\frac{1}{\lambda^4}+\sum_{\substack{\lambda\in L_w\\w\neq 0}}\frac{1}{(w\lambda)^4}+\sum_{\substack{\lambda\in L_w\\w\neq 0}}\frac{1}{(w^2\lambda)^4} = \sum_{\substack{\lambda\in L_w\\\lambda\neq 0}}\left(\frac{1}{\lambda^4}+\frac{1}{w^4\lambda^4}+\frac{1}{w^8\lambda^4}\right)$$
    $$\sum_{\substack{\lambda\in L_w\\\lambda\neq 0}}\left(\frac{1}{\lambda^4}+\frac{w^2}{w^6\lambda^4}+\frac{w}{w^9\lambda^4}\right)=\sum_{\substack{\lambda\in L_w\\\lambda\neq 0}}\left(\frac{1}{\lambda^4}+\frac{w^2}{\lambda^4}+\frac{w}{\lambda^4}\right) = (1+w+w^2)\sum_{\substack{\lambda\in L_w\\\lambda\neq 0}}\frac{1}{\lambda^4}$$
    Since $w=e^{2\pi i/3}$ is a primitive $3^{rd}$ root of unity, then $1+w+w^2=0$, showing that $3G_4(w)=0$ based on the above equation. Hence, $G_4(w)=0$, so $g_2(w)=60G_4(w)=0$ also.

    Finally, since $\Delta(w)=g_2^3(w)-27g_3^2(w) = -27g_3^2(w)$, and $\Delta(w)\neq 0$, then $g_3^2(w)\neq 0$.

    So, in conclusion, we have $g_2(w)=0$, $g_3(w)\neq 0$, and $g_3(w)\in\mathbb{R}$, so $g_3(w)\in\mathbb{R}^\times$.
\end{itemize}

\break

\section*{4}
\begin{myBox}[]{}
    \begin{question}
        Freitag Chap. V.8 Exercise 3:

        The Eisenstein series are "real" functions, i.e. $\overline{G_k(\mathcal{T})}=G_k(-\overline{\mathcal{T}})$. This implies 
        $$G_k\left(\frac{\alpha(-\overline{\mathcal{T}})+\beta}{\gamma(-\mathcal{T})+\delta}\right) = (\gamma(-\overline{\mathcal{T}})+\delta)^k\overline{G_k(\mathcal{T})} \quad \quad \textmd{and}$$
        $$j\left(\frac{\alpha(-\overline{\mathcal{T}})+\beta}{\gamma(-\overline{\mathcal{T}})+\delta}\right)=\overline{j(\mathcal{T})}\quad\quad \textmd{for } \begin{pmatrix}\alpha&\beta\\\gamma&\delta\end{pmatrix}\in \Gamma$$
        On the vertical half-lines $\textmd{Re}(\mathcal{T})=\pm\frac{1}{2}$ in $\mathbb{H}$ in $\mathbb{H}$ the Eisenstein series and the $j$-function are real. if $\mathcal{T}\in\mathbb{H}$ lies on the circle line $|\mathcal{T}|=1$, then $j(\mathcal{T})=\overline{j(\mathcal{T})}$. In particular, the $j$-function is real on the boundary of the modular figure, and on the imaginary axis.
    \end{question}
\end{myBox}

\textbf{Pf:}

For all $\mathcal{T}\in\mathbb{H}$ (and $k\geq 3$), since $G_k(\mathcal{T})$ is a series of functions that converges normally within $\mathbb{H}$, then the following is true:
$$\overline{G_k(\mathcal{T})}=\overline{\sum_{\substack{(a,b)\in\mathbb{Z}^2\\(a,b)\neq (0,0)}}\frac{1}{(a+b\mathcal{T})^k}} = \sum_{\substack{(a,b)\in\mathbb{Z}^2\\(a,b)\neq (0,0)}}\frac{1}{\overline{(a-b\mathcal{T})}^k} = \sum_{\substack{(a,b)\in\mathbb{Z}^2\\(a,b)\neq (0,0)}}\frac{1}{(a+b(-\overline{\mathcal{T}}))^k} = G_k(-\overline{\mathcal{T}})$$
This verifies the first property in the problem. 

Then, based on the relations given as follow:
$$\forall \begin{pmatrix}\alpha&\beta\\\gamma&\delta\end{pmatrix}\in \Gamma,\quad G_k\left(\frac{\alpha \mathcal{T}+\beta}{\gamma\mathcal{T}+\delta}\right) = (\gamma\mathcal{T}+\delta)^kG_k(\mathcal{T})$$
$$g_2(\mathcal{T})=60G_4(\mathcal{T}),\quad g_3(\mathcal{T})=140G_6(\mathcal{T}),\quad j(\mathcal{T})=\frac{g_2^3(\mathcal{T})}{g_2^3(\mathcal{T})-27g_3^2(\mathcal{T})}$$
We can conclude the following:
$$G_k\left(\frac{\alpha(-\overline{\mathcal{T}})+\beta}{\gamma(-\mathcal{T})+\delta}\right) = (\gamma(-\overline{\mathcal{T}})+\delta)^kG_k(-\overline{\mathcal{T}})=(\gamma(-\overline{\mathcal{T}})+\delta)^k\overline{G_k(\mathcal{T})}$$

$$j\left(\frac{\alpha(-\overline{\mathcal{T}})+\beta}{\gamma(-\mathcal{T})+\delta}\right) = \frac{\left(g_2\left(\frac{\alpha(-\overline{\mathcal{T}})+\beta}{\gamma(-\mathcal{T})+\delta}\right)\right)^3}{\left(g_2\left(\frac{\alpha(-\overline{\mathcal{T}})+\beta}{\gamma(-\mathcal{T})+\delta}\right)\right)^3-27\left(g_3\left(\frac{\alpha(-\overline{\mathcal{T}})+\beta}{\gamma(-\mathcal{T})+\delta}\right)\right)^2}$$
$$=\frac{\left(60G_4\left(\frac{\alpha(-\overline{\mathcal{T}})+\beta}{\gamma(-\mathcal{T})+\delta}\right)\right)^3}{\left(60G_4\left(\frac{\alpha(-\overline{\mathcal{T}})+\beta}{\gamma(-\mathcal{T})+\delta}\right)\right)^3-27\left(140G_6\left(\frac{\alpha(-\overline{\mathcal{T}})+\beta}{\gamma(-\mathcal{T})+\delta}\right)\right)^2}$$
$$=\frac{(60(\gamma(-\overline{\mathcal{T}})+\delta)^4\overline{G_4(\mathcal{T})})^3}{(60(\gamma(-\overline{\mathcal{T}})+\delta)^4\overline{G_4(\mathcal{T})})^3-27(140(\gamma(-\overline{\mathcal{T}})+\delta)^6\overline{G_6(\mathcal{T})})^2}$$
$$=\frac{(\gamma(-\overline{\mathcal{T}})+\delta)^{12}\overline{(60G_4(\mathcal{T}))}^3}{(\gamma(-\overline{\mathcal{T}})+\delta)^{12}\overline{(60G_4(\mathcal{T}))}^3-27\cdot (\gamma(-\overline{\mathcal{T}})+\delta)^{12}\overline{(140G_6(\mathcal{T}))}^2}$$
$$=\frac{\overline{g_2(\mathcal{T})}^3}{\overline{g_2(\mathcal{T})}^3-27\overline{g_3(\mathcal{T})}^2} = \overline{\frac{g_2^3(\mathcal{T})}{g_2^3(\mathcal{T})-27g_3^2(\mathcal{T})}} = \overline{j(\mathcal{T})}$$
So, the second and third properties are verified.

\hfil

Now, given $\mathcal{T}\in\mathbb{H}$ with $\textmd{Re}(\mathcal{T})=\pm\frac{1}{2}$ (WLOG, assume it is $\frac{1}{2}$, since if $\mathcal{T}=-\frac{1}{2}+yi$ for some $y>0$, then $\mathcal{T}+1 = \frac{1}{2}+yi = -\overline{\mathcal{T}}$, which since $-\overline{\mathcal{T}}=\frac{\mathcal{T}+1}{0\cdot\mathcal{T}+1}$, the two values are equivalent. Hence, swap $\mathcal{T}$ and $-\overline{\mathcal{T}}$, we still get the same case).

Then, since $\mathcal{T}=\frac{1}{2}+yi$ for some $y>0$, $-\overline{\mathcal{T}} = -\frac{1}{2}+yi$, so $-\overline{\mathcal{T}}+1 = \mathcal{T}$. Using the previous properties, we get:
$$G_k(\mathcal{T})=G_k\left(\frac{(-\overline{\mathcal{T}})+1}{0\cdot(-\overline{\mathcal{T}})+1}\right) = (0\cdot(-\overline{\mathcal{T}})+1)^k\overline{G_k(\mathcal{T})} = \overline{G_k(\mathcal{T})}$$
$$G_k(\mathcal{T})=\overline{G_k(\mathcal{T})}\implies G_k(\mathcal{T})\in \mathbb{R}$$

$$j(\mathcal{T})=j\left(\frac{(-\overline{\mathcal{T}})+1}{0\cdot(-\overline{\mathcal{T}})+1}\right) = \overline{j(\mathcal{T})}$$
$$j(\mathcal{T})=\overline{j(\mathcal{T})}\implies j(\mathcal{T})\in\mathbb{R}$$
This proves that both Eisenstein series and the $j$-function are real on $\textmd{Re}(\mathcal{T})=\pm\frac{1}{2}$.

Finally, given $\mathcal{T}\in\mathbb{H}$ with $|\mathcal{T}|=1$, then since $\mathcal{T}=\frac{1}{\overline{\mathcal{T}}}=\frac{-1}{(-\overline{\mathcal{T}})}$, we get the following:
$$j(\mathcal{T})=j\left(\frac{0\cdot(-\overline{\mathcal{T}})-1}{(-\overline{\mathcal{T}})+0}\right) = \overline{j(\mathcal{T})} \implies j(\mathcal{T})\in\mathbb{R}$$
So, on the half circle $|\mathcal{T}|=1$ in $\mathbb{H}$, $j$-function is also real.

\end{document}