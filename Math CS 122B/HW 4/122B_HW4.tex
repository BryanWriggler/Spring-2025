\documentclass{article}
\usepackage{graphicx} % Required for inserting images
\usepackage[margin = 2.54cm]{geometry}
\usepackage[most]{tcolorbox}

\newtcolorbox{myBox}[3]{
arc=5mm,
lower separated=false,
fonttitle=\bfseries,
%colbacktitle=green!10,
%coltitle=green!50!black,
enhanced,
attach boxed title to top left={xshift=0.5cm,
        yshift=-2mm},
colframe=blue!50!black,
colback=blue!10
}

\usepackage{amsmath}
\usepackage{amssymb}
\usepackage{verbatim}
\usepackage[utf8]{inputenc}
\linespread{1.2}

\newtheorem{definition}{Definition}
\newtheorem{proposition}{Proposition}
\newtheorem{theorem}{Theorem}
\newtheorem{question}{Question}

\title{Math CS 122B HW4}
\author{Zih-Yu Hsieh}

\begin{document}
\maketitle

\section*{1}
\begin{myBox}[]{}
    \begin{question}
        Freitag Chap. V.1 Exercise 10:

        Let $f$ be an entire function, and let $L$ be a lattice in $\mathbb{C}$. For any lattice point $w\in\mathbb{L}$ let there exists a number $C_w\in\mathbb{C}$ with the property 
        $$f(z+w)=C_wf(z)$$
        Then
        $$f(z)=Ce^{az}$$
        for suitable constantx $C$ and $a$.
    \end{question}
\end{myBox}

\textbf{Pf:}

We'll consider the meromorphic function $f'/f:\mathbb{C}\rightarrow\overline{\mathbb{C}}$: For all $z\in\mathbb{C}$ and $w\in L$, since $f(z+w)=C_wf(z)$, then $f'(z+w)=C_wf'(z)$. Then, $f'/f$ satisfies:
$$\frac{f'(z+w)}{f(z+w)}=\frac{C_wf'(z)}{C_wf(z)}=\frac{f'(z)}{f(z)}$$
This shows that $f'/f$ is in fact an elliptic function with respect to the given lattice $L$.

\hfil

Now, we'll consider the singularities of $f'/f$: Since $f$ is entire, then $f'$ is also entire, hence the only singularities possible for $f'/f$, are the zeros of $f$.

Since the singularities of $f'/f$ must be discrete, then we can choose a fundamental region $P$ of lattice $L$, such that its boundary $\partial P$ contains no singularities of $f'/f$. Then, by argument principle, we get:
$$\frac{1}{2\pi i}\int_{\partial P}\frac{f'(z)}{f(z)}dz = (\textmd{Number of zeros of $f$ in $P$})-(\textmd{Number of poles of $f$ in $P$})$$
Also, since $f'/f$ is an elliptic function, then we also get the following:
$$\int_{\partial P}\frac{f'(z)}{f(z)}dz = 0$$
So, this implies that the number of zeros of $f$ in $P$, is precisely the same as the number of poles of $f$ in $P$.
Because $f$ is entire, there are no poles in $\mathbb{C}$, hence number of poles in $P$ is $0$; this implies that the number of zeros of $f$ in $P$ is also $0$, showing that $f'/f$ is in fact entire in $P$, which further extends to be entire in $\mathbb{C}$ (since $f'/f$ is an elliptic function).

Hence, by the \textbf{First Liouville's Theorem}, $f'/f$ is in fact a constant.

Lastly, because $f'/f = a\in\mathbb{C}$, then $f'(z)=af(z)$, showing that $f(z)=Ce^{az}$.

\break

\section*{2}
\begin{myBox}[]{}
    \begin{question}
        Freitag Chap. V.2 Exercise 1:

        If $L\subset \mathbb{C}$ is a lattice, then the formula
        $$\sum_{w\in L}\frac{1}{(z-w)^n}$$
        defines for any $n\geq 3$ an elliptic function of order $n$. What is the connection with the Weierstrass $\wp$-function?
    \end{question}
\end{myBox}

\textbf{Pf:}

Recall that the Weierstrass $\wp$-function iwth lattice $L$ is given as:
$$\wp:\mathbb{C}\setminus L\rightarrow\mathbb{C},\quad \wp(z)=\frac{1}{z^2}+\sum_{\substack{w\in L \\ w\neq 0}}\left(\frac{1}{(z-w)^2}-\frac{1}{w^2}\right)$$
Which, the above series converges normally in $\mathbb{C}\setminus L$, hence the derivative to any order in fact can be performed termwise.

For any integer $n\geq 3$, $n-2\geq 1$, then the $(n-2)^{th}$ derivative is given as:
$$\frac{d^{(n-2)}}{dz^{(n-2)}}\wp(z)=\frac{d^{(n-2)}}{dz^{(n-2)}}\frac{1}{z^2}+\sum_{\substack{w\in L \\ w\neq 0}}\frac{d^{(n-2)}}{dz^{(n-2)}}\frac{d^{(n-2)}}{dz^{(n-2)}}\left(\frac{1}{(z-w)^2}-\frac{1}{w^2}\right)$$
$$ = \frac{(-1)^n\cdot (n-1)!}{z^n}+\sum_{\substack{w\in L \\ w\neq 0}}\frac{(-1)^n\cdot(n-1)!}{(z-w)^n} = (-1)^n\cdot(n-1)!\sum_{w\in L}\frac{1}{(z-w)^n}$$
Hence, for $n\geq 3$, the series $\sum_{\substack{w\in L}}\frac{1}{(z-w)^n}$ is in fact some multiple of the $(n-2)^{th}$ derivative of Weierstrass $\wp$-function.

\break

\section*{3}
\begin{myBox}[]{}
    \begin{question}
        Freitag Chap. V.2 Exercise 5:

        Let $L\subset \mathbb{C}$ be a latice. we denote by $\widehat{L}$ the set of all conformal maps $\mathbb{C}\rightarrow\mathbb{C}$ of the form 
        $$z\mapsto \pm z+w,\quad w\in L$$
        We identify (similar to the construction of the torus $\mathbb{C}/L$) two points in $\mathbb{C}$,
        iff they can be mapped into each other by suitable substitutions of $\widehat{L}$. After identiciation, we obtain $\mathbb{C}/\widehat{L}$, first as a set.
        Show that the $\wp$-function gives a bijection
        $$\mathbb{C}/\widehat{L}\rightarrow\overline{\mathbb{C}}$$
        The field of all $\widehat{L}$-invariant meromorphic functions is generated by $\wp$.
    \end{question}
\end{myBox}

\textbf{Pf:}

First, consider the surjectivity: Since $\wp:\mathbb{C}\rightarrow\overline{\mathbb{C}}$ is a nonconstant elliptic function with $L$ being the lattice, 
then it is in fact surjective:

For all $b\in\overline{\mathbb{C}}$, if $b=\infty$, we know $\wp$ satisfies $\wp(z)=\infty$ for all $z\in L$.

On the other hand, if $b\in\mathbb{C}$, consider the function $\wp(z)-b$, which is again an elliptic function with poles at all points of $L$.
Then, since its derivative is again given by $\wp'(z)$, consider the elliptic function $\frac{\wp'(z)}{\wp(z)-b}$, with a suitable fundamental region $P$ such that $\partial P$
contains no singularities. Integrate along the boundary $\partial P$, we get:
$$\frac{1}{2\pi i}\int_{\partial P}\frac{\wp'(z)}{\wp(z)-b}dz = 0$$
And, by argument principle, the above integral provides $(\textmd{Number of zeros of }(\wp-b)\textmd{ in }P)-(\textmd{Number of poles of }(\wp-b)\textmd{ in }P)$. Which, the integral is $0$
implies that the number of zeros and the number of poles in $P$ for $\wp-b$ must be the same.
 
Since in given fundamental region $P$, there exists precisely one double pole (the point $w\in L$ that's also contained in $P$),
hence this forces $\wp-b$ to have two zeros (including multiplicity) within the region $P$.
So, there exists $z\in P$, with  $\wp(z)-b=0$, or $\wp(z)=b$. This proves surjectivity of $\wp(z)$.

\hfil

Now, to prove injectivity of $\wp:\mathbb{C}/\widehat{L}\rightarrow\overline{\mathbb{C}}$, recall that $z_1,z_2\in\mathbb{C}$ satisfies $\wp(z_1)=\wp(z_2)$ iff $z_1\equiv z_2\mod\ L$ or $z_1\equiv -z_2\mod\ L$.
Hence, $z_1 = z_2+w$ or $z_1=-z_2+w$ for some $w\in L$, which, $z_1$ and $z_2$ have the same representation under $\mathbb{C}/\widehat{L}$.
This finishes the injectivity of $\wp$ when domain is given by $\mathbb{C}\setminus \widehat{L}$.

\hfil

As conclusion, $\wp:\mathbb{C}/\widehat{L}\rightarrow\overline{\mathbb{C}}$ is in fact a bijection.


\end{document}