\documentclass{article}
\usepackage{graphicx} % Required for inserting images
\usepackage[margin = 2.54cm]{geometry}
\usepackage[most]{tcolorbox}

\newtcolorbox{myBox}[3]{
arc=5mm,
lower separated=false,
fonttitle=\bfseries,
%colbacktitle=green!10,
%coltitle=green!50!black,
enhanced,
attach boxed title to top left={xshift=0.5cm,
        yshift=-2mm},
colframe=blue!50!black,
colback=blue!10
}

\usepackage{amsmath}
\usepackage{amssymb}
\usepackage{verbatim}
\usepackage[utf8]{inputenc}
\linespread{1.2}

\newtheorem{definition}{Definition}
\newtheorem{proposition}{Proposition}
\newtheorem{theorem}{Theorem}
\newtheorem{question}{Question}

\title{Math 118C HW2}
\author{Zih-Yu Hsieh}

\begin{document}
\maketitle

\section*{1}
\begin{myBox}[]{}
    \begin{question}
        Rudin Pg. 239 Problem 9:

        If $f$ is a differentiable mapping of a \emph{connected} open set $E\subset \mathbb{R}^n$ into $\mathbb{R}^m$, and if $f'(x)=0$ for every $x\in E$, prove that $f$ is constant in $E$.
    \end{question}
\end{myBox}

\textbf{Pf:}

To prove that $f(x)=(f_1(x),...,f_m(x))$ for $f_1,...,f_m:E\rightarrow\mathbb{R}$ is constant, is suffices to show that each individual $f_i$ is constant in $E$.

Since $f$ is differentiable, each $f_i$ is also differentiable, hence $Df_i$ exists for all $x\in E$; also, since $f'(x)=0$, this implies that each $Df_i(x)=0$ for all $x\in E$.

Now, for all index $i\in\{1,...,m\}$, the function $f_i:E\rightarrow \mathbb{R}$ satisfies:

\hfil

\textbf{1. $f_i$ is constant within a Neighborhood:}

Consider any $x\in E$, since $E$ is open, then there exists $r>0$, such that the open ball $B_r(x)\subseteq E$.
Since all $x\in E$ satisfies $Df_i(x)=0$, then $\|Df_i(x)\|\leq 0$ for all $x\in E$, in particular, it also applies to $B_r(x)$.

Now, since $B_r(x)$ is convex, while the differential $Df_i$ is uniformly bounded by $0$ in $B_r(x)$, then for all $y\in B_r(x)$, the following inequality is true:
$$0\leq |f_i(y)-f_i(x)| \leq 0\cdot |y-x| = 0$$
THis enforces $f_i(y)=f_i(x)$, hence $f_i(x)$ is a constant function when restricting to $B_r(x)$.

\hfil

\textbf{2. $f_i$ is constant in $E$:}

Now, fix any $x\in E$, and define $U_x, V_x\subseteq E$ as follow:
$$U_x=\{y\in E\ |\ f_i(y)=f_i(x)\},\quad V_x=\{z\in E\ |\ f_i(z)\neq f_i(x)\}$$
Since for all point $y\in E$, either $f_i(y)=f_i(x)$ or $f_i(y)\neq f_i(x)$, then $y\in U_x\cup V_x$, so $E\subseteq U_x\cup V_x$, or $E=U_x\cup V_x$. Also, by definition, $U_x\cap V_x = \emptyset$, the two sets are disjoint.

Also, both $U_x$ and $U_y$ must be open: 

For any $y\in U_x\subseteq E$, since there exists $r'>0$, such that the open ball $B_{r'}(y)\subseteq E$. Since in the first part, we've proven that $f_i$ is a constant when restricting to any open ball in $E$,
then all $y'\in B_{r'}(y)$ satisfies $f_i(y')=f_i(y)$, while by definition, $y\in U_x$ implies $f_i(y)=f_i(x)$, so $f_i(y')=f_i(x)$, or $y'\in U_x$.
Hence, $B_{r'(y)}\subseteq U_x$, proven that $U_x$ is open.

Similarly, for any $z\in V_x\subseteq E$, since there exists $r''>0$, such that the open ball $B_{r''}(z)\subseteq E$, then because $f_i$ restricting to any open ball in $E$ is a constant,
then all $z'\in B_{r''}(z)$ satisfies $f_i(z')=f_i(z)$, while by definition, $z\in V_x$ implies $f_i(z)\neq f_i(x)$, so $f_i(z')\neq f_i(x)$, showing that $z'\in V_x$.
Hence, $B_{r''}(z)\subseteq V_x$, proven that $V_x$ is also open.

Then, since $U_x,V_x$ are two open sets satisfying $U_x\cap V_x = \emptyset$, while $U_x\cap V_x=E$, then they form a separation of $E$. If both $U_x$ and $V_x$ are not empty, then it contradicts the assumption that $E$ is connected,
hence we must have one of the sets being empty.

Which, because there exists $r>0$, with $B_r(x)\subseteq E$, then from the above proof, we know $B_r(x)\subseteq U_x$, proving that $U_x$ is not empty. Hence, $V_x=\emptyset$, showing that $U_x=E$.
So, all $y\in E$ must have $f_i(y)=f_i(x)$, showing that $f_i$ is constant in $E$.

\hfil

Since all $f_i$ (with $i\in \{1,...,m\}$) must be constant, then the original function $f:E\rightarrow\mathbb{R}^m$ must also be constant.

\hfil

\hfil

\section*{2 (part (d) not done)}
\begin{myBox}[]{}
    \begin{question}
        Rudin Pg. 239-240 Problem 12:

        Fix two real numbers $a$ and $b$, $0<a<b$. Define a mapping $f=(f_1,f_2,f_3)$ of $\mathbb{R}^2$ into $\mathbb{R}^3$ by 
        $$f_1(s,t)=(b+a\cos(s))\cos(t),\quad f_2(s,t)=(b+a\cos(s))\sin(t),\quad f_3(s,t)=a\sin(s)$$
        Describe the range $K$ of $f$. (It is a certain compact subset of $\mathbb{R}^3$).

        \begin{itemize}
            \item[(a)] Show that there are exactly $4$ points $p\in K$ such that 
            $$(\nabla f_1)(f^{-1}(p))=0$$
            Find these points.
            \item[(b)] Determine the set of all $q\in K$ such that 
            $$(\nabla f_3)(f^{-1}(q))=0$$
            \item[(c)] Show that one of the points $p$ found in part (a) corresponds to a local maximum of $f_1$, one corresponds to a local minimum, and that the other two are neither (they are so-called "saddle points").
            
            Which of the points $q$ found in part (b) correspond to maxima or minima?
            \item[(d)] Let $\lambda$ be an irrational real number, and define $g(t)=f(t,\lambda t)$. Prove that $g$ is a $1-1$ mapping of $\mathbb{R}$ onto a dense subset of $K$. Prove that
            $$|g'(t)|^2 = a^2+\lambda^2(b+a\cos(t))^2$$
        \end{itemize}
    \end{question}
\end{myBox}

\textbf{Pf:}

First, for any $(s_1,t_1), (s_2,t_2)\in\mathbb{R}^2$, if $s_1\equiv s_2\mod\ 2\pi$, and $t_1\equiv t_2\mod\ 2\pi$, then $\sin(s_1)=\sin(s_2)$, $\cos(s_1)=\cos(s_2)$, $\sin(t_1)=\sin(t_2)$, and $\cos(t_1)=\cos(t_2)$,
which suggests that $f(s_1,t_1)=f(s_2,t_2)$ based on the given formula.

Else, if $f(s_1,t_1)=f(s_2,t_2)$, then the following equations are true:
$$(b+a\cos(s_1))\cos(t_1)=(b+a\cos(s_2))\cos(t_2)$$
$$(b+a\cos(s_1))\sin(t_1)=(b+a\cos(s_2))\sin(t_2)$$
$$a\sin(s_1)=a\sin(s_2)$$
Since $a\neq 0$, the third equation satisfies $\sin(s_1)=\sin(s_2)$, which happens iff $s_1\equiv s_2\mod\ 2\pi$, so also $\cos(s_1)=\cos(s_2)$.
And, since $b>a>0$, then for all $s\in\mathbb{R}$, $0<(b-a)\leq(b+a\cos(s))\leq (b+a)$. So, $(b+a\cos(s_1))=(b+a\cos(s_2))>0$,
which the first two equations imply that $\cos(t_1)=\cos(t_2)$ and $\sin(t_1)=\sin(t_2)$, which this occurs iff $t_1 \equiv t_2\mod\ 2\pi$.

Therefore, we can conclude that $f(s_1,t_1)=f(s_2,t_2)$ iff $s_1\equiv s_2\mod\ 2\pi$ and $t_1\equiv t_2\mod\ 2\pi$, which for all $(s',t')\in \mathbb{R}^2$,
there exists $(s,t)\in [0,2\pi]\times [0,2\pi]$ such that $f(s,t)=f(s',t')$, hence for all point in the image of $f$, one can find such $(s,t)\in [0,2\pi]\times [0,2\pi]$, such that $f(s,t)$ is the chosen point in the image.

So, $f([0,2\pi]\times [0,2\pi]) = f(\mathbb{R}^2)=K$, and since $f$ is continuous while $[0,2\pi]\times [0,2\pi]$ is compact, $K$ is in fact compact.

\hfil

Now, given $f(s,t)=(x,y,z)$, since $x=(a+b\cos(s))\cos(t)$ and $y=(a+b\cos(s))\sin(t)$, we have $\sqrt{x^2+y^2}=(a+b\cos(s))$.
Hence, when project down to the $xy$-plane, all points in the image satisfies $(b-a)\leq \sqrt{x^2+y^2}\leq (b+a)$.

Also, since $z=a\sin(s)$, then $-a\leq z\leq a$.

(Note: In fact, the first part suggests that the image can be described using $\mathbb{R}^2 / 2\pi\mathbb{Z}\times 2\pi\mathbb{Z}$, which is a torus).

\hfil

\begin{itemize}
    \item[(a)] Given the formula of $f_1$, its gradient is given by:
    $$\nabla f_1 = \left(\frac{\partial f_1}{\partial s},\frac{\partial f_1}{\partial t}\right)=(-a\sin(s)\cos(t), -(b+a\cos(s))\sin(t))$$
    Now, suppose $(s,t)\in \mathbb{R}^2$ satisfies $(\nabla f_1)(s,t)=0$, we need $-(b+a\cos(s))\sin(t)=0$. Since $(b+a\cos(s))>0$ is proven beforehand, then $\sin(t)=0$,
    so $t\equiv 0\mod\ 2\pi$, or $t\equiv \pi\mod\ 2\pi$.

    Then, at these possible choice of $t$, since $\cos(t)\neq 0$ and $a\neq 0$, then for $-a\sin(s)\cos(t)=0$ (the first coordinate), we need $\sin(s)=0$, so $s\equiv 0\mod\ 2\pi$, or $s\equiv \pi\mod\ 2\pi$.

    Concluding from above, there are the four following choices:
    $$(s,t)\equiv (0\mod\ 2\pi, 0\mod\ 2\pi),\quad (s,t)\equiv (0\mod\ 2\pi, \pi\mod\ 2\pi)$$
    $$(s,t)\equiv (\pi\mod\ 2\pi, 0\mod\ 2\pi),\quad (s,t)\equiv (\pi\mod\ 2\pi, \pi\mod\ 2\pi)$$
    For simplicity, we'll use $(0,0),(0,\pi),(\pi,0),(\pi,\pi)$ as representatives for each case. If label the image of the four cases as $p_1,p_2,p_3,p_4$ respectively, we get:
    $$p_1=f(0,0) = ((b+a\cos(0))\cos(0),(b+a\cos(0))\sin(0),a\sin(0)) = (b+a,0,0)$$
    $$p_2=f(0,\pi) = ((b+a\cos(0))\cos(\pi),(b+a\cos(0))\sin(\pi),a\sin(0)) = (-(b+a),0,0) = (-b-a,0,0)$$
    $$p_3=f(\pi,0) = ((b+a\cos(\pi))\cos(0),(b+a\cos(\pi))\sin(0),a\sin(\pi)) = (b-a,0,0)$$
    $$p_4=f(\pi,\pi) = ((b+a\cos(\pi))\cos(\pi),(b+a\cos(\pi))\sin(\pi),a\sin(\pi)) = (-(b-a),0,0)=(a-b,0,0)$$
    So, $p_1,p_2,p_3,p_4\in K$ above are the four points with $(\nabla f_1)(f^{-1}(p_i)) = 0$.

    \hfil

    \item[(b)] Given the formula of $f_3$, its gradient is given by:
    $$\nabla f_3 = \left(\frac{\partial f_3}{\partial s},\frac{\partial f_3}{\partial t}\right) = (a\cos(s),0)$$
    Hence, if $(s,t)\in\mathbb{R}^2$ satisfies $(\nabla f_3)(s,t)=0$, then we need $a\cos(s)=0$, or $\cos(s)=0$. Hence, $s\equiv \frac{\pi}{2}\mod\ 2\pi$, or $s\equiv \frac{3\pi}{2}\mod\ 2\pi$.

    For simplicity, use $s=\frac{\pi}{2}$ and $s=\frac{3\pi}{2}$ as representatives. 
    In the above two cases, we get:
    $$q_1 = f(\pi/2,t) = ((b+a\cos(\pi/2))\cos(t),(b+a\cos(\pi/2))\sin(t),a\sin(\pi/2)) = (b\cos(t),b\sin(t),a)$$
    $$q_1 = f(3\pi/2,t) = ((b+a\cos(3\pi/2))\cos(t),(b+a\cos(3\pi/2))\sin(t),a\sin(3\pi/2)) = (b\cos(t),b\sin(t),-a)$$
    Hence, if $q\in K$ satisfies $(\nabla f_3)(f^{-1}(q))=0$, it belongs to the following set:
    $$\{(b\cos(t),b\sin(t),a)\in\mathbb{R}^3\ |\ t\in\mathbb{R}\}\cup \{(b\cos(t),b\sin(t),-a)\in\mathbb{R}^3\ |\ t\in\mathbb{R}\}$$

    \hfil

    \item[(c)] \textbf{Points obtained in part (a):}
    \begin{itemize}
        \item For $p_1=(b+a,0,0)$, since all $(s,t)\in\mathbb{R}^2$ satisfies the following:
        $$f_1(s,t)=(b+a\cos(s))\cos(t) \leq b+a\cos(s) \leq b+a$$
        (Note: $(b+a\cos(s))>0$).

        Hence, $p_1$ with first entry being $b+a$ (or $f_1(f^{-1}(p_1))=b+a$) is in fact a local maximum of $f_1$.

        \hfil

        \item For $p_2=(-b-a,0,0)$, similarly, since all $(s,t)\in\mathbb{R}^2$ satisfies:
        $$f_1(s,t)=(b+a\cos(s))\cos(t) \geq -(b+a\cos(s)) = -b-a\cos(s) \geq -b-a$$
        Hence, $p_2$ with first entry being $-b-a$ (or $f_1(f^{-1}(p_2)) = -b-a$), is a local minimum of $f_1$.

        \hfil

        \item For $p_3 = (b-a,0,0)$, the points of its preimage is given as $(\pi\mod\ 2\pi, 0\mod\ 2\pi)$, 
        which any point in its preimage is given as $((2k+1)\pi, 2l\pi)$ for some $k,l\in\mathbb{Z}$.

        Then, for all radius $r>0$, choose $d=\min\{\pi/4, r/2\}>0$, since $0<d\leq r/2<r$, then both points $((2k+1)\pi+d, 2l\pi),\ ((2k+1)\pi, 2l\pi+d)\in B_r((2k+1)\pi, 2l\pi)$.
        Evaluate these points, we get:
        $$f_1((2k+1)\pi+d, 2l\pi) =(b+a\cos((2k+1)\pi+d))\cos(2l\pi) = (b-a\cos(d)) > b-a$$
        $$f_1((2k+1)\pi, 2l\pi+d)=(b+a\cos((2k+1)\pi))\cos(2l\pi +d) = (b-a)\cos(d) < b-a$$
        (Note: since $0<d<\frac{\pi}{4}$, then $ 0<\cos(d)<1$; and, $b-a>0$).

        Since for arbitrary preimage of $p_3$, and arbitrary open neighborhood of it, one can find some points within the neighborhood, such that the output of $f_1$ is greater than or less than $b-a$ (the value of $f_1$ for $p_3$),
        then $p_3$ is neither a local maximum nor a local minimum.

        \hfil

        \item For $p_4=(a-b,0,0)$, the points of its preimage is given as $(\pi\mod\ 2\pi, \pi\mod\ 2\pi)$, 
        which any point in its preimage is given as $((2k+1)\pi, (2l+1)\pi)$ for some $k,l\in\mathbb{Z}$.

        Again, for all radius $r>0$, choose $d=\min\{\pi/4,r/2\}>0$, both $((2k+1)\pi+d, (2l+1)\pi),((2k+1)\pi, (2l+1)\pi+d)\in B_r((2k+1)\pi, (2l+1)\pi)$, and their evaluation with $f_1$ are:
        $$f_1((2k+1)\pi+d, (2l+1)\pi)=(b+a\cos((2k+1)\pi+d))\cos((2l+1)\pi) = -(b-a\cos(d))<a-b$$
        $$f_1((2k+1)\pi, (2l+1)\pi+d)=(b+a\cos((2k+1)\pi))\cos((2l+1)\pi+d) = -(b-a)\cos(d)>a-b$$
        Hence, for arbitrary preimage of $p_4$, and arbitrary open neighborhood of it, there exists some points within the neighborhood, such that $f_1$ yields output either greater than or less than $a-b$ (the value of $f_1$ for $p_4$),
        then $p_4$ is neither a local maximum nor a local minimum.
    \end{itemize}

    \hfil

    \textbf{Points obtained in part (b):}

    Given that $q$ obtained in \textbf{part (b)} is collected into the set:
    $$\{(b\cos(t),b\sin(t),a)\in\mathbb{R}^3\ |\ t\in\mathbb{R}\}\cup \{(b\cos(t),b\sin(t),-a)\in\mathbb{R}^3\ |\ t\in\mathbb{R}\}$$
    For all $(s,t)\in\mathbb{R}^2$, since $f_3(s,t)=a\sin(s)$, then $-a\leq f_3(s,t)\leq a$.

    For points $q_1\in\{(b\cos(t),b\sin(t),a)\in\mathbb{R}^3\ |\ t\in\mathbb{R}\}$, since their third entry is $a$ (value of $f_3$ at these points), hence they're the local maxima of $f_3$;
    else, for points $q_2\in \{(b\cos(t),b\sin(t),-a)\in\mathbb{R}^3\ |\ t\in\mathbb{R}\}$, since their third entry is $-a$ (value of $f_3$ at these points), hence they're the local minima of $f_3$.

    \hfil

    \item[(d)] Given $g(t)=f(t,\lambda t)$ with $\lambda$ being irrational, there are some properties to prove:
    
    \hfil

    \textbf{$g$ is injective:}

    Suppose $t_1,t_2\in\mathbb{R}$ satisfies $g(t_1)=g(t_2)$, then, $f(t_1,\lambda t_1)=f(t_2,\lambda t_2)$, which implies the following equations:
    $$(b+a\cos(t_1))\cos(\lambda t_1)=(b+a\cos(t_2))\cos(\lambda t_2)$$
    $$(b+a\cos(t_1))\sin(\lambda t_1)=(b+a\cos(t_2))\sin(\lambda t_2)$$
    $$a\sin(t_1)=a\sin(t_2)$$
    The third equation implies that $t_1\equiv t_2\mod\ 2\pi$, which also implies that $\cos(t_1)=\cos(t_2)$. Then, since $(b+a\cos(t_1))=(b+a\cos(t_2))>0$, then the first two equations satisfy $\cos(\lambda t_1)=\cos(\lambda t_2)$ and $\sin(\lambda t_1)=\sin(\lambda t_2)$,
    then $\lambda t_1 \equiv \lambda t_2\mod\ 2\pi$.

    Which, $t_1 = 2k\pi + t_2$ for some $k\in\mathbb{Z}$ based on the first relation, so $\lambda t_1 = 2k\lambda \pi + \lambda t_2$, or $\lambda t_1-\lambda t_2 = 2k\lambda \pi$;
    and, based on the second relation, we know $\lambda t_1 - \lambda t_2 = 2l\pi$ for some $l\in\mathbb{Z}$, hence $2l\pi = 2k\lambda \pi$, showing that $2\pi (\lambda k-l)=0$, or $\lambda k=l$.

    Yet, if $k\neq 0$, we have $\lambda = \frac{l}{k}$ while $l,k\in\mathbb{Z}$, this violates the assumption that $\lambda$ is irrational. Hence, we must have $k=0$, showing that $t_1 = 2k\pi + t_2 = t_2$.

    This proves that $g$ is injective.

    \hfil

    \textbf{$g$ has a dense image:}

    Since for all $y\in K$, there exists $(s,t)\in [0,2\pi]\times [0,2\pi]$, such that $f(s,t)=y$, then to show $g(\mathbb{R})$ is dense in $K$, it suffices to show that the set $U=\{(s,t)\in [0,2\pi]\times [0,2\pi]\ |\ f(s,t)\in g(\mathbb{R})\}$ is dense in $[0,2\pi]\times [0,2\pi]$.
    
    If $U$ is dense in $[0,2\pi]\times [0,2\pi]$, any $(s,t)$ in this region has a sequence $(\bar{x}_n)_{n\in\mathbb{N}}\subset U$ converging to $(s,t)$, then if consider the image, because of continuity of $f$, $\lim_{n\rightarrow\infty}f(\bar{x}_n) = f(s,t)=y$, showing that $(f(\bar{x}_n))_{n\in\mathbb{N}}$
    is a sequence in $g(\mathbb{R})$ that converges to $y$ (note: since $\bar{x}_n\in U$ for all $n\in\mathbb{N}$, then $f(\bar{x}_n)\in g(\mathbb{R})$ by definition). Which, since this is true for arbitrary $y\in K$, then $y$ is a limit point of $g(\mathbb{R})$, showing that $g(\mathbb{R})$ is dense in $K$.

    \hfil

    To prove that $U$ is dense in $[0,2\pi]\times [0,2\pi]$, for any $(s,t)\in [0,2\pi]\times [0,2\pi]$, for all $n\in\mathbb{N}$, let $\bar{x}_n\in [0,2\pi]\times [0,2\pi]$ to satisfy $

    \hfil

    \textbf{Derivative of $g$:}

    Since $g(t)=f(t,\lambda t)$, with the linear map $h\in\mathcal{L}(\mathbb{R},\mathbb{R}^2)$ by $h(t) = (t,\lambda t)$, $g = f\circ h$, the derivative is given by chain rule:
    $$g'(t) = Df(h(t)) Dh(t)$$
    Which, $Dh(t) = \begin{pmatrix}1\\\lambda\end{pmatrix}$ (matrix representation of $h$, since $h$ is a linear map). Now, with $f_1(x,y)=(b+a\cos(x))\cos(y)$, $f_2(x,y)=(b+a\cos(x))\sin(y)$, $f_3(x,y)=a\sin(x)$, the derivative of $f(x,y)$ is given by:
    $$Df(x,y) = \begin{pmatrix}
        \frac{\partial f_1}{\partial x} & \frac{\partial f_1}{\partial y}\\
        \frac{\partial f_2}{\partial x} & \frac{\partial f_2}{\partial y}\\
        \frac{\partial f_3}{\partial x} & \frac{\partial f_3}{\partial y}
    \end{pmatrix} = \begin{pmatrix}
        -a\sin(x)\cos(y) & -(b+a\cos(x))\sin(y)\\
        -a\sin(x)\sin(y) & (b+a\cos(x))\cos(y)\\
        a\cos(x) & 0
    \end{pmatrix}$$
    Hence, $g'(t)$ is given as:
    $$g'(t) = Df(t,\lambda t) Dh(t) = \begin{pmatrix}
        -a\sin(t)\cos(\lambda t) & -(b+a\cos(t))\sin(\lambda t)\\
        -a\sin(t)\sin(\lambda t) & (b+a\cos(t))\cos(\lambda t)\\
        a\cos(t) & 0
    \end{pmatrix}\begin{pmatrix}1\\\lambda\end{pmatrix}$$
    $$=\begin{pmatrix}
        -a\sin(t)\cos(\lambda t) - \lambda(b+a\cos(t))\sin(\lambda t)\\
        -a\sin(t)\sin(\lambda t) + \lambda(b+a\cos(t))\cos(\lambda t)\\
        a\cos(t)
    \end{pmatrix}$$
    Whic, $|g'(t)|^2$ is given as:
    $$|g'(t)|^2 = (-a\sin(t)\cos(\lambda t) - \lambda(b+a\cos(t))\sin(\lambda t))^2 + (-a\sin(t)\sin(\lambda t) + \lambda(b+a\cos(t))\cos(\lambda t))^2 + (a\cos(t))^2$$
    
    $$ = (a^2\sin^2(t)\cos^2(\lambda t) + \lambda^2(b+a\cos(t))^2\sin^2(\lambda t) + 2a\lambda (b+a\cos(t))\sin(t)\sin(\lambda t)\cos(\lambda t))$$
    $$+ (a^2\sin^2(t)\sin^2(\lambda t) + \lambda^2(b+a\cos(t))^2\cos^2(\lambda t) - 2a\lambda (b+a\cos(t))\sin(t)\sin(\lambda t)\cos(\lambda t))$$
    $$ + a^2\cos^2(t)$$
    
    $$= a^2\sin^2(t) + \lambda^2(b+a\cos(t))^2 + a^2\cos^2(t)$$
    $$= a^2+\lambda^2(b+a\cos(t))^2$$

\end{itemize}

\break

\section*{3}
\begin{myBox}[]{}
    \begin{question}
        Rudin Pg. 240 Problem 13:

        Suppose $f$ is differentiable mapping of $\mathbb{R}$ into $\mathbb{R}^3$ such that $|f(t)|=1$ for every $5$.
        Prove that $f'(t)\cdot f(t)=0$. Interpret this result geometrically.
    \end{question}
\end{myBox}

\textbf{Pf:}

Since $|f(t)|=1$, then the function $g:\mathbb{R}\rightarrow\mathbb{R}$ given by $g(t)=f(t)\cdot f(t)=|f(t)|^2 = 1$, hence $g'(t)=0$.

On the other hand, since $g'(t) = \frac{d}{dt}(f(t)\cdot f(t))$, while the derivative given by product rule for real dot product is given by:
$$\frac{d}{dt}(f(t)\cdot f(t)) = f'(t)\cdot f(t) + f(t)\cdot f'(t) = 2f'(t)\cdot f(t)$$
Hence, $2f'(t)\cdot f(t)=0$, or $f'(t)\cdot f(t)=0$.

Geometrically, since $|f(t)|=1$, then $f$ is in fact a curve on the 2-dimensional sphere $S^2$; since $f'(t)$ is the tangent vector (the traveling direction) of the curve at any given point,
then $f'(t)\cdot f(t)=0$ implies the tangent vector and the position of the curve is always orthogonal to each other, showing that to travel on a sphere, the tangent vector is necessarily orthogonal to the surface.

\hfil

\hfil

\section*{4}
\begin{myBox}[]{}
    \begin{question}
        Rudin Pg. 240 Problem 16:

        Show that the continuity of $f'$ at the point $a$ is needed in the inverse function theorem, even in the case $n=1$: if
        $$f(t)=t+2t^2\sin\left(\frac{1}{t}\right)$$
        for $t\neq 0$, and $f(0)=0$, then $f'(0)=1$, $f'$ is bounded in $(-1,1)$, but $f$ is not $1-1$ in any neighborhood of $0$.
    \end{question}
\end{myBox}

\textbf{Pf:}

\textbf{Derivative at $0$:}

The derivative of the function at $0$ is given as follow:
$$f'(0)=\lim_{h\rightarrow 0}\frac{f(h)-f(0)}{h}=\lim_{h\rightarrow 0}\frac{h+2h^2\sin(1/h)}{h}=\lim_{h\rightarrow 0}(1+2h\sin(1/h))=1$$
(Note: since $|h\sin(1/h)|\leq |h|$ for all $0<h$, then $0\leq \lim_{h\rightarrow 0}|h\sin(1/h)|\leq \lim_{h\rightarrow 0}|h|=0$, so the limit is $0$).

\hfil

\textbf{Derivative is bounded in $(-1,1)$, but not continuous at $0$:}

For any nonzero $t\in (-1,1)$, based on differentiation rules, we get the following:
$$f'(t)=1+4t\sin\left(\frac{1}{t}\right) + 2t^2\cos\left(\frac{1}{t}\right)\cdot\frac{-1}{t^2} = 1+4t\sin\left(\frac{1}{t}\right)-2\cos\left(\frac{1}{t}\right)$$
Which, its bound is given as follow:
$$|f'(t)| \leq 1+\left|4t\sin\left(\frac{1}{t}\right)\right| + \left|2\cos\left(\frac{1}{t}\right)\right| \leq 1+4+2 = 7$$
(Note: $\sin,\cos$ are both bounded by $1$, while $t\in (-1,1)$ implies $|t|<1$).

So, conbine with the previous part that $f'(0)=1$, all $t\in (-1,1)$ satisfies $|f'(t)|\leq 7$, hence $f'$ is bounded in $(-1,1)$.

Yet, since $\lim_{t\rightarrow 0}f'(t)$ does not exist, then $f'(t)$ is not continuous at $0$.

\hfil

\textbf{Inverse Function Theorem doesn't apply:}

For any open neighborhood $U\subseteq \mathbb{R}$ of $0$, there exists $\epsilon>0$, such that $(-\epsilon,\epsilon)\subseteq U$. Now, by Archimedean's Property,
choose $n\in\mathbb{N}$ such that $0<\frac{1}{2n\pi}<\epsilon$ (which, since $2n\pi < 2n\pi +\pi/2 < 2n\pi +\pi$, then $0<\frac{1}{2n\pi +\pi}<\frac{1}{2n\pi +\pi/2}<\frac{1}{2n\pi}<\epsilon$, 
so all of these points are within $(-\epsilon,\epsilon)$).

Let $t_1=\frac{1}{2n\pi +\pi},\ t_2=,\frac{1}{2n\pi +\pi/2}\ t_3=\frac{1}{2n\pi}$. If we evaluate $f$ at these points, we get:
$$f_1=f\left(\frac{1}{2n\pi}\right)=\frac{1}{2n\pi}+2\left(\frac{1}{2n\pi}\right)^2\sin\left(\frac{1}{1/(2n\pi)}\right)=\frac{1}{2n\pi}+2\left(\frac{1}{2n\pi}\right)^2\sin(2n\pi) = \frac{1}{2n\pi}$$

$$f_2 = f\left(\frac{1}{2n\pi +\pi/2}\right)=\frac{1}{2n\pi +\pi/2}+2\left(\frac{1}{2n\pi +\pi/2}\right)^2\sin\left(\frac{1}{1/(2n\pi +\pi/2)}\right)$$
$$= \frac{1}{2n\pi +\pi/2}+2\left(\frac{1}{2n\pi +\pi/2}\right)^2\sin\left(2n\pi +\frac{\pi}{2}\right)=\frac{1}{2n\pi +\pi/2}+2\left(\frac{1}{2n\pi +\pi/2}\right)^2$$

$$f_3 = f\left(\frac{1}{2n\pi +\pi}\right)=\frac{1}{2n\pi +\pi}+2\left(\frac{1}{2n\pi +\pi}\right)^2\sin\left(\frac{1}{1/(2n\pi +\pi)}\right)$$
$$=\frac{1}{2n\pi +\pi}+2\left(\frac{1}{2n\pi +\pi}\right)^2\sin(2n\pi+\pi)=\frac{1}{2n\pi +\pi}$$
If we compare $f_2$ and $f_3$, we get:
$$f_2=\frac{1}{2n\pi +\pi/2}+2\left(\frac{1}{2n\pi +\pi/2}\right)^2>\frac{1}{2n\pi +\pi/2}>\frac{1}{2n\pi +\pi}=f_3$$
On the other hand, if we choose $n>\frac{\pi}{16-4\pi}>0$, we get the following inequality:
$$16n-4n\pi > \pi\implies 16n -4n\pi -\pi>0\implies \frac{16n -(4n\pi +\pi)}{4n(4n\pi +\pi)}>0 \implies \frac{4}{4n\pi +\pi}-\frac{1}{4n}>0 $$
$$\implies \frac{2}{2n\pi + \pi/2}-\frac{\pi/2}{2n\pi}>0 \implies 2\left(\frac{1}{2n\pi+\pi/2}\right)^2-\frac{\pi/2}{2n\pi(2n\pi+\pi/2)}>0$$
$$\implies 2\left(\frac{1}{2n\pi+\pi/2}\right)^2+\left(\frac{1}{2n\pi+\pi/2}-\frac{1}{2n\pi}\right)>0$$
$$\implies \frac{1}{2n\pi+\pi/2}+2\left(\frac{1}{2n\pi+\pi/2}\right)^2>\frac{1}{2n\pi}$$
Which, this inequality implies the following:
$$f_2 = \frac{1}{2n\pi+\pi/2}+2\left(\frac{1}{2n\pi+\pi/2}\right)^2 >\frac{1}{2n\pi}=f_1$$
So, $f_2>f_3$ and $f_2>f_1$

Now, choose any $y$ satisfying $f_3<y<f_2$ and $f_1<y<f_2$. Since by the notation, we have $t_1<t_2<t_3$, and $f_1=f(t_1)$, $f_2=f(t_2)$, and $f_3=f(t_3)$,
then because $f$ is a continuous function, by Intermediate Value Theorem, there exists $c\in (t_1,t_2)$ and $c'\in (t_2,t_3)$, such that $f(c)=f(c')=y$.

Which, because $c\neq c'$, then $f$ is not injective; also, since $0<t_1<c<t_2<c'<t_3<\epsilon$, then $c,c'\in (-\epsilon,\epsilon)\subseteq U$.
This shows that $f$ restricting to $U$ is not injective. Then, because $f|U$ for any open neighborhood $U$ of $0$ is not injective, then Inverse Function Theorem fails.

\break

\section*{5 (part (c) not done)}
\begin{myBox}[]{}
    \begin{question}
        Rudin Pg. 241 Problem 17:

        Lef $f=(f_1,f_2)$ be the mapping of $\mathbb{R}^2$ into $\mathbb{R}^2$ given by 
        $$f_1(x,y)=e^x\cos(y),\quad f_2(x,y)=e^x\sin(y)$$
        \begin{itemize}
            \item[(a)] What is the range of $f$?
            \item[(b)] Show that the Jacobian of $f$ is not zero at any point of $\mathbb{R}^2$. Thus every point of $\mathbb{R}^2$
            has a neighborhood in which $f$ is $1-1$. Nevertheless, $f$ is not $1-1$ on $\mathbb{R}^2$.
            \item[(c)] Put $a=(0,\pi/3)$, $b=f(a)$, let $g$ be the continuous inverse of $f$, defined in a neighborhood of $b$, such that $g(b)=a$.
            Find an explicit formula of $g$, compute $f'(a)$ and $g'(b)$, and verify the formula (52).

            (Note: Formula (52) states if $g$ is an inverse of $f$, then for any $y$ in the given domain of $f$, $g'(y)=(f'(g(y)))^{-1}$).
            \item[(d)] What are the images under $f$ of lines parallel to the coordinate axes?
        \end{itemize}
    \end{question}
\end{myBox}

\textbf{Pf:}
\begin{itemize}
    \item[(a)] We'll prove that the range of $f$ is given as $\mathbb{R}^2\setminus\{\bar{0}\}$.
    
    First, for all $v\in\mathbb{R}^2$ that's not a zero vector, one can fined $r>0$ and $\theta\in\mathbb{R}$ to represent $v$ under polar coordinates.
    Hence, $v = (r\cos(\theta),r\sin(\theta))$. Then, consider $f(\ln(r),\theta)$, we get:
    $$f(\ln(r),\theta)=\left(f_1(\ln(r),\theta),f_2(\ln(r),\theta)\right) = (e^{\ln(r)}\cos(\theta),e^{\ln(r)}\sin(\theta)) = (r\cos(\theta),r\sin(\theta))=v$$
    Hence, $v$ is in the range of $f$.

    Then, to prove that $\bar{0}\in \mathbb{R}^2$ is not in the range, suppose the contrary that there exists $(x,y)\in\mathbb{R}^2$ such that $f(x,y)=\bar{0}$.
    Then, we need $f_1(x,y)=e^x\cos(y)=0$ and $f_2(x,y)=e^x\sin(y)=0$.
    Since $^x\neq 0$ for all $x\in\mathbb{R}$, for both previous equality to hold, we need $\cos(y)=\sin(y)=0$. Yet, this is a contradiction, since this enforces $\cos^2(y)+\sin^2(y)=0$,
    violates the trigonometric identity. So, $\bar{0}$ is not in the range.

    This proves that range of $f$ is $\mathbb{R}^2\setminus\{\bar{0}\}$.

    \hfil

    \item[(b)] First, the partial derivatives of $f_1,\ f_2$ are given as follow:
    $$\frac{\partial f_1}{\partial x} = e^x\cos(y),\quad \frac{\partial f_1}{\partial y}=-e^x\sin(y),\quad \frac{\partial f_2}{\partial x}=e^x\sin(y),\quad \frac{\partial f_2}{\partial y}=e^x\cos(y)$$
    
    If we consider the differential of $f$ at any $(x,y)\in\mathbb{R}^2$, we get:
    $$Df(x,y) = \begin{pmatrix}
        \frac{\partial f_1}{\partial x} & \frac{\partial f_1}{\partial y}\\
        \frac{\partial f_2}{\partial x} & \frac{\partial f_2}{\partial y}
    \end{pmatrix} = \begin{pmatrix}
        e^x\cos(y) & -e^x\sin(y)\\
        e^x\sin(y) & e^x\cos(y)
    \end{pmatrix}$$
    Then, the Jacobian is given by:
    $$|Df(x,y)| = \begin{vmatrix}
        e^x\cos(y) & -e^x\sin(y)\\
        e^x\sin(y) & e^x\cos(y)
    \end{vmatrix} = e^{2x}\cos^2(y) - (-e^{2x}\sin^2(y)) = e^{2x}$$
    So, the Jacobian $|Df(x,y)|=e^{2x}\neq 0$ for all $(x,y)\in\mathbb{R}^2$. Also, since the above differential has all the entries being continuous functions,
    then the differential is in fact continuous, hence the Jacobian is nonvanishing at any point (or invertible at all point) while the differential is continuous,
    so the Inverse Function Theorem applies, all point $(x,y)\in\mathbb{R}^2$ has a small open neighborhood such that $f$ is invertible, which $f$ is $1-1$ in that neighborhood.

    But, $f$ is not $1-1$ on $\mathbb{R}^2$, since for arbitrary $x\in \mathbb{R}$, the points $(x,2\pi),\ (x,4\pi)\in\mathbb{R}^2$ satisfies:
    $$f(x,2\pi) = (f_1(x,2\pi),f_2(x,2\pi))=(e^x\cos(2\pi),e^x\sin(2\pi)) = (e^x,0)$$
    $$f(x,4\pi) = (f_1(x,4\pi),f_2(x,4\pi))=(e^x\cos(4\pi),e^x\sin(4\pi)) = (e^x,0)$$
    Though $(x,2\pi)\neq (x,4\pi)$, we still have $f(x,2\pi)=f(x,4\pi)$, showing that $f$ is not $1-1$.

    \hfil

    \item[(c)] 


    \hfil

    \item[(d)] First, for lines parallel to the $x$-axis (i.e. all collections of $H_y = \{(x,y)\in\mathbb{R}^2\ |\ x\in\mathbb{R}\}$ for arbitrary $y\in\mathbb{R}$), all point $(x,y)\in H_y$ satisfies:
    $$f(x,y) = (e^x\cos(y),e^x\sin(y)) = e^x(\cos(y),\sin(y))$$
    Let the $R_y = \{r(\cos(y),\sin(y))\ |\ r>0\}$ (a straight ray from origin with angle $y$, but not including the origin), since $e^x>0$, we have $f(x,y)\in R_y$, hence $f(H_y)\subseteq R_y$;
    on the other hand, for all $r>0$, since $\ln(r)$ satisfies $e^{\ln(r)}=r$, then:
    $$(\ln(r),y)\in H_y,\quad f(\ln(r),y) = (e^{\ln(r)}\cos(y),e^{\ln(r)}\sin(y)) = (r\cos(y),r\sin(y))=r(\cos(y),\sin(y))$$
    Hence, for all $r(\cos(y),\sin(y))\in R_y$ satisfies $f(\ln(r),y)=r(\cos(y),\sin(y))$, showing that $r(\cos(y),\sin(y))\in f(H_y)$, or $R_y\subseteq f(H_y)$.
    Therefore, $f(H_y)=R_y$, its image is a straight ray from the origin with angle $y$ (not including the origin).

    Then, for lines parallel to the $y$-axis (i.e. all collections of $V_x = \{(x,y)\in\mathbb{R}^2\ |\ y\in\mathbb{R}\}$ for arbitrary $x\in \mathbb{R}$), all point $(x,y)\in H_y$ satisfies:
    $$f(x,y)=e^x(\cos(y),\sin(y)),\quad |f(x,y)| = e^x\sqrt{\cos^2(y)+\sin^2(y)}=e^x$$
    Since $S^1 = \{v\in\mathbb{R}^2\ |\ |v|=1\}$, define $rS^1 = \{v\in\mathbb{R}^2\ |\ |v|=r\}$ for all $r\geq 0$, then since $|f(x,y)| = e^x$, we have $f(x,y)\in e^x S^1$, so $f(V_x)\subseteq e^xS^1$.
    Also, for all $v\in e^xS^1$, there exists angle $\theta$, such that $v = e^x(\cos(\theta),\sin(\theta))$ (so $(e^x,\theta)$ is a representation of $v$ under polar coordinates), then:
    $$(x,\theta)\in V_x,\quad f(x,\theta)=(e^x\cos(\theta),e^x\sin(\theta)) = e^x(\cos(\theta),\sin(\theta)) = v$$
    Hence, $v\in f(V_x)$, showing that $e^xS^1\subseteq f(V_x)$. Therefore, $f(V_x)=e^xS^1$, its image is a circle centered at the origin, with radius $e^x$.
\end{itemize}

\end{document}