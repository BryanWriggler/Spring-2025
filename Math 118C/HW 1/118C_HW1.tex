\documentclass{article}
\usepackage{graphicx} % Required for inserting images
\usepackage[margin = 2.54cm]{geometry}
\usepackage[most]{tcolorbox}

\newtcolorbox{myBox}[3]{
arc=5mm,
lower separated=false,
fonttitle=\bfseries,
%colbacktitle=green!10,
%coltitle=green!50!black,
enhanced,
attach boxed title to top left={xshift=0.5cm,
        yshift=-2mm},
colframe=blue!50!black,
colback=blue!10
}

\usepackage{amsmath}
\usepackage{amssymb}
\usepackage{verbatim}
\usepackage[utf8]{inputenc}
\linespread{1.2}

\newtheorem{definition}{Definition}
\newtheorem{proposition}{Proposition}
\newtheorem{theorem}{Theorem}
\newtheorem{question}{Question}

\title{Math 118C HW1}
\author{Zih-Yu Hsieh}

\begin{document}
\maketitle

\section*{1}
\begin{myBox}[]{}
    \begin{question}
        Rudin Pg. 239 Problem 1:

        If $S$ is a nonempty subset of a vector space $X$. prove that the span of $S$ is a vector space.
    \end{question}
\end{myBox}

\textbf{Pf:}

(Remark: The notation $\mathbb{F}$ denotes the base field of the vector space $X$).

Let $S'$ be the span of the set $S$. Then, $S'$ is a collection of all arbitrary linear combinations of vectors in any finite subcollection of $S$.

Hence, for all $x\in S'$, there exists $x_1,...,x_n\in S$, and $a_1,...,a_n\in\mathbb{F }$, where $x=\sum_{k=1}^{n}a_kx_k$.

Which, the zero vector $\bar{0}\in S'$, since $0 = 0 x$ for all $x\in S$.

For all $x,y\in S'$, there exists $x_1,...,x_n, y_1,...,y_m\in S$, and $a_1,...,a_n,b_1,...,b_m\in\mathbb{F}$, where $x=\sum_{k=1}^{n}a_kx_k$, and $y=\sum_{j=1}^{m}b_jy_j$.
THen, the sum $x+y=\sum_{k=1}^{n}a_kx_k+\sum_{j=1}^{m}b_jy_j\in S'$, since it is a linear combination of $x_1,...,x_n,y_1,...,y_m\in S$.

Finally, for any $\lambda\in\mathbb{F}$, given $x\in S'$ above, $\lambda x\in S'$, since $\lambda x=\lambda \sum_{k=1}^{n}a_kx_k = \sum_{k=1}^{n}(\lambda a_k)x_k$, where each index $k\in \{1,...,n\}$ satisfies $\lambda a_k\in\mathbb{F}$.
Hence, $\lambda x$ is again a linear combination of $x_1,...,x_n\in S$, showing that $\lambda x\in S'$.

\hfil

Since the zero vector $\bar{0}\in S'$, $S'$ is closed under addition (all $x,y\in S'$ has $x+y\in S'$), and it's closed under scalar multiplication (all $x\in S'$ and $\lambda\in\mathbb{F}$ satisfies $\lambda x\in S'$), hence $S'$ (the span of $S$) is a vector space.

\break

\section*{2}
\begin{myBox}[]{}
    \begin{question}
        Rudin Pg. 239 Problem 4:

        Prove that null spaces and ranges of linear transformations are vector spaces.
    \end{question}
\end{myBox}

\textbf{Pf:}

Let $\mathbb{F}$ be an arbitrary field, and $V, W$ be arbitrary two vector spaces over base field $\mathbb{F}$, and $T\in\mathcal{L}(V,W)$ (an arbitrary linear transformation from $V$ to $W$).

\hfil

\textbf{Null Space is a vector space:}

The null space of $T$, $null(T)\subseteq V$ satisfies the following properties:
\begin{itemize}
    \item $\bar{0}_V\in null(T)$: By definition, since $T\bar{0}_V = \bar{0}_W$, then $\bar{0}_V\in null(T)$.
    \item $null(T)$ is closed under addition: For all $u,v\in null(T)$, since $Tu, Tv = \bar{0}_W$, then $T(u+v)=Tu+Tv = \bar{0}_W+\bar{0}_W=\bar{0}_W$, hence $u+v$ also got mapped to $\bar{0}_W$, showing that $u+v\in null(T)$.
    \item $null(T)$ is closed under scalar multiplication: For all $v\in null(T)$ and $\lambda\in\mathbb{F}$, since $Tv=\bar{0}_W$, then $T(\lambda v)=\lambda Tv = \lambda\cdot \bar{0}_W = \bar{0}_W$, showing that $\lambda v$ also got mapped to $\bar{0}_W$, hence $\lambda v\in null(T)$.
\end{itemize}
With the above three conditions, $null(T)$ the null space of $T$, is a vector space.

\hfil

\textbf{Range is a vector space:}

The range of $T$, $range(T)\subseteq W$ satisfies the following properties:
\begin{itemize}
    \item $\bar{0}_W\in range(T)$: By definition, since $T\bar{0}_V=\bar{0}_W$, then $\bar{0}_W\in range(T)$.
    \item $range(T)$ is closed under addition: For all $u,v\in range(T)$, there exists $x,y\in V$, such that $Tx= u$, and $Ty = v$. Then, $T(x+y)=Tx+Ty = u+v$, showing that $u+v\in range(T)$.
    \item $range(T)$ is closed under scalar multiplication: For all $v\in range(T)$ and $\lambda\in \mathbb{F}$, since there exists $x\in V$, such that $Tx=v$, then $T(\lambda x)=\lambda (Tx)=\lambda v$, showing that $\lambda v\in range(T)$.
\end{itemize}
Again, with the above three conditions, $range(T)$ is a vector space.

\hfil

\hfil

\section*{3}
\begin{myBox}[]{}
    \begin{question}
        Rudin Pg. 239 Problem 5:

        Prove that to every $A\in\mathcal{L}(\mathbb{R}^n,\mathbb{R})$ corresponds to a unique $y\in\mathbb{R}^n$, such that $Ax = x\cdot y$.
        Prove also that $\|A\|=|y|$.
    \end{question}
\end{myBox}

\textbf{Pf:}

\textbf{Existence of $y$:}

If we pick the standard orthonormal basis $e_1,...,e_n\in\mathbb{R}^n$, which for every $A\in\mathcal{L}(\mathbb{R}^n,\mathbb{R})$, let $a_i=A e_i\in\mathbb{R}$ for all index $i\in\{1,...,n\}$.

Now, consider the vector $y=\sum_{i=1}^{n}a_ie_i$:

For any $x\in\mathbb{R}^n$, there exists unique $b_1,...,b_n\in\mathbb{R}$, such that $x=\sum_{i=1}^{n}b_ie_i$. Then, when apply the transformation and the dot product, we get the following:
$$Ax = A\left(\sum_{i=1}^{n}b_ie_i\right)=\sum_{i=1}^{n}b_i(Ae_i)=\sum_{i=1}^{n}b_ia_i$$
$$x\cdot y = \left(\sum_{i=1}^{n}b_ie_i\right)\cdot \left(\sum_{j=1}^{n}a_je_j\right)=\sum_{i=1}^{n}b_i\left(e_i\cdot \sum_{j=1}^{n}a_je_j\right) = \sum_{i=1}^{n}b_ia_i$$
(Note: Since $e_1,...,e_n\in\mathbb{R}^n$ is an orthonormal basis, then $e_i\cdot e_j=1$ if $i=j$, and $e_i\cdot e_j=0$ if $i\neq j$).
Hence, $Ax = x\cdot y$, showing that there exists such $y\in\mathbb{R}^n$, with $Ax = x\cdot y$.

\hfil

\textbf{Uniqueness of $y$:}

Suppose $y,z\in\mathbb{R}^n$ are two vectors satisfying $Ax = x\cdot y$ and $Ax = x\cdot z$ for all $x\in\mathbb{R}^n$. Then, by the bilinearity of real dot product, we have:
$$0=Ax-Ax = (x\cdot y)-(x\cdot z)=x\cdot (y-z)$$
However, notice that the choice of $x$ is arbitrary. In particular, we can choose $x=(y-z)\in\mathbb{R}^n$, and get the following:
$$0 = (y-z)\cdot (y-z)$$
By the property of dot product, any $x\in\mathbb{R}^n$ satisfies $x\cdot x\geq 0$, and $x\cdot x=0$ iff $x=\bar{0}$, hence the above equality implies $(y-z)=\bar{0}$, or $y=z$.
This proves the uniqueness of such corresponding vector $y$ of $A$.

\hfil

\textbf{Norm of $A$:}

First, we need to consider the special case where $A=0$ as a linear functional: For all $x\in\mathbb{R}^n$, since $Ax=0$, and $x\cdot \bar{0}=0$, then the unique vector corresponding to $A=0$ the zero map, is $\bar{0}$.
In this case, all $x\in\mathbb{R}^n$ with $|x|=1$ satisfies $|Ax|=0=|\bar{0}|$, hence $\|A\|=\sup_{|x|=1}|Ax|=0=|\bar{0}|$.

Now, suppose $A\neq 0$. For all $x\in\mathbb{R}^n$ with $|x|=1$, based on Cauchy-Schwartz Inequality, we can get the following relationship:
$$|Ax| = |x\cdot y|\leq |x|\cdot |y|=|y|$$
Hence, $\|A\|=\sup_{|x|=1}|Ax|\leq |y|$.

On the other hand, since $A\neq 0$, then the corresponding vector $y\neq\bar{0}$ (or else all $x\in\mathbb{R}^n$ would satisfy $Ax = x\cdot \bar{0}=0$, which is a contradiction). Then, $|y|>0$, which we can define a unit vector $\hat{y}=\frac{y}{|y|}$ with $|\hat{y}|=1$.
Because Cauchy-Schwartz Inequality achieves an equality when the two vectors are scalar multiple of each other, then since $\hat{y}$ is a scalar multiple of $y$, we get the following:
$$|A\hat{ y}|=|\hat{y}\cdot y| = |\hat{y}|\cdot |y|=|y|$$
Hence, $|A\hat{y}|=|y|\leq \|A\|$.

The above two inequalities show that $\|A\|=|y|$.

\break

\section*{4}
\begin{myBox}[]{}
    \begin{question}
        Rudin Pg 239 Problem 7:

        Suppose that $f$ is a real-valued function defined in an open se $E\subseteq \mathbb{R}^n$, and that the partial derivatives $D_1f, ..., D_nf$ are bounded in $E$.
        Prove that $f$ is continuous in $E$.
    \end{question}
\end{myBox}

\textbf{Pf:}

\break

\section*{5}
\begin{myBox}[]{}
    \begin{question}
        Rudin Pg. 239 Problem 8:

        Suppose that $f$ is a differentiable real function in an open set $E\subseteq \mathbb{R}^n$, 
        and that $f$ has a local maximum at a point $x\in E$. Prove that $f'(x)=Df(x)=0$.
    \end{question}
\end{myBox}

\textbf{Pf:}

First, since $f$ has a local maximum at $x$, then there exists a $\delta>0$, such that any $y\in B_\delta(x)$ (a small open neighborhood of $x$),
satisfies $f(y)\leq f(x)$. 

Then, since $f$ is differentiable implies the existence of all partial derivative and the uniqueness of the differential $Df(x)$, we know it is given as follow:
$$Df(x)=\left(\frac{\partial f}{\partial x_1}(x),...,\frac{\partial f}{\partial x_n}(x)\right)$$
So, to prove that $Df(x)=0$, it suffices to prove that each partial derivative is $0$ at $x$.

\hfil

Let $x=(a_1,...,a_n)\in\mathbb{R}^n$. For each $i\in\{1,...,n\}$, the partial derivative is given as follow:
$$\frac{\partial f}{\partial x_i}(x)=\lim_{h\rightarrow 0}\frac{f(a_1,...,a_i+h,...,a_n)-f(a_1,...,a_i,...,a_n)}{h}$$
Now, if we consider any $0<|h|<\delta$, since $|(a_1,...,a_i+h,...,a_n)-(a_1,...,a_i,...,a_n)|=|(0,...,h,...,0)| = |h|<\delta$, then the vector $(a_1,...,a_i+h,...,a_n)\in B_\delta(x)$. 
Hence, $f(a_1,...,a_i+h,...,a_n) \leq f(x)=f(a_1,...,a_i,...,a_n)$, so the difference $f(a_1,...,a_i+h,...,a_n)-f(a_1,...,a_i,...,a_n)\leq 0$.

Then, there are two cases to consider:
\begin{itemize}
    \item For all $h>0$, the following is true:
    $$\frac{f(a_1,...,a_i+h,...,a_n)-f(a_1,...,a_i,...,a_n)}{h}\leq 0\implies \lim_{h\rightarrow 0}\frac{f(a_1,...,a_i+h,...,a_n)-f(a_1,...,a_i,...,a_n)}{h}\leq 0$$
    \item Else, for all $h<0$, the following is true:
    $$\frac{f(a_1,...,a_i+h,...,a_n)-f(a_1,...,a_i,...,a_n)}{h}\geq 0\implies \lim_{h\rightarrow 0}\frac{f(a_1,...,a_i+h,...,a_n)-f(a_1,...,a_i,...,a_n)}{h}\geq 0$$
\end{itemize}
(Note: the above two inequalities are followed by the properties of limit).

Then, we can conclude the following:
$$\frac{\partial f}{\partial x_i}(x)=\lim_{h\rightarrow 0}\frac{f(a_1,...,a_i+h,...,a_n)-f(a_1,...,a_i,...,a_n)}{h}=0$$
So, because each partial derivative is $0$, the differential $Df(x)=0$.

Therefore, $f$ is differentiable over $E$ and $x\in E$ is a local maximum, implies that $Df(x)=0$.

\break

\section*{6}
\begin{myBox}[]{}
    \begin{question}
        Rudin Pg. 239 Problem 11:

        If $f$ and $g$ are differentiable real functions in $\mathbb{R}^n$, prove that 
        $$D(fg)=f(Dg)+g(Df)$$
        and that $D(1/f)=-f^{-2}(Df)$ wherever $f\neq 0$.
    \end{question}
\end{myBox}

\textbf{Pf:}

\end{document}