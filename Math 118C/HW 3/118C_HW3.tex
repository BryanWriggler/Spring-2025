\documentclass{article}
\usepackage{graphicx} % Required for inserting images
\usepackage[margin = 2.54cm]{geometry}
\usepackage[most]{tcolorbox}

\newtcolorbox{myBox}[3]{
arc=5mm,
lower separated=false,
fonttitle=\bfseries,
%colbacktitle=green!10,
%coltitle=green!50!black,
enhanced,
attach boxed title to top left={xshift=0.5cm,
        yshift=-2mm},
colframe=blue!50!black,
colback=blue!10
}

\usepackage{amsmath}
\usepackage{amssymb}
\usepackage{verbatim}
\usepackage[utf8]{inputenc}
\linespread{1.2}

\newtheorem{definition}{Definition}
\newtheorem{proposition}{Proposition}
\newtheorem{theorem}{Theorem}
\newtheorem{question}{Question}

\title{Math 118C HW3}
\author{Zih-Yu Hsieh}

\begin{document}
\maketitle

\section*{1}
\begin{myBox}[]{}
    \begin{question}
        Rudin Pg. 241 Problem 19:

        Show that the system of equations
        $$3x+y-z+u^2=0$$
        $$x-y+2z+u=0$$
        $$2x+2y-3z+2u=0$$
        can be solved for $x,y,u$ in terms of $z$; for $x,z,u$ in terms of $y$; for $y,z,u$ in terms of $x$; but not for $x,y,z$ in terms of $u$.
    \end{question}
\end{myBox}

\textbf{Pf:}

\break

\section*{2}
\begin{myBox}[]{}
    \begin{question}
        Rudin Pg. 242 Problem 23:

        Define $f$ in $\mathbb{R}^3$ by
        $$f(x,y_1,y_2)=x^2y_1+e^x+y_2$$
        Show that $f(0,1,-1)=0$,$(D_1f)(0,1,-1)\neq 0$, and that there exists therefore a differentiable function $g$ in some neighborhood of $(1,-1)$ in $\mathbb{R}^2$,
        such that $g(1,-1)=0$ and 
        $$f(g(y_1,y_2),y_1,y_2)=0$$
        Find $(D_1g)(1,-1)$ and $(D_2g)(1,-1)$.
    \end{question}
\end{myBox}

\textbf{Pf:}

\break

\section*{3}
\begin{myBox}[]{}
    \begin{question}
        Rudin Pg. 242 Problem 24:

        For $(x,y)\neq (0,0)$, define $f=(f_1,f_2)$ by 
        $$f_1(x,y)=\frac{x^2-y^2}{x^2+y^2},\quad f_2(x,y)=\frac{xy}{x^2+y^2}$$
        Compute the rank of $f'(x,y)$, and find the range of $f$.
    \end{question}
\end{myBox}

\textbf{Pf:}

\break

\section*{4}
\begin{myBox}[]{}
    \begin{question}
        Rudin Pg. 242 Problem 25:

        Suppose $A\in\mathcal{L}(\mathbb{R}^n,\mathcal{R}^m)$, let $r$ be the rank of $A$.
        \begin{itemize}
            \item[(a)] Define $S$ as in the proof of Theorem $9.32$. Show that $SA$ is a projection in $\mathbb{R}^n$ whose null space is $\textmd{null}(A)$ and whose range is $\textmd{range}(S)$.
            \item[(b)] Use (a) to show that 
            $$\dim(\textmd{null}(A)+\dim(\textmd{range}(A))=n$$
        \end{itemize}
    \end{question}
\end{myBox}

\textbf{Pf:}
\begin{itemize}
    \item[(a)] Given that $A$ has rank $r$, then its range $\textmd{range}(A)\subseteq \mathbb{R}^m$ is an $r$-dimensional linear subspace, hence there exists $y_1,...,y_r\in \textmd{range}(A)$ that forms a basis of it.
    
    Then, by the text in Rudin, choose $z_1,...,z_r\in\mathbb{R}^n$, so for each index $i\in \{1,...,r\}$, $Az_i=y_i$. Which, the collection $z_1,...,z_r\in\mathbb{R}^n$ is linearly independent, since if $a_1,...,a_r\in\mathbb{R}$ satisfies $\sum_{i=1}^{r}a_iz_i = \bar{0}$, then the following is true:
    $$A\left(\sum_{i=1}^{r}a_iz_i\right) = \sum_{i=1}^{r}a_i(Az_i) = \sum_{i=1}^{r}a_iy_i$$
    By the linear independence of $y_1,...,y_r\in \textmd{range}(A)$, each $a_i = 0$, which proves the linear independence of $z_1,...,z_r\in\mathbb{R}^n$.

    Finally, define $S\in\mathcal{L}(\textmd{range}(A),\mathbb{R}^n)$ the same as in the text, which has the following formula:
    $$\forall c_1,...,c_r\in\mathbb{R},\quad S\left(\sum_{i=1}^{r}c_iy_i\right)=\sum_{i=1}^{r}c_iz_i$$
    \item[(b)] 
\end{itemize}

\break

\section*{5}
\begin{myBox}[]{}
    \begin{question}
        Rudin Pg. 242 Problem 26:

        Show that the existence (and even the continuity) of $D_{12}f$ does not imlpy the existence of $D_1f$. For example, let $f(x,y)=g(x)$, where $g$ is nowhere differentiable.
    \end{question}
\end{myBox}

\textbf{Pf:}

Consider the Weierstrass Functon $g:\mathbb{R}\rightarrow\mathbb{R}$, which is uniformly continuous, while being differentiable nowhere.

Then, given the function $f:\mathbb{R}^2\rightarrow\mathbb{R}$ by $f(x,y)=g(x)$, since $g$ is not differentiable with respect to its variable $x$, then $D_1f$ does not exist; yet, since $D_2f \equiv 0$ (due to the fact that $g$ is a constant when $x$ is fixed), then $D_{12}f = D_1(D_2f) = 0$.

Hence, even though $D_{12}f$ is continuous, $D_1f$ doesn't exist in this case.

\end{document}