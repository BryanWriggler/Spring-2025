\documentclass[20pt,,margin=1in,innermargin=-4.5in,blockverticalspace=-0.25in]{tikzposter}
\geometry{paperwidth=43in,paperheight=32.5in}
\usepackage[utf8]{inputenc}
\usepackage{amsmath}
\usepackage{amsfonts}
\usepackage{amsthm}
\usepackage{amssymb}
\usepackage{mathrsfs}
\usepackage{graphicx}
\usepackage{adjustbox}
\usepackage{enumitem}
\usepackage{wrapfig}
\usepackage[backend=biber,style=numeric]{biblatex}
\usepackage{SUtheme}

\usepackage{mwe} % for placeholder images

\usepackage{comment}

\addbibresource{refs.bib}

% set colored boxes
\usepackage[most]{tcolorbox}
\newtcolorbox{definitionBox}{
  colback=blue!10,
  colframe=blue!50!black,
  fonttitle=\bfseries,
  title=Definition,
  arc=3mm,
  boxrule=0.8pt
}

\newtcolorbox{propertyBox}{
  colback=teal!10,
  colframe=teal!50!black,
  fonttitle=\bfseries,
  title=Property,
  arc=3mm,
  boxrule=0.8pt
}

\newtcolorbox{exampleBox}{
  colback=cyan!10,
  colframe=cyan!50!black,
  fonttitle=\bfseries,
  title=Example,
  arc=3mm,
  boxrule=0.8pt
}

% set theme parameters
\tikzposterlatexaffectionproofoff
\usetheme{SUTheme}
\usecolorstyle{SUStyle}
\usetitlestyle{Filled}

\usepackage[scaled]{helvet}
\renewcommand\familydefault{\sfdefault} 
\usepackage[T1]{fontenc}


\title{Lie Algebra of a Lie Group}
\author{Zih-Yu Hsieh, mentored by Arthur Jiang}
\institute{University of California Santa Barbara}
\titlegraphic{\includegraphics[width=0.06\textwidth]{logo.png}}

% begin document
\begin{document}
\maketitle
\centering
\begin{columns}
    \column{0.32}
    \block{Tangent Vectors as Derivations}{
        When embedding smooth manifolds into Euclidean Space, tangent vectors are associated with directional derivatives.

        \textbf{Insert geometric object}

        To generalize such notion into abstract smooth manifold, we need an analogy:
        
        \begin{definitionBox}
            Any point $u\in M$, a \textbf{Derivation at $u$}, is a linear map $v_u:C^\infty(M)\rightarrow\mathbb{R}$, that satisfies the product rule:
            $$\forall f,g\in C^\infty(M),\quad v_u(fg) = f(u)(v_u g)+g(u)(v_u f)$$ 
            Which, the set of all derivations at $u$, denoted as $T_u(M)$, is the \textbf{Tangent Space} of $M$ at $u$, and each derivation $v_u\in T_u(M)$ is a \textbf{Tangent Vector} of $u$.
        \end{definitionBox}
    }
    \block{Vector Fields \& Smooth Conditions}{
        Given smooth manifold $M$, a vector field $X$ is a function assigning each point $u\in M$ with a tangent vector of $u$. More precisely:
        \begin{definitionBox}
            a vector field is a map $X:M\rightarrow TM$ (where $TM$ denotes the \textbf{Tangent Bundle} of $M$), with $X(u) = X_u\in T_u(M)$.

            Which, $X$ is a \textbf{Smooth Vector Field}, if $X:M\rightarrow TM$ is a smooth map. 
            
            A collection of smooth vector fields on $M$ is denoted as $\mathfrak{X}(M)$, which is an $\mathbb{R}$-vector space.
        \end{definitionBox}
        
        \textbf{insert image}

        An equivalent condition of saying a vector field $X$ is smooth, is through smooth functions $f\in C^\infty(M)$: Since for all $u\in M$, $X(u)= X_u\in T_u(M)$ is a derivation at $u$, define $Xf:M\rightarrow\mathbb{R}$ by $Xf(u) = X_u(f)$. Then, $X$ is a smooth vector field iff $Xf\in C^\infty(M)$.

        Based on such condition, a smooth vector field is also a \textbf{Derivation:}
        \begin{propertyBox}
            For all $f,g\in C^\infty(M)$, given $X\in\mathfrak{X}(M)$, any $u\in M$ satisfies product rule:
            $$X(fg)(u) = X_u(fg) = f(u)(X_ug) + g(u)(X_uf) = f(u)Xg(u)+g(u)Xf(u)$$
            $$\implies X(fg) = f(Xg) + g(Xf)$$
        \end{propertyBox}
    }
    \block{Vector Fields of Different Manifolds}{
        Given $M,N$ two smooth manifolds, and smooth map $F:M\rightarrow N$. Let $X\in\mathfrak{X}(M)$, it would be ideal if we can send vector field $F$ maps $X$ to a vector field of $N$. Yet, this requires $F$ to be both injective and surjective:
        
        \textbf{Insert an example for both injectivity and surjectivity}

        So, we'll consider a weaker notion: 
        \begin{definitionBox}
            Given $X\in\mathfrak{X}(M)$ and $Y\in\mathfrak{X}(N)$, the two are $F$-related, if for all $u\in M$, the following is true:
            $$dF_u(X_u) = Y_{F(u)}$$
            Simply speaking, $F$ maps the tangent vectors collected by $X$, to be compatible with tangent vectors collected by $Y$.
        \end{definitionBox}

        \textbf{Insert another example}

        %\textbf{Thm:} If $F$ is a diffeomorphism, then for every $X\in\mathfrak{X}(M)$, there exists a unique $Y\in\mathfrak{X}(N)$, such that $X$ and $Y$ are $F$-related.
    }

    \column{0.36}

    \block{Lie Brackets of Vector Fields}{
        The initial motivation is to combine two vector fields $X,Y\in \mathfrak{X}(M)$ to be another vector field. Which, for all $f\in C^\infty(M)$, since $Yf\in C^\infty(M)$, then $XYf = X(Yf)\in C^\infty(M)$. But, in general $XY$ is not a derivation, hence not a vector field:
        \begin{exampleBox}
            Define vector fields $X=\frac{\partial}{\partial x}$, $y=x\frac{\partial}{\partial y}$ on $\mathbb{R}^2$. Take smooth functions $f(x,y)=x$ and $g(x,y)=y$, then we get the following:
            $$XY(fg) = X(x\frac{\partial}{\partial y}(xy)) = \frac{\partial}{\partial x}(x^2) = 2x$$
            But, product rule doesn't hold for this example:
            $$f(XY g)+g(XY f)=x(X(x\frac{\partial}{\partial y}(y))) + y(X(x\frac{\partial}{\partial y}(x))) = x$$
        \end{exampleBox}
        So, we need to define a new operation on vector fields: 
        \begin{definitionBox}
            The \textbf{Lie Bracket} $[\cdot,\cdot]:\mathfrak{X}(M)\times \mathfrak{X}(M)\rightarrow \mathfrak{X}(M)$, is defined as:
        $$\forall X,Y\in\mathfrak{X}(M),\quad [X,Y]=XY-YX$$
        Which, the output $[X,Y]\in\mathfrak{X}(M)$, and also satisfies these properties:
        \begin{center}
            \begin{itemize}
                \item $\textbf{Bilinearity:}\quad [aX+bY,Z]=a[X,Z]+b[Y,Z]$
                \item $\textbf{Antisymmetry:}\quad [X,Y]=-[Y,X]$
                \item $\textbf{Jacobi's Identity:}\quad \left[X,[Y,Z]\right]+ \left[Y,[Z,X]\right]+ \left[Z,[X,Y]\right]=0$
            \end{itemize}
        \end{center}
        \end{definitionBox}
        Moreover, Lie Bracket inherits relation of smooth maps:
        \begin{propertyBox}
            Given smooth map $F:M\rightarrow N$, if $X_1,X_2\in\mathfrak{X}(M)$ and $Y_1,Y_2\in\mathfrak{X}(N)$ are $F$-related respectively, then $[X_1,X_2]\in \mathfrak{X}(M)$ and $[Y_1,Y_2]\in\mathfrak{X}(N)$ are also $F$-related. 
        \end{propertyBox}
    }
    \block{Lie Group \& Left-Invariant Vector Fields}{
        The initial motivation, is to study group structures occuring in specific classes of smooth manifolds.
        
        \begin{definitionBox}
            A \textbf{Lie Group} $G$, is a smooth manifold along with group structure, such that the group operation $P:G\times G\rightarrow G$ by $P(g,h) = gh$, and the inversion map $i:G\rightarrow G$ by $i(g)=g^{-1}$ are both smooth maps between manifolds.
        \end{definitionBox}

        For all $g\in G$, denote the left multiplication $L_g:G\rightarrow G$ by $L_g(h)=gh$,
        since $L_g = P\bigm|_{\{g\}\times G}$, all left multiplication is a smooth map; also, since $L_{g^{-1}}$ is a smooth inverse of $L_g$, it's a \textbf{Diffeomorphism}. 
        Since $L_g$ is a diffeomorphism, there's a notion of $X$ being $L_g$-related to itself:

        \begin{definitionBox}
            Given any $X\in\mathfrak{X}(G)$ and all $g\in G$, $X$ is a \textbf{Left-Invariant Vector Field}, if for all $g\in G$, $X$ is $L_g$-related to itself.

            The collection of Left-Invariant vector fields $\mathfrak{g}\subseteq \mathfrak{X}(G)$, is a linear subspace.
        \end{definitionBox}

        Recall that Lie Bracket of vector field preserves an $F$-relation between manifolds. Which, such property exists for Lie Groups:
        \begin{propertyBox}
            For all $X,Y\in\mathfrak{X}(G)$ that are left-invariant, since for all $g\in G$, $X$ and $Y$ are $L_g$ related to themselves, then the Lie Bracket $[X,Y]$ is also $L_g$ related to $[X,Y]$. Hence, $[X,Y]$ is also left-invariant, so Left-Invariant vector fields $\mathfrak{g}$ is closed under Lie Bracket's operation.
        \end{propertyBox}
    }

    \column{0.32}
    \block{Lie Algebra on a Lie Group}{
        \begin{definitionBox}
            Given a vector space $\mathfrak{g}$ over $\mathbb{R}$ or $\mathbb{C}$, associates with a binary operation $[\cdot,\cdot]:\mathfrak{g}\times \mathfrak{g}\rightarrow \mathfrak{g}$, such that the following holds:
            \begin{itemize}
                \item \textbf{Bilinearity:} $[aX+bY,Z]=a[X,Z]+b[Y,Z]$
                \item \textbf{Antisymmetry:} $[X,Y]=-[Y,X]$
                \item \textbf{Jacobi's Identity:} $\left[X,[Y,Z]\right]+\left[Y,[Z,X]\right]+\left[Z,[X,Y]\right]=0$
            \end{itemize}
            Then, the pair $(\mathfrak{g},[\cdot,\cdot])$ is a \textbf{Lie Algebra}.
        \end{definitionBox}
        In general, Lie Algebra is non-associative, which Jacobi's Identity is an alternative condition for Lie Algebra.

        Finally, we can define \textbf{Lie Algebra of a Lie Group:}
        \begin{definitionBox}
            Given a lie group $G$, since the subset of left-invariant vector fields $\mathfrak{g}\subseteq \mathfrak{X}(G)$ forms a linear subspace, while closed under Lie Bracket's operation, then the pair $(\mathfrak{g},[\cdot,\cdot])$ forms a \textbf{Lie Algebra}, denoted as $\textmd{Lie}(G)$.
        \end{definitionBox}
    }

    \block{Example}{

    }
    
   \block{Acknowledgements}{
        I want to thank my mentor Arthur Jiang for the effort and kindness of guiding me through the materials, and provide helpful information on the project. I'd also like to thank the UCSB Math Department Directed Reading Program for this opportunity.
    }
    
    \block{References}{
        John M. Lee, Introduction to Smooth Manifolds, 2nd Edition
    }
\end{columns}
\end{document}