\documentclass[20pt,,margin=1in,innermargin=-4.5in,blockverticalspace=-0.25in]{tikzposter}
\geometry{paperwidth=43in,paperheight=32.5in}
\usepackage[utf8]{inputenc}
\usepackage{amsmath}
\usepackage{amsfonts}
\usepackage{amsthm}
\usepackage{amssymb}
\usepackage{mathrsfs}
\usepackage{graphicx}
\usepackage{adjustbox}
\usepackage{enumitem}
\usepackage{wrapfig}
\usepackage[backend=biber,style=numeric]{biblatex}
\usepackage{SUtheme}

\usepackage{mwe} % for placeholder images

\usepackage{comment}

\addbibresource{refs.bib}

% set theme parameters
\tikzposterlatexaffectionproofoff
\usetheme{SUTheme}
\usecolorstyle{SUStyle}
\usetitlestyle{Filled}

\usepackage[scaled]{helvet}
\renewcommand\familydefault{\sfdefault} 
\usepackage[T1]{fontenc}


\title{Lie Algebra of a Lie Group}
\author{Zih-Yu Hsieh, mentored by Arthur Jiang}
\institute{University of California Santa Barbara}
\titlegraphic{\includegraphics[width=0.06\textwidth]{logo.png}}

% begin document
\begin{document}
\maketitle
\centering
\begin{columns}
    \column{0.32}
    \block{Tangent Space, Tangent Vectors and Derivations}{
        In simplest case, if embedd manifold $M^n$ into $\mathbb{R}^m$, for any chart $(U,\phi)$ of $M$, since $\phi:U\rightarrow\phi(U)\subseteq \mathbb{R}^n$ has its inverse $\phi^{-1}$ being smooth, for any $u\in U\subseteq M$, a tangent vector $v_u$ associates with vector $v\in\mathbb{R}^n$, is characterized by differential of $\phi^{-1}$:
        $$v_u := D\phi^{-1}(\phi(u))(v) = \lim_{t\rightarrow 0}\frac{\phi^{-1}(\phi(u)+tv)-\phi^{-1}(\phi(u))}{t}$$
        A collection of all such vector is the \textbf{Geometric Tangent Space} of $u$, denoted as $T_u(M)$.

        \textbf{insert image}

        Notice that for any smooth function $f\in C^\infty(M)$, it has a notion of directional derivative at $u$ depending on the tangent vector $v_u\in T_u(M)$, and such derivative satisfies genral differentiation rules (for instance, product rule). 
        
        \hfil

        To generalize such notion into abstract manifold (space with no definition of vectors), we need a notion of \textbf{Derivation}: For any point $u\in M$, a \textbf{Derivation at $u$}, is a linear map $v_u:C^\infty(M)\rightarrow\mathbb{R}$, that satisfies the product rule:
        $$\forall f,g\in C^\infty(M),\quad v_u(fg) = f(u)(v_u g)+g(u)(v_u f)$$ 
        Which, the set of all derivations at $u$, denoted as $T_u(M)$, is the \textbf{Tangent Space} of $M$ at $u$, and each derivation $v_u\in T_u(M)$ is called the \textbf{Tangent Vector} of $u$.
    }
    \block{Vector Fields \& Smooth Conditions}{
        Given smooth manifold $M$, a vector field $X$ is a function associating each point $u\in M$ with a tangent vector of $u$, so $X(u)\in T_u(M)$. More precisely, a vector field is a map $X:M\rightarrow TM$ (where $TM$ denotes the \textbf{Tangent Bundle} of $M$), such that with the canonical projection map $\pi:TM\rightarrow M$, $\pi\circ X:M\rightarrow M$ is an identity.

        Which, $X$ is a \textbf{Smooth Vector Field}, if $X:M\rightarrow TM$ is a smooth map. And, a collection of smooth vector fields on $M$ is denoted as $\mathfrak{X}(M)$.
        
        \textbf{insert image}

        An equivalent condition of saying a vector field $X$ is smooth, is through smooth functions $f\in C^\infty(M)$: Since for all $u\in M$, $X(u)= X_u\in T_u(M)$ is a derivation at $u$, define $Xf:M\rightarrow\mathbb{R}$ by $Xf(u) = X_u(f)$. Then, $X$ is a smooth vector field iff $Xf\in C^\infty(M)$.
    }
    \block{Vector Fields of Different Manifolds}{
        Given $M,N$ two smooth manifolds, and smooth map $F:M\rightarrow N$. Let $X\in\mathfrak{X}(M)$, it would be ideal if we can send vector field $X$ to be a vector field of $N$. Yet, this requires both injectivity and surjectivity, which is too much to assume.
        
        \textbf{Insert an example}

        So, we'll consider a weaker notion, called an \textbf{$F$-Relation}: Given $X\in\mathfrak{X}(M)$ and $Y\in\mathfrak{X}(N)$, the two are $F$-related, if for all $u\in M$, the following is true:
        $$dF_u(X_u) = Y_{F(u)}$$
        Simply speaking, $F$ maps the tangent vectors collected by $X$, to be compatible with tangent vectors collected by $Y$.

        \textbf{Insert another example}
    }

    \column{0.36}

    \block{Lie Brackets on Vector Fields}{
        The initial motivation is to combine two vector fields $X,Y\in \mathfrak{X}(M)$ to be another vector field. Which, for all $f\in C^\infty(M)$, since $Yf\in C^\infty(M)$ from previous characterization, then $XYf = X(Yf)\in C^\infty(M)$. But, if consider function $XY$, it is possibly not 
    }

    \column{0.32}
    \block{Surfaces}{
        
    }
    
   \block{Acknowledgements}{

    }
    
    \block{References}{
    }
\end{columns}
\end{document}