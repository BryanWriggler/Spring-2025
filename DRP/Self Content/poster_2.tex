\documentclass[20pt,margin=0.9in,innermargin=-4.5in,blockverticalspace=-0.25in]{tikzposter}
\geometry{paperwidth=43in,paperheight=32.5in}
\usepackage[utf8]{inputenc}
\usepackage{amsmath}
\usepackage{amsfonts}
\usepackage{amsthm}
\usepackage{amssymb}
\usepackage{mathrsfs}
\usepackage{graphicx}
\usepackage{adjustbox}
\usepackage{enumitem}
\usepackage{wrapfig}
\usepackage[backend=biber,style=numeric]{biblatex}
\usepackage{SUtheme}

\usepackage{mwe} % for placeholder images

\usepackage{comment}

\addbibresource{refs.bib}

% set colored boxes
\usepackage[most]{tcolorbox}
\newtcolorbox{definitionBox}{
  colback=blue!10,
  colframe=blue!50!black,
  fonttitle=\bfseries,
  title=Definition,
  arc=3mm,
  boxrule=0.8pt
}

\newtcolorbox{theoremBox}{
  colback=teal!10,
  colframe=teal!50!black,
  fonttitle=\bfseries,
  title=Theorem,
  arc=3mm,
  boxrule=0.8pt
}

\newtcolorbox{exampleBox}{
  colback=cyan!10,
  colframe=cyan!50!black,
  fonttitle=\bfseries,
  title=Example,
  arc=3mm,
  boxrule=0.8pt
}

% set theme parameters
\tikzposterlatexaffectionproofoff
\usetheme{SUTheme}
\usecolorstyle{SUStyle}
\usetitlestyle{Filled}

\usepackage[scaled]{helvet}
\renewcommand\familydefault{\sfdefault} 
\usepackage[T1]{fontenc}

\linespread{0.84}

\title{Lie Algebra of a Lie Group}
\author{Zih-Yu Hsieh \quad\quad Mentor: Arthur Jiang}
\institute{University of California Santa Barbara}
\titlegraphic{\includegraphics[width=0.06\textwidth]{logo.png}}

% begin document
\begin{document}
\maketitle
\centering
\begin{columns}
    \column{0.32}
    \block{Tangent Vectors as Derivations}{
        \vspace*{-1em}
        When embedding smooth manifolds into $\mathbb{R}^n$, tangent vectors are associated with directional derivatives. To generalize tangent vectors into abstract smooth manifold, we need an analogy:
        
        \begin{definitionBox}
            Any point $u\in M$, a \textbf{Derivation at $u$}, is a linear map $v_u:C^\infty(M)\rightarrow\mathbb{R}$, that satisfies the product rule:
            $$\forall f,g\in C^\infty(M),\quad v_u(fg) = f(u)(v_u g)+g(u)(v_u f)$$ 
            The vector space of all derivations at $u$, or $T_u(M)$, is the \textbf{Tangent Space} of $M$ at $u$, and each derivation $v_u\in T_u(M)$ is a \textbf{Tangent Vector} of $u$.
        \end{definitionBox}
    }
    \block{Vector Fields \& Smooth Condition}{
        \vspace*{-1em}
        \begin{definitionBox}
            a vector field is a map $X:M\rightarrow TM$ ($TM$ denotes the \textbf{Tangent Bundle}), with $X(u) = X_u\in T_u(M)$.

            Which, $X$ is a \textbf{Smooth Vector Field}, if $X:M\rightarrow TM$ is a smooth map. 
            
            A collection of smooth vector fields on $M$ is $\mathfrak{X}(M)$, which is an $\mathbb{R}$-vector space.
        \end{definitionBox}

        Another equivalent condition of saying $X$ is smooth, is through smooth functions $f\in C^\infty(M)$: For all $u\in M$, $X(u)= X_u\in T_u(M)$ is a derivation at $u$, define $Xf:M\rightarrow\mathbb{R}$ by $Xf(u) = X_u(f)$. Which, the \textbf{Derivation} is an equivalent condition for smooth vector field:
        \begin{theoremBox}
            Given vector field $X$, $X\in\mathfrak{X}(M)$ iff it satisfies product rule. i.e. For all $u\in M$, and all $f,g\in C^\infty(M)$:
            $$X(fg)(u) = X_u(fg) = f(u)(X_ug) + g(u)(X_uf) = f(u)Xg(u)+g(u)Xf(u)$$
            $$\implies X(fg) = f(Xg) + g(Xf)$$
        \end{theoremBox}
    }
    \block{Vector Fields of Different Manifolds}{
        \vspace*{-1em}
        Given $M,N$ two smooth manifolds, and smooth map $F:M\rightarrow N$. Let $X\in\mathfrak{X}(M)$, an ideal situation is mapping $X$ to a smooth vector field of $N$ through $F$. Yet, this requires $F$ to be bijective:
        
        \begin{center}
            \includegraphics[width=0.24\textwidth]{example 1.png} 

            \textbf{Figure 1:} Example of Conflicting Tangent Vectors
        \end{center}

        \hfil

        So, we'll consider a weaker notion: 
        \begin{definitionBox}
            Given $X\in\mathfrak{X}(M)$ and $Y\in\mathfrak{X}(N)$, the two are $\boldsymbol{F}$\textbf{-related}, if for all $u\in M$, the following is true:
            $$dF_u(X_u) = Y_{F(u)}$$
            Simply speaking, $F$ maps the tangent vectors collected by $X$, to be compatible with tangent vectors collected by $Y$.
        \end{definitionBox}

        \begin{center}
            \includegraphics[width=0.24\textwidth]{example 2.png} 

            \textbf{Figure 2:} A demonstration of $F$-Relation
        \end{center}
    }

    \column{0.36}

    \block{Lie Bracket of Vector Fields}{
        \vspace*{-1em}
        The initial motivation is to combine two vector fields $X,Y\in \mathfrak{X}(M)$ to be another vector field. For all $f\in C^\infty(M)$, since $Yf\in C^\infty(M)$, then $XYf := X(Yf)\in C^\infty(M)$. But, in general $XY$ is not a derivation, hence not a vector field:
        \begin{exampleBox}
            Define vector fields $X=\frac{\partial}{\partial x}$, $Y=x\frac{\partial}{\partial y}$ on $\mathbb{R}^2$. Take smooth functions $f(x,y)=x$ and $g(x,y)=y$, then we get the following:
            $$XY(fg) = X\left(x\frac{\partial}{\partial y}(xy)\right) = \frac{\partial}{\partial x}(x^2) = 2x$$
            But, product rule doesn't hold for this example:
            $$f(XY g)+g(XY f)=x\left(X\left(x\frac{\partial}{\partial y}(y)\right)\right) + y\left(X\left(x\frac{\partial}{\partial y}(x)\right)\right) = x$$
        \end{exampleBox}
        So, we need to define a new operation on vector fields: 
        \begin{definitionBox}
            The \textbf{Lie Bracket} $[\cdot,\cdot]:\mathfrak{X}(M)\times \mathfrak{X}(M)\rightarrow \mathfrak{X}(M)$, is defined as:
        $$\forall X,Y\in\mathfrak{X}(M),\quad [X,Y]=XY-YX$$
        Which, the output $[X,Y]\in\mathfrak{X}(M)$, since it satisfies product rule:
        $$[X,Y](fg) = X(Y(fg))-Y(X(fg))= X(f(Yg)+g(Yf))-Y(f(Xg)+g(Xf))$$
        $$ = f(XYg)+(Yg)(Xf)+g(XYf)+(Yf)(Xg)-f(YXg)-(Xg)(Yf)-g(YXf)-(Xf)(Yg)$$
        $$ = f(XYg-YXg)+g(XYf-YXf) = f[X,Y](g)+g[X,Y](f)$$
        Lie Bracket also satisfies these properties:
        \begin{itemize}
            \item \textbf{Bilinearity:} $[aX+bY,Z]=a[X,Z]+b[Y,Z]$
            \item \textbf{Antisymmetry:} $[X,Y]=-[Y,X]$
            \item \textbf{Jacobi's Identity:} $\left[X,[Y,Z]\right]+ \left[Y,[Z,X]\right]+ \left[Z,[X,Y]\right]=0$
        \end{itemize}
        \end{definitionBox}
        Moreover, Lie Bracket inherits relation of smooth maps:
        \begin{theoremBox}
            Given smooth map $F:M\rightarrow N$, if $X_1,X_2\in\mathfrak{X}(M)$ and $Y_1,Y_2\in\mathfrak{X}(N)$ are $F$-related respectively, then $[X_1,X_2]\in \mathfrak{X}(M)$ and $[Y_1,Y_2]\in\mathfrak{X}(N)$ are also $F$-related. 
        \end{theoremBox}
    }
    \block{Lie Groups \& Left-Invariant Vector Fields}{
        \vspace*{-1em}
        The initial motivation is to study group structures in some smooth manifolds.
        
        \begin{definitionBox}
            A \textbf{Lie Group} $G$, is a smooth manifold along with group structure, such that the group operation $P:G\times G\rightarrow G$ by $P(g,h) = gh$, and the inversion map $i:G\rightarrow G$ by $i(g)=g^{-1}$ are both smooth maps between manifolds.
        \end{definitionBox}

        For all $g\in G$, denote the left multiplication $L_g:G\rightarrow G$ by $L_g(h)=gh$,
        since $L_g = P\bigm|_{\{g\}\times G}$, it is a smooth map. Hence, there's a notion of $X$ being $L_g$-related to itself:

        \begin{definitionBox}
            Given any $X\in\mathfrak{X}(G)$ and all $g\in G$, $X$ is a \textbf{Left-Invariant Vector Field}, if for all $g\in G$, $X$ is $L_g$-related to itself. Which, for all $g\in G$: 
            $$d(L_g)_e(X_e) = X_{L_g(e)} = X_g$$ 
            So, $X$ is uniquely determined by its tangent vector at identity, $X_e\in T_e(G)$. In fact, each $v_e\in T_e(G)$ also corresponds to a unique Left-Invariant vector field.

            The collection of Left-Invariant vector fields $\mathfrak{g}\subseteq \mathfrak{X}(G)$, is itself a linear subspace, and $\mathfrak{g}\cong T_e(G)$ as vector spaces, based on the above relation.
            
            %Also, as vector spaces, $\mathfrak{g}\cong T_e(G)$.
        \end{definitionBox}

        Recall that Lie Bracket of vector field preserves $F$-relation between manifolds, so:
        \begin{theoremBox}
            For all $X,Y\in\mathfrak{g}$, since for all $g\in G$, $X$ and $Y$ are $L_g$ related to themselves, then the Lie Bracket $[X,Y]$ is also $L_g$-related to $[X,Y]$. Hence, $[X,Y]$ is also left-invariant, or $[X,Y]\in \mathfrak{g}$. So, $\mathfrak{g}$ is closed under Lie Bracket's operation.
        \end{theoremBox}
    }

    \column{0.32}
    \block{Lie Algebra on a Lie Group}{
        \vspace*{-1em}
        \begin{definitionBox}
            Given a vector space $\mathfrak{g}$ over $\mathbb{R}$ or $\mathbb{C}$, with a binary operation $[\cdot,\cdot]:\mathfrak{g}\times \mathfrak{g}\rightarrow \mathfrak{g}$, such that the following holds:
            \begin{itemize}
                \item \textbf{Bilinearity:} $[aX+bY,Z]=a[X,Z]+b[Y,Z]$
                \item \textbf{Antisymmetry:} $[X,Y]=-[Y,X]$
                \item \textbf{Jacobi's Identity:} $\left[X,[Y,Z]\right]+\left[Y,[Z,X]\right]+\left[Z,[X,Y]\right]=0$
            \end{itemize}
            Then, the pair $(\mathfrak{g},[\cdot,\cdot])$ is a \textbf{Lie Algebra}.
        \end{definitionBox}
        In general, Lie Algebra is non-associative, so Jacobi's Identity is an alternative condition. Finally, we can define \textbf{Lie Algebra of a Lie Group:}
        \begin{definitionBox}
            Given a lie group $G$, since the subset of left-invariant vector fields $\mathfrak{g}\subseteq \mathfrak{X}(G)$ forms a linear subspace, while closed under Lie Bracket's operation, then the pair $(\mathfrak{g},[\cdot,\cdot])$ forms a \textbf{Lie Algebra} of $G$, denoted as $Lie(G)$.
        \end{definitionBox}

        Here's an example of Lie Algebra on a Lie Group:

        \begin{exampleBox}
            \textbf{General Linear Group \& its Lie Algebra:}
            
            Given $M_n(\mathbb{R})\cong \mathbb{R}^{n^2}$, and $GL_n(\mathbb{R})\subset M_n(\mathbb{R})$ as an open subset, it's a natural smooth manifold with dimension $n^2$. The product of matrices and the inversion are smooth maps, so $GL_n(\mathbb{R})$ is a Lie Group.

            Now, consider $\mathfrak{g}=Lie(GL_n(\mathbb{R}))$: Each $X\in\mathfrak{g}$ is uniquely characterized by $X_{I_n}\in T_{I_n}(GL_n(\mathbb{R}))$. And, as vector spaces, $\mathfrak{g}\cong T_{I_n}(GL_n(\mathbb{R}))$.

            \hfil

            \textbf{Lie Algebra on $M_n(\mathbb{R})$:}

            Given $M_n(\mathbb{R})$ as $\mathbb{R}$-vector space and the commutator $[A,B]=AB-BA$, the pair $(M_n(\mathbb{R}),[\cdot,\cdot])$ in fact forms a Lie Algebra, denoted as $\mathfrak{gl}_n(\mathbb{R})$.

            

            \hfil

            \textbf{Lie Algebra Isomorphism between $\mathfrak{g}$ and $\mathfrak{gl}_n(\mathbb{R})$:}

            $GL_n(\mathbb{R})$ has a global coordinate provided by $M_n(\mathbb{R})$, denote as $(X^i_j)_{1\leq i,j\leq n}$.
            
            For each $A\in \mathfrak{gl}_n(\mathbb{R})$, it corresponds to a tangent vector in $T_{I_n}(GL_n(\mathbb{R}))$:
            $$A = (A^i_j)\mapsto A^i_j\frac{\partial}{\partial X^i_j}\bigg|_{I_n}$$
            The above tangent vector defines a Left-Invariant vector field $A^L\in \mathfrak{g}$. For all $X\in \mathfrak{g}$, the left multiplication $L_X$ is in fact a linear operator on $M_n(\mathbb{R})$, so its differential is identical to itself. Which, it provides the following relation:
            $$A^L_X=d(L_X)_{I_n}\left(A^i_j\frac{\partial}{\partial X^i_j}\bigg|_{I_n}\right) = X^i_j A^j_k\frac{\partial}{\partial X^i_k}\bigg|_{X},\quad A^L=X^i_j A^j_k\frac{\partial}{\partial X^i_k}$$
            Then, for arbitrary $A,B\in\mathfrak{gl}_n(\mathbb{R})$, Lie Bracket of $A^L,\ B^L\in\mathfrak{g}$ generates:
            $$\left[A^L,B^L\right] = X^i_jA^j_k\frac{\partial}{\partial X^i_k}(X^p_qB^q_r)\frac{\partial}{\partial X^p_r}-X^p_qB^q_r\frac{\partial}{\partial X^p_r}(X^i_jA^j_k)\frac{\partial}{\partial X^i_k}$$
            Because each $A^j_k,\ B^q_r$ are constants, while $\frac{\partial}{\partial X^i_k}X^p_q = 1$ iff $(i,k)=(p,q)$ and is $0$ otherwise. Then, match up related indices, we get:
            $$\left[A^L,B^L\right] = X^i_j(A^j_kB^k_r-B^j_kA^k_r)\frac{\partial}{\partial X^i_r} = (AB-BA)^L=[A,B]^L$$
            Hence, the map $\mathfrak{gl}_n(\mathbb{R})\rightarrow \mathfrak{g}$ by $A\mapsto A^L$ is a Lie Algebra Isomorphism.
        \end{exampleBox}
    }

   \block{Acknowledgements \& Reference}{
        \vspace*{-1em}
        I really thank my mentor Arthur Jiang for the effort, guidance, and insights on the materials and this project, and also UCSB Math DRP for this opportunity. Finally, check out my peer \textbf{Siyu Chen's} cool poster about Lie Group and Lie Algebra's applications in physics!

        \textrm{\textbf{Reference:} Lee, J.M. \textit{Introduction to Smooth Manifolds}; 2nd ed.; Springer: New York, 2012; 9781441999825}
    }
\end{columns}
\end{document}