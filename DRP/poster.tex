\documentclass[30pt,margin=1in,innermargin=-4.5in,blockverticalspace=-0.25in]{tikzposter}
\geometry{paperwidth=43in,paperheight=32.5in}
\usepackage[utf8]{inputenc}
\usepackage{amsmath}
\usepackage{amsfonts}
\usepackage{amsthm}
\usepackage{amssymb}
\usepackage{mathrsfs}
\usepackage{graphicx}
\usepackage{adjustbox}
\usepackage{float}
\usepackage{enumitem}
\usepackage[backend=biber,style=numeric]{biblatex}
\usepackage{uwtheme}

\graphicspath{ {./images/} }

\usepackage{mwe} % for placeholder images

\addbibresource{refs.bib}

% set theme parameters
\tikzposterlatexaffectionproofoff
\usetheme{Rays}
\usecolorstyle{ColorOne}

\usepackage[scaled]{helvet}
\renewcommand\familydefault{\sfdefault} 
\usepackage[T1]{fontenc}


\title{Introduction to Lie Groups}
\author{Zih-Yu Hsieh, mentored by Arthur Jiang}
\institute{\textsuperscript{}University of California - Santa Barbara}
\titlegraphic{\includegraphics[width=0.6\textwidth]{logo.png}}

\makeatletter
\renewcommand\TP@maketitle{%
   \begin{minipage}{0.6\linewidth}
        \centering
        \color{titlefgcolor}
        {\bfseries \Huge \sc \@title \par}
        \vspace*{1em}
        {\huge \@author \par}
        \vspace*{1em}
        {\LARGE \@institute}
    \end{minipage}%
    \hfill
    \begin{minipage}{.2\linewidth}
       \centering
       \@titlegraphic
    \end{minipage}
}
\makeatother

% begin document
\begin{document}

\maketitle
\centering
\begin{columns}
    \column{0.32}    
    \block{History of Lie Groups}{
       Back in nineteenth century, the mathematician Sophus Lie was inspired by Galois' development of group theory and its application on analyzing polynomials. Which, he intended to develop symmetries in PDEs and geometry, specifically using group actions. With further development on manifolds, Lie's groundwork had become the current form of lie groups, a tool used to study specific structures of manifolds.
    }

    \block{Preliminaries - Smooth Manifold}{
        A manifold $M$ with dimension $n$, is a topological space that is locally like $\mathbb{R}^n$. More precisely, for any point $m\in M$, there exists an open neighborhood $U\subseteq M$ along with a homeomorphism $\phi:U\rightarrow \mathbb{R}^n$ (i.e. $\phi$ is a continuous bijective map, with its inverse $\phi^{-1}$ also being continuous).

        A smooth structure 
    }

    \block{Preliminaries - Group}{

    }

    \column{0.36}

    \block{An Example in Python}{
        
    }
    
    \block{Primality and Factorization}{ 

    }
    
    \column{0.32}

    \block{The Foundations of Modern Cryptography: Elliptic Curves}{

    }
    
    \block{Reference and Acknowledgements}{
        \textbf{Reference Material:} "Introduction to Smooth Manifold" by John M. Lee \\
        I want to thank my mentor Arthur Jiang for providing an initiation and constant support for this project.
    }
    
\end{columns}
\end{document}