\documentclass{article}
\usepackage{graphicx} % Required for inserting images
\usepackage[margin = 2.54cm]{geometry}
\usepackage[most]{tcolorbox}

\newtcolorbox{myBox}[3]{
arc=5mm,
lower separated=false,
fonttitle=\bfseries,
%colbacktitle=green!10,
%coltitle=green!50!black,
enhanced,
attach boxed title to top left={xshift=0.5cm,
        yshift=-2mm},
colframe=blue!50!black,
colback=blue!10
}

\usepackage{amsmath}
\usepackage{amssymb}
\usepackage{verbatim}
\usepackage[utf8]{inputenc}
\linespread{1.2}

\newtheorem{definition}{Definition}
\newtheorem{proposition}{Proposition}
\newtheorem{theorem}{Theorem}
\newtheorem{question}{Question}

\title{Math 118C HW4}
\author{Zih-Yu Hsieh}

\begin{document}
\maketitle

\section*{1}
\begin{myBox}[]{}
    \begin{question}
        Rudin Pg. 242 Problem 27:

        Put $f(0,0)=0$, and 
        $$f(x,y)=\frac{xy(x^2-y^2)}{x^2+y^2}$$
        if $(x,y)\neq (0,0)$. Prove that 
        \begin{itemize}
            \item[(a)] $f,\ D_1f,\ D_2f$ are continuous in $\mathbb{R}^2$.
            \item[(b)] $D_{12}f$ and $D_{21}f$ exist at every point of $\mathbb{R}^2$, and are continuous except at $(0,0)$.
            \item[(c)] $D_{12}f(0,0)=1$, and $D_{21}f(0,0)=-1$.  
        \end{itemize}
    \end{question}
\end{myBox}

\textbf{Pf:}

\break

\section*{2}
\begin{myBox}[]{}
    \begin{question}
        Rudin Pg. 242 Problem 28:

        For $t\geq 0$, put 
        $$\varphi(x,t)=\begin{cases}
            x & 0\leq x\leq \sqrt{t}\\
            -x+2\sqrt{t} & \sqrt{t}\leq x\leq 2\sqrt{t}\\
            0 & \textmd{otherwise}
        \end{cases}$$
        and put $\varphi(x,t)=-\varphi(x,|t|)$ if $t<0$.

        Show that $\varphi$ is continuous on $\mathbb{R}^2$, and $D_2\varphi(x,0)=0$ for all $x$. Define 
        $$f(t)=\int_{-1}^{1}\varphi(x,t)dx$$
        Show that $f(t)=t$ if $|t|<\frac{1}{4}$. Hence 
        $$f'(0)\neq \int_{-1}^{1}D_2\varphi(x,0)dx$$
    \end{question}
\end{myBox}

\textbf{Pf:}

\break

\section*{3}
\begin{myBox}[]{}
    \begin{question}
        Rudin Pg. 243 Problem 30:

        Let $f\in \mathcal{C}^{(m)}(E)$, where $E$ is an open subset of $\mathbb{R}^n$. Fix $a\in E$, and suppose $x\in\mathbb{R}^n$ is so close to $0$ that the points $p(t)=a+tx$ lie in $E$ whenever $0\leq t\leq 1$. Define $h(t)=f(p(t))$ for all $t\in\mathbb{R}$ for which $p(t)\in E$.
        \begin{itemize}
            \item[(a)] For $1\leq k\leq m$, show (by repeated application of the chain rule) that 
            $$h^{(k)}(t)=\sum (D_{l_1...l_k}f)(p(t))x_{l_1}...x_{l_k}$$
            The sum extends over all order $k$-tuples $(l_1,...,l_k)$ in which each $l_j$ is one of the integers $1,...,n$.
            \begin{comment}
            \item[(b)] By Taylor's Theorem:
            $$h(1)=\sum_{k=0}^{m-1}\frac{h^{(k)}(0)}{k!}+\frac{h^{(m)}(t)}{m!}$$
            for some $t\in (0,1)$. Use this to prove Taylor's Theorem in $n$ variables by showing that the formula 
            $$f(a+x)=\sum_{k=0}^{m-1}\frac{1}{k!}\left(\sum(D_{l_1...l_k}f)(a)x_{l_1}...x_{l_k}\right)+r(x)$$
            represents $f(a+x)$ as the sum of its so-called "Taylor polynomial of degree $m-1$" plus a remainder that satisfies 
            $$\lim_{x\rightarrow 0}\frac{r(x)}{|x|^{m-1}}=0$$
            Each of the inner sums extends over all ordered $k$-tuples $(l_1,...,l_k)$, as in part (a); as usual, the zero-order derivative of $f$ is simply $f$, so that the constant term of the Taylor polynomial of $f$ at $a$ is $f(a)$.
            \item[(c)] Exercise $29$ shows that repetition occurs in the Taylor polynomial as written in part (b). For instance, $D_{113}$ occurs three times, as $D_{113}, D_{131}, D_{311}$. The sum of the corresponding three terms can be written in the form 
            $$3(D_1^2D_3f)(a)x_1^2x_3$$
            Prove (by calculating how often each derivative occurs) that the Taylor polynomial in (b) can be written in the form 
            $$\sum\frac{(D_1^{s_1}...D_n^{s_n}f)(a)}{s_1!...s_n!}x_1^{s_1}...x_n^{s_n}$$
            Here the summation extends over all ordered $n$-tuples $(s_1,...,s_n)$ such that each $s_i$ is a nonnegative integer and $s_1+...+s_n\leq m-1$.
            \end{comment}
        \end{itemize}
    \end{question}
\end{myBox}

\textbf{Pf:}

\break

\section*{4}
\begin{myBox}[]{}
    \begin{question}
        Rudin Pg. 288 Problem 2:

        For $i=1,2,3,...$, let $\varphi_i\in \mathcal{C}(\mathbb{R})$ have support in $(2^{-i},2^{1-i})$, such that $\int\varphi_i=1$. Put 
        $$f(x,y)=\sum_{i=1}^{\infty}(\varphi_i(x)-\varphi_{i+1}(x))\varphi_i(y)$$
        Then $f$ has compact support in $\mathbb{R}^2$, $f$ is cotinuous except at $(0,0)$, and 
        $$\int dy\int f(x,y)dx = 0,\quad \textmd{but } \int dx\int f(x,y)dy = 1$$
        Observe that $f$ is unbounded in every neighborhood of $(0,0)$.
    \end{question}
\end{myBox}

\textbf{Pf:}

\end{document}