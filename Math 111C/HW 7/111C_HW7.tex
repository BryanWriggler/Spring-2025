\documentclass{article}
\usepackage{graphicx} % Required for inserting images
\usepackage[margin = 2.54cm]{geometry}
\usepackage[most]{tcolorbox}

\newtcolorbox{myBox}[3]{
arc=5mm,
lower separated=false,
fonttitle=\bfseries,
%colbacktitle=green!10,
%coltitle=green!50!black,
enhanced,
attach boxed title to top left={xshift=0.5cm,
        yshift=-2mm},
colframe=blue!50!black,
colback=blue!10
}

\usepackage{amsmath}
\usepackage{amssymb}
\usepackage{verbatim}
\usepackage[utf8]{inputenc}
\linespread{1.2}

\newtheorem{definition}{Definition}
\newtheorem{proposition}{Proposition}
\newtheorem{theorem}{Theorem}
\newtheorem{question}{Question}

\title{Math 111C HW7}
\author{Zih-Yu Hsieh}

\begin{document}
\maketitle

\section*{1}
\begin{myBox}[]{}
    \begin{question}
        Let $F\subseteq K\subseteq L$ be field extensions. Prove or disprove the following statements:
        \begin{itemize}
            \item[(i)] If $L/F$ is Galois, then so is $K/F$.
            \item[(ii)] If $L/K$ and $K/F$ are both Galois, then so is $L/F$. 
        \end{itemize}
    \end{question}
\end{myBox}

\textbf{Pf:}
\begin{itemize}
    \item[(i)] To disprove the first statement, here is a counterexample:
    
    Consider the three fields, $\mathbb{Q}\subseteq \mathbb{Q}(2^{1/3})\subseteq \mathbb{Q}(2^{1/3},\ \zeta_3)$, where $\zeta_3 = e^{2\pi i/3}$ (a $3^{rd}$ primitive root of unity).

    First, since $\textmd{char}(\mathbb{Q})=0$, $\mathbb{Q}$ is a perfect field, hence any finite extension of $\mathbb{Q}$ is separable. Then, since $\mathbb{Q}(2^{1/3},\ \zeta_3)/\mathbb{Q}$ is a finite extension, it is separable.

    Then, if we consider $2^{1/3}\in\mathbb{Q}(2^{1/3},\ \zeta_3)$, since it is a root of $x^3-2\in\mathbb{Q}[x]$ (proven above), while this polynomial satisfies the Eisenstein Criterion for prime $p=2$, hence $x^3-2$ is irreducible over $\mathbb{Q}$. So, because it is both monic and irreducible over $\mathbb{Q}$ while $2^{1/3}$ is a root, then it must be the minimal polynomial of $2^{1/3}$.
    \begin{comment}
    Now, notice that over $\mathbb{C}$, $2^{1/3}\cdot e^{2\pi i\cdot k/3} = 2^{1/3}\cdot \zeta_3^k$ (for integers $k=0,1,2$) creates three distinct roots of $x^3-2$, while $x^3-2$ has at most $3$ distinct roots, hence these are all the roots of $x^3-2$ over $\mathbb{C}$. Which, because they can be generated through finite products of $2^{1/3}$ and $\zeta_3$, then all roots of $x^3-2$ over $\mathbb{C}$ exist in $\mathbb{Q}(2^{1/3},\ \zeta_3)$, showing that $x^3-2$ splits completely over $\mathbb{Q}(2^{1/3},\ \zeta_3)$; on the other hand, since the splitting field of $x^3-2$ over $\mathbb{C}$ can be generated as $\mathbb{Q}(2^{1/3},\ 2^{1/3}\zeta_3,\ 2^{1/3}\zeta_3^2)$ (adjoining $\mathbb{Q}$ with all the roots of $x^3-2$), because $\zeta_3 = (2^{1/3}\zeta_3)/2^{1/3}\in \mathbb{Q}(2^{1/3},\ 2^{1/3}\zeta_3,\ 2^{1/3}\zeta_3^2)$, this shows that $\mathbb{Q}(2^{1/3},\zeta_3)\subseteq \mathbb{Q}(2^{1/3},\ 2^{1/3}\zeta_3,\ 2^{1/3}\zeta_3^2)$, which by the definition of splitting field, $\mathbb{Q}(2^{1/3},\zeta_3)= \mathbb{Q}(2^{1/3},\ 2^{1/3}\zeta_3,\ 2^{1/3}\zeta_3^2)$. So, $\mathbb{Q}(2^{1/3},\ \zeta_3)$ is indeed a splitting field of $x^3-2$, the minimal polynomial of $2^{1/3}$ over $\mathbb{Q}$. 
    \end{comment}
    Now, recall that in \textbf{HW 3 Question 4}, we've proven that for any $n\in\mathbb{N}$, $\mathbb{Q}(2^{1/n},\ \zeta_n)$ is a splitting field of $x^n-2$ (where $\zeta_n$ is a primitive $n^{th}$ root of unity), hence $\mathbb{Q}(2^{1/3},\zeta_3)$ is a splitting field of $x^3-2$. Which, since being a normal extension of $\mathbb{Q}$ is equivalent to be a splitting field of some subset of $\mathbb{Q}[x]$, then $\mathbb{Q}(2^{1/3},\ \zeta_3)/\mathbb{Q}$ (as a splitting field of $x^3-2$) is a normal extension.

    Together with both information above, $\mathbb{Q}(2^{1/3},\ \zeta_3)/\mathbb{Q}$ is both Normal and Separable, hence it is a Galois Extension. 
    Yet, if consider $\mathbb{Q}(2^{1/3})/\mathbb{Q}$, since $2^{1/3}\in \mathbb{Q}(2^{1/3})$ has minimal polynomial $x^3-2$ (proven above), while it is clear that $\mathbb{Q}(2^{1/3})\subsetneq \mathbb{Q}(2^{1/3},\zeta_3)$, where the larger field here is a splitting field of $x^3-2$ contained in $\mathbb{C}$ (since the larger field contains $\zeta_3\notin \mathbb{R}$, while $\mathbb{Q}(2^{1/3})\subset \mathbb{R}$). So, as a proper subfield, $\mathbb{Q}(2^{1/3})$ is not a splitting field of $x^3-2$. Because it is the minimal polynomial of $2^{1/3}\in \mathbb{Q}(2^{1/3})$, then there is an element with its minimal polynomial not splitting over $\mathbb{Q}(2^{1/3})$, showing that $\mathbb{Q}(2^{1/3})$ is not normal, hence not Galois.

    So, given $\mathbb{Q}\subseteq \mathbb{Q}(2^{1/3})\subseteq \mathbb{Q}(2^{1/3},\ \zeta_3)$, even though $\mathbb{Q}(2^{1/3},\ \zeta_3)/\mathbb{Q}$ is Galois, but $\mathbb{Q}(2^{1/3})/\mathbb{Q}$ is not Galois, showing that given $F\subseteq K\subseteq L$, $L/F$ being Galios doesn't imply $K/F$ is Galois.

    \hfil

    \item[(ii)] For the second statement, consider the counterexample provided by \textbf{Question 4} in this HW:
    
    Given the field extension $\mathbb{Q}\subseteq \mathbb{Q}(\sqrt{1+\sqrt{2}})$, which the adjoining element $\sqrt{1+\sqrt{2}}$ satisfies:
    $$\left(\sqrt{1+\sqrt{2}}\right)^2 = 1+\sqrt{2},\quad \sqrt{2} = \left(\sqrt{1+\sqrt{2}}\right)^2-1 \in\mathbb{Q}\left(\sqrt{1+\sqrt{2}}\right)$$
    Hence, this implies that $\mathbb{Q}\subseteq \mathbb{Q}(\sqrt{2})\subseteq \mathbb{Q}(\sqrt{1+\sqrt{2}})$.

    \textbf{The first claim} is that $\mathbb{Q}(\sqrt{1+\sqrt{2}})/\mathbb{Q}$ is not Galois: From what we've proven in \textbf{Question 4} (can check the proof), the minimal polynomial of $\sqrt{1+\sqrt{2}}$ over $\mathbb{Q}$ is $x^4-2x^2-1$, which has roots $\pm\sqrt{1+\sqrt{2}},\ \pm\sqrt{1-\sqrt{2}}\in\mathbb{C}$. So, if we fix $\mathbb{C}$ as the large algebraically closed field, then the unique splitting field of $x^4-2x^2-1$ is given by $\mathbb{Q}(\sqrt{1+\sqrt{2}},\sqrt{1-\sqrt{2}})\subset \mathbb{C}$ (Note: $\sqrt{1-\sqrt{2}}\notin \mathbb{R}$). Yet, because $\mathbb{Q}(\sqrt{1+\sqrt{2}})\subset\mathbb{R}$, then $\mathbb{Q}(\sqrt{1+\sqrt{2}})\subsetneq \mathbb{Q}(\sqrt{1+\sqrt{2}},\sqrt{1-\sqrt{2}})$, which is not a splitting field of $x^4-2x^2-1$. Since $\sqrt{1+\sqrt{2}}$ has its minimal polynomial not splitting completely over $\mathbb{Q}(\sqrt{1+\sqrt{2}})$, then $\mathbb{Q}(\sqrt{1+\sqrt{2}})/\mathbb{Q}$ is not normal, hence not Galois.

    \textbf{The second claim} is that both $\mathbb{Q}(\sqrt{2})/\mathbb{Q}$ and $\mathbb{Q}(\sqrt{1+\sqrt{2}})/\mathbb{Q}(\sqrt{2})$ are degree $2$ extensions: Since $\sqrt{2}$ satisfies $(\sqrt{2})^2-2=0$, then it is a root of $x^2-2\in\mathbb{Q}[x]$; and since this polynomial satisfies the Eisenstein Criterion for prime $p=2$, then it is irreducible. Hence, because $x^2-2$ is both monic and irreducible, it is the minimal polynomial of $\sqrt{2}$ over $\mathbb{Q}$, which implies the following:
    $$\mathbb{Q}(\sqrt{2})\cong \mathbb{Q}[x]/(x^2-2),\quad [\mathbb{Q}(\sqrt{2}):\mathbb{Q}] = 2$$
    On the other hand, since $x^4-2x^2-1$ is said to be the minimal polynomial of $\sqrt{1+\sqrt{2}}$ over $\mathbb{Q}$, then we get the following:
    $$\mathbb{Q}\left(\sqrt{1+\sqrt{2}}\right)\cong \mathbb{Q}[x]/(x^4-2x^2-1),\quad \left[\mathbb{Q}\left(\sqrt{1+\sqrt{2}}\right):\mathbb{Q}\right]=4$$
    Which, by the relations of field extension, we get:
    $$4=\left[\mathbb{Q}\left(\sqrt{1+\sqrt{2}}\right):\mathbb{Q}\right] = \left[\mathbb{Q}\left(\sqrt{1+\sqrt{2}}\right):\mathbb{Q}(\sqrt{2})\right]\cdot [\mathbb{Q}(\sqrt{2}):\mathbb{Q}] = 2\left[\mathbb{Q}\left(\sqrt{1+\sqrt{2}}\right):\mathbb{Q}(\sqrt{2})\right]$$
    $$\implies \left[\mathbb{Q}\left(\sqrt{1+\sqrt{2}}\right):\mathbb{Q}(\sqrt{2})\right] = 2$$
    So, this shows that both $\mathbb{Q}(\sqrt{2})/\mathbb{Q}$, $\mathbb{Q}(\sqrt{1+\sqrt{2}})/\mathbb{Q}(\sqrt{2})$ are degree $2$ extensions.

    \textbf{The final claim} is that over a perfect field $F$, a degree $2$ extension is Galois: Since $F$ is perfect, any degree $2$ extension (which is finite) is separable. Also, if $K/F$ is the given degree $2$ extension, choose any $\alpha\in K\setminus F$, then the list $1,\alpha,\alpha^2$ has $3$ elements, which is linearly dependent (since $K$ is a $2$-dimensional $F$-vector space). Hence, WLOG, there exists $b,c\in F$, such that $\alpha^2+b\alpha+c=0$. Then, $m_{\alpha,F}(x)\mid x^2+bx+c\in F[x]$, showing that $\deg(m_{\alpha,F})\leq 2$; on the other hand, since $\alpha\notin F$, then $\deg(m_{\alpha,F})>1$, which enforces $\deg(m_{\alpha,F})=2$. Because $x^2+bx+c$ is chosen to be monic, we must have $m_{\alpha,F}(x)=x^2+bx+c$ (since their degree matches up, being one others' factor, and are both monic). As a consequence, $F(\alpha)\cong F[x]/(x^2+bx+c)$, which $[F(\alpha):F]=2$. Furthermore, since $F(\alpha)\subseteq K$, while $[K:F]=2$, then $F(\alpha)$ as a linear subspace of $K$ has the same dimension with $K$, showing that $F(\alpha)=K$. Finally, since $\alpha\in K=F(\alpha)$ has its minimal polynomial being degree $2$, while $\alpha$ is a root of $m_{\alpha,F}(x)$, which implies that $m_{\alpha,F}(x)$ splits completely over $K$; now, suppose $K'\subseteq K$ is the splitting field of $m_{\alpha,F}(x)$, then $K'$ must contain all roots of it, showing that $\alpha\in K'$, or $K=F(\alpha)\subseteq K'$, hence we can deduce that $K=K'$, or $K$ is the splitting field of $m_{\alpha,F}(x)$. Since $K$ is a splitting field of some subset of $F[x]$, then it is normal.
    So, $K/F$ as a degree $2$ extension is both separable and normal, which is Galois.

    To conclude for the counterexample, since $\mathbb{Q}\subseteq \mathbb{Q}(\sqrt{2})\subseteq \mathbb{Q}(\sqrt{1+\sqrt{2}})$, and $\mathbb{Q}$ is perfect, which implies that its algebraic extension $\mathbb{Q}(\sqrt{2})$ is also perfect (proven in \textbf{HW 5 Question 4}), hence $\mathbb{Q}(\sqrt{2})/\mathbb{Q}$ and $\mathbb{Q}(\sqrt{1+\sqrt{2}})/\mathbb{Q}(\sqrt{2})$ (which are both degree $2$ extensions over a perfect field) by our claims above, are both Galois.
    However, our first claim shows that $\mathbb{Q}(\sqrt{1+\sqrt{2}})/\mathbb{Q}$ is not Galois. Hence, given $F\subseteq K\subseteq L$, even if $K/F$ and $L/K$ are Galois, it doesn't guarantee that $L/F$ is Galois.
\end{itemize}

\hfil

\hfil

\section*{2}
\begin{myBox}[]{}
    \begin{question}
        Let $K/F$ be a finite extensions. Prove that:
        \begin{itemize}
            \item[(a)] $K/F$ is normal if and only if $K$ is a splitting field of some polynomial $p(x)\in F[x]$ over $F$.
            \item[(b)] $K/F$ is Galois if and only if $K$ is a splitting field of some separable poynomial $p(x)\in F[x]$ over $F$. 
        \end{itemize}
    \end{question}
\end{myBox}

\textbf{Pf:}

Before starting the proof, since $K/F$ is a finite extension, there exists distinct $\alpha_1,...,\alpha_n\in K$, such that $K=F(\alpha_1,...,\alpha_n)$. Where, each index $i\in\{1,...,n\}$ has $\alpha_i$ being algebraic over $F$, so $m_{\alpha_i,F}(x)$ exists.
\begin{itemize}
    \item[(a)]
    \begin{itemize}
        \item[$\implies:$] Suppose $K/F$ is normal, then for every index $i\in\{1,...,n\}$, the minimal polynomial of $\alpha_i$, $m_{\alpha_i,F}(x)\in F[x]$ splits completely over $K$. Which, if define $p(x)\in F[x]$ as follow:
        $$p(x)=\prod_{i=1}^{n}m_{\alpha_i,F}(x)$$
        then since each polynomial component in the above product splits completely over $K$, then $p(x)$ also splits completely over $K$.

        Now, it suffices to show that $K$ is a splitting field of $p(x)\in F[x]$: Suppose $F\subseteq K'\subseteq K$, where $K'$ is the splitting field of $p(x)$ contained in $K$. Then, it implies that $K'$ must necessarily contain all the roots of $p(x)$ in $K$. Because for each index $i\in\{1,...,n\}$, the definition of $p(x)$ above shows that $m_{\alpha_i,F}(x)\mid p(x)$, then since $m_{\alpha_i,F}(\alpha_i)=0$, we must have $p(\alpha_i)=0$. So, each $\alpha_i$ is a root of $p(x)$. Hence, with $K'$ containing all the roots of $p(x)$ in $K$, each $\alpha_i\in K'$, showing that $K= F(\alpha_1,...,\alpha_n)\subseteq K'$. Hence, $K=K'$, or $K$ is a splitting field of $p(x)\in F[x]$.

        \item[$\impliedby:$] Recall that $K/F$ is normal iff $K$ is a splitting field of some collections of polynomials $A\subseteq F[x]$.
        
        Now, suppose that $K/F$ is a splitting field of $p(x)\in F[x]$, then let $A=\{p(x)\}$, apply the statement from above, we get that $K/F$ is indeed Normal.
    \end{itemize}

    \hfil

    \item[(b)] 
    \begin{itemize}
        \item[$\implies:$] Suppose $K/F$ is Galois, then it is both a normal and separable extension. As a consequence, for each index $i\in\{1,...,n\}$, not only $m_{\alpha_i,F}(x)\in F[x]\subseteq K[x]$ splits completely over $K$, and it necessarily has simple roots. 
        
        Now, let $A = \{m_{\alpha_i,F}(x)\ |\ 1\leq i\leq n\}\subset F[x]$ (the collection of all $\alpha_i$'s minimal polynomial, which if two $\alpha_i,\alpha_j$ share the same minimal polynomial, it counts only once in $A$). Which, given any $f(x),g(x)\in A$, since they're both minimal polynomials of some $\alpha_i,\alpha_j$ respectively, then they're both monic and irreducible; which, suppose $f(x),g(x)$ share some roots $\beta\in K$, then being monic and irreducible polynomial, it enforces $f(x),g(x)$ to both be the minimal polynomial of $\beta$, or $f(x)=g(x)$. Take the contrapositive, if $f(x)\neq g(x)$, then they share no roots.

        Which, let $p(x)\in F[x]$ be defined as follow:
        $$p(x)=\prod_{f(x)\in A}f(x)$$
        Which, $p(x)$ is the product of distinct polynomials in $A$. 
        
        First, notice that each $f(x)\in A$ splits completely over $K$ (since it is a minimal polynomial of some $\alpha_i$), then $p(x)$ as the product of them must split completely.
        
        Then, since each $f(x)\in A$ only has simple roots (again since it is a minimal polynomial of some $\alpha_i$), while for any other $g(x)\in A$ with $g(x)\neq f(x)$, they share no roots, then as product of distinct polynomials in $A$, $p(x)$ must also have simple roots (suppose $\beta$ is a root of $p(x)$, it must be a root for some $f(x)\in A$; but then, for any other $g(x)\in A$, since $f(x)\neq g(x)$ implies they share no roots, then $g(\beta)\neq 0$, so $\beta$ must necessarily have multiplicity $1$, since it is a root for only $f(x)$, and $f(x)$ only has simple roots).

        The above proves that $p(x)\in F[x]$ is a separable polynomial, while $p(x)\in K[x]$ splits completely, hence it suffices to show that $K$ is indeed a splitting field of $p(x)$:
        Following from similar methods used in \textbf{part (a)}, if $F\subseteq K'\subseteq K$, where $K'$ is the splitting field of $p(x)$ contained in $K$, then it must contain all roots of $p(x)$; on the other hand, for all $i\in\{1,...,n\}$, since $m_{\alpha_i,F}(x)\in A$, then $m_{\alpha_i,F}(x)\mid p(x)$, showing that $p(\alpha_i)=0$. So, $\alpha_i$ is a root of $p(x)$, or $\alpha_i\in K'$. Hence, $K=F(\alpha_1,...,\alpha_n)\subseteq K'$, showing that $K=K'$.

        Therefore, we can conclude that $K$ is a splitting field of $p(x)\in F[x]$, where $p(x)$ is separable.
        

        \item[$\impliedby:$] Recall that $K/F$ is Galois iff $K$ is a splitting field of some collections of separable polynomials $A\subseteq F[x]$. 
        
        Now, suppose that $K/F$ is a splitting field of $p(x)\in F[x]$ (where $p(x)$ is separable), then let $A=\{p(x)\}$, apply the statement from above, we get that $K/F$ is indeed Galois.
    \end{itemize}
\end{itemize}

\break

\section*{3}
\begin{myBox}[]{}
    \begin{question}
        Let $K/F$ be "separable" and $K=F(\alpha_1,...,\alpha_n)$. For a fixed algebraic closure $\overline{F}$ such that $F\subseteq K\subseteq \overline{F}$, suppose that $\phi_1,\phi_2,...,\phi_m$ are all the $F$-embeddings from $K$ to $\overline{F}$. Prove that $F(S)$ is a Galois Closure of $K/F$ where $S=\{\phi_i(\alpha_j)\ |\ 1\leq i\leq m,\ 1\leq j\leq n\}$.
    \end{question}
\end{myBox}

\textbf{Pf:}

Let $A=\{m_{\alpha_j,F}(x)\ |\ 1\leq j\leq n\}\subset F[x]$.

\textbf{1. $S$ contains all roots of all polynomials in $A$:}

First, it is clear that all element in $S$ is a root of some polynomials in $A$, since any $s\in S$ satisfies $s=\phi_i(\alpha_j)$ for some $1\leq i\leq m$ and $1\leq j\leq n$. Which, because $m_{\alpha_j,F}(x)\in F[x]$, all of its coefficients are fixed by $\phi_i$ (which is an $F$-embedding), hence $0 = \phi_i(0)=\phi_i(m_{\alpha_j,F}(\alpha_j)) = m_{\alpha_j,F}(\phi_i(\alpha_j))$, showing that $s=\phi_i(\alpha_j)$ is also a root of $m_{\alpha_j,F}(x)\in A$.

Then, for any $m_{\alpha_j,F}(x)\in A$, let $s\in \overline{F}$ be one of its roots, since $m_{\alpha_j,F}(x)$ is assumed to be monic and irreducible over $F$, then having $s$ being a root, implies that it is also the minimal polynomial of $s$. Hence, $F(\alpha_j)\cong F[x]/(m_{\alpha_j,F}(x))\cong F(s)$ via an explicit isomorphism given as follow:
$$\varphi:F(\alpha_j)\tilde{\rightarrow}F(s),\quad \forall a_0,a_1,...,a_n\in F,\ \varphi(a_0+a_1\alpha_j+...+a_n\alpha_j^n)=a_0+a_1s+...+a_ns^n$$
Notice that $\varphi$ fixes all the elements of $F$, and also $F(s)\subseteq\overline{F}$, so composing with the inclusion map, the new map $\varphi:F(\alpha_j)\rightarrow\overline{F}$ is in fact an $F$-embedding.
Since $K/F$ is algebraic, while $F\subseteq F(\alpha_j)\subseteq K$, then this guarantees that $K/F(\alpha_j)$ is algebraic; on the other hand, since $F(\alpha_j)\subseteq K\subseteq \overline{F}$, this ensures that $\overline{F}$ is also an algebraic closure of $F(\alpha_j)$ (since $\overline{F}/F$ is algebraic, so $\overline{F}/F(\alpha_j)$ is also algebraic; then since $\overline{F}$ is algebraically closed, it is an algebraic closure of $F(\alpha_j)$). Hence, as $K/F(\alpha_j)$ is an algebraic extension, the above embedding $\varphi:F(\alpha_j)\rightarrow\overline{F}$ can be extended to $\overline{\varphi}:K\rightarrow\overline{F}$, with $\overline{\varphi}\bigm|_{F(\alpha_j)}=\varphi$. Because $\varphi$ is already an $F$-embedding, $\overline{\varphi}$ is also an $F$-embedding. Hence, $\overline{\varphi} = \phi_i$ for some $1\leq i\leq n$.
So, we get that $\phi_i(\alpha_j) = \overline{\varphi}(\alpha_j) = \varphi(\alpha_j) = s$, which shows that $s\in S$.

The above implications show that $S$ must contain (and only contains) all roots of all polynomials in $A$.

\vspace*{0.5em}

\textbf{2. $F(S)/F$ is Galois:}

First, since $K = F(\alpha_1,...,\alpha_n)$ is assumed to be a separable extension of $F$, then $\alpha_1,...,\alpha_n$ must have their minimal polynomials being separable, hence $A$ as a collection of all $\alpha_j$'s minimal polynomial, is a subset of $F[x]$ containing only separable polynomials.

Then, because $S\subset F(S)$ contains all the roots of all polynomials in $A$, every $f(x)\in A$ splits completely over $F(S)$. Now, suppose $K'\subseteq F(S)$ is the splitting field of the subset $A$, then $K'$ must contain all roots of all polynomials in $A$; because $S$ only contains the roots of polynomials in $A$, this implies that $S\subset K'$, which further implies $F(S)\subseteq K'$, or $F(S)=K'$. Hence, $F(S)$ is a splitting field of $A$.

Finally, because $F(S)$ is a splitting field of $A$ (which is a collection of separable polynomials), then $F(S)/F$ is a Galois Extension.

\vspace*{0.5em}

\textbf{3. $F(S)$ is a Galois Closure of $K$:}

Suppose $K'\subseteq F(S)$ is the Galois Closure of $K/F$, then since $K=F(\alpha_1,...,\alpha_n)$, for each index $j\in\{1,...,n\}$, we must have $m_{\alpha_j,F}(x)$ split completely over $K'$. Hence, because $A$ collects all $m_{\alpha_j,F}(x)$ (and only contains these polynomials), every polynomial in $A$ splits completely over $K'$. However, in the previous section, when proving that $F(S)/F$ is a Galois Extension, we've shown that $F(S)$ is a splitting field of $A$, then since every polynomial in $A$ splits completely over $K'\subseteq F(S)$, this implies that $K'=F(S)$ by the definition of splitting field. Hence, $F(S)$ is a Galois Closure of $K/F$.

\break

\section*{4}
\begin{myBox}[]{}
    \begin{question}
        Find the Galois Closure of $\mathbb{Q}(\sqrt{1+\sqrt{2}})$ over $\mathbb{Q}$.
    \end{question}
\end{myBox}

\textbf{Pf:}

First, since Galois Closure is a normal extension, every element must have its minimal polynomial over $\mathbb{Q}$ splits completely. Then, $\sqrt{1+\sqrt{2}}$ must also have its minimal polynomial splits completely. Hence, the first goal is to find the minimal polynomial of $\alpha=\sqrt{1+\sqrt{2}}$ over $\mathbb{Q}$. Notice that it satisfies the following:
$$\alpha=\sqrt{1+\sqrt{2}}\implies \alpha^2=1+\sqrt{2} \implies \alpha^2-1 = \sqrt{2} \implies (\alpha^4-2\alpha^2+1)=2\implies \alpha^4-2\alpha^2-1 = 0$$
This shows that $\alpha$ is a root of $x^4-2x^2-1 \in\mathbb{Q}[x]$. Then, to prove that it is irreducible, consider the ring automorphism on $\mathbb{Q}[x]$ by $x\mapsto (x+1)$. Which, we get:
$$x^4-2x^2-1\mapsto (x+1)^4-2(x+1)^2-1$$
$$(x+1)^4-2(x+1)^2-1 = (x^4+4x^3+6x^2+4x+1)-2(x^2+2x+1)-1 = x^4+4x^3 + 4x^2 - 2$$
Notice that after the shift, $(x+1)^4-2(x+1)^2-1 = x^4+4x^3+4x^2-2$ satisfies the Eisenstien's Criterion for prime $p=2$, which is irreducible over $\mathbb{Q}$. This implies that $x^4-2x^2-1$ is also irreducible over $\mathbb{Q}$.

Because $x^4-2x^2-1\in\mathbb{Q}[x]$ is monic and irreducible, while $\alpha=\sqrt{1+\sqrt{2}}$ is a root, hence $x^4-2x^2-1$ is necessarily the minimal polynomial of $\sqrt{1+\sqrt{2}}$ over $\mathbb{Q}$.

\hfil

Then, since the minimal polynomial of $\sqrt{1+\sqrt{2}}$ over $\mathbb{Q}$ (namely $x^4-2x^2-1$), should split completely in a Galois Closure, then it must contain a splitting field of $x^4-2x^2-1$. So, the second goal is to find the roots of $x^4-2x^2-1$ over $\mathbb{C}$. Let $y=x^2$, then $x^4-2x^2-1 = y^2-2y-1$. Which, by Quadratic Formula, we get:
$$y = \frac{-(-2)\pm\sqrt{(-2)^2-4\cdot 1\cdot (-1)}}{2} = \frac{2\pm\sqrt{8}}{2} = 1\pm\sqrt{2}$$
Hence, solving for $y=x^2=1\pm\sqrt{2}$ would provide the roots for the polynomial. Which for this equation, $x = \pm\sqrt{1+\sqrt{2}},\ \pm\sqrt{1-\sqrt{2}}\in\mathbb{C}$. Hence, these four distinct roots all satisfy $x^4-2x^2-1=0$, while the polynomial can have at most $4$ distinct roots, so these must be all the roots of $x^4-2x^2-1$.

As a consequence, we get $\mathbb{Q}(\sqrt{1+\sqrt{2}},\sqrt{1-\sqrt{2}})\subset \mathbb{C}$ is the splitting field of $x^4-2x^2-1$ under $\mathbb{C}$.

\hfil

Finally, we can show that $\mathbb{Q}(\sqrt{1+\sqrt{2}},\sqrt{1-\sqrt{2}})$ is a Galois Closure of $\mathbb{Q}(\sqrt{1+\sqrt{2}})$:

Since $\textmd{char}(\mathbb{Q})=0$, then it is a perfect field, hence $\mathbb{Q}(\sqrt{1+\sqrt{2}},\sqrt{1-\sqrt{2}})/\mathbb{Q}$ as a finite extension must be separable. On the other hand, since $\mathbb{Q}(\sqrt{1+\sqrt{2}},\sqrt{1-\sqrt{2}})$ is also a splitting field of $x^4-2x^2-1\in \mathbb{Q}[x]$, then this enforces $\mathbb{Q}(\sqrt{1+\sqrt{2}},\sqrt{1-\sqrt{2}})$ to also be a normal extension. Therefore, $\mathbb{Q}(\sqrt{1+\sqrt{2}},\sqrt{1-\sqrt{2}})/\mathbb{Q}$ is a Galois Extension.

Now, suppose $\mathbb{Q}(\sqrt{1+\sqrt{2}})\subseteq K\subseteq \mathbb{Q}(\sqrt{1+\sqrt{2}},\sqrt{1-\sqrt{2}})$, where $K$ is the Galois Closure of $\mathbb{Q}(\sqrt{1+\sqrt{2}})$ under 
$\mathbb{Q}(\sqrt{1+\sqrt{2}},\sqrt{1-\sqrt{2}})$. Then, since $\sqrt{1+\sqrt{2}}\subseteq K$, its minimal polynomial $x^4-2x^2-1\in\mathbb{Q}[x]$ must split completely over $K$. So, $K$ must contain all the roots of $x^4-2x^2-1$ in $\mathbb{Q}(\sqrt{1+\sqrt{2}},\sqrt{1-\sqrt{2}})$, which is given by $\pm\sqrt{1+\sqrt{2}}$ and $\pm\sqrt{1-\sqrt{2}}$. Hence, this implies that $\mathbb{Q}(\sqrt{1+\sqrt{2}},\sqrt{1-\sqrt{2}})\subseteq K$, or $K=\mathbb{Q}(\sqrt{1+\sqrt{2}},\sqrt{1-\sqrt{2}})$.

Therefore, $\mathbb{Q}(\sqrt{1+\sqrt{2}},\sqrt{1-\sqrt{2}})$ is indeed a Galois Closure of $\mathbb{Q}(\sqrt{1+\sqrt{2}})$.

\break

\section*{5}
\begin{myBox}[]{}
    \begin{question}
        Prove that, if $K_1/F$ and $K_2/F$ are Galois, so is $(K_1\cap K_2)/F$.
    \end{question}
\end{myBox}

\textbf{Pf:}

First, since by assumption $K_2/F$ is Galois implies it is algebraic, then any $\alpha\in K_2$ is algebraic over $F$, which is also algebraic over $K_1$. As a consequence, consider the field $K_1K_2 = K_1(K_2)$: Since each element in $K_2$ is algebraic over $K_1$, then $K_1(K_2)/K_1$ is algebraic; together with $K_1/F$ being algebraic (since it is a Galois Extension), then $K_1(K_2)/F$ is also algebraic. So, we can fix an algebraic closure $\overline{F}$ such that $F\subseteq K_1(K_2)\subseteq \overline{F}$, which as sets, the three fields $K_1, K_2, (K_1\cap K_2)\subseteq \overline{F}$.

\hfil

Then, since $K_1/F$ and $K_2/F$ are both Galois, then they're separable extensions; hence, every element $\alpha\in (K_1\cap K_2)\subseteq K_1$ is separable over $F$, showing that $(K_1\cap K_2)/ F$ is a separable extension.

\hfil

Now, since $K_1/F$ and $K_2/F$ are both normal (since they're Galois), then any of their element has the minimal polynomial splits completely over the given field itself. Which, for all $\alpha\in (K_1\cap K_2)$, since $m_{\alpha,F}(x)$ splits completely over $K_1$ and $K_2$, it must also split over $K_1(K_2)$; on the other hand, since $K_1[x]\subseteq K_1(K_2)[x]$ and $K_2[x]\subseteq K_1(K_2)[x]$, while all three polynomials rings are UFDs, then the factorization in $K_1[x]$ and $K_2[x]$ must necessarily be the same as the factorization in $K_1(K_2)[x]$. So, for any $\beta\in K_1$ that is a root of $m_{\alpha,F}(x)$, since $(x-\beta)\mid m_{\alpha,F}(x)$ in $K_1[x]\subseteq K_1(K_2)[x]$, then $(x-\beta)$ is one of its irreducible factors; by the unique factorization mentioned, we must have $(x-\beta)\in K_2[x]$, showing that $\beta\in K_2$, or $\beta\in K_1\cap K_2$.

Which, since $m_{\alpha,F}(x)$ splits completely in $K_1[x]$, which implies that $K_1$ contains all roots of $m_{\alpha,F}(x)$; then, since all roots of $m_{\alpha,F}(x)$ in $K_1$ also appears in $K_1\cap K_2$, then $m_{\alpha,F}(x)$ splits completely over $K_1\cap K_2$. This proves that $(K_1\cap K_2)/F$ is a normal extension.

\hfil

Combining both information above, $(K_1\cap K_2)/F$ is in fact Galois.


\end{document}