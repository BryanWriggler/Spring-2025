\documentclass{article}
\usepackage{graphicx} % Required for inserting images
\usepackage[margin = 2.54cm]{geometry}
\usepackage[most]{tcolorbox}

\newtcolorbox{myBox}[3]{
arc=5mm,
lower separated=false,
fonttitle=\bfseries,
%colbacktitle=green!10,
%coltitle=green!50!black,
enhanced,
attach boxed title to top left={xshift=0.5cm,
        yshift=-2mm},
colframe=blue!50!black,
colback=blue!10
}

\usepackage{amsmath}
\usepackage{amssymb}
\usepackage{verbatim}
\usepackage[utf8]{inputenc}
\linespread{1.2}

\newtheorem{definition}{Definition}
\newtheorem{proposition}{Proposition}
\newtheorem{theorem}{Theorem}
\newtheorem{question}{Question}

\title{Math 111C HW6}
\author{Zih-Yu Hsieh}

\begin{document}
\maketitle

\section*{1}
\begin{myBox}[]{}
    \begin{question}
        Let $E$ be a splitting field of $f(x)\in F[x]$ and $G=\textmd{Aut}(E/F)$. Prove that:
        \begin{itemize}
            \item[(a)] If $f(x)$ is irreducible, then $G$ acts transitively on the set of all roots of $f(x)$, i.e. if $\alpha,\beta$ are two roots of $f(x)$ in $E$, there exists $\sigma\in G$ with $\sigma(\alpha)=\beta$.
            \item[(b)] If $f(x)$ has no repeated roots and $G$ acts transitively on the roots, then $f(x)$ is irreducible. 
        \end{itemize}
    \end{question}
\end{myBox}

\textbf{Pf:}
\begin{itemize}
    \item[(a)] Suppose $f(x)\in F[x]$ is irreducible, then let $a\in F$ be the leading coefficient of $f(x)$ (which $a\neq 0$), then $a^{-1}f(x)$ is a monoic polynomial. For any roots of $f(x)$ in $E$, denoted as $\alpha,\beta\in E$, since $a^{-1}f(\alpha)=a^{-1}f(\beta)=0$, while $a^{-1}f(x)$ is an irreducible monic polynomial in $F[x]$, then it must be the minimal polynomial of $\alpha$ and $\beta$. Hence, $F(\alpha)\cong F[x]/(a^{-1}f(x))\cong F(\beta)$, and an explicit isomorphism is given as $\varphi:F(\alpha)\Tilde{\rightarrow }F(\beta)$ by:
    $$\forall a_0,a_1,...,a_n\in F,\quad \varphi(a_n\alpha^n+...+a_1\alpha+a_0)=a_n\beta^n+...+a_1\beta+a_0$$
    Now, notice the following information:
    \begin{itemize}
        \item For all $k\in F$, $\varphi(k)=k$ (which $\varphi$ fixes $F$, or $\varphi\bigm|_{F}=\textmd{Id}_F$).
        \item Since $E/F$ is a splitting field of $f(x)$, then $E/F$ is an algebraic extension, hence there exists an algebraic closure $\overline{F}$ of $F$, such that $F\subseteq E\subseteq \overline{F}$.
        \item Because $\alpha,\beta\in E$ are roots of $f(x)\in F[x]$, then they're algebraic over $F$, hence $F(\alpha),F(\beta)$ are algebraic extensions of $F$, with $F(\alpha),F(\beta)\subseteq E$. With the fact that $E/F$ is algebraic and $F\subseteq F(\alpha)\subseteq E$, then $E/F(\alpha)$ is also an algebraic extension.
        \item Because $F\subseteq F(\alpha)\subseteq \overline{F}$, while $\overline{F}/F$ is an algebraic extension, then $\overline{F}/F(\alpha)$ is an algebraic extension; since $\overline{F}$ is itself algebraically closed, it is also an algebraic closure of $F(\alpha)$.
    \end{itemize}
    So, composing the inclusion $F(\beta)\hookrightarrow \overline{F}$, the isomorphism $\varphi:F(\alpha)\tilde{\rightarrow}F(\beta)$ can be extended to an embedding $\varphi:F(\alpha)\rightarrow \overline{F}$, which is also an $F$-embedding (since $\varphi$ fixes $F$). Furthermore, since $E/F(\alpha)$ is an algebraic extension, while $F(\alpha)$ is embedded into $\overline{F}$ (its own algebraic closure also), then such embedding $\varphi:F(\alpha)\rightarrow\overline{F}$ can be extended to an embedding $\sigma:E\rightarrow \overline{F}$, such that $\sigma\bigm|_{F(\alpha)} = \varphi$. So, $\sigma\bigm|_{F} = \varphi\bigm|_{F} = \textmd{Id}_F$.

    Then, the final step is to claim that $\sigma(E)=E$, or $\sigma\in \textmd{Aut}(E/F)$ after restricting the codomain.

    Since $f(x)\in F[x]$, while $\sigma$ fixes $F$, then after extending the $F$-embedding $\sigma:E\rightarrow\overline{F}$ to a canonical ring homomorphism $\overline{\sigma}:E[x]\rightarrow\overline{F}[x]$ (which $\overline{\sigma}(a)=\sigma(a)$ and $\overline{\sigma}(x)=x$ for all $a\in E$, so this map is injective), then $\overline{\sigma}(f(x)) = f(x)$ (since all of its coefficients are in $F$, while $\sigma$ fixes $F$). So, since $\sigma(E)\cong E$, while $\overline{\sigma}$ preserves $f(x)$, then $\sigma(E)$ is also a splitting field of $f(x)$ under $\overline{F}$. However, since $\overline{F}$ is chosen in a way such that $E\subseteq \overline{F}$, then because a splitting field of a polynomial $f(x)\in F[x]$ is unique when chosen a larger algebraic extension of $F$ such that $f$ splits completely, then $E,\sigma(E)\subseteq \overline{F}$ while both being a splitting field of $f(x)$ implies that $E=\sigma(E)$, hence restricting the codomain of $\sigma$ to the range, we get that $\sigma:E\tilde{\rightarrow} E$, which $\sigma\in \textmd{Aut}(E)$. Finally, since we've proven that $\sigma\bigm|_F = \textmd{Id}_F$, then $\sigma\in \textmd{Aut}(E/F) = G$.

    Hence, because $\sigma\in G$, while $\sigma\bigm|_{F(\alpha)}=\varphi$, and $\varphi(\alpha)=\beta$ based on the setup, then $\sigma(\alpha)=\beta$. So, this proves that $G$ acts transitively on the set of all roots of $f(x)$.

    \hfil

    \item[(b)] Suppose $f(x)$ has no repeated roots, while $G$ acts transitively on the roots, then for any two roots of $f(x)$, denote as $\alpha,\beta \in E$, there exists $\sigma\in G=\textmd{Aut}(E/F)$, such that $\sigma(\alpha)=\beta$. Then, since $F(\alpha),F(\beta)\subseteq E$ are finite extensions of $F$, let $\varphi=\sigma\bigm|_{F(\alpha)}$, then we get the following:
    $$\forall a_0,a_1,...,a_n\in F,\quad \varphi(a_0+a_1\alpha+...+a_n\alpha^n) = \sigma(a_0+a_1\alpha+...+a_n\alpha^n) = a_0+a_1\beta + ... + a_n\beta^n$$
    This proves that $\varphi(F(\alpha))=F(\beta)$ (since choice of $a_0,a_1,...,a_n\in F$ are arbitrary), so after restricting the codomain, $\varphi:F(\alpha)\tilde{\rightarrow}F(\beta)$ is in fact an isomorphism.

    Now, consider the minimal polynomials $m_{\alpha,F}(x),m_{\beta,F}(x)\in F[x]$: First, since $m_{\alpha,F}(\alpha)=0$, while all of its coefficients are in $F$, then we get:
    $$\varphi(0)=\varphi(m_{\alpha,F}(\alpha)) = m_{\alpha,F}(\beta) = 0$$
    So, this implies that $m_{\beta,F}(x)\mid m_{\alpha,F}(x)$ (since $\beta$ is a root of $m_{\alpha,F}(x)$). Then, because both $m_{\alpha,F}(x)$ and $m_{\beta,F}(x)$ are irreducible in $F[x]$, then $m_{\beta,F}(x)\mid m_{\alpha,F}(x)$ implies that $m_{\alpha,F}(x)=k\cdot m_{\beta,F}(x)$, where $k\in (F[x])^\times = F^\times$; but since both polynomials are monic, then $k=1$. So, $m_{\beta,F}(x)=m_{\alpha,F}(x)$. This implies that all roots of $f(x)$ must have the same minimal polynomial. Hence, WLOG, let $m(x)\in F[x]$ be the minimal polynomial of all roots of $f(x)$ in $E$.

    Finally, since all roots of $f(x)$ has minimal polynomial $m(x)\in F[x]$, then $m(x)\mid f(x)$, so $\deg(m)\leq \deg(f)$; on the other hand, since $E$ is a splitting field of $f$, while $f$ is assumed to have no repeated roots, then let $n=\deg(f)$, it implies that there are $n$ distinct roots of $f$ in $E$. Since they're all having $m(x)$ as the minimal polynomial, they're all roots of $m(x)$, so $m$ has at least $n$ distinct roots, showing that $\deg(m)\geq n = \deg(f)$.
    So, this enforces $\deg(m)=\deg(f)$. And, because $m(x)\mid f(x)$, then $f(x)=k\cdot m(x)$ for some $k\in F^\times$. Which, since $m(x)\in F[x]$ is irreducible (because it's a minimal polynomial of some elements in $E$), then $f(x)$ as a nonzero constant multiple of $m(x)$ is also irreducible. This finishes the proof.
\end{itemize}

\break

\section*{2}
\begin{myBox}[]{}
    \begin{question}
        In each part, find the degree of the extension $K/F$.
        \begin{itemize}
            \item[(a)] Splitting field $K\subseteq\mathbb{C}$ of $f(x)=x^4-4$ over $F=\mathbb{Q}$.
            \item[(b)] Splitting field $K\subseteq\mathbb{C}$ of $f(x)=x^6-2$ over $F=\mathbb{Q}$.
            \item[(c)] Splitting field $K$ of $f(x)=x^{10}-2$ over $F=\mathbb{F}_5$.  
        \end{itemize}
    \end{question}
\end{myBox}

\textbf{Pf:}
\begin{itemize}
    \item[(a)] Notice that $\sqrt{2},-\sqrt{2},\sqrt{2}i,-\sqrt{2}i\in\mathbb{C}$ all satisfies $x^4 = 4$, so they're all the roots of $x^4-4\in\mathbb{Q}[x]$ over $\mathbb{C}$. Hence, the splitting field of $x^4-4\in\mathbb{Q}[x]$ is $K=\mathbb{Q}(\sqrt{2},-\sqrt{2},\sqrt{2}i,-\sqrt{2}i)$.
    
    Now, notice that $-\sqrt{2}\in \mathbb{Q}(\sqrt{2})$; on the other hand, since $\sqrt{2}i\in K$, then $i=\frac{\sqrt{2}i}{\sqrt{2}} \in K$, which indicates that the field $\mathbb{Q}(\sqrt{2},i)\subseteq K$.

    Furthermore, since $\sqrt{2},-\sqrt{2},\sqrt{2}i,-\sqrt{2}i$ can all be generated by $\sqrt{2}$ and $i$, then we can also deduce that $K=\mathbb{Q}(\sqrt{2},-\sqrt{2},\sqrt{2}i,-\sqrt{2}i)\subseteq \mathbb{Q}(\sqrt{2},i)$. So, we can conclude that $K=\mathbb{Q}(\sqrt{2},i)$.

    Then, consider the relation $\mathbb{Q}\subseteq \mathbb{Q}(\sqrt{2})\subseteq K=\mathbb{Q}(\sqrt{2},i)$: Since $\sqrt{2}\notin \mathbb{Q}$, and it satisfies $(\sqrt{2})^2-2 = 0$, then it is a root of $x^2-2\in\mathbb{Q}[x]$. Since this polynomial satisfies Eisenstein Criterion for prime $p=2$, it is irreducible; and since it is also monic, while $\sqrt{2}$ is its root, then $x^2-2$ is the minimal polynomial of $\sqrt{2}$ over $\mathbb{Q}$. Hence, $\mathbb{Q}(\sqrt{2})\cong \mathbb{Q}[x]/(x^2-2)$, which indicates that $[\mathbb{Q}(\sqrt{2}):\mathbb{Q}]=2$.

    Also, If consider $\mathbb{Q}(\sqrt{2},i)=\mathbb{Q}(\sqrt{2})(i)$, since $i\notin \mathbb{Q}(\sqrt{2})$, while $i$ satisfies $(i)^2+1 = 0$, then it is a root of $x^2+1\in\mathbb{Q}[x]$. Which, since this polynomial has no roots in $\mathbb{Q}(\sqrt{2})$ (since all $q\in\mathbb{Q}(\sqrt{2})\subset\mathbb{R}$ has $q^2>0$, so no $q$ satisfies $q^2=-1$, or $q^2+1=0$), then it is irreducible over $\mathbb{Q}(\sqrt{2})$. Together with the fact that $x^2+1$ is monic, it is the minimal polynomial of $i$ over $\mathbb{Q}(\sqrt{2})$. So, $\mathbb{Q}(\sqrt{2})(i)\cong \mathbb{Q}(\sqrt{2})[x]/(x^2+1)$, which indicates that $[\mathbb{Q}(\sqrt{2},i):\mathbb{Q}(\sqrt{2})]=2$.

    Finally, with the above two degrees of field extension, we get:
    $$[K:\mathbb{Q}] = [\mathbb{Q}(\sqrt{2},i):\mathbb{Q}(\sqrt{2})]\cdot [\mathbb{Q}(\sqrt{2}):\mathbb{Q}] = 2\cdot 2 = 4$$

    \hfil

    \item[(b)] Notice that for all integer $0\leq k\leq 5$ ($6$ distinct entries), we have $2^{1/6}e^{2\pi i\cdot k/6}\in\mathbb{C}$ satisfies the equation $(2^{1/6}e^{2\pi i\cdot k/6})^6 -2 = 0$. Then, since these are all distinct ($6$ distinct roots), while $x^6-2\in\mathbb{Q}[x]$ can have at most $6$ distinct roots, then these must be all the roots of $x^6-2$ over $\mathbb{C}$. Which, since $K$ is the splitting field of $x^6-2$, it is precisely the field obtained by $\mathbb{Q}$ adjoining all the above roots.
    
    Now, since $2^{1/6},2^{1/6}\cdot e^{2\pi i/6}\in K$, then let $\zeta_6 = e^{2\pi i/6}$, we get that $\zeta_6 = \frac{2^{1/6}\cdot e^{2\pi i/6}}{2^{1/6}}\in K$. Hence, the field $\mathbb{Q}(2^{1/6},\zeta_6)\subseteq K$. On the other hand, each root of $x^6-2$ over $\mathbb{C}$ is in the form $2^{1/6}\cdot e^{2\pi i\cdot k/6} = 2^{1/6}\cdot \zeta_6^k$ for some integer $0\leq k\leq 5$, this shows that each root $2^{1/6}\cdot e^{2\pi i\cdot k/6}\in \mathbb{Q}(2^{1/6},\zeta_6)$, hence $K\subseteq\mathbb{Q}(2^{1/6},\zeta_6)$, which together with the previous inclusion shows that $K = \mathbb{Q}(2^{1/6},\zeta_6)$.

    Then, consider the relation $\mathbb{Q}\subseteq \mathbb{Q}(2^{1/6})\subseteq K=\mathbb{Q}(2^{1/6},\zeta_6)$: First, since $2^{1/6}\notin \mathbb{Q}$ is a root of $x^6-2\in\mathbb{Q}[x]$, and this polynomial satisfies the Eisenstein Criterion for prime $p=2$, then it is irreducible; together with the fact that it is monic, then it must be the minimal polynomial of $2^{1/6}$ over $\mathbb{Q}$. Hence, $\mathbb{Q}(2^{1/6})\cong \mathbb{Q}[x]/(x^6-2)$, showing that $[\mathbb{Q}(2^{1/6}):\mathbb{Q}]=6$.

    Also, if consider the $\mathbb{Q}(2^{1/6},\zeta_6)=\mathbb{Q}(2^{1/6})(\zeta_6)$, since $\zeta_6\notin\mathbb{R}$, while $\mathbb{Q}(2^{1/6})\subset \mathbb{R}$, then $\zeta_6$ must have its minimal polynomial with degree $\geq 2$; on the other hand, given the polynomial $x^2-x+1\in\mathbb{Q}(2^{1/6})[x]$, using quadratic formula, we get the roots are given by:
    $$\alpha = \frac{-(-1)\pm\sqrt{(-1)^2-4\cdot 1\cdot 1}}{2\cdot 1} = \frac{1\pm\sqrt{-3}}{2}$$
    Which, since $\zeta_6 = e^{2\pi i/6} = \frac{1+\sqrt{3}}{2}$, then $\zeta_6$ is a root of $x^2-x+1$. Then, let $m(x)\in\mathbb{Q}(2^{1/6})[x]$ be the minimal polynomial of $\zeta_6$ over $\mathbb{Q}(2^{1/6})$, $\zeta_6$ being a root of $x^2-x+1$ implies $m(x)\mid (x^2-x+1)$, which $\deg(m)\leq 2$; on the other hand, we know $m(x)$ is proven to have degree $\geq 2$, this enforces $\deg(m)=2$. Which, $m(x)$ divides $x^2-x+1$, thw two polynomials are both degree 2, while $x^2-x+1$ is monic, indicates that $m(x)=x^2-x+1$.
    So, $\mathbb{Q}(2^{1/6})(\zeta_6) \cong \mathbb{Q}(2^{1/6})[x]/(m(x))=\mathbb{Q}(2^{1/6})[x]/(x^2-x+1)$, which shows that $[\mathbb{Q}(2^{1/6},\zeta_6):\mathbb{Q}(2^{1/6})]=2$.

    Finally, combine all the degree of extensions from above, we get:
    $$[K:\mathbb{Q}] = [\mathbb{Q}(2^{1/6},\zeta_6):\mathbb{Q}(2^{1/6})]\cdot [\mathbb{Q}(2^{1/6}):\mathbb{Q}] = 2\cdot 6 = 12$$

    \hfil

    \item[(c)] Given $f(x)=x^{10}-2$ over $\mathbb{F}_5$. Notice that within $\mathbb{F}_5$, the following equality is true:
    $$2^5 = (2^2)^2\cdot 2=4^2\cdot 2 = (16\mod\ 5)\cdot 2 = 1\cdot 2 =  2$$
    Hence, with the fact that $\mathbb{F}_5[x]$ has characteristic $5$, using Frobenius Endomorphism, we get:
    $$x^{10}-2 = (x^2)^5 - 2^5 = (x^2-2)^5$$
    Hence, all the roots of $x^{10}-2$ are precisely the roots of $x^{2}-2$, which $K$ as a splitting field of $x^{10}-2$, is the same field as $\mathbb{F}_5$ adjoining the roots of $x^{10}-2$ that's within $K$, which is the same as $\mathbb{F}_5$ adjoining the roots of $x^2-2$, so it is also a splitting field of $x^2-2$. Hence, it suffices to show the degree of any splitting field $K$ of $x^2-2$ as a field extension of $\mathbb{F}_5$.

    Now, notice that within $\mathbb{F}_5$, $0^2=0$, $1^2=1$, $2^2=4$, $3^2=(9\mod\ 5) = 4$, and $4^2 = (16\mod\ 5) = 1$. So, since no element in $\mathbb{F}_5$ satisfies $x^2 = 2$, then $x^2-2\in\mathbb{F}_5[x]$ has no roots in $\mathbb{F}_5$, showing that it is irreducible over $\mathbb{F}_5$. This shows that the splitting field $K\neq \mathbb{F}_5$, which $[K:\mathbb{F}_5]\geq 2$.

    Then, since the field extension $K' = \mathbb{F}_5[x]/(x^2-2)$ contains a root of $x^2-2\in\mathbb{F}_5[x]\subset K'[x]$, namely the element $\theta=\overline{x}\in K'$ (and  $[K':\mathbb{F}_5]=2$). Then, $x^2-2$ has a linear factor in $K'[x]$, showing that it splits completely over $K'$. So, if consider the splitting field of $x^2-2$, $K''\subseteq K'$, then we know $[K'':\mathbb{F}_5]\leq [K':\mathbb{F}_5] = 2$.

    Which, since all splitting fields of $x^2-2\in\mathbb{F}_5[x]$ are all isomorphic, then $K\cong K''$, which shows that $[K:\mathbb{F}_5] = [K'':\mathbb{F}_5]\leq 2$. Which, with both inequalities of $[K:\mathbb{F}_5]$ above, we can conclude the following:
    $$[K:\mathbb{F}_5] = 2$$
\end{itemize}

\break

\section*{3}
\begin{myBox}[]{}
    \begin{question}
        Let $L$ be the splitting field of $f(x)=x^3+x+1$ over $\mathbb{Q}$ contained in $\mathbb{C}$. Prove that $\textmd{Aut}(L/\mathbb{Q})\cong S_3$.
    \end{question}
\end{myBox}

\textbf{Pf:}

First, we'll get some information about the roots of $f(x)$: Since it is a cubic polynomial, it must have at least a real root; On the other hand, as a real differentiable function, its derivative is $3x^2+1$, which is always positive, this indicates that $f(x)$ is strictly increasing on the whole domain $\mathbb{R}$, which is injective. Therefore, it can have at most $1$ real root. This indicates that $f$ has a real root (denoted as $\alpha\in\mathbb{R}$), while the other two roots must be complex roots, and being the conjugate of each other (denoted as $\beta,\overline{\beta}\in\mathbb{C}$). Which, $L=\mathbb{Q}(\alpha,\beta,\overline{\beta})$ (since the splitting field of $f(x)$ is the same as adjoining the roots to the base field).

\hfil

Notice that for any $\sigma\in\textmd{Aut}(L/\mathbb{Q})$, it is purely determined by the permutation of the three roots of $f(x)$: Suppose $k\in\{\alpha,\beta,\overline{\beta}\}$ (a root of $f(x)$), then since $\sigma$ fixes coefficients $\mathbb{Q}$ while $f(x)\in\mathbb{Q}[x]$, then $0=\sigma(0)=\sigma(f(k)) = f(\sigma(k))$, which shows that $\sigma(k)$ is a root of $f(x)$. On the other hand,if $k\in L$ satisfies $f(\sigma(k))=0$, then by similar reasoning, $\sigma(f(k))=0$, implying that $f(k)=0$ (since $\sigma$ is an automorphism), then $k$ is a root of $f(x)$. So, $\sigma$ can only send roots of $f(x)$ to roots of $f(x)$, and since $\sigma$ is bijective, it in fact forms a permutation of the $3$ roots of $f(x)$. On the other hand, since $L=\mathbb{Q}(\alpha,\beta,\overline{\beta})$, then the automorphism $\sigma$ is purely determined by where the three roots go.

So, if we define the map $\textmd{Aut}(L/\mathbb{Q})\rightarrow S_3$, by associating $\{\alpha,\beta,\overline{\beta}\}$ with $\{1,2,3\}$ in order, and $\sigma\mapsto \tau\in S_3$ iff $\sigma$ permutes $\{\alpha,\beta,\overline{\beta}\}$ in the same way as $\tau$ acts on $\{1,2,3\}$ in the associated order. Then, this map is injective, because if two automorphisms $\sigma_1,\sigma_2$ permute the roots $\{\alpha,\beta,\overline{\beta}\}$ in the same way, since any automorphism in $\textmd{Aut}(L/\mathbb{Q})$ is purely determined on how they permute the given three roots, then $\sigma_1=\sigma_2$. So, this shows that $\textmd{Aut}(L/\mathbb{Q})$ can be identified as a subgroup of $S_3$ (since the composition of two automorphisms acts on the roots in the same way of composing their associated permutations in $S_3$).

\hfil

Then, notice that the conjugation map $\mathbb{C}\rightarrow\mathbb{C}$ when restricting to $L$ is certainly a field automorphism of $L$, since it is an injective map (that is a field automorphism of $\mathbb{C}$), while for every finite $\mathbb{Q}$-combination of the finite products of $\alpha,\beta,\overline{\beta}$, the conjugation only affects each term of $\alpha,\beta,$ and $\overline{\beta}$. However, $\overline{\alpha}=\alpha$, while $\beta$ and $\overline{\beta}$ are conjugates of each other, then for any $k\in L$, since it is a finite $\mathbb{Q}$-combination of products of $\alpha,\beta,\overline{\beta}$, then the conjugation maps $k$ into $L$ (since it acts on each individual $\alpha,\beta,\overline{\beta}$, which their conjugation is closed under $L$).

Based on the given order of the roots $\{\alpha,\beta,\overline{\beta}\}$, the conjugation map can be identified as the transposition $(2\ 3)\in S_3$. Which, since $(2\ 3)$ has order $2$, then $2$ divides $|\textmd{Aut}(L/\mathbb{Q})|$.

Also, from \textbf{Question 1} in this HW, we know because $f(x)=x^3+x+1$ has no roots in $\mathbb{Q}$ (based on rational root theorem, the only rational roots are $\pm 1$, but none of them are the roots of $f(x)$, so $f(x)$ is a degree 3 polynomial with no roots, showing that it is irreducible over $\mathbb{Q}$), then because $L$ is a splitting field of $f(x)\in \mathbb{Q}[x]$, then $\textmd{Aut}(L/\mathbb{Q})$ acts transitively on the set of roots $\{\alpha,\beta,\overline{\beta}\}$, so there exists $\sigma\in\textmd{Aut}(L/\mathbb{Q})$, such that $\sigma(\alpha)=\beta$. There are two scenarios:
\begin{itemize}
    \item[1.] If $\sigma(\beta)=\alpha$, then $\sigma(\overline{\beta})=\overline{\beta}$. This indicates that $\sigma$ only permutes $\alpha$ and $\beta$, which with the order of the roots identified as $\{\alpha,\beta,\overline{\beta}\}$, $\sigma$ corresponds to the transposition $(1\ 2)\in S_3$. Then, since there exists an automorphism that corresponds to $(2\ 3)$ (namely the conjugation map), based on the following relationship, we get:
    $$(1\ 2)\circ (2\ 3)\circ (1\ 2) = (1\ 2)\circ (1\ 3\ 2)= (1\ 3)$$
    This indicates that we get automorphisms in $\textmd{Aut}(L/\mathbb{Q})$ that corresponds to $(1\ 2),\ (2\ 3),\ (1\ 3)\in S_3$ respectively. Hence, with all three transpositions, it generates $S_3$. This indicates that $\textmd{Aut}(L/\mathbb{Q})\cong S_3$.

    \item[2.] If $\sigma(\beta)=\overline{\beta}$ instead, then since $\sigma$ doesn't fix any roots $\{\alpha,\beta,\overline{\beta}\}$, then it corresponds to a $3$-cycle in $S_3$. So, $\sigma$ must have order $3$, showing that $3$ divides $|\textmd{Aut}(L/\mathbb{Q})|$. But, together with the fact that $2$ divides $|\textmd{Aut}(L/\mathbb{Q})|$, then $6$ divides $|\textmd{Aut}(L/\mathbb{Q})|$, showing that $|\textmd{Aut}(L/\mathbb{Q})|\geq 6$. However, since there exists an injection from $|\textmd{Aut}(L/\mathbb{Q})|$ into $S_3$ while $|S_3|=6$, then $|\textmd{Aut}(L/\mathbb{Q})|\leq 6$. With these two facts, $|\textmd{Aut}(L/\mathbb{Q})|=6$, and since $\textmd{Aut}(L/\mathbb{Q})$ can be identified as a subgroup of $S_3$, the order of $\textmd{Aut}(L/\mathbb{Q})$ and $S_3$ match implies that $\textmd{Aut}(L/\mathbb{Q})\cong S_3$.
\end{itemize}
In either case, we'll eventually derive the fact that $\textmd{Aut}(L/\mathbb{Q})\cong S_3$, which finishes the proof.



\break

\section*{4}
\begin{myBox}[]{}
    \begin{question}
        Calculate the splitting field of $f(x)=x^3+x+1\in\mathbb{F}_2[x]$.
    \end{question}
\end{myBox}

\textbf{Pf:}

First, since $0^3+0+1 = 1 \neq 0$, and $1^3+1+1 = 1+1+1 = 1\neq 0$, then $f(x)=x^3+x+1\in\mathbb{F}_2[x]$ has no roots in $\mathbb{F}_2$. Because it is a degree $3$ polynomial, having no roots implies that it is irreducible over $\mathbb{F}_2$.

Now, consider the field $K=\mathbb{F}_2[x]/(x^3+x+1)$: Taken $\theta=\overline{x}\neq 0$, then it satisfies $\overline{x}^3 +\overline{x}+1 = \overline{x^3+x+1} = 0$. Hence, for the same polynomial $f(y)=y^3+y+1\in \mathbb{F}_2[y]\subset K[y]$, the element $\theta\in K$ is a root. This implies that $(y-\theta)\in K[x]$ is a linear factor of $f(y)$, so there exists $\alpha,\beta\in K$, such that the following factorization holds:
$$f(y)=y^3+y+1 = (y-\theta)(y^2+\alpha y + \beta) = y^3 + (-\theta+\alpha)y^2 + (-\theta\alpha+\beta)y + (-\theta)\beta$$
Which, solving for coefficients, the constant coefficient is $1 = (-\theta)\beta = \theta\beta$, which indicates that $\beta = \theta^{-1}\in K$. Going deeper, we know that the following equation holds:
$$\theta^3 + \theta +1 = \overline{x}^3+\overline{x}+1 = \overline{x^3+x+1}=0,\implies \theta^3+\theta = -1 = 1$$
$$\implies \theta(\theta^2+1)=1\implies \theta^{-1} = \theta^2+1$$
So, we can conclude that $\beta = \theta^2+1\in K$.

On the other hand, if solving for the coefficient of $y^2$, we get that $0 = (-\theta) + \alpha$, which $\alpha =\theta$. 

Hence, the first factorization is given by:
$$f(y)=y^3+y+1 = (y-\theta)(y^2+\alpha y+\beta)= (y-\theta)(y^2+\theta y + (\theta^2+1))$$

\hfil

Now, consider the element $\theta^2 \in K$: If we plug it into the polynomial $y^2+\theta y + (\theta^2+1)\in K[x]$, we get:
$$(\theta^2)^2 + \theta\cdot \theta^2 + (\theta^2+1) = \theta^3\cdot \theta + \theta^3\cdot 1 + (\theta+1)^2 = \theta^3(\theta+1)+(\theta+1)^2= (\theta^3+\theta+1)(\theta+1) = 0$$
(Note: the second equality is true with $(\theta^2+1)=(\theta+1)^2$ is because $K/\mathbb{F}_2$ is a characteristic-$2$ field, and the last equality is true since $\theta$ is a root of $y^3+y+1\in K[x]$).

This shows that $\theta^2\in K$ is a root of $y^2+\theta y+(\theta^2+1)\in K[x]$, hence $(y-\theta^2)$ is a linear factor of it, which there exists $\gamma\in K$, such that the following holds:
$$y^2+\theta y + (\theta^2+1) = (y-\theta^2)(y-\gamma)$$
So, within $K$, $f(y)=y^3+y+1$ can be factored as:
$$f(y)=y^3+y+1 = (y-\theta)(y+\theta y+(\theta^2+1)) = (y-\theta)(y-\theta^2)(y-\gamma)$$
Hence, $f(y)$ splits completely over $K$.

\hfil

Finally, let $K'\subseteq K$ be the splitting field of $f(y)$ under $K$. Then, let $\alpha\in K'$ be the root of $f(y)$, we know $\mathbb{F}_2(\alpha)\subseteq K'$, which since $f(x)=x^3+x+1\in \mathbb{F}_2[x]$ is proven to be irreducible, while being monic, then $f(\alpha) = 0$ implies that $f(x)$ is the minimal polynomial of $\alpha$ over $\mathbb{F}_2$. Hence, $\mathbb{F}_2(\alpha)\cong \mathbb{F}_2[x]/(x^3+x+1)$, which shows that $[F(\alpha):\mathbb{F}_2]=3$. However, since $K=\mathbb{F}_2[x]/(x^3+x+1)$, then $[K:\mathbb{F}_2]=3$. So, since $\mathbb{F}_2(\alpha)\subseteq K'\subseteq K$, this enforces $\mathbb{F}_2(\alpha) = K$ (since they have the same dimension with base field $\mathbb{F}_2$), which further enforces $K'=K$. So, $K=\mathbb{F}_2[x]/(x^3+x+1)$ is a splitting field of $f(x)=x^3+x+1\in\mathbb{F}_2[x]$.

\break

\section*{5}
\begin{myBox}[]{}
    \begin{question}
        Let $f(x)\in\mathbb{F}_p[x]$ be an irreducible polynomial of degree $3$. Prove that $f(x)$ is irreducible over $\mathbb{F}_{p^5}$.
    \end{question}
\end{myBox}

\textbf{Pf:}

Since $|\mathbb{F}_{p^5}| = p^5$, then as a field extension of $\mathbb{F}_p$, we must have $[\mathbb{F}_{p^5}:\mathbb{F}_p]=5$ (Note: Given any finite extension $K/\mathbb{F}_p$, if $[K:\mathbb{F}_p]=k\in \mathbb{N}$, then the number of elements $|K| = p^k$).

Now, given $f(x)\in\mathbb{F}_p[x]$ that is irreducible with degree $3$, we'll prove by contradiction that it is irreducible over $\mathbb{F}_{p^5}$.
Suppose the contrary that it is reducible over $\mathbb{F}_{p^5}$, then since it is degree $3$, being reducible in $\mathbb{F}_{p^5}$ implies there exists a root in $\mathbb{F}_{p^5}$. 

Let $\alpha\in\mathbb{F}_{p^5}$ be a root of $f(x)$, then since $f(x)\in\mathbb{F}_p[x]$ is irreducible, and WLOG can assume it is monic (by multiplying by $a^{-1}$, where $a$ is the leading coefficient of $f$), so we can treat $f(x)$ as the minimal polynomial of $\alpha$ over $\mathbb{F}_p$. Hence, $\mathbb{F}_p[x]/(f(x))\cong \mathbb{F}_p(\alpha)\subseteq \mathbb{F}_{p^5}$.

However, since $f(x)$ has degree $3$, then $[\mathbb{F}_p(\alpha):\mathbb{F}_p] = 3$ based on the fact that $\mathbb{F}_p[x]/(f(x))\cong \mathbb{F}_p(\alpha)$; however, since $\mathbb{F}_p\subseteq \mathbb{F}_p(\alpha)\subseteq \mathbb{F}_{p^5}$, while $\mathbb{F}_p(\alpha)/\mathbb{F}_p$ and $\mathbb{F}_{p^5}/\mathbb{F}_p$ are proven to be finite extensions from above, then we get the following:
$$5=[\mathbb{F}_{p^5}:\mathbb{F}_p] = [\mathbb{F}_{p^5}:\mathbb{F}_p(\alpha)]\cdot [\mathbb{F}_p(\alpha):\mathbb{F}_p] = [\mathbb{F}_{p^5}:\mathbb{F}_p(\alpha)]\cdot 3$$
This indicates that $3\mid 5$, yet this is a contradiction since $3$ and $5$ are coprime. So, our assumption is false, $f(x)$ must be irreducible over $\mathbb{F}_{p^5}$.

\end{document}