\documentclass{article}
\usepackage{graphicx} % Required for inserting images
\usepackage[margin = 2.54cm]{geometry}
\usepackage[most]{tcolorbox}

\newtcolorbox{myBox}[3]{
arc=5mm,
lower separated=false,
fonttitle=\bfseries,
%colbacktitle=green!10,
%coltitle=green!50!black,
enhanced,
attach boxed title to top left={xshift=0.5cm,
        yshift=-2mm},
colframe=blue!50!black,
colback=blue!10
}

\usepackage{amsmath}
\usepackage{amssymb}
\usepackage{verbatim}
\usepackage[utf8]{inputenc}
\linespread{1.2}

\newtheorem{definition}{Definition}
\newtheorem{proposition}{Proposition}
\newtheorem{theorem}{Theorem}
\newtheorem{question}{Question}

\title{Math 111C HW4}
\author{Zih-Yu Hsieh}

\begin{document}
\maketitle

\section*{1}
\begin{myBox}[]{}
    \begin{question}
        Let $F$ be a field and $f\in F[x]$ be an irreducible polynomial. Prove that all roots of $f(x)$ in $\overline{F}$ have the same multiplicity.
    \end{question}
\end{myBox}

\textbf{Pf:}

\break

\section*{2}
\begin{myBox}[]{}
    \begin{question}
        \begin{itemize}
            \item[(a)] Let $\zeta_6\in\mathbb{C}$ be a primitive $6^{th}$ root of unity. Find $m_{\zeta_6,\mathbb{Q}}(x)$.
            \item[(b)] Let $m,n\in\mathbb{N}$ such that $m\equiv 2(\mod\ 6)$ and $n\equiv 4(\mod\ 6)$. Prove that $f(x)=x^m+x^n+1$ is not irreducible over $\mathbb{Q}$. 
        \end{itemize}
    \end{question}
\end{myBox}

\textbf{Pf:}

\break

\section*{3}
\begin{myBox}[]{}
    \begin{question}
        Prove that if $F$ is an infinite field, then its multiplicative group $F^\times$ is never cyclic.
    \end{question}
\end{myBox}

\textbf{Pf:}

Suppose the contrary, that $F$ is infinite while $F^\times$ is cyclic, then there exists $a\in F^\times$, such that $F^\times = \left<a\right>$ (under multiplication).

First, notice that $a\neq 0$ (since $a\in F^\times$) and $a\neq 1$ (since $\left<1\right> = \{1\}$, if $a=1$, then $F^\times$ is finite, contradicting the assumption that $F=F^\times\cup \{0\}$ is infinite). Also, since $F^\times$ must be infinite based on similar reason, then $a\neq -1$ (since $(-1)^2=1$, then $|\left<a\right>| = |a|=2$ as the order of $a$, showing that $F^\times$ is again finite, which is a contradiction).

\break

\section*{4}
\begin{myBox}[]{}
    \begin{question}
        Let $K/F$ be a field extension and $m,n\in\mathbb{N}$. Let $\alpha,\beta\in K$ with $[F(\alpha):F]=m$ and $[F(\beta):F]=n$.
        \begin{itemize}
            \item[(a)] Show that $[F(\alpha,\beta):F]\leq mn$.
            \item[(b)] If $\gcd(m,n)=1$, show that $[F(\alpha,\beta):F]=mn$.
        \end{itemize}
    \end{question}
\end{myBox}

\textbf{Pf:}

\break

\section*{5}
\begin{myBox}[]{}
    \begin{question}
        Let $K$ be a finite field. Show that $K$ is not algebraically closed.
    \end{question}
\end{myBox}

\textbf{Pf:}

\end{document}