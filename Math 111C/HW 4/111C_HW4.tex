\documentclass{article}
\usepackage{graphicx} % Required for inserting images
\usepackage[margin = 2.54cm]{geometry}
\usepackage[most]{tcolorbox}

\newtcolorbox{myBox}[3]{
arc=5mm,
lower separated=false,
fonttitle=\bfseries,
%colbacktitle=green!10,
%coltitle=green!50!black,
enhanced,
attach boxed title to top left={xshift=0.5cm,
        yshift=-2mm},
colframe=blue!50!black,
colback=blue!10
}

\usepackage{amsmath}
\usepackage{amssymb}
\usepackage{verbatim}
\usepackage[utf8]{inputenc}
\linespread{1.2}

\newtheorem{definition}{Definition}
\newtheorem{proposition}{Proposition}
\newtheorem{theorem}{Theorem}
\newtheorem{question}{Question}

\title{Math 111C HW4}
\author{Zih-Yu Hsieh}

\begin{document}
\maketitle

\section*{1}
\begin{myBox}[]{}
    \begin{question}
        Let $F$ be a field and $f\in F[x]$ be an irreducible polynomial. Prove that all roots of $f(x)$ in $\overline{F}$ have the same multiplicity.
    \end{question}
\end{myBox}

\textbf{Pf:}

Given $f(x)\in F[x]$ an irreducible polynomial, which WLOG, can assume $f$ is monic (by dividing the nonzero leading coefficient). Then, for any root $\alpha$ of $f$ in some field extension of $F$, since $f$ is monic and irreducible, it is in fact a minimal polynomial of $\alpha$.

Hence, the following forms a well-defined field isomorphism that fixes $F$:
$$\phi_\alpha:F[x]/(f(x))\rightarrow F(\alpha),\quad \forall c_0,c_1,...,c_n\in F,\quad \phi_\alpha(c_0+c_1\overline{x}+...+c_n\overline{x}^n) = c_0+c_1\alpha+...+c_n\alpha^n$$
Whih, given $\alpha,\beta$ two roots of $f$, the following map is a well-defined field isomorphism that fixes $F$ also:
$$\psi = \phi_\beta\circ \phi_\alpha^{-1}:F(\alpha)\rightarrow F(\beta),\quad \forall c_0,c_1,...,c_n\in F$$
$$\psi(c_0+c_1\alpha+...+c_n\alpha^n) = \phi_\beta\circ \phi_\alpha^{-1}(c_0+c_1\alpha+...+c_n\alpha^n) = \phi_\beta(c_0+c_1\overline{x}+...+c_n\overline{x}^n) = c_0+c_1\beta+...+c_n\beta^n$$
Notice that such field isomorphism $psi:F(\alpha)\rightarrow F(\beta)$ can be extended to a ring isomorphism $\overline{\psi}:F(\alpha)[x]\rightarrow F(\beta)[x]$, given as:
$$\forall a_0,a_1,...,a_n\in F(\alpha),\quad \overline{\psi}(a_0+a_1x+...+a_nx^n)=\psi(a_0)+\psi(a_1)x+...+\psi(a_n)x^n$$
So, $\overline{\psi}\bigm|_{F(\alpha)} = \psi$. Which, because $f(x)\in F[x]$, it has all the coefficients being in $F$, then $\overline{\psi}(f(x))=f(x)\in F(\beta)[x]$.

\hfil

Now, given that $\alpha$ has multiplicity $k$, and $\beta$ has multiplicity $l$, this implies that $(x-\alpha)^k\mid f(x)$ over $F(\alpha)$ (with any $n>k$ fails to satisfy this condition), while $(x-\beta)^l\mid f(x)$ over $F(\beta)$ (with any $m>l$ fails to satisfy this condition).

Then, since $f(x)=(x-\alpha)^kp_1(x)$ for some $p_1(x)\in F(\alpha)[x]$, we have the following:
$$f(x)=\overline{\psi}(f(x)) = \overline{\psi}((x-\alpha)^k)\overline{\psi}(p_1(x)) = (x-\beta)^k\overline{\psi}(p_1(x))$$
(Note: since $\overline{\psi}(x-\alpha) = x-\psi(\alpha)=x-\beta$, the above equality holds).

Which, the above equation shows that $(x-\beta)^k\mid f(x)$, hence $k\leq l$;
on the other hand, if consider $\overline{\psi}^{-1}$, since $f(x)=(x-\beta)^lp_2(x)$ for some $p_2(x)\in F(\beta)[x]$, we have the following:
$$f(x)=\overline{\psi}^{-1}(f(x))=\overline{\psi}^{-1}((x-\beta)^l)\overline{\psi}^{-1}(p_2(x)) = (x-\alpha)^l\overline{\psi}^{-1}(p_2(x))$$
Hence, $(x-\alpha)^l\mid f(x)$, showing that $l\leq k$. Which, we can conclude that $l=k$, so $\alpha,\beta$ have the same multiplicity.

\break

\section*{2}
\begin{myBox}[]{}
    \begin{question}
        
        \hfil

        \begin{itemize}
            \item[(a)] Let $\zeta_6\in\mathbb{C}$ be a primitive $6^{th}$ root of unity. Find $m_{\zeta_6,\mathbb{Q}}(x)$.
            \item[(b)] Let $m,n\in\mathbb{N}$ such that $m\equiv 2\mod\ 6$ and $n\equiv 4\mod\ 6$. Prove that $f(x)=x^m+x^n+1$ is not irreducible over $\mathbb{Q}$. 
        \end{itemize}
    \end{question}
\end{myBox}

\textbf{Pf:}

\begin{itemize}
    \item[(a)] Since $\zeta_6$ satisfies $(\zeta_6)^6-1 = 0$, then $\zeta_6$ is a root of the polynomial $x^6-1 \in\mathbb{Q}[x]$.
    
    Notice that $x^6-1$ has the following factorization in $\mathbb{Q}$:
    $$x^6-1 = (x^3-1)(x^3+1)=(x-1)(x^2+x+1)(x+1)(x^2-x+1)$$
    (Note: the two above quadratic polynomials are irreducible over $\mathbb{Q}$, since the only possible rational roots are $\pm 1$, while none of them are actually the root of the quadratic polynomials).

    Which, $\zeta_6$ cannot be the root of $(x-1)$ or $(x+1)$ (since $\zeta_6\notin\mathbb{Q}$), and $\zeta_6$ cannot be a root of $x^2+x+1$ either:
    Suppose the contrary that $\zeta_6$ is a root of $x^2+x+1$, then it implies that $0 = (\zeta_6-1)\cdot 0 = (\zeta_6-1)((\zeta_6)^2+\zeta_6+1) = (\zeta_6)^3-1$. So, $\zeta_6 \in \mu_3$ (where $\mu_3$ is the multiplicative group of the $3^{rd}$ roots of unity). Then, the multiplicative group of the $6^{th}$ roots of unity, $\mu_6 = \left<\zeta_6\right> \subseteq \mu_3$, which is a contradiction (since $\mu_6$ contains more elements than $\mu_3$), hence the assumption is false, $\zeta_6$ cannot be a root of $x^2+x+1$.

    Then, since $\zeta_6$ is a root of $x^6-1$, while not a root for $(x-1),(x+1),$ and $(x^2+x+1)$, then it must be a root of $x^2-x+1$.

    Since $(x^2-x+1)$ is irreducible (since it has no roots over $\mathbb{Q}$, and has degree 2) while being monic, then it must be the irreduible polynomial of $\zeta_6$. So:
    $$m_{\zeta_6,\mathbb{Q}}(x) = x^2-x+1$$

    \hfil

    \item[(b)] Given that $m=6k+2$ and $n=6l+4$ for some $k,l\in\mathbb{Z}$. Notice that since $(\zeta_6)^6 = ((\zeta_6)^2)^3 = 1$, then $(\zeta_6)^2\neq 1$ is in fact a $3^{rd}$ root of unity. Then, plug $\zeta_6$ into the polynomial $x^m+x^n+1$, we get:
    $$(\zeta_6)^m+(\zeta_6)^n+1 = (\zeta_6)^{6k+2}+(\zeta_6)^{6l+4}+1 = (\zeta_6)^2 + (\zeta_6)^4 +1 = ((\zeta_6)^2)^2+(\zeta_6)^2+1$$
    Which, from the relation $(\zeta_6)^6-1=0$, we get:
    $$0=((\zeta_6)^2)^3-1 = ((\zeta_6)^2-1)(((\zeta_6)^2)^2+(\zeta_6)^2+1)$$
    And, since $(\zeta_6)^2 \neq 1$, the first linear term is not zero. Therefore, for the above expression to be $0$, we need:
    $$((\zeta_6)^2)^2+(\zeta_6)^2+1 = 0$$
    Hence, $\zeta_6$ is a root of $x^m+x^n+1$, showing that $m_{\zeta_6,\mathbb{Q}}(x)\mid (x^m+x^n+1)$.
    Also, because both $m,n\in\mathbb{N}$, then $m\equiv 2\mod 6$ enforces $m\geq 2$, and $n\equiv 4\mod 6$ enforces $n\geq 4$, so $\deg(x^m+x^n+1)\geq 4$, while $\deg(m_{\zeta_6,\mathbb{Q}}) = 2$ (given in \textbf{part (a)}), so $m_{\zeta_6,\mathbb{Q}}\neq x^m+x^n+1$. Hence, $x^m+x^n+1$ is reducible over $\mathbb{Q}$ (since $m_{\zeta_6,\mathbb{Q}}$ is a proper factor of it).
\end{itemize}

\break

\section*{3}
\begin{myBox}[]{}
    \begin{question}
        Prove that if $F$ is an infinite field, then its multiplicative group $F^\times$ is never cyclic.
    \end{question}
\end{myBox}

\textbf{Pf:}

Suppose the contrary, that $F$ is infinite while $F^\times$ is cyclic, then there exists $a\in F^\times$, such that $F^\times = \left<a\right>$ (under multiplication). There are two cases to consider:

\hfil

\textbf{Characteristic $0$ Field:}

Given that $\textmd{char}(F)=0$, then $-1 \neq 1$ (since if $-1=1$ in $F$, then $1+1=0$, showing that $1$ has order $2$ under addition, or $\textmd{char}(F)=2$). So, since $-1\in F^\times$, then there exists $l\in\mathbb{Z}$, such that $a^l = -1$.

Yet, this implies that $a^{2l} = (-1)^2 = 1$, so $|a|\leq 2l$, which further implies that $|\left<a\right>| \leq 2l$, so $F^\times=\left<a\right>$ is in fact finite. And, this is a contradiction.

\hfil

\textbf{Characteristic $p>0$ field:}

For all such field $F$, the prime subfield is $\mathbb{F}_p$. Hence, can view $F$ as a field extension of $\mathbb{F}_p$.

First, notice that $a\neq 0$ (since $a\in F^\times$) and $a\neq 1$ (since $\left<1\right> = \{1\}$, if $a=1$, then $F^\times$ is finite, contradicting the assumption that $F=F^\times\cup \{0\}$ is infinite). 

Also, since $F^\times$ must be infinite based on similar reason, then $a\neq -1$ (since $(-1)^2=1$, if $a=-1$, then $|\left<a\right>| = |a|=2$ as the order of $a$, showing that $F^\times$ is again finite, which is a contradiction). So, it implies that $a+1\neq 0$, hence $a+1\in F^\times$.

Then, there exists $l\in\mathbb{Z}$, such that $a^l = a+1$, or $a^l-a-1 =0$. Which, there are several situations:
\begin{itemize}
    \item Suppose $l=0$, then $a^0 = 1$, so $a+1 = 1$, or $a=0$, which contradicts the fact that $a\neq 0$, so we don't need to consider this case.
    \item Suppose $l>0$, then $a$ is a root of the polynomial $x^l-x-1 \in \mathbb{F}_p[x]$.
    \item Else if $l<0$, then $(-l)>0$. So, $a^{(-l)}(a^l-a-1) = 1-a^{1-l}-a^{-l} = 0$, showing that $a$ is a root of the polynomial $1-x^{1-l}-x^{-l}\in\mathbb{F}_p[x]$.
\end{itemize}
So, in either cases, there exists a polynomial $p(x)\in \mathbb{F}_p[x]$, such that $p(a)=0$, hence $a\in F/\mathbb{F}_p$ is algebraic, its minimal polynomial $m_{a,\mathbb{F}_p}(x)\in\mathbb{F}_p[x]$ exists.

Then, $\mathbb{F}_p(a) \cong \mathbb{F}_p[x]/(m_{a,\mathbb{F}_p}(x))$ is a finite extension, which further implies that $\mathbb{F}_p(a)$ is finite (finite extension of a finite field is finite).

However, for all $b\in F$, if $b=0$, $b\in\mathbb{F}_p(a)$; on the other hand, if $b\neq 0$, since $b\in F^\times = \left<a\right>$, then $b=a^l\in\mathbb{F}_p(a)$ for some $l\in\mathbb{Z}$. Hence, $F\subseteq\mathbb{F}_p(a)$, while $\mathbb{F}_p(a)\subseteq F$, showing that $F=\mathbb{F}_p(a)$. This implies that $F$ is finite, which again contradicts the assumption that $F$ is an infinite field.

\hfil

Since in all cases, $F^\times$ being cyclic would lead to a contradiction, then if $F$ is infinite, $F^\times$ cannot be cyclic.

(Note: The proof fo $\textmd{char}(F)=p$ is designed for $p=2$ specifically, since in that case $-1=1$, the proof used for $\textmd{char}(F)=0$ cannot work. If $p>2$, the proof for $\textmd{char}(F)=0$ works perfectly fine).

\break

\section*{4}
\begin{myBox}[]{}
    \begin{question}
        Let $K/F$ be a field extension and $m,n\in\mathbb{N}$. Let $\alpha,\beta\in K$ with $[F(\alpha):F]=m$ and $[F(\beta):F]=n$.
        \begin{itemize}
            \item[(a)] Show that $[F(\alpha,\beta):F]\leq mn$.
            \item[(b)] If $\gcd(m,n)=1$, show that $[F(\alpha,\beta):F]=mn$.
        \end{itemize}
    \end{question}
\end{myBox}

\textbf{Pf:}

Given the condition, since $F(\alpha),F(\beta)$ are both finite extensions of $F$, then $\alpha,\beta$ are algebraic over $F$. Also, with the degree given, we know $m = \deg(m_{\alpha,F})$, while $n=\deg(m_{\beta,F})$.
\begin{itemize}
    \item[(a)]
    Given $F\subseteq F(\alpha)\subseteq F(\alpha,\beta)$, we have the following relation:
    $$[F(\alpha,\beta):F(\alpha)]\cdot [F(\alpha):F]=[F(\alpha,\beta):F]$$
    Which, since $m_{\beta,F}(x)\in F[x]\subseteq F(\alpha)[x]$, then $\beta$ is also algebraic over $F(\alpha)$. Hence, $m_{\beta,F(\alpha)}(x)\in F(\alpha)[x]$ exists, while $m_{\beta,F(\alpha)}(x)\mid m_{\beta,F}(x)$ (since $m_{\beta,F}(\beta)=0$ by definition).
    This implies that $\deg(m_{\beta,F(\alpha)})\leq \deg(m_{\beta,F}) = n$.

    Which, since $F(\alpha,\beta) = F(\alpha)(\beta)$, then $[F(\alpha,\beta):F(\alpha)] = \deg(m_{\beta,F(\alpha)}) \leq n$, hence we get the following inequality:
    $$[F(\alpha,\beta):F]=[F(\alpha,\beta):F(\alpha)]\cdot [F(\alpha):F]\leq mn$$

    \hfil

    \item[(b)] Now suppose $\gcd(m,n)=1$, then $\textmd{lcm}(m,n)=mn$. Which, notice that both $F(\alpha),F(\beta)$ are subfields of $F(\alpha,\beta)$, hence the following two equality holds:
    $$[F(\alpha,\beta):F]=[F(\alpha,\beta):F(\alpha)]\cdot [F(\alpha):F] = [F(\alpha,\beta):F(\alpha)]\cdot m$$
    $$[F(\alpha,\beta):F]=[F(\alpha,\beta):F(\beta)]\cdot [F(\beta):F] = [F(\alpha,\beta):F(\beta)]\cdot n$$
    Hence, since $m\mid [F(\alpha,\beta):F]$ and $n\mid [F(\alpha,\beta):F]$, then $\textmd{lcm}(m,n)=mn$ divides $[F(\alpha,\beta):F]$; on the other hand, since in \textbf{part (a)} we've shown that $[F(\alpha,\beta):F]\leq mn$, then $[F(\alpha,\beta):F] = mn$.
\end{itemize}

\break

\section*{5}
\begin{myBox}[]{}
    \begin{question}
        Let $K$ be a finite field. Show that $K$ is not algebraically closed.
    \end{question}
\end{myBox}

\textbf{Pf:}

Suppose the contrary that some finite field $K$ is algebraically closed, it implies that all polynomial in $K[x]$ has a root in $K$. Hence, the goal is to find a polynomial with no roots in $K$.

Consider the following example:
$$f(x)=1+\prod_{k\in K}(x-k)\in K[x]$$
Since $K$ is finite, the above polynomial is well-defined. Also, for any $a\in K$, if plug into $f(x)$, we get:
$$f(a)=1+(a-a)\prod_{\substack{k\in K\\k\neq a}}(a-k) = 1+0\cdot \prod_{\substack{k\in K\\k\neq a}}(a-k) = 1$$
This shows that none of the element $a\in K$ is a root of $f(x)\in K[x]$, which contradicts the assumption that $K$ is algebraically closed.

Hence, the assumption is false, any finite field $K$ is not algebraically closed.

\end{document}