\documentclass{article}
\usepackage[margin = 2.54cm]{geometry} % set margin to traditional doc

%packages
\usepackage{graphicx} % Required for inserting images
\usepackage[most]{tcolorbox} %for creating environments
\usepackage{amsmath}
\usepackage{amssymb}
\usepackage{verbatim}
\usepackage[utf8]{inputenc}
\usepackage[dvipsnames]{xcolor} %for importing multiple colors
\usepackage{hyperref} %for creating links to different sections

\linespread{1.2} %controlling line spread

%define colors i like
\definecolor{myTeal}{RGB}{0,128,128}
\definecolor{myGreen}{RGB}{34,170,34}
\definecolor{mySapphire}{RGB}{15,82,186}
\definecolor{myEmerald}{RGB}{50.4, 130, 90}

%create math environments, can add [section] or [subsection] to add index counter based on sections/subsections
\newtheorem{define}{Definition}
\newtheorem{prop}{Proposition}
\newtheorem{thm}{Theorem}
\newtheorem{question}{Question}
\newtheorem{lemma}{Lemma}

%setup colored box environment for each math env above
\tcolorboxenvironment{define}{
    enhanced, colframe=myTeal!50!teal, colback=myTeal!10,
    arc=5mm, lower separated=false, fonttitle=\bfseries
}
\tcolorboxenvironment{prop}{
    enhanced, colframe=myGreen!50!black, colback=myGreen!15,
    arc=5mm, lower separated=false, fonttitle=\bfseries
}
\tcolorboxenvironment{thm}{
    enhanced, colframe=mySapphire!50!mySapphire, colback=mySapphire!15,
    arc=5mm, lower separated=false, fonttitle=\bfseries
}
\tcolorboxenvironment{question}{
    enhanced, colframe=blue!50!black, colback=blue!10,
    arc=5mm, lower separated=false, fonttitle=\bfseries
}
\tcolorboxenvironment{lemma}{
    enhanced, colframe=myEmerald!50!myEmerald, colback=myEmerald!10,
    arc=5mm, lower separated=false, fonttitle=\bfseries
}

%setup hyperlink within pdf
\hypersetup{
    colorlinks=true,
    linkcolor=blue,
    filecolor=magenta,      
    urlcolor=cyan,
    pdftitle={Overleaf Example},
    pdfpagemode=FullScreen,
}

%common command (add to template)
%general
\newcommand{\FF}{\mathbb{F}}
\newcommand{\QQ}{\mathbb{Q}}
\newcommand{\RR}{\mathbb{R}}
\newcommand{\CC}{\mathbb{C}}

%algebra
\newcommand{\Gal}{\textmd{Gal}}
\newcommand{\Aut}{\textmd{Aut}}

%analysis
\newcommand{\Vol}{\textmd{Vol}}


\title{Math 111C HW8}
\author{Zih-Yu Hsieh}

\begin{document}
\maketitle

\section{}
\begin{question}\label{q1}
    Show that every finite separable extension $K/F$ has only finitely many sub-extensions.
\end{question}

\textbf{Pf:}

\break

\section{}
\begin{question}\label{q2}
    Let $L\subseteq\mathbb{C}$ be the splitting field of $f(x)=x^3-3x+1$ over $\mathbb{Q}$. Let $\alpha,\beta,\gamma\in L$ be roots of $f(x)$.
    \begin{itemize}
        \item[(a)] Calculate $\Gal(L/\QQ)$ as a group of permutations of $\{\alpha,\beta,\gamma\}$.
        \item[(b)] Is there an automorphism of $L$ that acts on $\{\alpha,\beta,\gamma\}$ as the transposition $(\alpha,\beta)$?
    \end{itemize}
    (\textbf{Hint:-} For a polynomial $f(x)=x^3+ax^2+bx+c\in\QQ[x]$ wth roots $\alpha,\beta,\gamma\in \CC$, the discriminant of $f(x)$, $D$ is defined as 
    $$D=(\alpha-\beta)^2(\beta-\gamma)^2(\gamma-\alpha)^2$$
    It is known that $D=18abc+a^2b^2-4b^3-4a^3c-27c^2$).
\end{question}

\textbf{Pf:}

\break

\section{}
\begin{question}\label{q3}
    Repeat the above question with $f(x)=x^3-4x+1$.
\end{question}

\textbf{Pf:}

\break

\section{}
\begin{question}\label{q4}
    Let $L\subseteq \CC$ be the splitting field of $f(x)=(x^3-2)(x^2-3)$ over $\QQ$.
    \begin{itemize}
        \item[(a)] Show that $L=\QQ(\sqrt{2}+\sqrt{3})$ and $[L:\QQ]=4$.
        \item[(b)] Find $\Gal(L/\QQ)$ as a group of permutations of the roots of $f$.
        \item[(c)] Which elements of your ansewr to (b) belong to the subgroup $\Gal(L/\QQ(\sqrt{6}))$?  
    \end{itemize}
\end{question}

\textbf{Pf:}

\break

\section{}
\begin{question}\label{q5}
    The Galois group of a polynomial $f(x)$ over a perfect field $F$ is defined as $\Gal(K/F)$ where $K$ is a splitting field of $f(x)$. Find the Galois groups of $x^6-1$ over $\FF_5$, $\FF_{5^2}$, and $\FF_{5^3}$.
\end{question}

\textbf{Pf:}

\end{document}