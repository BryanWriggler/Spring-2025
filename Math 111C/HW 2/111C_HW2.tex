\documentclass{article}
\usepackage{graphicx} % Required for inserting images
\usepackage[margin = 2.54cm]{geometry}
\usepackage[most]{tcolorbox}

\newtcolorbox{myBox}[3]{
arc=5mm,
lower separated=false,
fonttitle=\bfseries,
%colbacktitle=green!10,
%coltitle=green!50!black,
enhanced,
attach boxed title to top left={xshift=0.5cm,
        yshift=-2mm},
colframe=blue!50!black,
colback=blue!10
}

\usepackage{amsmath}
\usepackage{amssymb}
\usepackage{verbatim}
\usepackage[utf8]{inputenc}
\linespread{1.2}

\newtheorem{definition}{Definition}
\newtheorem{proposition}{Proposition}
\newtheorem{theorem}{Theorem}
\newtheorem{question}{Question}

\title{Math 111C HW2}
\author{Zih-Yu Hsieh}

\begin{document}
\maketitle

\section*{1}
\begin{myBox}[]{}
    \begin{question}
        Let $F\subseteq K\subseteq L$ be fields and let $\alpha\in L$ be algebraic over $F$. Prove that $[K(\alpha):K]\leq [F(\alpha):F]$.
    \end{question}
\end{myBox}

\textbf{Pf:}

Given that $\alpha\in L$ is algebraic over $F$, there exists minimal polynomial $f_{\alpha,F}(x)\in F[x]$, such that $f_{\alpha, F}(\alpha)=0$.
On the other hand, since $F[x]\subseteq K[x]$, then $f_{\alpha, F}(x)$ as a polynomial over $K$ has $\alpha$ being a root, implies that $\alpha$ is also algebraic over $K$,
hence its minimal polynomial $p_{\alpha, K}(x)\in K[x]$ exists, with $p_{\alpha,K}(\alpha)=0$.

Now, since $f_{\alpha,F}(x)\in K[x]$ has $\alpha$ being a root, then it implies that $p_{\alpha,K}(x)\mid f_{\alpha,F}(x)$ (since minimal polynomial of $\alpha$ divides all polynomials having $\alpha$ being a root),
hence $\deg(p_{\alpha,K})\leq \deg(f_{\alpha,F})$.

Lastly, since $\alpha$ is algebraic over both $F$ and $K$, then $F(\alpha)\cong F[x]/(f_{\alpha,F}(x))$ (which the extension has degree being $\deg(f_{\alpha,F})$) and $K(\alpha)\cong K[x]/(p_{\alpha,K}(x))$ (which the extension has degree being $\deg(p_{\alpha,K})$),
then:
$$[K(\alpha):K]=[K[x]/(p_{\alpha,K}(x)):K]=\deg(p_{\alpha,K})\leq\deg(f_{\alpha,F})=[F[x]/(f_{\alpha,K}(x)):F]=[F(\alpha):F]$$
Hence, $[K(\alpha):K]\leq [F(\alpha):F]$.


\break

\section*{2}
\begin{myBox}[]{}
    \begin{question}
        Let $K/F$ be a field extension and $\alpha_1,...,\alpha_n\in K$ be algebraic over $F$. Prove that $F(\alpha_1,...,\alpha_n)=F[\alpha_1,...,\alpha_n]$.
    \end{question}
\end{myBox}

\textbf{Pf:}

We'll prove this statement by induction on the number of elements $n$.

\hfil

First, given arbitrary field $F$, for $n=1$, if $\alpha_1\in K/F$ is algebraic over $F$, if we consider its minimal polynomial $m_{\alpha,F}(x)\in F[x]$, since it is irreducible while $F[x]$ is a PID,
then $F[x]/(m_{\alpha,F}(x))$ is a field. 

Now, consider the following ring homomorphism $\phi:F[x]/(m_{\alpha,F}(x))\rightarrow F[\alpha_1]$ by $\phi(\bar{x})=\alpha_1$. Then, since it is a nonzero map, while $F[x]/(m_{\alpha,F}(x))$ is a field, then $\phi$ is injective;
also, for all $a_k\alpha^k+...+a_0\in F[\alpha]$, let $f(x)=a_kx^k+...+a_0\in F[x]$, then $\overline{f(x)}=a_k\bar{x}^k+...+a_0\in F[x]/(m_{\alpha,F}(x))$, it satisfies: 
$$\phi(\overline{f(x)})=\phi(a_k\bar{x}^k+...+a_0)=a_k\phi(\bar{x})^k+...+a_0=a_k\alpha_1^k+...+a_0$$
This shows that $\phi$ is also surjective.

Then, because $\phi$ is bijctive, $F[x]/(m_{\alpha,F}(x))\cong F[\alpha_1]$, hence $F[\alpha_1]$ is a field.

Given that $F(\alpha_1)$ is a field containing all operations of $F$ and $\alpha_1$, we know $F[\alpha_1]\subseteq F(\alpha_1)$; on the other hand, since $F(\alpha_1)$ is defined to be the smallest field containing both $\alpha_1$ and $F$,
because $F[\alpha_1]$ also satisfies this property, then $F(\alpha_1)\subseteq F[\alpha_1]$. This shows that $F(\alpha_1)=F[\alpha_1]$.

\hfil

Now, suppose for given $n\in\mathbb{N}$, given arbitrary field $F$, any $\alpha_1,...,\alpha_n\in K/F$ that are algebraic over $F$ satisfy $F(\alpha_1,...,\alpha_n)=F[\alpha_1,...,\alpha_n]$.
Then, for the case $(n+1)$, given arbitrary $\alpha_1,...,\alpha_n,\alpha_{n+1}\in K$ that are algebraic over $F$, we know $F(\alpha_1,...,\alpha_n,\alpha_{n+1})=F(\alpha_1,...,\alpha_n)(\alpha_{n+1})$,
which by induction hypothesis, $F(\alpha_1,...,\alpha_n)=F[\alpha_1,...,\alpha_n]=K''$. Then, since $F\subseteq K''$, while $\alpha_{n+1}$ is algebraic over $F$, then $\alpha_{n+1}$ is also algebraic over $K''$, so its minimal polynomial $m(x)\in K''[x]$ exists (with respect to field $K''$).

Then, using the same logic for the case $n=1$ (since initially we're using arbitrary field, then let $F=K''$ for the case $n=1$), we know $K''[x]/(m(x))\cong K''[\alpha_{n+1}]=K''(\alpha_{n+1})$.

Hence, we have the following:
$$F(\alpha_1,...,\alpha_n,\alpha_{n+1})=F(\alpha_1,...,\alpha_n)(\alpha_{n+1})=K''(\alpha_{n+1})=K''[\alpha_{n+1}] = F(\alpha_1,...,\alpha_n)[\alpha_{n+1}]$$
$$= F[\alpha_1,...,\alpha_n][\alpha_{n+1}]=F[\alpha_1,...,\alpha_n,\alpha_{n+1}]$$
This proves that $F(\alpha_1,...,\alpha_n,\alpha_{n+1})=F[\alpha_1,...,\alpha_n,\alpha_{n+1}]$ for the case $(n+1)$.

\hfil

Finally, by the principle of mathematical induction, given arbitrary field $F$ and field extension $K/F$, given any $\alpha_1,...,\alpha_n\in K$ that are algebraic over $F$, we have $F(\alpha_1,...,\alpha_n)=F[\alpha_1,...,\alpha_n]$.

\break

\section*{3}
\begin{myBox}[]{}
    \begin{question}
        Let $F$ be a field of caracteristic $p$, where $p$ is a prime number. Suppose that $x^p-a$ where $a\in F$, does not have a root in $F$. 
        Show that $x^p-a$ is irreducible in $F[x]$.
    \end{question}
\end{myBox}

\textbf{Pf:}

We'll prove by contradiction, that if $x^p-a$ has no roots in $F$, then $x^p-a\in F[x]$ is irreducible. Suppose it is reducible, then there exists nonconstant polynomials $q(x),r(x)\in F[x]$, with $x^p-a=q(x)r(x)$.
(Note: it also implies $p=\deg(x^p-a)>\deg(q),\deg(r)$).

\hfil

First, given $x^p-a\in F[x]$, we know there exists a field extension $K/F$, such that $x^p-a$ splits completely over $K$. In particular, there exists $b\in K$, such that $(x-b)$ is a linear factor of $x^p-a$.
This implies that $b^p-a=0$, or $b^p=a$ (since $(x-b)$ is a factor of the polynomial iff $b$ is a root of the polynomial, in a polynomial ring $R[x]$ over a commutative ring $R$).

Now, since $F\subseteq K$, while $K$ is an integral domain, then they share the same unity element $1$; also, because $char(K)$ is dependent on the order of $1$ under addition, while $char(F)=p$ is the order of $1$,
then $char(K)=char(F)=p$. Similar argument applies to $K[x]$, where $char(K[x])=char(K)=p$. Hence, we can apply fresher's dream to the element $(x-b)\in K[x]$,
and get that $(x-b)^p=x^p-b^p=x^p-a$.

Hence, $(x-b)^p$ is a factorization of $x^p-a$, while $x^p-a=q(x)r(x)$. Then, since $q(x),r(x)$ divides $x^p-a$ while $x^p-a$ splits completely over $K$, then $q(x),r(x)$ must also split completely over $K$; 
also, for the factorization to be unique, we must have $q(x)=(x-b)^k$ and $r(x)=(x-b)^l$, where $\deg(q)=k$ and $\deg(r)=l$.

\hfil

Now, if we focus on $q(x)=(x-b)^l\in K[x]$, since $q(x)\in F[x]$, all of its coefficients should be in $F$, then the constant term of $q(x)$ (represented as $(-b)^l\in K$) should also be in $F$. But, here is a contradiction:

Since $b^p=a\in F$, there exists a minimum positive integer $n\leq p$, where $b^n\in F$, but if $n<p$, since $p$ is a prime, then $n\nmid p$, hence $p=qn+r$, for $q,r\in \mathbb{N}$, and $0<r<n$.
Which, we get the following:
$$b^p=b^{nq+r}=(b^n)^q\cdot b^r \in F$$
Also, since $b^p=a$, while the equation $x^p-a$ has no roots in $F$, then $a\neq 0$, hence $b\neq a$. THis implies that $b^n \neq 0$, which is invertible in $F$. Hence:
$$((b^n)^{-1})^\cdot(b^n)^q\cdot b^r=b^r\in F$$
However, since $b^r\in F$ and $r<n$, while $n$ is assumed to be the smallest positive integer satisfying $b^n\in F$, we reach a contradiction.

Therefore or initial assumption must be false, $x^p-a\in F[x]$ must be irreducible.



\break

\section*{4}
\begin{myBox}[]{}
    \begin{question}

        \hfil

        \begin{itemize}
            \item[(a)] Find all ring homomorphisms $\Psi:\mathbb{Q}[x]\rightarrow \mathbb{C}$.
            \item[(b)] Find all ring homomorphisms $\Psi_1:\mathbb{Q}[x]/(x^3-2)\rightarrow\mathbb{C}$ and $\Psi_2:\mathbb{Q}[x]/(x^3-2)\rightarrow\mathbb{R}$.
        \end{itemize}
    \end{question}
\end{myBox}

\textbf{Pf:}

\begin{itemize}
    \item[(a)] First, the zero map $\Psi=0$ is an ansewr.
    
    Now, suppose $\Psi\neq 0$, then because $\Psi(\mathbb{Q}[x])\subset \mathbb{C}$ is a nontrivial subring, while $\mathbb{C}$ is an integral domain, then all its nontrivial subring must have the same identity.
    Hence, $\Psi(1)=1\in\mathbb{C}$. 

    Notice that this also implies that for all $q\in\mathbb{Q}$, $\Psi(q)=q$: Since $\Psi(1)=1$, then for all $a\in \mathbb{Z}$, $\Psi(a)=a$;
    now, if represent $q=\frac{a}{b}$ for $a,b\in\mathbb{Z},\ b\neq 0$, then we get:
    $$\Psi(q)=\Psi(a\cdot b^{-1})=\Psi(a)\cdot\Psi(b)^{-1}=a\cdot b^{-1}=q$$
    Hence, all the rationals are fixed by the homomorphism.

    Now, we can simply define $\Psi(x)=a$ for arbitrary $a\in\mathbb{C}$. Then, for all $f(x)=\sum_{k=0}^{n}f_kx^k\in\mathbb{Q}[x]$, we get:
    $$\Psi(f(x))=\Psi\left(\sum_{k=0}^{n}f_kx^k\right)=\sum_{k=0}^{n}\Psi(f_k)\cdot\Psi(x)^k=\sum_{k=0}^{n}f_ka^k$$
    So, all the nonzero possibilities are characterized by $\Psi(1)=1$, and $\Psi(x)=a$ for arbitrary $a\in\mathbb{C}$.

    \hfil

    \item[(b)] Notice that since $x^3-2$ has no roots over $\mathbb{Q}$, then it must be irreducible. Hence, $\mathbb{Q}[x]/(x^3-2)$ is in fact a field (since the ideal $(x^3-2)\subset \mathbb{Q}[x]$ is maximal).
    
    \textbf{Possibility of $\Psi_1$:}
    
    First, the zero map $\Psi_1=0$ is always an answer. 
    
    For other possibilities, suppose $\Psi_1\neq 0$, because $\mathbb{Q}[x]/(x^3-2)$ is a field, then $\Psi_1$ must be injective, hence $\mathbb{Q}[x]/(x^3-2)\cong \Psi_1(\mathbb{Q}[x]/(x^3-2))\subseteq \mathbb{C}$.
    In this case, since the image is nontrivial, while $\mathbb{C}$ in particular is an integral domain, then all its subring (including the image of $\Psi_1$) must have the same identity as $\mathbb{C}$, so $\Psi_1(1)=1\in\mathbb{C}$.

    Then, since $\bar{x}\in \mathbb{Q}[x]/(x^3-2)$ satisfies $\bar{x}^3-2=\overline{x^3-2}=0$, hence $\Psi_1(\bar{x})\in\mathbb{C}$ must also satisfy this relationship, namely:
    $$0=\Psi_1(\bar{x}^3-2)=\Psi_1(\bar{x})^3-\Psi_1(2)=\Psi_1(\bar{x})^3-2$$
    (Note: since $\Psi_1(1)=1$, then $\Psi_1(2)=\Psi_1(1+1)=\Psi_1(1)+\Psi_1(1)=2$).

    So, $\alpha=\Psi_1(\bar{x})\in\mathbb{C}$ must satisfy $\alpha^3-2=0$, which is a root of $x^3-2\in\mathbb{C}[x]$. Then, the only possibilities of $\alpha\in \mathbb{C}$ is $\alpha=\sqrt[3]{2},\ \sqrt[3]{2}e^{i\cdot 2\pi/3},\ \sqrt[3]{2}e^{i\cdot 4\pi/3}$.

    Therefore, if $\Psi_1\neq 0$, then it must satisfy $\Psi_1(1)=1$, and $\Psi_1(\bar{x})=\sqrt[3]{2},\ \sqrt[3]{2}e^{i\cdot 2\pi/3},$ or $\sqrt[3]{2}e^{i\cdot 4\pi/3}$.

    \hfil

    \textbf{Possibility of $\Psi_2$:}

    Again, the zero map $\Psi_2=0$ is an answer.

    Now, suppose $\Psi_2\neq 0$, again because $\mathbb{Q}[x]/(x^3-2)$ is a field, $\Psi_2$ must be injective, and $\Psi_2(1)=1\in\mathbb{R}$ (based on the same reason as described in $\Psi_1$).

    Again, since $\bar{x}\in\mathbb{Q}[x]/(x^3-2)$ satisfies $\bar{x}^3-2=0$, hence $0=\Psi_2(\bar{x}^3-2)=\Psi_2(\bar{x})^3-\Psi_2(2)=\Psi(\bar{x})^3-2$.
    So, $\beta=\Psi_2(\bar{x})\in\mathbb{R}$ satisfies $\beta^3-2=0$, which is a root of $x^3-2\in\mathbb{R}[x]$. Then, the only possibility of $\beta\in\mathbb{R}$ is $\beta=\sqrt[3]{2}$.

    Therefore, if $\Psi_2\neq 0$, then it must satisfy $\Psi_2(1)=1$, and $\Psi_2(\bar{x})=\sqrt[3]{2}$.

\end{itemize}

\hfil

\hfil

\section*{5}
\begin{myBox}[]{}
    \begin{question}
        Let $p$ be  prime number and $F$ be a finite field with $q=p^k$ elements. Prove that
        $$x^{q-1}-1=\prod_{\alpha\in F^\times}(x-\alpha)$$
        in $F(x)$. By comparing coefficients of suitable powers of $x$, conclude that
        \begin{itemize}
            \item[(a)]$$\sum_{\alpha\in F^\times}\alpha =0$$
            \item[(b)] $$\prod_{\alpha\in F^{\times}}\alpha=-1$$
        \end{itemize}
    \end{question}
\end{myBox}

\textbf{Pf:}

Since $F$ is a finite field wit order $|F|=q$, then $F^\times$ is a group under multiplication with $|F^\times|=q-1$ (without $0$).
Hence, for all $\alpha\in F^\times$, $\alpha^{q-1}=1$ (since $1$ is the identity of $F^\times$, and all element's order of a finite group divides the order of the group itself).

So, given $x^{q-1}-1\in F[x]$, since $F$ is a field, and the polynomial has degree $q-1$, it has at most $q-1$ roots counting multiplicity; on the other hand,
all $\alpha\in F^\times$ satisfies $\alpha^{q-1}-1=1-1=0$, hence $\alpha$ is a root of $x^{q-1}-1$. And, since there are $q-1$ distinct elements in $F^\times$,
then $F^\times$ in fact contains (and only contains) all the roots of $x^{q-1}-1$.

Now, because each $\alpha\in F^\times$ is a root of $x^{q-1}-1$, then $(x-\alpha)\mid (x^{q-1}-1)$, $x^{q-1}-1=(x-\alpha)q_1(x)$. Then, for a distinct $\beta\in F^\times$,
since $0=\beta^{q-1}-1=(\beta-\alpha)q_1(\beta)$, while $\beta\neq \alpha$, then $q_1(\beta)=0$, showing that $(x-\beta)\mid q_1(x)$.
So, inductively, we can factor out all $(x-\alpha)$ (with $\alpha\in F^\times$) as a linear term of $x^{q-1}-1$, showing the following:
$$x^{q-1}-1=q(x)\prod_{\alpha\in F^\times}(x-a),\quad q(x)\in F[x],\quad q(x)\neq 0$$
On the other hand, the left hand side above has degree $q-1$, while the right hand side has degree $\deg(q)+\deg(\prod_{\alpha\in F^\times}(x-a))\geq \deg(\prod_{\alpha\in F^\times}(x-a))=q-1$ 
(Note: since there are $q-1$ elements in $F^\times$, then $\prod_{\alpha\in F^\times}(x-a)$ is a product of $q-1$ linear factors, hence has degree $q-1$).
So, this enforces $\deg(q)=0$, which $q(x)\in F$ must be an invertible element.

Lastly, if we consider the leading coefficient, the left hand side has coefficient $1$, while the right side has coefficient $q(x)$, hence $q(x)=1$, showing the following:
$$x^{q-1}-1=\prod_{\alpha\in F^\times}(x-a)$$
\begin{itemize}
    \item[(a)] Given the above statement, if we consider the coefficient of degree $q-2$, we get $0$ for $x^{q-1}-1$; on the right side, since $x^{q-2}$ can only be obtained by choosing $1$ factor to be constant, 
    while the other factors are $x$, then the $x^{q-2}$ has the coefficient $\sum_{\alpha\in F^\times}\alpha$. Hence, $\sum_{\alpha\in F^\times}\alpha=0$.

    \item[(b)] Again, if we consider the constant term, we get $-1$ for $x^{q-1}-1$; on the right side, since constant term can only be obtained by the product of all constant terms, then it has constant term $\prod_{\alpha\in F^\times}\alpha$.
    Hence, $\prod_{\alpha\in F^\times}\alpha=-1$.
\end{itemize}

\end{document}