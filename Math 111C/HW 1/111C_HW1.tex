\documentclass{article}
\usepackage{graphicx} % Required for inserting images
\usepackage[margin = 2.54cm]{geometry}
\usepackage[most]{tcolorbox}

\newtcolorbox{myBox}[3]{
arc=5mm,
lower separated=false,
fonttitle=\bfseries,
%colbacktitle=green!10,
%coltitle=green!50!black,
enhanced,
attach boxed title to top left={xshift=0.5cm,
        yshift=-2mm},
colframe=blue!50!black,
colback=blue!10
}

\usepackage{amsmath}
\usepackage{amssymb}
\usepackage{verbatim}
\usepackage[utf8]{inputenc}
\linespread{1.2}

\newtheorem{definition}{Definition}
\newtheorem{proposition}{Proposition}
\newtheorem{theorem}{Theorem}
\newtheorem{question}{Question}

\title{Math 111C HW1}
\author{Zih-Yu Hsieh}

\begin{document}
\maketitle

\section*{1}
\begin{myBox}[]{}
    \begin{question}
        Show, using Eisenstein's criterion, that $f(X)=X^3-3X-1$ is irreducible over $\mathbb{Q}$. Let $\alpha$ be a root of $f$ in $\mathbb{C}$. 
        Express $\frac{1}{\alpha}$ and $\frac{1}{\alpha+3}$ as linear combinations of $1$, $\alpha$ and $\alpha^2$.
    \end{question}
\end{myBox}

\textbf{Pf:}

Consider the ring homomorphism $\phi:\mathbb{Z}[X]\rightarrow \mathbb{Z}[X]$ by $\phi(f)=f(X+1)$. Which, $\phi$ is injective, since if $f(X+1)$ is constantly $0$, 
its leading coefficient $a_n=0$, which helps us inductively prove that $f=0$.

Then, for $f(X)=X^3-3X-1$ given above:
$$\phi(f)=f(X+1)=(X+1)^3-3(X+1)-1 = (X^3+3X^2+3X+1)-(3X+3)-1 = X^3+3X^2-3$$
Then, since leading coefficient is $1$, while the rest of the coefficients (namely $3,0,-3$) are divisible by $3$, and $-3$ is not divisible by $3^2$,
so by Eisenstein's criterion, $\phi(f)=X^3+3X^2-3$ is irreducible over $\mathbb{Q}$. Then, since $\phi(f)=f(X+1)$ is irreducible over $\mathbb{Q}$, then $f$ itself must also be irreducible:

Since $f(x)=x^3-3x-1$ is irreducible over $\mathbb{Q}$, then $(f(x))\subseteq\mathbb{Q}[x]$ is in fact a maximal ideal, hence $K=\mathbb{Q}[x]/(f(x))$ is a field, where $\overline{x}\in K$ is a zero of 
$f(\theta)\in K[\theta]$.

\hfil

Now, for the rest of the part, consider the ring homomorphism $\phi:K\rightarrow \mathbb{C}$ by $\phi(\overline{x})=\alpha$. Which, since $f$ is irreducible over $\mathbb{Q}$, $0\in\mathbb{Q}$ is not a zero of $f$,
hence if $f(\alpha)=0$ over $\mathbb{C}$, then $\alpha\neq 0$. This implies that $\phi$ is not a zero map, hence because $K$ is a field, $\phi$ must be injective.

So, because $\mathbb{C}$ is also a field (an integral domain), then $\phi(1)=1\in\mathbb{C}$, showing that for all nonero $k\in K$, $\phi(k^{-1})\phi(k)=\phi(1)=1$, with $\phi(k)\neq 0$, then $\phi(k^{-1})=(\phi(k))^{-1}$.

\hfil

\textbf{Expression of $\frac{1}{\alpha}$:}

Since $\frac{1}{\alpha}=\alpha^{-1}$, and $\phi(\overline{x})=\alpha$, then $\alpha^{-1}=\phi(\overline{x})^{-1}=\phi(\overline{x}^{-1})$. It suffices to find the inverse of $\overline{x}\in K$.

Given that $\overline{f(x)}=\overline{x^3-3x-1}=0\in K$, then $\overline{x^3-3x}=\overline{1}\in K$, hence $\overline{x}\cdot\overline{x^2-3}=\overline{1}$, showing that $\overline{x^2-3}=(\overline{x})^{-1}$. 
Hence, the following is true:
$$\alpha^{-1}=\phi(\overline{x}^{-1})=\phi(\overline{x^2-3})=\alpha^2-3$$

\hfil

\textbf{Expression of $\frac{1}{\alpha+3}$:}

Again, since $\frac{1}{\alpha+3}=(\alpha+3)^{-1}=(\phi(\overline{x+3}))^{-1}=\phi((\overline{x+3})^{-1})$, it suffices to find the inverse of $\overline{x+3}\in K$.

Since $K=\mathbb{Q}[x]/(f(x))$ is a degree 2 field extension of $\mathbb{Q}$ with basis $\{\overline{1},\overline{x},\overline{x^2}\}$, guess $\overline{x+3}^{-1}=\overline{ax^2+bx+c}$ for some $a,b,c\in\mathbb{Q}$.
Then, the following equation is satisfied:
$$\overline{(x+3)(ax^2+bx+c)}=\overline{1},\quad (x+3)(ax^2+bx+c)\mod (f(x))=1\mod (f(x))$$
$$\exists q(x)\in\mathbb{Q}[x],\quad (x+3)(ax^2+bx+c) = q(x)f(x)+1 = q(x)(x^3-3x-1)+1$$ 
Since $(x+3)(ax^2+bx+c)=q(x)(x^3-3x-1)+1$, while $ax^2+bx+c\neq 0$ (since over $K$, it is the inverse of $\overline{x+3}$), then $1=\deg(x+3)\leq \deg((x+3)(ax^2+bx+c))\leq 3$.

Hence, in case for $q(x)(x^3-3x-1)+1$ to have degree at least $1$, we need $q(x)\neq 0$ (if $q=0$, then the expression is just $1$, which violates the degree $\geq 1$); 
also, for its degree to be at most $3$ while $(x^3-3x-1)$ has degree $3$, the only possibility is $q(x)$ being a constant (since $q(x)(x^3-3x-1)$ is nonconstant, then $\deg(q(x)(x^3-3x-1)+1)=\deg(q(x)(x^3-3x-1))=\deg(q)+ \deg(x^3-3x-1)\geq 3$).

So, $q(x)=f\in \mathbb{Q}$, and $f\neq 0$.

Now, expand the above equation of polynomials, we get:
$$(x+3)(ax^2+bx+c)=q(x)(x^3-3x-1)+1=f(x^3-3x-1)+1$$
$$ax^3+(3a+b)x^2+(3b+c)x+3c = fx^3-3fx+(-f+1)$$
Which, the coefficient of $x^3$ provides $a=f$; coefficient of $x^2$ provides $(3a+b)=0$, so $b=-3a$; coefficient of $x$ provides $(3b+c)=-3f=-3a$, then $c=-3b-3a = -3(-3a)-3a=6a$;
finally, the constant term provides $3c=(-f+1)=(-a+1)$, hence $18a=(-a+1)$, $19a=1$, so $a=\frac{1}{19}$.

Plug all the coefficients back, we get:
$$ax^3+bx+c = ax^3-3ax+6a = a(x^2-3x+6)=\frac{1}{19}(x^2-3x+6)$$
Which, multiply by $(x+3)$, we get:
$$\frac{1}{19}(x+3)(x^2-3x+6)=\frac{1}{19}(x^3-3x+18) = \frac{1}{19}((x^3-3x-1)+19)=\frac{1}{19}(x^3-3x-1)+1$$
$$\frac{1}{19}(x+3)(x^2-3x+6)\mod (f(x))=1\mod (f(x))$$
The above is true since $f(x)=x^3-3x-1$. Hence, this shows that $\overline{\frac{1}{19}(x^2-3x+6)}$ is in fact the inverse of $\overline{x+3}\in K$.

Then, return to the original equation, $\frac{1}{\alpha+3}$ can be expressed as:
$$\frac{1}{\alpha+3}=\phi((\overline{x+3})^{-1})=\phi\left(\overline{\frac{1}{19}(x^2-3x+6)}\right) = \frac{1}{19}(\alpha^2-3\alpha+6)$$

\break

\section*{2}
\begin{myBox}[]{}
    \begin{question}
        Let $K=F(\alpha)$, where $\alpha$ is a root of the irreducible polynomial
        $$f(x)=x^n+a_{n-1}x^{n-1}+...+a_1x+a_0$$
        Express $\frac{1}{\alpha}$ in terms of $\alpha$ and the coefficients $a_i$.
    \end{question}
\end{myBox}

\textbf{Pf:}

First, since $f$ is irreducible in $F[x]$, then $f$ has no zeroes in $F$. Hence, $0$ cannot be a zero of $f$, so $\alpha\neq 0$. This also implies that $a_0\neq 0$ (or else if $a_0=0$, $0$ is a zero of $f$).

\hfil

Then, consider $K'=F[x]/(f(x))$: Since $f$ is an irreducible polynomial over $F$, then since $F[x]$ is a PID,
the ideal $(f(x))$ is in fact maximal, hence $K'=F[x]/(f(x))$ is a field.

Now, consider $\overline{x} = x\mod (f(x))\in K'$: since it satisfies the following: 
$$f(\overline{x})=\overline{x}^n+a_{n-1}\overline{x}^{n-1}+...+a_1\overline{x}+a_0 = \overline{x^n+a_{n-1}x^{n-1}+...+a_1x+a_0} = 0\mod (f(x))\in K'$$
then $K'$ is a field containing a zero of $f(x)$.

Then, consider the ring homomorphism $\phi: K'\rightarrow K$, by $\phi(\overline{x})=\alpha$: Since $0$ is not a zero of $f$, then $\alpha\neq 0\in F(\alpha)$.
Hence, the ring homomorphism $\phi$ is not the zero map, showing that $\ker(\phi)\neq K'$;
then, since $K'$ is a field, while $\ker(\phi)\neq K'$, the map is injective.

\hfil

Lastly, consider the inverse of $\overline{x}\in K'$: Since $a_0\neq 0$ in $F$, then $a_0^{-1}f(x)=a_0^{-1}(x^n+a_{n-1}x^{n-1}+...+a_1x)+1$.
Hence, the following is true:
$$0=a_0^{-1}f(x)\mod (f(x)) = (a_0^{-1}(x^n+a_{n-1}x^{n-1}+...+a_1x)+1)\mod (f(x))$$
$$\implies \overline{1}=\overline{a_0^{-1}(x^n+a_{n-1}x^{n-1}+...+a_1x)}\in K' = F[x]/(f(x))$$
So, $\overline{1}=\overline{a_0^{-1}(x^n+a_{n-1}x^{n-1}+...+a_1x)}=\overline{x}\cdot \overline{a_0^{-1}(x^{n-1}+a_{n-1}x^{n-2}+...+a_1)}$, hence the inverse of $\overline{x}$ in $K'$ is $\overline{a_0^{-1}(x^{n-1}+a_{n-1}x^{n-2}+...+a_1)}$.
Then, since ring homomorphism maps an element's inverse to the output's inverse, then $\phi(\overline{x})=\alpha$ implies the following: 
$$\frac{1}{\alpha}=\alpha^{-1}=\phi((\overline{x})^{-1})=\phi\left(\overline{a_0^{-1}(x^{n-1}+a_{n-1}x^{n-2}+...+a_1)}\right)=a_0^{-1}(\alpha^{n-1}+a_{n-1}\alpha^{n-2}+...+a_1)$$


\break

\section*{3}
\begin{myBox}[]{}
    \begin{question}
        Show that $x^4+1$ is irreducible over $\mathbb{Q}$, but not over $\mathbb{Q}(\sqrt{2})$.
    \end{question}
\end{myBox}

\textbf{Pf:}

If consider $x^4+1\in \mathbb{Z}[x]$, if we do a substitution $x\mapsto (x+1)$, then we get the following:
$$(x^4+1)\mapsto (x+1)^4+1 =(x^4+4x^3+6x^2+4x+1)+1 = x^4+4x^3+6x^2+4x+2$$
Notice that since leading coefficient $1$ is not divisible by $2$, the other coefficients $4,6,4,2$ are divisible by $2$,
while the constant term $2$ is not divisible by $2^2$, then by Eisenstein's criterion, $(x+1)^4+1$ is irreducible over $\mathbb{Q}$.
Hence, the original polynomial $x^4+1$ is also irreducible over $\mathbb{Q}$.

\hfil

Now, consider $x^4+1$ over $\mathbb{Q}(\sqrt{2})$: since $\sqrt{2}$ is an element in the given field, then the following is a factorization of $x^4+1$:
$$((x^2+1)-\sqrt{2}x)((x^2+1)+\sqrt{2}x)=(x^2+1)^2-(\sqrt{2}x)^2=(x^4+2x^2+1)-(2x^2)=x^4+1$$
Since $x^4+1$ can be factored into smaller degree nonconstant polynomial, this indicates $x^4+1$ is reducible over $\mathbb{Q}(\sqrt{2})$.

\break

\section*{4}
\begin{myBox}[]{}
    \begin{question}
        
    \end{question}
\end{myBox}

\textbf{Pf:}

\break

\section*{5}
\begin{myBox}[]{}
    \begin{question}
        Is $\mathbb{Q}(\sqrt{2})$ isomorphic to $\mathbb{Q}(\sqrt{3})$?
    \end{question}
\end{myBox}

\textbf{Pf:}

We'll prove by contradiction that the two fields are not isomorphic.

Suppose the contrary, that the two fields are isomorphic, then there exists bijective ring homomorphism $\phi:\mathbb{Q}(\sqrt{2})\rightarrow\mathbb{Q}(\sqrt{3})$.

First, since $\phi(\mathbb{Q}(\sqrt{2}))= \mathbb{Q}(\sqrt{3})$ by assumption that $\phi$ is a bijection, then $\phi(1)=1\in\mathbb{Q}(\sqrt{3})$.
Which, this implies that $\phi(2)=\phi(1+1)=\phi(1)+\phi(1)=1+1=2\in\mathbb{Q}(\sqrt{3})$. Hence, since $\sqrt{2}\in\mathbb{Q}(\sqrt{2})$ satisfies $(\sqrt{2})^2=2$,
then $2=\phi(2)=\phi((\sqrt{2})^2)=\phi(\sqrt{2})^2\in\mathbb{Q}(\sqrt{3})$.

Now, let $\phi(\sqrt{2})=a+b\sqrt{3}\in\mathbb{Q}(\sqrt{3})$, which $a,b\in\mathbb{Q}$. Then, it satisfies the following:
$$2+0\sqrt{3}=2=\phi(\sqrt{2})^2=(a+b\sqrt{3})^2=(a^2+3b^2)+2ab\sqrt{3}$$
Hence, for the coefficients to match up, we need $2ab = 0$, which $a=0$ or $b=0$.

Yet, both leads to a contradiction:
\begin{itemize}
    \item Suppose $a=0$, then $2=(a+b\sqrt{3})^2=(b\sqrt{3})^2=3b^2$. Since $b=\frac{p}{q}$ for some $p,q\in\mathbb{Z}$ with $q\neq 0$ (WLOG, assume $\gcd(p,q)=1$),
    then $2=3b^2=3(\frac{p}{q})^2$, hence $3p^2=2q^2$.
    
    Since $3p^2$ is divisible by $2$, while $3$ is coprime with $2$, then $2$ divides $p^2$, hence $2$ divides $p$. So, $p=2k$ for some $k\in\mathbb{Z}$.

    Which, $2q^2=3p^2=3(2k)^2=4\cdot 3k^2$, so $q^2=2\cdot 3k^2$. Since $q^2$ is now divisible by $2$, this implies that $2$ divides $q$.

    Yet, since both $p,q$ are divisible by $p$, $\gcd(p,q)\geq 2$, which violates the assumption that $\gcd(p,q)=1$, so we reach a contradiction.

    \hfil

    \item Else, suppose $b=0$, then $2=(a+b\sqrt{3})^2=a^2$, where $a\in\mathbb{Q}$. However, this violates the fact that $2$ has no square root in $\mathbb{Q}$, which is again a contradiction.
\end{itemize}

Since both leads to a contradiction, our initial assumption must be false. Hence, the two fields $\mathbb{Q}(\sqrt{2})$ and $\mathbb{Q}(\sqrt{3})$ can't be isomorphic.

\break

\section*{6}
\begin{myBox}[]{}
    \begin{question}
        Prove that $\mathbb{R}$ is not a simple extension of $\mathbb{Q}$.
    \end{question}
\end{myBox}

\textbf{Pf:}

Recall that $\mathbb{R}$ is an uncountable set. So, it suffices to show that all simple extension of $\mathbb{Q}$ is countable.

Every simple extension of $\mathbb{Q}$ is in the form $K=\mathbb{Q}(\theta) = \{p(\theta)/q(\theta)\ |\ p,q\in \mathbb{Q}[\theta],\ q\neq 0\}$.

Which, there are several cases to consider:

\begin{itemize}
    \item[1.] Suppose $\theta\in\mathbb{Q}$, then $K=\mathbb{Q}(\theta)=\mathbb{Q}$, which is countable.

    \item[2.] Suppose $\theta\notin\mathbb{Q}$, but it is algebraic over $\mathbb{Q}$, then there exists a minimal polynomial $p(x)\in\mathbb{Q}[x]$ that is irreduible,
    such that $p(\theta)=0\in K$. In this case, since $(p(x))\subset \mathbb{Q}[x]$ is maximal, then $\mathbb{Q}[x]/(p(x))$ is a field extension of $\mathbb{Q}$ containing a zero of $p(x)$,
    and it is isomorphic to $K=\mathbb{Q}(\theta)$.

    Then, because $K'=\mathbb{Q}[x]/(p(x))$ is a field extension of $\mathbb{Q}$ with degree $[K':\mathbb{Q}]=\deg(p)=n$, where $n$ is finite,
    it is also a $\mathbb{Q}$-vector space with dimension $n$, hence isomorphic to $\mathbb{Q}^n$.

    However, since $\mathbb{Q}$ is countable, for finite $n\in\mathbb{N}$, $\mathbb{Q}^n$ is also countable. Hence, $K\cong K'\cong \mathbb{Q}^n$ is also countable.

    \item[3.] Suppose $\theta\notin \mathbb{Q}$, and is transcendental over $\mathbb{Q}$, then for all nonzero $p(x)\in\mathbb{Q}[x]$, $p(\theta)\neq 0\in K$,
    hence the map $\mathbb{Q}[x]\rightarrow \mathbb{Q}(\theta)$ by $x\mapsto\theta$ is injective (since for all nonzero $p(x)\in\mathbb{Q}[x]$, $p(x)\mapsto p(\theta)\neq 0$),
    hence $\mathbb{Q}(\theta)$ contains $\mathbb{Q}[x]$; furthermore, since every element can be expressed as $\frac{p(\theta)}{q(\theta)}$ for $p,q\in\mathbb{Q}[x]$, and $q\neq 0$,
    then $\mathbb{Q}(\theta)$ is in fact isomorphic to $F(\mathbb{Q}[x])$, the field of fraction of $\mathbb{Q}[x]$ (since the ring homomorphism $F(\mathbb{Q}[x])\rightarrow \mathbb{Q}(\theta)$ by $x\mapsto \theta$ has $\frac{p(x)}{q(x)}\mapsto \frac{p(\theta)}{q(\theta)}$, 
    showing the map is surjective; also, since $F(\mathbb{Q}[x])$ is a field, the nonzero map is guaranteed to be injective).

    So, for this case it suffices to prove that $F(\mathbb{Q}[x])$ is countable.

    \hfil

    First, $\mathbb{Q}[x]$ is countable: For all $n\in\mathbb{N}$, let $P_n\subset \mathbb{Q}[x]$ be a collection of all polynomials with degree at most $n$. 
    Which, as a $\mathbb{Q}$-vector space, $P_n$ is isomorphic to $\mathbb{Q}^n$, so it is countable.

    Now, consider $\bigcup_{n\in\mathbb{N}}P_n\subseteq \mathbb{Q}[x]$: For app $p(x)\in\mathbb{Q}[x]$, since its degree $\deg(p)=n$ is finite, then $p(x)\in P_n\subset \bigcup_{n\in\mathbb{N}}P_n$,
    hence $\mathbb{Q}[x]=\bigcup_{n\in\mathbb{N}}P_n$. Now, since $\bigcup_{n\in\mathbb{N}}P_n$ is a countable union of all $P_n,\ n\in\mathbb{N}$, while each $P_n$ is countable,
    then the union is also countable. Hence, $\mathbb{Q}[x]$ is countable.

    \hfil

    Now, consider $F(\mathbb{Q}[x])=\{\frac{p(x)}{q(x)}\ |\ p,q\in\mathbb{Q}[x],\ q\neq 0\}$: Since $\mathbb{Q}[x]$ is also a UFD, then $\gcd$ for any finite collection of elements exist. For $\frac{p(x)}{q(x)}$ with $p,q\neq 0$, we'll assume $\gcd(p(x),q(x))=1$ (so the fraction is irreducible),
    and for $0\in F(\mathbb{Q}[x])$, assume it's in the form $\frac{0}{1}$.

    Then, if we do the map $F(\mathbb{Q}[x])\rightarrow (\mathbb{Q}[x]\times \mathbb{Q}[x])$ by $\frac{p(x)}{q(x)}\mapsto (p(x),q(x))$, the map is injective, since if $\frac{p(x)}{q(x)},\frac{f(x)}{g(x)}\in F(\mathbb{Q}[x])$ (both in irreducible forms) get mapped to the same element, we neec $(p(x),q(x))=(f(x),g(x))$,
    showing that the two fractions are the same. Hence, $F(\mathbb{Q}[x])$ is set isomorphic to a subset of $\mathbb{Q}[x]\times \mathbb{Q}[x]$, a set that is countable since $\mathbb{Q}[x]$ is countable. Hence, $F(\mathbb{Q}[x])$ is also countable.

    \hfil

    Finally, since $F(\mathbb{Q}[x])$ is countable, $\mathbb{Q}(\theta)$ that is isomorphic to $F(\mathbb{Q}[x])$, then it is also countable.
\end{itemize}

Since regardless of the case, the simple extension $\mathbb{Q}(\theta)$ is a countable set, because $\mathbb{R}$ is not countable, it cannot be a simple extension of $\mathbb{Q}$.

\break

\section*{7}
\begin{myBox}[]{}
    \begin{question}
        Let $E/F$ be a field extension, and let $\alpha\in E$. Show that multiplication by $\alpha$ is a
        linear transformation of $E$ considered as a vector space over $F$. When is this linear
        transformation non-singular?
    \end{question}
\end{myBox}

\textbf{Pf:}

To verify the multiplication by $\alpha$ being a linear transformation of $E$ as a vector space over $F$, consider all $f,g\in E$, and scalar $\lambda\in F$:

By distributive property of multiplication, we know $\alpha(f+g)=\alpha f+\alpha g$; similarly, since $E$ is a field, the multiplication is commutative, hence $\alpha(\lambda f)=\lambda (\alpha f)$,
showing that the multiplication is in fact a linear transformation of $E$ as a vector space over $F$.

\hfil

Now, suppose $\alpha$ as a linear transformation is non-singular (i.e. invertible), which we'll verify that such transformation is non-singular iff $\alpha\neq 0$:

$\implies$: Suppose $\alpha\neq 0$, then $\alpha^{-1}\in E$ exists since $E$ is a field. Based on the fact that multiplication of any element in $E$ is a linear transformation of $E$,
any $f\in E$ satisfies $\alpha^{-1}(\alpha f) = \alpha(\alpha^{-1}f)=f$, which $\alpha^{-1}$ as a linear transformation over $E$ composes with $\alpha$ to be identity on both sides,
this shows that $\alpha^{-1}$ is the inverse transformation of $\alpha$, hence $\alpha$ is non-singular.

$\impliedby:$ We'll prove the contrapositive. Suppose $\alpha=0$, then since all nonzero $f\in E$ satisfies $\alpha f = 0$, then the transformation $\alpha$ is not injective,
hence non-invertible. This shows that $\alpha$ is a singular linear transformation.
Then, the contrapositive states that if $\alpha$ is non-singular, the $\alpha\neq 0$.

The above two implication states that $\alpha$ as a linear transformation is non-singular, iff $\alpha\neq 0$.

\hfil

\hfil

\section*{8}
\begin{myBox}[]{}
    \begin{question}
        Let $E/F$ be a field extension, and let $p(x)$ be an irreducible polynomial over $F$. 
        Show that if the degree of $p(x)$ and $[E:F]$ are coprime, then $p(x)$ has no zeros in $E$.
    \end{question}
\end{myBox}

\textbf{Pf:}

We'll prove the contrapositive. 
Suppose $p(x)$ has a zero in $E$, say $\alpha\in E$, and $m=[E:F]$ is finite. 

Given that $p(x)$ is irreducible over $F$, then it has no zero in $F$. Hence, $p(0)\neq 0$. Then, since $p(\alpha)=0$, $\alpha\neq 0$.

\hfil

First, we'll consider the ring $K'=F[x]/(p(x))$: Since $p(x)\in F[x]$ is irreducible, and $F[x]$ is a PID,
the ideal $(p(x))\subset F[x]$ is in fact maximal. Hence, $K'=F[x]/(p(x))$ is a field.

Now, given that $p(x)=a_nx^n+...+a_1x+a_0$, since $\overline{x} = x\mod (p(x))\in K'$ satisfies the following:
$$p(\overline{x})=a_n\overline{x}^n+...+a_1\overline{x}+a_0 = (a_nx^n+...+a_1x+a_0)\mod (p(x))=p(x)\mod (p(x))=0\in K'$$
Hence, $p(x)$ has a zero over the field $K'$.

\hfil

Then, consider the ring homomorphism $\phi:K'\rightarrow E$ given by $\phi(\overline{x})=\alpha$: since $\alpha\neq 0$ in $E$ and $\overline{x}\neq 0$ in $K'$,
then such ring homomorphism is nonzero, hence $\ker(\phi)\neq K'$. Now, because $K'$ is a field, then it enforces $\phi$ to be injective.
Then, since $K'\cong \phi(K')\subseteq E$, this shows that $K'$ is isomorphic to a subfield of $E$. Hence, $E/K'$ is also a field extension.

\hfil

\textbf{Relationships of $E,\ K',$ and $F$:}

Now, given that $\deg(p)=n$, then $K'$ as a vector space of $F$, has dimension $n$ (i.e. $[K':F]=n$); on the other hand, given that $m=[E:F]$ is finite, then $E$ as a vector space of $F$ has dimension $m$.

The above implies that $q=[E:K']$ is in fact finite, since $K'$ is a finite-dimensional subspace of vector space $E$ over field $F$. \textbf{(Need to verify)}

Lastly, given $E/K'$ as a field extension, since $q=[E:K']$ by assumption, then there exists distinct nonzero $e_1,...,e_q\in E$ that represents a basis of $E$ as a vector space over $K'$. 

Also, since $n=[K':F]$, then there exists distinct nonzero $k_1,...,k_n\in K'$ that represents a basis of $K'$ as a vector space over $F$. 

Our goal is to prove that the collection $\{k_je_i\ |\ 1\leq j\leq n,\ 1\leq i\leq q\}$, actually represents a basis of $E$ as a vector space over $F$: Based on the given bases of $E/K'$ and $K'/F$ above, for all $f\in E$,
there exists unique $f_1,...,f_q\in K'$, with $f=\sum_{i=1}^{q}f_ie_i$. And, for each $f_i\in K'$, there exists unique $l_1^{(i)},...,l_n^{(i)}\in F$, with $f_i=\sum_{j=1}^{n}l_j^{(i)}k_j$. Hence, the following is true:
$$f=\sum_{i=1}^{q}f_ie_i = \sum_{i=1}^{q}\left(\sum_{j=1}^{n}l_j^{(i)}k_j\right)e_i = \sum_{i=1}^{q}\sum_{j=1}^{n}l_j^{(i)}k_je_i$$
Hence, the collection $\{k_je_i\ |\ 1\leq j\leq n,\ 1\leq i\leq q\}$ actually is a basis of $E/F$.

On the other hand, suppose the collection of scalars $\{l_j^{(i)}\ |\ 1\leq j\leq n,\ 1\leq i\leq q\}$ satisfies $0=\sum_{i=1}^{q}\sum_{j=1}^{n}l_j^{(i)}k_je_i$, then after regrouping, we get the following:
$$0=\sum_{i=1}^{q}\sum_{j=1}^{n}l_j^{(i)}k_je_i= \sum_{i=1}^{q}\left(\sum_{j=1}^{n}l_j^{(i)}k_j\right)e_i$$
Since $e_1,...,e_q\in E/K'$ is a basis of $E$ over field $K'$, the above equation implies that for each $1\leq i\leq q$, the coefficient $\sum_{j=1}^{n}l_{j}^{(i)}k_j=0\in K'$;
similarly, since $k_1,...,k_n\in K'/F$ is a basis of $K'$ over field $f$, the above equation implies that $l_{1}^{(i)},...,l_n^{(i)}=0\in F$, for all $i$ given.

Hence, this proves the linear independence of the collection $\{k_je_i\ |\ 1\leq j\leq n,\ 1\leq i\leq q\}\subset E/F$.

Since the collection is linearly independent while spanning $E/F$, then it is in fact a basis of $E/F$. Hence, as a vector space over $F$, $E$ has dimension $n\cdot q = m$.

\hfil

Since $n=\deg(p)$, while $nq = m = [E:F]$, this proves that $n,m$ are not coprime. Hence, the contrapositive states the following:
Given $p(x)$ an irreducible polynomial over $F$, and $[E:F]$ is finite, then degree of $p(x)$ and $[E:F]$ are coprime implies $p(x)$ has no zeros in $E$.

(However, if $\deg(p)=1$, then the above breaks, since $p(x)$ is guaranteed to have a root in $\mathbb{F}$, while $\deg(p)$ is coprime to $[E:F]$).



\break

\section*{9}
\begin{myBox}[]{}
    \begin{question}
        Express $\sqrt[3]{28}-3$ as a square in $\mathbb{Q}(\sqrt[3]{28})$.
    \end{question}
\end{myBox}

\textbf{Pf:}

Since $\alpha=\sqrt[3]{28}$ satisfies $\alpha^3=28$, so it is a zero of $\alpha^3-28$.
Notice that given $x^3-28\in\mathbb{Z}[x]$, since with prime $p=7$, it satisfies the Eisenstein Criterion 
(leading coefficient $1$ is not divisible by $7$; the other coefficients $0,0,28$ are divisible by $7$, while $28$ is not divisible by $7^2$).
Hence, $x^3-28$ is irreducible over $\mathbb{Q}$. Then, $(x^3-28)\subset \mathbb{Q}[x]$ is a maximal ideal, which $K=\mathbb{Q}[x]/(x^3-28)$
is a field containing a zero of $x^3-28$.

\hfil

Now, consider the ring homomorphism $\phi:K\rightarrow\mathbb{Q}(\sqrt[3]{28})$ by $\phi(\overline{x})=\sqrt[3]{28}$. Which, since all $k\in K$ has $\phi(k^2)=\phi(k)^2$,
and $\phi(\overline{x-3})=\sqrt[3]{28}-3$, it suffices to find the element $k\in K$, with $k^2=\overline{x-3}$.

\hfil

Consider the element $k=\overline{\frac{1}{6}(x^2-2x-2)}\in K$: If we take the square of the element, we get the following:
$$\left(\frac{1}{6}(x^2-2x-2)\right)^2=\frac{1}{36}((x^4-2x^3-2x^2)+(-2x^3+4x^2+4x)+(-2x^2+4x+4))$$
$$=\frac{1}{36}(x^4-4x^3+8x+4)=\frac{1}{36}((x^4-28x)+(-4x^3+112)+(36x-108))$$
$$=\frac{1}{36}((x-4)(x^3-28)+36(x-3)) = \frac{1}{36}(x-4)(x^3-28)+(x-3)$$

$$\overline{\left(\frac{1}{6}(x^2-2x-2)\right)^2}=\left(\frac{1}{36}(x-4)(x^3-28)+(x-3)\right)\mod (x^3-28) = \overline{x-3}$$
Hence, since the above element satisfies $k^2=\overline{x-3}$, then $\phi(k^2)=\phi(k)^2=\phi(\overline{x-3})=\sqrt[3]{28}-3$.

Since $\phi(k)=\phi(\overline{\frac{1}{6}(x^2-2x-2)})=\frac{1}{6}((\sqrt[3]{28})^2-2\sqrt[3]{28}-2)$, then we can conclude the following:
$$\left(\frac{1}{6}((\sqrt[3]{28})^2-2\sqrt[3]{28}-2)\right)^2=\sqrt[3]{28}-3$$



\break

\section*{10}
\begin{myBox}[]{}
    \begin{question}
        Let $\beta=\omega \sqrt[3]{2}$, where $\omega=e^{2\pi i/3}$, and let $K=\mathbb{Q}(\beta)$. 
        Prove that $-1$ cannot be written as a sum of squares in $K$.
    \end{question}
\end{myBox}

\textbf{Pf:}



\end{document}