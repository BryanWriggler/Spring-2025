\documentclass{article}
\usepackage{graphicx} % Required for inserting images
\usepackage[margin = 2.54cm]{geometry}
\usepackage[most]{tcolorbox}

\newtcolorbox{myBox}[3]{
arc=5mm,
lower separated=false,
fonttitle=\bfseries,
%colbacktitle=green!10,
%coltitle=green!50!black,
enhanced,
attach boxed title to top left={xshift=0.5cm,
        yshift=-2mm},
colframe=blue!50!black,
colback=blue!10
}

\usepackage{amsmath}
\usepackage{amssymb}
\usepackage{verbatim}
\usepackage[utf8]{inputenc}
\linespread{1.2}

\newtheorem{definition}{Definition}
\newtheorem{proposition}{Proposition}
\newtheorem{theorem}{Theorem}
\newtheorem{question}{Question}

\title{Math 111C HW1}
\author{Zih-Yu Hsieh}

\begin{document}
\maketitle

\section*{1 (Not Done)}
\begin{myBox}[]{}
    \begin{question}
        Show, using Eisenstein's criterion, that $f(X)=X^3-3X-1$ is irreducible over $\mathbb{Q}$. Let $\alpha$ be a root of $f$ in $\mathbb{C}$. 
        Express $\frac{1}{\alpha}$ and $\frac{1}{\alpha+3}$ as linear combinations of $1$, $\alpha$ and $\alpha^2$.
    \end{question}
\end{myBox}

\textbf{Pf:}

Consider the ring homomorphism $\phi:\mathbb{Z}[X]\rightarrow \mathbb{Z}[X]$ by $\phi(f)=f(X+1)$. Which, $\phi$ is injective, since if $f(X+1)$ is constantly $0$, 
its leading coefficient $a_n=0$, which helps us inductively prove that $f=0$.

Then, for $f(X)=X^3-3X-1$ given above:
$$\phi(f)=f(X+1)=(X+1)^3-3(X+1)-1 = (X^3+3X^2+3X+1)-(3X+3)-1 = X^3+3X^2-3$$
Then, since leading coefficient is $1$, while the rest of the coefficients (namely $3,0,-3$) are divisible by $3$, and $-3$ is not divisible by $3^2$,
so by Eisenstein's criterion, $\phi(f)=X^3+3X^2-3$ is irreducible over $\mathbb{Q}$. Then, since $\phi(f)=f(X+1)$ is irreducible over $\mathbb{Q}$, then $f$ itself must also be irreducible:

Suppose not, then there exists $d$ fuck it

\break

\section*{2}
\begin{myBox}[]{}
    \begin{question}
        Let $K=F(\alpha)$, where $\alpha$ is a root of the irreducible polynomial
        $$f(x)=x^n+a_{n-1}x^{n-1}+...+a_1x+a_0$$
        Express $\frac{1}{\alpha}$ in terms of $\alpha$ and the coefficients $a_i$.
    \end{question}
\end{myBox}

\textbf{Pf:}

First, since $f$ is irreducible in $F[x]$, then $f$ has no zeroes in $F$. Hence, $0$ cannot be a zero of $f$, so $\alpha\neq 0$. This also implies that $a_0\neq 0$ (or else if $a_0=0$, $0$ is a zero of $f$).

\hfil

Then, consider $K'=F[x]/(f(x))$: Since $f$ is an irreducible polynomial over $F$, then since $F[x]$ is a PID,
the ideal $(f(x))$ is in fact maximal, hence $K'=F[x]/(f(x))$ is a field.

Now, consider $\overline{x} = x\mod (f(x))\in K'$: since it satisfies the following: 
$$f(\overline{x})=\overline{x}^n+a_{n-1}\overline{x}^{n-1}+...+a_1\overline{x}+a_0 = \overline{x^n+a_{n-1}x^{n-1}+...+a_1x+a_0} = 0\mod (f(x))\in K'$$
then $K'$ is a field containing a zero of $f(x)$.

Then, consider the ring homomorphism $\phi: K'\rightarrow K$, by $\phi(\overline{x})=\alpha$: Since $0$ is not a zero of $f$, then $\alpha\neq 0\in F(\alpha)$.
Hence, the ring homomorphism $\phi$ is not the zero map, showing that $\ker(\phi)\neq K'$;
then, since $K'$ is a field, while $\ker(\phi)\neq K'$, the map is injective.

\hfil

Lastly, consider the inverse of $\overline{x}\in K'$: Since $a_0\neq 0$ in $F$, then $a_0^{-1}f(x)=a_0^{-1}(x^n+a_{n-1}x^{n-1}+...+a_1x)+1$.
Hence, the following is true:
$$0=a_0^{-1}f(x)\mod (f(x)) = (a_0^{-1}(x^n+a_{n-1}x^{n-1}+...+a_1x)+1)\mod (f(x))$$
$$\implies \overline{1}=\overline{a_0^{-1}(x^n+a_{n-1}x^{n-1}+...+a_1x)}\in K' = F[x]/(f(x))$$
So, $\overline{1}=\overline{a_0^{-1}(x^n+a_{n-1}x^{n-1}+...+a_1x)}=\overline{x}\cdot \overline{a_0^{-1}(x^{n-1}+a_{n-1}x^{n-2}+...+a_1)}$, hence the inverse of $\overline{x}$ in $K'$ is $\overline{a_0^{-1}(x^{n-1}+a_{n-1}x^{n-2}+...+a_1)}$.
Then, since ring homomorphism maps an element's inverse to the output's inverse, then $\phi(\overline{x})=\alpha$ implies the following: 
$$\frac{1}{\alpha}=\alpha^{-1}=\phi((\overline{x})^{-1})=\phi\left(\overline{a_0^{-1}(x^{n-1}+a_{n-1}x^{n-2}+...+a_1)}\right)=a_0^{-1}(\alpha^{n-1}+a_{n-1}\alpha^{n-2}+...+a_1)$$


\break

\section*{3}
\begin{myBox}[]{}
    \begin{question}
        Show that $x^4+1$ is irreducible over $\mathbb{Q}$, but not over $\mathbb{Q}(\sqrt{2})$.
    \end{question}
\end{myBox}

\textbf{Pf:}

If consider $x^4+1\in \mathbb{Z}[x]$, if we do a substitution $x\mapsto (x+1)$, then we get the following:
$$(x^4+1)\mapsto (x+1)^4+1 =(x^4+4x^3+6x^2+4x+1)+1 = x^4+4x^3+6x^2+4x+2$$
Notice that since leading coefficient $1$ is not divisible by $2$, the other coefficients $4,6,4,2$ are divisible by $2$,
while the constant term $2$ is not divisible by $2^2$, then by Eisenstein's criterion, $(x+1)^4+1$ is irreducible over $\mathbb{Q}$.
Hence, the original polynomial $x^4+1$ is also irreducible over $\mathbb{Q}$.

\hfil

Now, consider $x^4+1$ over $\mathbb{Q}(\sqrt{2})$: since $\sqrt{2}$ is an element in the given field, then the following is a factorization of $x^4+1$:
$$((x^2+1)-\sqrt{2}x)((x^2+1)+\sqrt{2}x)=(x^2+1)^2-(\sqrt{2}x)^2=(x^4+2x^2+1)-(2x^2)=x^4+1$$
Since $x^4+1$ can be factored into smaller degree nonconstant polynomial, this indicates $x^4+1$ is reducible over $\mathbb{Q}(\sqrt{2})$.

\break

\section*{4}
\begin{myBox}[]{}
    \begin{question}
        
    \end{question}
\end{myBox}

\textbf{Pf:}

\break

\section*{5}
\begin{myBox}[]{}
    \begin{question}
        
    \end{question}
\end{myBox}

\textbf{Pf:}

\break

\section*{6}
\begin{myBox}[]{}
    \begin{question}
        
    \end{question}
\end{myBox}

\textbf{Pf:}

\break

\section*{7}
\begin{myBox}[]{}
    \begin{question}
        Let $E/F$ be a field extension, and let $\alpha\in E$. Show that multiplication by $\alpha$ is a
        linear transformation of $E$ considered as a vector space over $F$. When is this linear
        transformation non-singular?
    \end{question}
\end{myBox}

\textbf{Pf:}

To verify the multiplication by $\alpha$ being a linear transformation of $E$ as a vector space over $F$, consider all $f,g\in E$, and scalar $\lambda\in F$:

By distributive property of multiplication, we know $\alpha(f+g)=\alpha f+\alpha g$; similarly, since $E$ is a field, the multiplication is commutative, hence $\alpha(\lambda f)=\lambda (\alpha f)$,
showing that the multiplication is in fact a linear transformation of $E$ as a vector space over $F$.

\hfil

Now, suppose $\alpha$ as a linear transformation is non-singular (i.e. invertible), which we'll verify that such transformation is non-singular iff $\alpha\neq 0$:

$\implies$: Suppose $\alpha\neq 0$, then $\alpha^{-1}\in E$ exists since $E$ is a field. Based on the fact that multiplication of any element in $E$ is a linear transformation of $E$,
any $f\in E$ satisfies $\alpha^{-1}(\alpha f) = \alpha(\alpha^{-1}f)=f$, which $\alpha^{-1}$ as a linear transformation over $E$ composes with $\alpha$ to be identity on both sides,
this shows that $\alpha^{-1}$ is the inverse transformation of $\alpha$, hence $\alpha$ is non-singular.

$\impliedby:$ We'll prove the contrapositive. Suppose $\alpha=0$, then since all nonzero $f\in E$ satisfies $\alpha f = 0$, then the transformation $\alpha$ is not injective,
hence non-invertible. This shows that $\alpha$ is a singular linear transformation.
Then, the contrapositive states that if $\alpha$ is non-singular, the $\alpha\neq 0$.

The above two implication states that $\alpha$ as a linear transformation is non-singular, iff $\alpha\neq 0$.

\break

\section*{8}
\begin{myBox}[]{}
    \begin{question}
        Let $E/F$ be a field extension, and let $p(x)$ be an irreducible polynomial over $F$. 
        Show that if the degree of $p(x)$ and $[E:F]$ are coprime, then $p(x)$ has no zeros in $E$.
    \end{question}
\end{myBox}

\textbf{Pf:}

We'll prove the contrapositive. Given that $p(x)$ is irreducible over $F$, then it has no zero in $F$. Hence, $p(0)\neq 0$.

Suppose $p(x)$ has a zero in $E$, say $\alpha\in E$. Then, since $p(\alpha)=0$, $\alpha\neq 0$.

\hfil

First, we'll consider the ring $K'=F[x]/(p(x))$: Since $p(x)\in F[x]$ is irreducible, and $F[x]$ is a PID,
the ideal $(p(x))\subset F[x]$ is in fact maximal. Hence, $K'=F[x]/(p(x))$ is a field.

Now, given that $p(x)=a_nx^n+...+a_1x+a_0$, since $\overline{x} = x\mod (p(x))\in K'$ satisfies the following:
$$p(\overline{x})=a_n\overline{x}^n+...+a_1\overline{x}+a_0 = (a_nx^n+...+a_1x+a_0)\mod (p(x))=p(x)\mod (p(x))=0\in K'$$
Hence, $p(x)$ has a zero over the field $K'$.

\hfil

Then, consider the ring homomorphism $\phi:K'\rightarrow E$ given by $\phi(\overline{x})=\alpha$: since $\alpha\neq 0$ in $E$ and $\overline{x}\neq 0$ in $K'$,
then such ring homomorphism is nonzero, hence $\ker(\phi)\neq K'$. Now, because $K'$ is a field, then it enforces $\phi$ to be injective.
Then, since $K'\cong \phi(K')\subseteq E$, this shows that $K'$ is isomorphic to a subfield of $E$. Hence, $E/K'$ is also a field extension.

\hfil

Now, given that 

\break

\section*{9}
\begin{myBox}[]{}
    \begin{question}
        
    \end{question}
\end{myBox}

\textbf{Pf:}

\break

\section*{10}
\begin{myBox}[]{}
    \begin{question}
        
    \end{question}
\end{myBox}

\textbf{Pf:}

\end{document}