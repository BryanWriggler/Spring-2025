\documentclass{article}
\usepackage{graphicx} % Required for inserting images
\usepackage[margin = 2.54cm]{geometry}
\usepackage[most]{tcolorbox}

\newtcolorbox{myBox}[3]{
arc=5mm,
lower separated=false,
fonttitle=\bfseries,
%colbacktitle=green!10,
%coltitle=green!50!black,
enhanced,
attach boxed title to top left={xshift=0.5cm,
        yshift=-2mm},
colframe=blue!50!black,
colback=blue!10
}

\usepackage{amsmath}
\usepackage{amssymb}
\usepackage{verbatim}
\usepackage[utf8]{inputenc}
\linespread{1.2}

\newtheorem{definition}{Definition}
\newtheorem{proposition}{Proposition}
\newtheorem{theorem}{Theorem}
\newtheorem{question}{Question}

\title{Math 111C HW5}
\author{Zih-Yu Hsieh}

\begin{document}
\maketitle

\section*{1}
\begin{myBox}[]{}
    \begin{question}
        Let $F$ be a finite field. Prove that $|F|=p^n$ for some prime $p$ and $n\in\mathbb{N}$.
    \end{question}
\end{myBox}

\textbf{Pf:}

Since $F$ is a finite field, then $\textmd{char}(F)=p$ for some prime $p$. It suffices to show that $|F|=p^n$ for some $n\in\mathbb{N}$.

Suppose the contrary that the above statement doesn't hold, then there exists some distinct prime number $q\neq p$, such that $q$ divides $|F|$. Recall that $F$ is a finite abelian group under addition, hence \textbf{Cauchy's Theorem} applies, there exists $a\in F$, such that its order with respect to addition (denoted as $\textmd{order}(a)$) is $q$.

However, since $p,q$ are distinct primes, then by \textbf{Bezout's Lemma}, there exists $s,t\in\mathbb{Z}$, with $sp+tq = 1$. Then, let $n\cdot a$ denotes the addition of $a$ total of $n$ times (if $n$ is negative, do the addition of $-a$ total of $|n|$ times instead) and let $1_p$ denote the identity of $F$, then we get the following:
$$a = (sp+tq)\cdot a = (s\cdot (p\cdot 1_p))\cdot a + t(q\cdot a) = (s\cdot 0)\cdot a + t\cdot 0 = 0$$
Which shows that $a=0$. But, if $a=0$, then $\textmd{order}(a)=1$, which contradicts the statement that $\textmd{order}(a)=q>1$. 

So, our assumption is false, $|F|=p^n$ for some $n\in\mathbb{N}$.

\break

\section*{2 (insert commutative diagram)}
\begin{myBox}[]{}
    \begin{question}
        Show that $\mathbb{F}_2[x]/(x^3+x+1)\cong \mathbb{F}_2[y]/(y^3+y^2+1)$ and find an explicit isomorphism.
    \end{question}
\end{myBox}

\textbf{Pf:}

Let $K_1=\mathbb{F}_2[x]/(x^3+x+1)$, and $K_2=\mathbb{F}_2[y]/(y^3+y^2+1)$. Which, since the extensions are based on two degree $3$ polynomial, then $[K_1:\mathbb{F}_2]=[K_2:\mathbb{F}_2]=3$, which implies that $|K_1| = |K_2| = 2^3=8$.

Now, consider $\overline{\mathbb{F}}_2$: Since both $K_1,K_2$ are finite extensions of $\mathbb{F}_2$, they're algebraic extensions of $\mathbb{F}_2$. Hence, there exists embeddings $\phi_1:K_1\rightarrow \overline{\mathbb{F}}_2$ and $\phi_2:K_2\rightarrow\overline{\mathbb{F}}_2$.

Now, since $\phi_1(K_1)\cong K_1$ and $\phi_2(K_2)\cong K_2$, then $|\phi_1(K_1)|=|K_1|=8=|K_2|=|\phi_2(K_2)|$. Then, since $8=2^3$, under $\overline{\mathbb{F}}_2$, there exists a unique finite field $\mathbb{F}_{2^3}\subset\overline{\mathbb{F}}_2$ with order $|\mathbb{F}_{2^3}|=2^3$. Hence, this enforces $\phi_1(K_1)=\phi_2(K_2)=\mathbb{F}_{2^3}$.

So, after restriction, we get the following relationships of isomorphisms:
$$\phi_1:K_1\ \tilde{\rightarrow}\ \mathbb{F}_{2^3},\quad \phi_2:K_2\ \tilde{\rightarrow}\ \mathbb{F}_{2^3}$$
Hence, $\phi_2^{-1}\circ \phi_1:K_1\rightarrow K_2$ is an isomorphism, showing that $K_1\cong K_2$.

\hfil

\textbf{Construction of Isomorphism:}

Now, consider the element $(y+1)\in \mathbb{F}_2[y]$, it satisfies the following:
$$(y+1)^3+(y+1)+1 = (y+1)(y+1)^2+(y+1)+1 = (y+1)(y^2+1^2)+(y+1)\cdot 1+1$$
$$ = (y+1)(y^2+1+1)+1 = (y+1)y^2 + 1 = y^3+y^2+1$$
So, this implies that $(\overline{y+1})^3+\overline{y+1}+1 = \overline{y^3+y^2+1}= 0$ in $K_2$.

Hence, consider the ring isomorphism by $\phi:\mathbb{F}_2[x]\rightarrow\mathbb{F}_2[y]$ by $\phi(x)=(y+1)$, the maximal ideal $(x^3+x+1)\subset \mathbb{F}_2[x]$ has its image $\phi((x^3+x+1)) = ((y+1)^3+(y+1)+1) = (y^3+y^2+1)$, hence if take the projection $\pi_y:\mathbb{F}_2[y]\rightarrow K_2$ by $\pi_y(p(y))=\overline{p(y)}=p(y)\mod\ (y^3+y^2+1)$, the composition $\pi_y\circ\phi:\mathbb{F}_2[x]\rightarrow K_2$ becomes a ring homomorphism where the kernel is valid.

Which, since $\phi(x^3+x+1)=(y+1)^3+(y+1)+1 = y^3+y^2+1$, then $\pi_y\circ\phi(x^3+x+1)=\overline{y^3+y^2+1}=0$, hence $x^3+x+1\in \ker(\pi\circ\phi)$, or $(x^3+x+1)\subseteq \ker(\pi\circ\phi)$. Then, by \textbf{Generalized First Isomorphism Theorem}, there exists unique well-defined ring homomorphism $\overline{\phi}:\mathbb{F}_2[x]/(x^3+x+1)\rightarrow K_2$, such that with the projection $\pi_x:\mathbb{F}_2[x]\rightarrow K_1$ by $\pi(p(x))=\overline{p(x)}=p(x)\mod\ (x^3+x+1)$, the following diagram commutes:

\textbf{Insert commutative diagram}

Or, $\overline{\phi}\circ \pi_x = \pi_y\circ \phi$.

Then, since $\pi_y\circ\phi$ is surjective (since both $\pi_y$ and $\phi$ are surjective), while $\pi_x$ is surjective, then in case for $\overline{\phi}\circ \pi_x$ to be surjective, $\overline{\phi}$ is surjective. On the other hand, since $\overline{\phi}:K_1\rightarrow K_2$ with $K_1$ being a field, this map is injective. 

So, $\overline{\phi}$ is a well-defined isomorphism between $K_1$ and $K_2$, with the following formula:
$$\overline{\phi}(1)=1,\quad \overline{\phi}(\overline{x}) = \overline{y+1}\in K_2$$

\break

\section*{3}
\begin{myBox}[]{}
    \begin{question}
        Let $F$ be a perfect field with $\textmd{char}(F)=p$. Prove that $F=F^p$.
    \end{question}
\end{myBox}

\textbf{Pf:}

We'll prove by contradiction. Suppose $F$ is a perfect field, while $F\neq F^p$, then since $F^p\subsetneq F$, there exists $\alpha\in F\setminus F^p$, which implis that for all $\beta\in F$, $\beta^p\neq \alpha$.

So, the polynomial $x^p-\alpha \in F[x]$ has no solution in $F$, which based on \textbf{HW 2 Question 3}, this polynomial is in fact irreducible in $F[x]$.

\hfil

Now, consider $K=F[x]/(x^p-\alpha)$ a finite extension, and take $\theta=\overline{x}\in K$: since it satisfies $\overline{x}^p-\alpha = \overline{(x^p-\alpha)}=0$, then $\overline{x}^p=\alpha$, and $\theta = \overline{x}$ is a root of the monic polynomial $x^p-\alpha\in F[x]\subset K[x]$; also, since $x^p-\alpha$ is proven to be irreducible, then $m_{\theta,F}(x)=x^p-\alpha$.

But, because $\textmd{char}(F)=p$, then $\textmd{char}(K)=p$, which $\textmd{char}(K[x])=p$. So, based on Frobenius Endomorphism, $(x-\theta)^p = x^p-\theta^p$, showing that $(x-\theta)^p$ is a factorization of $x^p-\alpha$ in $K[x]$; then, since $K[x]$ is a UFD, such factorization is unique. Hence, $m_{\theta,F}(x) = (x-\theta)^p$, showing that the minimal polynomial of $\theta$ over $F$ has $\theta$ as a root with multiplicity $p>1$, so $\theta\in K$ is not separable over $F$, or $K/F$ is not a separable extension.

Yet, recall that $F$ is assumed to be a perfect field, while $K/F$ is a finite extension, then $K/F$ should be a separable extension by the definition of perfect field. So, we reach a contradiction, therefore the initial assumption is false, if $F$ is a perfect field, then $F=F^p$.

\break

\section*{4}
\begin{myBox}[]{}
    \begin{question}
        Show that an algebraic extension of a perfect field is perfect.
    \end{question}
\end{myBox}

\textbf{Pf:}

Suppose $F$ is a perfect field, then all finite extension is a separable extension. Which, for any algebraic extension $K/F$, there are two cases to consider:

\hfil

\textbf{1. When $K$ is a finite extension:}

Given any finite extension $K/F$, and consider any finite extension $L/K$" Since both extensions are finite (with $F\subseteq K\subseteq L$), then $L/F$ is also a finite extension. Based on the assumption that $F$ is perfect, $L/F$ is a separable extension.

Which, for all $\alpha\in L$, its minimal polynomial $m_{\alpha,F}(x)\in F[x]$ must have simple roots in $\overline{F}$.

\hfil

Since $L/F$ is a finite extension, then it is also algebraic, hence there exists embedding $\phi:L\rightarrow\overline{F}$ that fixes $F$, which can be extended to an injective ring homomorphism $\overline{\phi}:L[x]\rightarrow\overline{F}[x]$, by the following:
$$\forall a_n,...,a_0\in L,\quad \overline{\phi}(a_nx^n+...+a_0)=\phi(a_n)x^n+...+\phi(a_0)$$
(Note: it is injective, since if the output is $0$, then each coefficient $a_i$ must satisfy $\phi(a_i)=0$, and since $\phi$ is a field embedding, it is injective, so each $a_i=0$, showing the input is $0$).

Now, since $\alpha\in L$ is a root of $m_{\alpha,F}(x)\in F[x]\subseteq L[x]$, then let $k\in\mathbb{N}$ be the multiplicity of $\alpha$ as a root of $m_{\alpha,F}(x)$, we get $(x-\alpha)^k\mid m_{\alpha,F}(x)$, or $m_{\alpha,F}(x)=(x-\alpha)^kq(x)$ for some $q(x)\in L[x]$. Then, since $m_{\alpha,F}(x)\in F[x]$, we know $\overline{\phi}(m_{\alpha,F}(x)) = m_{\alpha,F}(x)$ (since $\phi$ fixes $F$, $\overline{\phi}$ also fixes $F$. Apply the extended ring homomorphism, we get:
$$m_{\alpha,F}(x)=\overline{\phi}(m_{\alpha,F}(x))=\overline{\phi}((x-\alpha)^kq(x)) = (x-\phi(\alpha))^k\overline{\phi}(q(x))\in \overline{F}[x]$$
This shows that $\phi(\alpha)$ is a root of $m_{\alpha,F}(x)$ in $\overline{F}$ with multiplicity $\geq k$. Then, because $m_{\alpha,F}(x)$ has simple roots in $\overline{F}$, $\phi(\alpha)$ as a root must have multiplicity of $1$, hence $k\leq 1$. This implies that $k=1$, which $\alpha$ as a root of $m_{\alpha,F}(x)$ must have multiplicity $1$.

\hfil

Finally, since $\alpha$ is also algebraic over $K$ (since $L/K$ are finite extensions), then $m_{\alpha, K}(x)\in K[x]$ exists; and since $m_{\alpha,F}(x)\in F[x]\subseteq K[x]$, then $m_{\alpha,K}(x)\mid m_{\alpha,F}(x)$ in $K[x]$.

Because $\alpha$ is a root of $m_{\alpha,K}(x)$, let $l\in \mathbb{N}$ be its multiplicity, we get $(x-\alpha)^l\mid m_{\alpha,K}(x)$ in $L[x]$;also, since $m_{\alpha,K}(x)\mid m_{\alpha,F}(x)$ in $K[x]\subseteq L[x]$, this implies $(x-\alpha)^l\mid m_{\alpha,F}(x)$ in $L[x]$. Hence, since $\alpha$ is proven to be a root of $m_{\alpha,F}(x)$ with multiplicity $1$, this implies that $l\leq 1$, or $l=1$.

So, $\alpha$ as a root of $m_{\alpha,K}(x)$ has multiplicity $1$, and since $m_{\alpha,K}(x)$ is irreducible in $K[x]$, all its root in $\overline{K}$ must have the same multiplicity. Which, they must all have multiplicity $1$ (or being a simple root), showing that $\alpha$ is actually separable over $K$.

This shows that $L/K$ is in fact a separable extension, which proves that $K$ is also perfect. So, all finite extension $K/F$ is also perfect.

\hfil

\textbf{2. When $[K:F]=\infty$:}

Suppose $K/F$ is an infinite algebraic extension, then for all finite extension $L/K$ (which is also algebraic), we have $L/F$ also being an algebraic extension. Then for all $\alpha \in L$, there exists $m_{\alpha,K}(x)\in K[x]$, say $m_{\alpha,K}(x)=a_nx^n+...+a_0$ for some $a_0,...,a_n\in K$. Then, since $K/F$ is an algebraic extension, all elements in $K$ is algebraic over $F$, showing that $K'=F(a_0,...,a_n)$ is a finite extension over $F$. By the proof in finite case, $F$ is a perfect field implies $K'/F$ is also a perfect field. Then, since $K'(\alpha)/K'$ is again a finite extension, it is a separable extension. Hence, $\alpha$ is separable over $K'$, which $m_{\alpha,K'}(x)\in K'[x]$ must have simple roots in $\overline{K'}$.

However, since $K'\subseteq K$, then $m_{\alpha,K}(x)\mid m_{\alpha,K'}(x)$; on the other hand, since $m_{\alpha,K}(x)\in K'[x]$ (since all the coefficients are contained in $K'$), then this enforces $m_{\alpha,K}(x) = m_{\alpha,K'}(x)$. So, $m_{\alpha,K}(x)$ has simple roots in $\overline{K'}$, while $K/K'$ is an algebraic extension (since $K/F$ is, $K'\subseteq K$, and $K'/F$ is also algebraic), then $\overline{K}\cong \overline{K'}$ via some field homomorphism fixing $K'$, so $m_{\alpha,K}(x)$ is also having simple roots in $\overline{K}$.

This proves that $\alpha$ is separable over $K$, hence $L/K$ is in fact a separable extension, hence this proves that $K$ is perfect.

\hfil

So, regardless of the case, if $F$ is perfect, its algebraic extension $K/F$ is perfect.

\break

\section*{5}
\begin{myBox}[]{}
    \begin{question}
        Let $K=\mathbb{F}_p(t,w)$ be the rational function field with two indeterminates $t,w$ over $\mathbb{F}_p$. Let $L$ be the splitting field over $K$ of the polynomial $h(x)=f(x)g(x)$ where $f(x)=x^p-t$ and $g(x)=x^p-w$. Prove the following:
        \begin{itemize}
            \item[(a)] $f$ and $g$ are irreducible over $K$.
            \item[(b)] $[L:K]=p^2$.
            \item[(c)] $L/K$ is not seperable.
            \item[(d)] $a^p\in K$ for all $a\in L$.
        \end{itemize}
    \end{question}
\end{myBox}

\textbf{Pf:}

Before starting, let $\mathbb{F}_p(w) = F_1$, and $F_2 = \mathbb{F}_p(t)$, then $K=\mathbb{F}_p(t)(w) = F_2(w)$, and $K=\mathbb{F}_p(w)(t) = F_1(t)$.

\begin{itemize}
    \item[(a)] Based on what we've proven in \textbf{HW 2 Question 3}, since $\textmd{char}(K)=p$, for any $\alpha\in K$, if $x^p-\alpha$ has no solution in $K$, then it is irreducible in $K[x]$. Hence, to prove $f,g$ are irreducible in $K[x]$, it suffices to show there's no solution in $K$.
    
    First, suppose the contrary that there exists $\alpha\in K$, such that $\alpha^p-w = 0$, then since $K=F_2(w)$, there exists $f(w),g(w)\in F_2[w]$, such that $\alpha = \frac{f(w)}{g(w)}$. Then, it implies the following:
    $$\alpha^p-w = \left(\frac{f(w)}{g(w)}\right)^p-w = 0,\quad (f(w))^p = w(g(w))^p$$
    Let $k = \deg_w(f)$, and $l=\deg_w(g)$, then $\deg_w(f^p) = kp$, while $\deg_w(wq^p) = \deg_w(w)+\deg_w(q^p) = 1+lp$. Since $(f(w))^p=w(g(w))^p$, then $kp = 1+lp$; however, the left side is divisible by $p$, while the right side is not divisible by $p$, so we reach a contradiction.
    Hence, the assumption is false, there doesn't exist $\alpha\in K$, satisfying $\alpha^p-w=0$. So, $x^p-w\in K[x]$ has no solution in $K$, showing that it is irreducible.

    Now, using the same proof on $x^p-t$ by viewing $K=F_1(t)$, we can also prove that $x^p-t$ has no solution in $K$, which $x^p-t$ is also irreducible over $K$.

    \hfil
    
    \item[(b)] Since $L/K$ is a splitting field of $h(x)=f(x)g(x)$ (where $f(x)=x^p-t$, and $g(x)=x^p-w$), then both $f(x),g(x)$ splits completely over $L$. Hence, there exists $\alpha\in L$, such that $f(\alpha)=0$. Then, since $x^p-t$ is monic, while proven to be irreducible in $K[x]$ by \textbf{part (a)}, then $m_{\alpha,K}(x) = x^p-t$.
    
    Now, because $\alpha^p-t = 0$, $\alpha^p=t$. However, since $K$ has characteristic $p$, then $\textmd{char}(L)=p$, so $\textmd{char}(L[x]) = p$. Then, within $L[x]$, since $(x-\alpha)^p = x^p-\alpha^p = x^p-t$, then $(x-\alpha)^p$ is a factorization of $x^p-t$; on the other hand, since $L[x]$ is a UFD, such factorization must be unique. Hence, $(x-\alpha)^p$ is the factorization of $x^p-t$, $\alpha$ is the only root of $x^p-t$. 

    Let $\beta\in L$ be a root of $g(x) = x^p-w$, then using similar logic we can deduce that $x^p-w = (x-\beta)^p$, so $\beta$ is the only root of $x^p-w$.

    Which, since $h(x)=f(x)g(x)=(x^p-t)(x^p-w)$, then $h(x)$ only has roots $\alpha,\beta$ in $L$. Hence, since $L/K$ is a spliting field of $h(x)\in K[x]$, then $L=K(\alpha,\beta)$. So, we'll consider the extensions $K\subseteq K(\alpha)\subseteq K(\alpha,\beta)$.

    \hfil

    Since $\alpha$ has its minimal polynomial over $K$ being $x^p-t\in K[x]$, then $K(\alpha)\cong K[x]/(x^p-t)$, hence $[K(\alpha):K]=p$.
    So, given that $[L:K]=[K(\alpha,\beta):K(\alpha)]\cdot [K(\alpha):K]$, to prove that $[L:K]=p^2 = [K(\alpha,\beta):K(\alpha)]\cdot [K(\alpha):K] = [K(\alpha,\beta):K(\alpha)]\cdot p$, it suffices to show $[K(\alpha,\beta):K(\alpha)] = p$. 
    
    And, if showing that $x^p-w\in K(\alpha)[x]$ is irreducible, since it is monic and $\beta$ is assumed to be a root of it, then $\beta$ must have its minimal polynomial over $K(\alpha)$ being $x^p-w$, hence $K(\alpha,\beta) = K(\alpha)(\beta)\cong K(\alpha)[x](^p-w)$, showing that $[K(\alpha,\beta):K(\alpha)]=p$. So, the last goal is to prove $x^p-w$ is irreducible over $K(\alpha)$ (Note: since $K(\alpha)$ is again having characteristic $p$, it suffices to show that $x^p-w$ has no roots in $K(\alpha)$).

    \hfil

    Suppose the contrary that there exists $\gamma\in K(\alpha)$ which satisfies $\gamma^p-w = 0$, then since $K(\alpha)\cong K[x]/(x^p-t)$, there exists $a_0,...,a_{p-1}\in K = F_2(w)$, such that the following is true:
    $$\gamma = a_{p-1}\alpha^{p-1}+...+a_0$$
    Which, each $a_i$ can be expressed as $\frac{f_i(w)}{g_i(w)}$ for some $f_i(w), g_i(w)\in F_2[w]$. Then, using Frobenius Endomorphism, we get the following:
    $$\gamma^p = (a_{p-1}\alpha^{p-1}+...+a_0)^p = a_{p-1}^p(\alpha^p)^{p-1} + ... + a_0^p$$
    $$ = \frac{f_{p-1}(w)^p}{g_{p-1}(w)^p}t^{p-1}+...+\frac{f_0(w)^p}{g_0(w)^p}$$
    Also, since $\gamma^p-w = 0$, then $\gamma^p=w$. So, if we take $q(w) = \prod_{i=0}^{p-1}g_i(w)^p\in F_2[w]$, we know that $\deg_w(q) = kp$ for some $k\in \mathbb{N}$ (since its product of polynomials, each to the power of $p$), and $q(w)\cdot \gamma^p\in F_2[w]$, since $t\in F_2=\mathbb{F}_p(t)$, and all the denominators $g_i(w)^p$ were cancelled out by $q(w)$.
    
    Hence, we get:
    $$q(w)\cdot \gamma^p = w \cdot q(w),\quad \deg_w(q\cdot \gamma^p) = \deg_w(w\cdot q(w)) =\deg_w(w)+\deg(q)= 1+kp$$
    On the other hand, each term $\frac{f_i(w)^p}{g_i(w)^p}t^i$ in $\gamma^p$ after multiplied by $q(w)$ would become:
    $$q(w)\cdot \frac{f_i(w)^p}{g_i(w)^p}t^i = t^i\cdot f_i(w)^p\cdot \prod_{\substack{j=1\\j\neq i}}^{p-1}g_j(w)^p\in F_2[w]$$
    (Note: the $g_i(w)^p$ in $q(w)$ got cancelled out by the denominator).

    Hence, $q(w)\cdot \frac{f_i(w)^p}{g_i(w)^p}t^i$ as a polynomial of $w$, is in fact having degree $l_ip$ for some $l_i\in\mathbb{N}$ (since it is also product of polynomials, each raised to the power of $p$).

    Then, $q(w)\cdot \gamma^p$ as the summation of all $q(w)\cdot \frac{f_i(w)^p}{g_i(w)^p}t^i$ (with index $i\in \{0,...,n\}$, since $q(w)\cdot \gamma^p = q(w)\left(\frac{f_{p-1}(w)^p}{g_{p-1}(w)^p}t^{p-1}+...+\frac{f_0(w)^p}{g_0(w)^p}\right)$), then since it's a sum of polynomials of $w$ with degree being multiples of $p$, then the sum $q(w)\cdot \gamma^p$ must have its degree $\deg_w(q(w)\cdot \gamma^p) = lp$ for some $l\in\mathbb{N}$.

    Hence, we must have $lp = 1+kp$ (since they're the degree of the same polynomial). But again, since the left side is divisible by $p$, while the right side is not divisible by $p$, we reach a contradiction.
    Hence, our assumption must be false, $K(\alpha)$ can't contain a root of $x^p-w$. Hence, followed from the prove before this section, $[K(\alpha,\beta):K(\alpha)] = p$,
    showing that $[L:K]=p^2$.

    \hfil
    
    \item[(c)] Using the results from \textbf{part (b)}, we know that $(x-\alpha)^p = x^p-t$ is the unique factorization. Hence, $\alpha$ as a root of $x^p-t$ with multiplicity $p>1$, while $x^p-t=m_{\alpha,K}(x)\in K[x]$ is also proven, then $m_{\alpha,K}(x)$ has roots with multiplicity $>1$, showing that $\alpha$ is not separable over $K$, hence $L/K$ is not a separable extension.

    \hfil

    \item[(d)] In \textbf{part (b)}, we've proven that $K(\alpha,\beta)=K(\alpha)(\beta) \cong K(\alpha)[x]/(x^p-w)$, hence for all $a\in K(\alpha,\beta)$, there exists $a_0,...,a_{p-1}\in K(\alpha)$, such that the following holds:
    $$a = a_{p-1}\alpha^{p-1}+...+a_0$$
    Which, applying Frobenius Endomorphism, we get:
    $$a^p = (a_{p-1}\alpha^{p-1}+...+a_0)^p = a_{p-1}^p(\alpha^p)^{p-1}+...+a_0^p$$
    $$ = a_{p-1}^p(t)^{p-1}+...+a_0^p$$
    Since $t\in K\subset K(\alpha)$, while each $a_i\in K(\alpha)$, then $a^p\in K(\alpha)$.

    Now, for all $\delta\in K(\alpha)$, since $K(\alpha)\cong K[x]/(x^p-t)$, there exists $b_0,...,b_{p-1}\in K$, such that the following holds:
    $$\delta = b_{p-1}\alpha^{p-1}+...+b_0$$
    Then again, applying Frobenius Endomorphism, we get:
    $$\delta^p = (b_{p-1}\alpha^{p-1}+...+b_0)^p = b_{p-1}^p(\alpha^p)^{p-1}+...+b_0^p$$
    $$ = b_{p-1}^pt^{p-1}+...+b_0^p$$
    Since each $b_i^p \in K$, while $t\in K$, this shows that $\delta^p \in K$.

    Hence, going back to $a^p = a_{p-1}^p(t)^{p-1}+...+a_0^p$, since each $a_i\in K(\alpha)$, then $a_i^p\in K$, showing that $a^p$ as a finite sum and product of elements in $K$, is in $K$.

    So, $a^p\in K$, showing that all element $a\in K(\alpha,\beta)=L$ satisfies $a^p\in K$.
\end{itemize}

\end{document}