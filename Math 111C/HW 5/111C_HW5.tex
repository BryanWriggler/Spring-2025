\documentclass{article}
\usepackage{graphicx} % Required for inserting images
\usepackage[margin = 2.54cm]{geometry}
\usepackage[most]{tcolorbox}

\newtcolorbox{myBox}[3]{
arc=5mm,
lower separated=false,
fonttitle=\bfseries,
%colbacktitle=green!10,
%coltitle=green!50!black,
enhanced,
attach boxed title to top left={xshift=0.5cm,
        yshift=-2mm},
colframe=blue!50!black,
colback=blue!10
}

\usepackage{amsmath}
\usepackage{amssymb}
\usepackage{verbatim}
\usepackage[utf8]{inputenc}
\linespread{1.2}

\newtheorem{definition}{Definition}
\newtheorem{proposition}{Proposition}
\newtheorem{theorem}{Theorem}
\newtheorem{question}{Question}

\title{Math 111C HW5}
\author{Zih-Yu Hsieh}

\begin{document}
\maketitle

\section*{1}
\begin{myBox}[]{}
    \begin{question}
        Let $F$ be a finite field. Prove that $|F|=p^n$ for some prime $p$ and $n\in\mathbb{N}$.
    \end{question}
\end{myBox}

\textbf{Pf:}

Since $F$ is a finite field, then $\textmd{char}(F)=p$ for some prime $p$. It suffices to show that $|F|=p^n$ for some $n\in\mathbb{N}$.

Suppose the contrary that the above statement doesn't hold, then there exists some distinct prime number $q\neq p$, such that $q$ divides $|F|$. Recall that $F$ is a finite abelian group under addition, hence \textbf{Cauchy's Theorem} applies, there exists $a\in F$, such that its order with respect to addition (denoted as $\textmd{order}(a)$) is $q$.

However, since $p,q$ are distinct primes, then by \textbf{Bezout's Lemma}, there exists $s,t\in\mathbb{Z}$, with $sp+tq = 1$. Then, let $n\cdot a$ denotes the addition of $a$ total of $n$ times (if $n$ is negative, do the addition of $-a$ total of $|n|$ times instead) and let $1_p$ denote the identity of $F$, then we get the following:
$$a = (sp+tq)\cdot a = (s\cdot (p\cdot 1_p))\cdot a + t(q\cdot a) = (s\cdot 0)\cdot a + t\cdot 0 = 0$$
Which shows that $a=0$. But, if $a=0$, then $\textmd{order}(a)=1$, which contradicts the statement that $\textmd{order}(a)=q>1$. 

So, our assumption is false, $|F|=p^n$ for some $n\in\mathbb{N}$.

\break

\section*{2 (not done)}
\begin{myBox}[]{}
    \begin{question}
        Show that $\mathbb{F}_2[x]/(x^3+x+1)\cong \mathbb{F}_2[y]/(y^3+y^2+1)$ and find an explicit isomorphism.
    \end{question}
\end{myBox}

\textbf{Pf:}

Let $K_1=\mathbb{F}_2[x]/(x^3+x+1)$, and $K_2=\mathbb{F}_2[y]/(y^3+y^2+1)$. Which, since the extensions are based on two degree $3$ polynomial, then $[K_1:\mathbb{F}_2]=[K_2:\mathbb{F}_2]=3$, which implies that $|K_1| = |K_2| = 2^3=8$.

Now, consider $\overline{\mathbb{F}}_2$: Since both $K_1,K_2$ are finite extensions of $\mathbb{F}_2$, they're algebraic extensions of $\mathbb{F}_2$. Hence, there exists embeddings $\phi_1:K_1\rightarrow \overline{\mathbb{F}}_2$ and $\phi_2:K_2\rightarrow\overline{\mathbb{F}}_2$.

Now, since $\phi_1(K_1)\cong K_1$ and $\phi_2(K_2)\cong K_2$, then $|\phi_1(K_1)|=|K_1|=8=|K_2|=|\phi_2(K_2)|$. Then, since $8=2^3$, under $\overline{\mathbb{F}}_2$, there exists a unique finite field $\mathbb{F}_{2^3}\subset\overline{\mathbb{F}}_2$ with order $|\mathbb{F}_{2^3}|=2^3$. Hence, this enforces $\phi_1(K_1)=\phi_2(K_2)=\mathbb{F}_{2^3}$.

So, after restriction, we get the following relationships of isomorphisms:
$$\phi_1:K_1\ \tilde{\rightarrow}\ \mathbb{F}_{2^3},\quad \phi_2:K_2\ \tilde{\rightarrow}\ \mathbb{F}_{2^3}$$
Hence, $\phi_2^{-1}\circ \phi_1:K_1\rightarrow K_2$ is an isomorphism, showing that $K_1\cong K_2$.

\hfil



\break

\section*{3}
\begin{myBox}[]{}
    \begin{question}
        Let $F$ be a perfect field with $\textmd{char}(F)=p$. Prove that $F=F^p$.
    \end{question}
\end{myBox}

\textbf{Pf:}

We'll prove by contradiction. Suppose $F$ is a perfect field, while $F\neq F^p$, then since $F^p\subsetneq F$, there exists $\alpha\in F\setminus F^p$, which implis that for all $\beta\in F$, $\beta^p\neq \alpha$.

So, the polynomial $x^p-\alpha \in F[x]$ has no solution in $F$, which based on \textbf{HW 2 Question 3}, this polynomial is in fact irreducible in $F[x]$.

\hfil

Now, consider $K=F[x]/(x^p-\alpha)$ (a finite extension, hence $K/F$ is algebraic), and take $\theta=\overline{x}\in K$: since it satisfies $\overline{x}^p-\alpha = \overline{(x^p-\alpha)}=0$, then $\overline{x}^p=\alpha$, and $\theta = \overline{x}$ is a root of the monic polynomial $x^p-\alpha\in F[x]\subset K[x]$; also, since $x^p-\alpha$ is proven to be irreducible, then $m_{\theta,F}(x)=x^p-\alpha$.

But, because $\textmd{char}(F)=p$, then $\textmd{char}(K)=p$, which $\textmd{char}(K[x])=p$. So, based on Frobenius Endomorphism, $(x-\theta)^p = x^p-\theta^p$, showing that $(x-\theta)^p$ is a factorization of $x^p-\alpha$ in $K[x]$; then, since $K[x]$ is a UFD, such factorization is unique. Hence, $m_{\theta,F}(x) = (x-\theta)^p$, showing that the minimal polynomial of $\theta$ over $F$ has $\theta$ as a root with multiplicity $p>1$, so $\theta\in K$ is not separable over $F$, or $K/F$ is not a separable extension.

Yet, recall that $F$ is assumed to be a perfect field, while $K/F$ is an algebraic extension, then $K/F$ should be a separable extension by the definition of perfect field. So, we reach a contradiction, therefore the initial assumption is false, if $F$ is a perfect field, then $F=F^p$.

\break

\section*{4}
\begin{myBox}[]{}
    \begin{question}
        Show that an algebraic extension of a perfect field is perfect.
    \end{question}
\end{myBox}

\textbf{Pf:}

\break

\section*{5}
\begin{myBox}[]{}
    \begin{question}
        Let $K=\mathbb{F}_p(t,w)$ be the rational function field with two indeterminates $t,w$ over $\mathbb{F}_p$. Let $L$ be the splitting field over $K$ of the polynomial $h(x)=f(x)g(x)$ where $f(x)=x^p-t$ and $g(x)=x^p-w$. Prove the following:
        \begin{itemize}
            \item[(a)] $f$ and $g$ are irreducible over $K$.
            \item[(b)] $[L:K]=p^2$.
            \item[(c)] $L/K$ is not seperable.
            \item[(d)] $a^p\in K$ for all $a\in L$.
        \end{itemize}
    \end{question}
\end{myBox}

\textbf{Pf:}

\end{document}