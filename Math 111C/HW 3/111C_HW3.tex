\documentclass{article}
\usepackage{graphicx} % Required for inserting images
\usepackage[margin = 2.54cm]{geometry}
\usepackage[most]{tcolorbox}

\newtcolorbox{myBox}[3]{
arc=5mm,
lower separated=false,
fonttitle=\bfseries,
%colbacktitle=green!10,
%coltitle=green!50!black,
enhanced,
attach boxed title to top left={xshift=0.5cm,
        yshift=-2mm},
colframe=blue!50!black,
colback=blue!10
}

\usepackage{amsmath}
\usepackage{amssymb}
\usepackage{verbatim}
\usepackage[utf8]{inputenc}
\linespread{1.2}

\newtheorem{definition}{Definition}
\newtheorem{proposition}{Proposition}
\newtheorem{theorem}{Theorem}
\newtheorem{question}{Question}

\title{Math 111C HW3}
\author{Zih-Yu Hsieh}

\begin{document}
\maketitle

\section*{1}
\begin{myBox}[]{}
    \begin{question}
        Let $H/F$ be a field extension and $f(x), g(x) \in F[x]$. Suppose $K_1,\ K_2$ are splitting
        fields of $f(x)$ and $g(x)$ respectively, contained in $H$. Prove that $K_1K_2:=K_1(K_2)$ is a
        splitting field of the polynomial $f(x)g(x)$.
    \end{question}
\end{myBox}

\textbf{Pf:}

Since $K_1$ is a splitting field of $f(x)$ and $K_2$ is a splitting field of $g(x)$, then the following two statements are true:
$$\exists a\in F,\ a_1,...,a_n\in K_1,\quad f(x)=a(x-a_1)...(x-a_n)$$
$$\exists b\in F,\ b_1,...,b_m\in K_2,\quad g(x)=b(x-b_1)...(x-b_m)$$
In particular, $K_1=F(a_1,...,a_n)$ and $K_2=F(b_1,...,b_m)$ based on what we've proven in class.

Then, since $K_1\subseteq K_1(K_2)$, $a_1,...,a_n\in K_1(K_2)$; similarly, for all $q\in K_2$, since $1\cdot q=q\in K_1(K_2)$, then $K_2\subseteq K_1(K_2)$, hence $b_1,...,b_m\in K_1(K_2)$.

Because $K_1(K_2)$ contains all roots of $f(x)$ in $K_1$, and all roots of $g(x)$ in $K_2$, hence $f(x)g(x)$ also splits completely over $K_1(K_2)$ (since $f(x),g(x)$ both split completely due to the existence of all roots).
In particular, the factorization is given as follow, up to unit associates:
$$f(x)g(x)=ab(x-a_1)...(x-a_n)\cdot (x-b_1)...(x-b_m)$$

\hfil

Now, to consider the splitting field of $f(x)g(x)$, say $E/F\subseteq K_1(K_2)/F$. From the statements proven in class, given the roots of $f(x)g(x)$ above, we know $E=F(a_1,...,a_n,b_1,...,b_m) = \left(F(a_1,...,a_n)\right)(b_1,...,b_m)=K_1(b_1,...,b_m)$.

Which, $K_1(b_1,...,b_m)$ by definition, is the smallest field within $K_1(K_2)$ that's containing both $K_1$ and the set $\{b_1,...,b_m\}$; however, since $F\subseteq K_1(b_1,...,b_m)$, while $K_2=F(b_1,...,b_m)$ is defined to be the smallest field containing both $F$ and $\{b_1,...,b_m\}$,
then since $K_1(b_1,...,b_2)$ contains both, we must have $K_2=F(b_1,...,b_m)\subseteq K_1(b_1,...,b_2)$.

Lastly, since $K_1,\ K_2\subseteq K_1(b_1,...,b_2)$, while $K_1(K_2)$ is the smallest field in $K_1(K_2)$ containing both $K_1$ and $K_2$, then $K_1(b_1,...,b_m)$ containing both $K_1,K_2$ implies $K_1(K_2)\subseteq K_1(b_1,...,b_m)$,
showing that $K_1(K_2)=K_1(b_1,...,b_m)=E$.

Hence, $K_1(K_2)$ is in fact a splitting field of $f(x)g(x)\in F[x]$.

\break

\section*{2}
\begin{myBox}[]{}
    \begin{question}
        Define the set of algebraic numbers $\mathbb{A}$ to be the set of all complex numbers which are
        algebraic over $\mathbb{Q}$. Show that $\mathbb{A}/\mathbb{Q}$ is an infinite algebraic extension.
    \end{question}
\end{myBox}

\textbf{Pf:}

We'll prove this via contradiction. Suppose $\mathbb{A}/\mathbb{Q}$ is a finite extension, say $[\mathbb{A}:\mathbb{Q}]=n<\infty$.
Then, for any $\alpha\in\mathbb{A}$, since the list $1,\alpha,\alpha^2,...,\alpha^n\in\mathbb{A}/\mathbb{Q}$ has length $(n+1)$, while $\mathbb{A}$ as a $\mathbb{Q}$-vector space
has dimension $n$, then the above list is linearly dependent, showing that there exists $a_0,a_1,...,a_n\in\mathbb{Q}$, such that the following is true:
$$f(x)=a_0+a_1x+....+a_nx^n\in\mathbb{Q}[x],\quad f(\alpha)=\sum_{k=0}^{n}a_k\alpha^k=0$$
Now, take the minimal polynomial of $\alpha$ over $\mathbb{Q}$ (denoted as $m_{\alpha,\mathbb{Q}}(x)\in\mathbb{Q}[x]$), since $\alpha$ is a root of $f(x)$ defined above,
then $m_{\alpha,\mathbb{Q}}(x)\mid f(x)$, showing that $\deg(m_{\alpha,\mathbb{Q}})\leq \deg(f)\leq n$. So, all $\alpha\in\mathbb{A}/\mathbb{Q}$ should have minimal polynomial with degree at most $n$.

\hfil

However, here is a counterexample: Consider $k>n$, and the polynomial $x^k-2\in\mathbb{Q}[x]$:
Since over $\mathbb{Z}$, it satisfies the Eisenstein Criterion (the leading coefficient is $1$, not divisible by $2$; the rest of the coefficients are $0$ and $2$, which are divisible by $2$; and $2$ as the constant is not divisible by $2^2$),
hence $x^k-2$ is irreducible over $\mathbb{Q}$ (and it is also monic).

Now, consider the element $2^{1/k}\in\mathbb{A}$: Since it satisfies $(2^{1/k})^k-2 = 2-2=0$, then it is a root of $x^k-2$.
Then, because $x^k-2$ is monnic and irreducible over $\mathbb{Q}$, it is in fact the minimal polynomial of $2^{1/k}\in\mathbb{A}$.

Yet, the proposed polynomial has degree $k>n$, while it is a minimal polynomial of some elements in $\mathbb{A}$, which supposedly should have degree at most $n$,
then this forms a contradiction.

Hence, the assumption is false, $\mathbb{A}/\mathbb{Q}$ must be an infinite extension.

\break

\section*{3}
\begin{myBox}[]{}
    \begin{question}
        Let $n\in\mathbb{N}$ and $\mu_n$ be the (multiplicative) group of $n^{th}$ roots of unity in $\mathbb{C}$.
        A generator of $\mu_n$ is called a primitive $n^{th}$ root of unity. Let $F_n\subseteq\mathbb{C}$ be the splitting field  of $x^n-1$ over $\mathbb{Q}$.
        \begin{itemize}
            \item[(a)] If $\zeta_n$ is any primitive $n^{th}$ root of unity, prove that $F_n=\mathbb{Q}(\zeta_n)$.
            \item[(b)] Prove that any complex root of $m_{\zeta_n,\mathbb{Q}}(x)$ is also a primitive $n^{th}$ root of unity.
            \item[(c)] Prove that $[F_n:\mathbb{Q}]\leq \phi(n)$ where $\phi$ is the famous Euler's totient function.
        \end{itemize}
    \end{question}
\end{myBox}

\textbf{Pf:}
\begin{itemize}
    \item[(a)] Given that $\zeta_n$ is a primitive $n^{th}$ root of unity, then $\left<\zeta_n\right>=\mu_n$ (since it generates the whole $\mu_n$).
    So, for any $\alpha\in \mu_n$, $\alpha=\zeta_n^k$ for some $k\in\mathbb{Z}$, proving that $\alpha\in\mathbb{Q}(\zeta_n)$. Hence, since $\mathbb{Q}(\zeta_n)$ contains all $n^{th}$ roots of unity (all roots of $x^n-1$ over $\mathbb{C}$),
    then $x^n-1$ can be splitted completely over $\mathbb{Q}(\zeta_n)$, which the splitting field of $x^n-1$, $F_n\subseteq \mathbb{Q}(\zeta_n)$.

    On the other hand, since $\mathbb{Q}\subseteq F_n$, while $F_n\subseteq\mathbb{C}$ is the splitting field of $x^n-1$, in particular, it must contain all roots of $x^n-1$, which $\zeta_n\in F_n$.

    Then, since $\mathbb{Q}(\zeta_n)$ is the smallest field in $\mathbb{C}$, containing both $\mathbb{Q}$ and $\zeta_n$, then because $F_n$ contains both collections, $\mathbb{Q}(\zeta_n)\subseteq F_n$.

    This proves that $F_n=\mathbb{Q}(\zeta_n)$.

    \hfil

    \item[(b)] Suppose $\alpha\in\mathbb{C}$ is also a root of $m_{\zeta_n,\mathbb{Q}}(x)$, then since $m_{\zeta_n,\mathbb{Q}(x)}\mid x^n-1$ due to the fact that $\zeta_n^n-1=0$ and $m_{\zeta_n,\mathbb{Q}}(x)$ is the minimal polynomial of $\zeta_n$ over $\mathbb{Q}$, then $\alpha$ is also an $n^{th}$ root of unity.
    
    However, if $\alpha$ is not a primitive $n^{th}$ root of unity, there exists $k<n$, such that $\alpha^k-1=0$. Choose the smallest $k$, then $\alpha$ is in fact a primitive $k^{th}$ root of unity.
    Because $m_{\zeta_n,\mathbb{Q}}(x)\in\mathbb{Q}[x]$ is both monic and irreducible, then since $\alpha$ is its root, it must also be the minimal polynomial of $\alpha$. Which, $\alpha^k-1=0$ implies that $m_{\zeta_n,\mathbb{Q}}(x)\mid x^k-1$ (since $\alpha$ is also a root of $x^k-1$),
    which all roots $\beta$ of $m_{\zeta_n,\mathbb{Q}}(x)$ satisfies $\beta^k-1=0$, including $\zeta_n$.

    However, this is a contradiction, since $\zeta_n$ as a primitive $n^{th}$ root of unity of $\mu_n$, supposedly has order $n$ (or else it can't generate all the elements), while now $\zeta_n^k-1=0$, showing that order of $\zeta_n$ is at most $k<n$.
    So, our assumption is false, if $\alpha$ is a root of $m_{\zeta_n,\mathbb{Q}}(x)$, it must also be a primitive $n^{th}$ root of unity.
    
    \hfil

    \item[(c)] Recall that in group theory, given a finite cyclic group $\left<a\right>$ with order $|\left<a\right>|=n$, then any $a^k\in \left<a\right>$ is a generator of $\left<a\right>$ iff $\gcd(k,n)=1$.
    Then, since $|\mu_n|=|\left<\zeta_n\right>|=n$, all the primitive $n^{th}$ roots of unity in $\mu_n$ (the generators) must be in the form $\zeta_n^k$, where $k\in\mathbb{Z}_n$ satisfies $\gcd(k,n)=1$.

    This implies that the number of primitive roots of unity for $\mu_n$ is precisely given by $\phi(n)$ (or, the number of elements in $\mathbb{Z}_n$ that is coprime to $n$).

    Now, from \textbf{part (b)}, since we've proven that $m_{\zeta_n,\mathbb{Q}}(x)$ must have all of its roots being primitive $n^{th}$ roots of unity,
    then it can have at most $\phi(n)$ roots, showing that $\deg(m_{\zeta_n,\mathbb{Q}})\leq \phi(n)$.

    Finally, since $F_n=\mathbb{Q}(\zeta_n)$, while $\mathbb{Q}(\zeta_n)\cong \mathbb{Q}[x]/(m_{\zeta_n,\mathbb{Q}}(x))$, then since $[\mathbb{Q}[x]/(m_{\zeta_n,\mathbb{Q}}(x)):\mathbb{Q}]=\deg(m_{\zeta_n,\mathbb{Q}})\leq \phi(n)$,
    then $[F_n:\mathbb{Q}]=[\mathbb{Q}(\zeta_n):\mathbb{Q}]\leq \phi(n)$.
    
\end{itemize}

\break

\section*{4}
\begin{myBox}[]{}
    \begin{question}
        Let $f(x)=x^n-2\in\mathbb{Q}[x]$ with $n\in\mathbb{N}$. Let $\alpha\in\mathbb{C}$ be any $n^{th}$ root of $2$ and $\zeta_n=e^{2\pi i/n}$.
        Let $E\subseteq\mathbb{C}$ be the splitting field of $f(x)$ over $\mathbb{Q}$.
        \begin{itemize}
            \item[(a)] Show that $E=\mathbb{Q}(\alpha,\zeta_n)$.
            \item[(b)] Let $n\geq 3$. Prove that $E\neq\mathbb{Q}(\alpha)$ and $E\neq\mathbb{Q}(\zeta_n)$. 
        \end{itemize}
    \end{question}
\end{myBox}

\textbf{Pf:}

\begin{itemize}
    \item[(a)] First, if we consider $x^n-2\in\mathbb{C}[x]$, it has at most $n$ roots counting multiplicity; now, for all integer $1\leq k\leq n$,
    since $\sqrt[n]{2}e^{2\pi i\cdot k/n}$ satisfies $(\sqrt[n]{2}e^{2\pi i\cdot k/n})^n-2 = 2-2=0$, while each $k$ corresponds to a distinct element in $\mathbb{C}$,
    then there are in fact $n$ distinct roots for $x^n-2$ in $\mathbb{C}$, given as the above form. And, $E=\mathbb{Q}(\alpha_1,...,\alpha_n)$, where each $\alpha_k=\sqrt[n]{2}e^{2\pi i\cdot k/n}$.

    First, for all root $\alpha_k = \sqrt[n]{2}e^{2\pi i\cdot k/n}$, since $\alpha = \sqrt[n]{2}e^{2\pi i\cdot m/n}$ for some integer $0\leq m <n$, then $\alpha_k = \sqrt[n]{2}e^{2\pi i\cdot m/n}\cdot e^{2\pi i\cdot (k-m)/n} = \alpha\cdot (\zeta_n)^{(k-m)}$,
    hence $\alpha_k\in \mathbb{Q}(\alpha,\zeta_n)$. Since all generators of $E$ are contained in $\mathbb{Q}(\alpha,\zeta_n)$, then $E\subseteq \mathbb{Q}(\alpha,\zeta_n)$.

    Then, since $\sqrt[n]{2},\ \sqrt[n]{2}e^{2\pi i/n}\in\mathbb{C}$ are two roots of $x^n-2$, then they're also contained in $E$; hence, $\zeta_n = e^{2\pi i/n}=(\sqrt[n]{2}e^{2\pi i/n})/(\sqrt[n]{2})  \in E$.
    Also, because $\alpha\in\mathbb{C}$ is a root of $x^n-2$, $\alpha\in E$.
    Hence, this implies $\mathbb{Q}(\alpha,\zeta_n)\subseteq E$.

    So, $E=\mathbb{Q}(\alpha,\zeta_n)$.
    
    \hfil

    \item[(b)] We'll prove the two cases separately, given that $n\geq 3$.
    \begin{itemize}
        \item First, to prove that $E\neq \mathbb{Q}(\zeta_n)$, we'll consider its dimension: 
        
        Since $\mathbb{Q}(\alpha)\subseteq E$ (since $\alpha\in E$), and $\alpha^n-2=0$, this shows that $\alpha$ has its minimal polynomial over $\mathbb{Q}$ divides $x^n-2\in\mathbb{Q}[x]$; on the other hand, since $x^n-2$ is monic, and it satisfies Eisenstein Criterion with prime $p=2$,
        it is in fact irreducible, so $x^n-2$ is the minimal polynomial of $\alpha$, hence $\mathbb{Q}(\alpha)\cong \mathbb{Q}[x]/(x^n-2)$, which has dimension $n$ when $\mathbb{Q}$ is a base field.
        Hence, $E$ as a $\mathbb{Q}$-vector space has dimension at least $n$ (since $\mathbb{Q}(\alpha)$ is a $\mathbb{Q}$-linear subspace with dimension $n$), so $n \leq [E:\mathbb{Q}]$.

        On the other hand, since $\zeta_n$ satisfies $\zeta_n^n-1 = 0$, or $(\zeta_n-1)(\sum_{k=0}^{n-1}(\zeta_n)^k)=0$. Since $\zeta_n\neq 1$ for $n\geq 3$, then $(\zeta_n-1)\neq 0$. for the equality to hold, we need $\sum_{k=0}^{n-1}(\zeta_n)^k=0$.
        Hence, $\zeta_n$ is a root of the polynomial $\sum_{k=0}^{n-1}x^k\in\mathbb{Q}[x]$, showing that its minimal polynomial $m_{\zeta_n,\mathbb{Q}}(x)\in\mathbb{Q}[x]$ has degree at most $n-1$ (since $m_{\zeta_n,\mathbb{Q}}(x)\mid (\sum_{k=0}^{n-1}x^k)$).
        Therefore, $\mathbb{Q}(\zeta_n)\cong \mathbb{Q}[x]/(m_{\zeta_n,\mathbb{Q}}(x))$ has dimension at most $n-1$, or $[\mathbb{Q}(\zeta_n):\mathbb{Q}]\leq (n-1)$.

        Because $[\mathbb{Q}(\zeta_n):\mathbb{Q}]\leq (n-1) < n \leq [E:\mathbb{Q}]$ from above, this implies that $\mathbb{Q}(\zeta_n)\neq E$.

        \item Then, to prove that $E\neq \mathbb{Q}(\alpha)$, we'll consider the following:
        
        Since all roots of $x^n-2$ have minimal polynomial being $x^n-2$ (proven in the first part of (b), that $x^n-2$ is both monic and irreducible over $\mathbb{Q}$), then $\mathbb{Q}(\alpha)\cong \mathbb{Q}[x]/(x^n-2)\cong \mathbb{Q}(\sqrt[n]{2})$. And, an explicit field isomorphism is given by $\phi:\mathbb{Q}(\alpha)\rightarrow\mathbb{Q}(\sqrt[n]{2})$, $\phi(1)=1$, and $\phi(\alpha)=\sqrt[n]{2}$.
        (Note: The above is based on the fact that the maps $\bar{x}\mapsto \alpha$ and $\bar{x}\mapsto \sqrt[n]{2}$ are two field isomorphisms from the intermediate field to the field of $\mathbb{Q}$ adjoint with $\alpha$ or $\sqrt[n]{2}$ respectively, then taking an inverse in the first one and compose with the second one yields the desired isomorphism).

        Now, notice that such field isomorphism can be generalized to a ring isomorphism $\bar{\phi}:\mathbb{Q}(\alpha)[x]\rightarrow \mathbb{Q}(\sqrt[n]{2})[x]$, such that $\bar{\phi}(a_0+a_1x+...+a_nx^n) = \phi(a_0)+\phi(a_1)x+...+\phi(a_n)x^n$.

        If we suppose the contrary that $E=\mathbb{Q}(\alpha)$, then $x^n-2 = (x-a_1)...(x-a_n)\in \mathbb{Q}(\alpha)[x]$ for some $a_1,...,a_n\in\mathbb{Q}(\alpha)$, which:
        $$x^n-2=\bar{\phi}(x^n-2) = \bar{\phi}((x-a_1)...(x-a_n))=\bar{\phi}(x-a_1)...\bar{\phi}(x-a_n) = (x-\phi(a_1))...(x-\phi(a_n))\in\mathbb{Q}(\sqrt[n]{2})[x]$$
        This shows that $x^n-2$ in fact splits completely over the field $\mathbb{Q}(\sqrt[n]{2})$.

        However, since $\mathbb{Q}(\sqrt[n]{2})\subseteq\mathbb{C}$ is a field that $x^n-2$ splits completely, by definition, $E\subseteq \mathbb{Q}(\sqrt[n]{2})$, showing that $\zeta_n\in E\subseteq \mathbb{Q}(\sqrt[n]{2})$.
        Yet, if $n\geq 3$, since $\zeta_n\notin \mathbb{R}$, then $\zeta_n\in\mathbb{Q}(\sqrt[n]{2})\subseteq\mathbb{R}$ is a contradiction.
        Hence, our assumption is false, $E\neq \mathbb{Q}(\alpha)$.
    \end{itemize}
\end{itemize}

\break

\section*{5}
\begin{myBox}[]{}
    \begin{question}

        \hfil

        \begin{itemize}
            \item[(a)] Find $[\mathbb{Q}(\sqrt[10]{2}):\mathbb{Q}(\sqrt{2})]$.
            \item[(b)] Prove that $x^5-2$ is irreducible over $\mathbb{Q}(\sqrt{2})[x]$. 
        \end{itemize}
    \end{question}
\end{myBox}

\textbf{Pf:}

\begin{itemize}
    \item[(a)] First, consider the extension $\mathbb{Q}(\sqrt[10]{2})/\mathbb{Q}$: Given $x^{10}-2\in\mathbb{Z}[x]\subseteq\mathbb{Q}[x]$, since with prime $p=2$, $x^{10}-2$ satisfies the Eisenstein Criterion,
    then it is in fact irreducible over $\mathbb{Q}[x]$.

    Now, since $\sqrt[10]{2}$ satisfies $(\sqrt[10]{2})^{10}-2=2-2=0$, it is a root of $x^{10}-2$.
    Then, because $x^{10}-2$ is monic and irreudicible, it must be the minimal polynomial of $\sqrt[10]{2}$ over $\mathbb{Q}$.
    So, this implies that $\mathbb{Q}(\sqrt[10]{2})\cong \mathbb{Q}[x]/(x^{10}-2)$, which $[\mathbb{Q}(\sqrt[10]{2}):\mathbb{Q}]=[\mathbb{Q}[x]/(x^{10}-2):\mathbb{Q}]=10$ (the degree of $x^{10}-2$).

    Then, because $\mathbb{Q}\subseteq \mathbb{Q}(\sqrt{2})\subseteq \mathbb{Q}(\sqrt[10]{2})$ (since $(\sqrt[10]{2})^5 = \sqrt{2}$, proving that both $\mathbb{Q},\{\sqrt{2}\}\subseteq \mathbb{Q}(\sqrt[10]{2})$,
    then $\mathbb{Q}(\sqrt{2})\subseteq \mathbb{Q}(\sqrt[10]{2})$), then by the theorem proven in class, we know:
    $$10=[\mathbb{Q}(\sqrt[10]{2}):\mathbb{Q}]=[\mathbb{Q}(\sqrt[10]{2}):\mathbb{Q}(\sqrt{2})]\cdot [\mathbb{Q}(\sqrt{2}):\mathbb{Q}]$$
    Now, because of the same arument about Eisenstein Criterion, $x^2-2\in\mathbb{Q}[x]$ is irreducible, and since $\sqrt{2}$ is a root of it, while $x^2-2$ is being both irreducible and monic,
    then it is in fact the minimal polynomial of $\sqrt{2}$ over $\mathbb{Q}$. Then, $\mathbb{Q}(\sqrt{2})\cong\mathbb{Q}[x]/(x^2-2)$, showing that $[\mathbb{Q}(\sqrt{2}):\mathbb{Q}]=[\mathbb{Q}[x]/(x^2-2):\mathbb{Q}]=2$.

    Hence, we get the following:
    $$10=[\mathbb{Q}(\sqrt[10]{2}):\mathbb{Q}(\sqrt{2})]\cdot [\mathbb{Q}(\sqrt{2}):\mathbb{Q}]=[\mathbb{Q}(\sqrt[10]{2}):\mathbb{Q}(\sqrt{2})]\cdot 2$$
    $$\implies [\mathbb{Q}(\sqrt[10]{2}):\mathbb{Q}(\sqrt{2})]=5$$

    \hfil

    \item[(b)] First, consider the element $\sqrt[5]{2}=(\sqrt[10]{2})^2 \in \mathbb{Q}(\sqrt[10]{2})$: Since it satisfies $(\sqrt[5]{2})^5-2 = 2-2 = 0$, then it is a root of $x^5-2 \in \mathbb{Q}(\sqrt{2})[x]$.
    Hence, let $m_{\sqrt[5]{2}}(x)\in\mathbb{Q}(\sqrt{2})[x]$ be the minimal polynomial of $\sqrt[5]{2}$ over $\mathbb{Q}$, it satisfies $m_{\sqrt[5]{2}}(x)\mid x^5-2$.

    Now, to prove the statement, suppose the contrary that $x^5-2\in\mathbb{Q}(\sqrt{2})[x]$ is not the minimal polynomial of $m_{\sqrt[5]{2}}(x)$ over $\mathbb{Q}(\sqrt{2})$, then this enforces $\deg(m_{\sqrt[5]{2}})<\deg(x^5-2)=5$
    (or else since $x^5-2 = k(x)\cdots m_{\sqrt[5]{2}}(x)$, if $\deg(m_{\sqrt[5]{2}})=5$, then $k(x)$ is a constant $k\in\mathbb{Q}(\sqrt{2})$; but since both $x^5-2$ and $m_{\sqrt[5]{2}}(x)$ are monic, $k=1$, showing that $x^5-2 = m_{\sqrt[5]{2}}(x)$).
    Which, $\mathbb{Q}(\sqrt{2})(\sqrt[5]{2})$ is field isomorphic to $\mathbb{Q}(\sqrt{2})[x]/(m_{\sqrt[5]{2}}(x))$, which has degree given as follow:
    $$[\mathbb{Q}(\sqrt{2})(\sqrt[5]{2}):\mathbb{Q}(\sqrt{2})]=[\mathbb{Q}(\sqrt{2})[x]/(m_{\sqrt[5]{2}}(x)):\mathbb{Q}(\sqrt{2})]=\deg(m_{\sqrt[5]{2}}(x))<5$$
    However, since $(\sqrt{2})/(\sqrt[5]{2})^2 = \sqrt[10]{2}\in \mathbb{Q}(\sqrt{2})(\sqrt[5]{2})$, then as a $\mathbb{Q}(\sqrt{2})$-vector space, we have $\mathbb{Q}(\sqrt[10]{2})\subseteq\mathbb{Q}(\sqrt{2})(\sqrt[5]{2})$;
    yet, in \textbf{part (a)} we've proven that $[\mathbb{Q}(\sqrt[10]{2}):\mathbb{Q}(\sqrt{2})]=5$, showing that $\mathbb{Q}(\sqrt[10]{2})$ is a $5$-dimensional $\mathbb{Q}(\sqrt{2})$-linear subspace of $\mathbb{Q}(\sqrt{2})(\sqrt[5]{2})$, a $\mathbb{Q}(\sqrt{2})$-vector space with dimension strictly less than $5$,
    which is a contradiction.

    So, our assumption is false, $x^5-2\in\mathbb{Q}(\sqrt{2})[x]$ is in fact the minimal polynomial of $\sqrt[5]{2}$ over $\mathbb{Q}(\sqrt{2})$, which must be irreducible.
\end{itemize}

\end{document}