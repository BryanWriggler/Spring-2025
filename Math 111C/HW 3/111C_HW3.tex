\documentclass{article}
\usepackage{graphicx} % Required for inserting images
\usepackage[margin = 2.54cm]{geometry}
\usepackage[most]{tcolorbox}

\newtcolorbox{myBox}[3]{
arc=5mm,
lower separated=false,
fonttitle=\bfseries,
%colbacktitle=green!10,
%coltitle=green!50!black,
enhanced,
attach boxed title to top left={xshift=0.5cm,
        yshift=-2mm},
colframe=blue!50!black,
colback=blue!10
}

\usepackage{amsmath}
\usepackage{amssymb}
\usepackage{verbatim}
\usepackage[utf8]{inputenc}
\linespread{1.2}

\newtheorem{definition}{Definition}
\newtheorem{proposition}{Proposition}
\newtheorem{theorem}{Theorem}
\newtheorem{question}{Question}

\title{Latex Template}
\author{Zih-Yu Hsieh}

\begin{document}
\maketitle

\section*{1}
\begin{myBox}[]{}
    \begin{question}
        Let $H/F$ be a field extension and $f(x), g(x) \in F[x]$. Suppose $K_1,\ K_2$ are splitting
        fields of $f(x)$ and $g(x)$ respectively, contained in $H$. Prove that $K_1K_2 \coloneqq K_1(K_2)$ is a
        splitting field of the polynomial $f(x)g(x)$.
    \end{question}
\end{myBox}

\textbf{Pf:}

\break

\section*{2}
\begin{myBox}[]{}
    \begin{question}
        Define the set of algebraic numbers $\mathbb{A}$ to be the set of all complex numbers which are
        algebraic over $\mathbb{Q}$. Show that $\mathbb{A}/\mathbb{Q}$ is an infinite algebraic extension.
    \end{question}
\end{myBox}

\textbf{Pf:}

\break

\section*{3}
\begin{myBox}[]{}
    \begin{question}
        Let $n\in\mathbb{N}$ and $\mu_n$ be the (multiplicative) group of $n^{th}$ roots of unity in $\mathbb{C}$. A
        generator of $\mu_n$ is called a primitive $n^{th}$ root of unity. Let $F_n\subseteq\mathbb{C}$ be the splitting
        field of $x^n-1$ over $\mathbb{Q}$.
        \begin{itemize}
            \item[(a)]If $\zeta_n$ is any primitive $n^{th}$ root of unity, prove that $F_n=\mathbb{Q}(\zetan)$
            \item[(b)]Prove that any complex root of $m_{\zeta_n,\mathbb{Q}}(x)$ is also a primitive $n^{th}$ root of unity.
            \item[(c)]Prove that $[F_n:\mathbb{Q}]\leq\phi(n)$ where $\phi$ is the famous Euler's totient function.
        \end{itemize}
    \end{question}
\end{myBox}

\textbf{Pf:}

\break

\section*{4}
\begin{myBox}[]{}
    \begin{question}
        Let $f(x)=x^n-2\in\mathbb{Q}[x]$ with $n\in\mathbb{N}$. Let $\alpha\in\mathbb{C}$ be any $n^{th}$ root of $2$ and $\zeta_n=e^{2\pi i/n}$.
        Let $E\subseteq \mathbb{C}$ be the splitting field of $f(x)$ over $\mathbb{Q}$.
        \begin{itemize}
            \item[(a)] Show that $E=\mathbb{Q}(\alpha,\zeta_n)$
            \item[(b)] Let $n\geq 3$. Prove that $E\neq\mathbb{Q}(\alpha)$ and $E\neq\mathbb{Q}(\zeta_n)$. 
        \end{itemize}
    \end{question}
\end{myBox}

\textbf{Pf:}

\break

\section*{5}
\begin{myBox}[]{}
    \begin{question}
        \begin{itemize}
            \item[(a)] Fine $[\mathbb{Q}(\sqrt[10]{2}):\mathbb{Q}(\sqrt{2})]$
            \item[(b)] Prove that $x^5-2$ is irreducible over $\mathbb{Q}(\sqrt{2})[x]$ 
        \end{itemize}
    \end{question}
\end{myBox}

\textbf{Pf:}

\end{document}